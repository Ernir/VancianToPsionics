\subsection[Paladin]{The Paladin}
\label{sec:Paladin}
\begin{quote}
\emph{Where evil lurks, that is where I stand vigilant.}
-Tulkas, half-giant Paladin
\end{quote}
A Paladin is a hero who has dedicated his life and soul to the promotion of good and destruction of evil, and gained divine powers in return.

\begin{table*}
\caption{The Paladin}
\label{tab:Paladin}
\makebox[\textwidth]{
\begin{tabular}{llccclccc}
\toprule
	&	&	&	&	&					&\multicolumn{3}{c}{Spellcasting}\\ \cmidrule(r){7-9}
Level	&BAB	&Fort 	&Ref 	&Will 	&Special				&SP/day	&Known&Max level\\
\midrule
1st &+1			&+2 &+0 &+0	&Aura of Good, Divine			&0&1+CMW&1st\\
    &			&&&		&Grace, Smite				&&&\\
2nd &+2 		&+3 &+0 &+0 	&Holy Gift 				&1 &2 &1st\\
3rd &+3 		&+3 &+1 &+1 	&-    					&3 &3 &1st\\
4th &+4 		&+4 &+1 &+1 	&-    					&5 &4 &2nd\\
5th &+5 		&+4 &+1 &+1 	&Holy Gift   				&7 &5 &2nd\\
6th &+6/+1 		&+5 &+2 &+2 	&-    					&11 &6 &2nd\\
7th &+7/+2 		&+5 &+2 &+2 	&-    					&15 &7 &3rd\\
8th &+8/+3 		&+6 &+2 &+2 	&Holy Gift   				&19 &8 &3rd\\
9th &+9/+4 		&+6 &+3 &+3 	&-    					&23 &9 &3rd\\
10th &+10/+5		&+7 &+3 &+3 	&-    					&27 &10 &4th\\
11th &+11/+6/+1		&+7 &+3 &+3 	&Holy Gift   				&35 &11 &4th\\
12th &+12/+7/+2 	&+8 &+4 &+4 	&-    					&43 &12 &4th\\
13th &+13/+8/+3 	&+8 &+4 &+4 	&-    					&51 &13 &5th\\
14th &+14/+9/+4 	&+9 &+4 &+4 	&Holy Gift   				&59 &14 &5th\\
15th &+15/+10/+5	&+9 &+5 &+5 	&-    					&67 &15 &5th\\
16th &+16/+11/+6/+1 	&+10 &+5 &+5 	&-    					&79 &16 &6th\\
17th &+17/+12/+7/+2 	&+10 &+5 &+5 	&Holy Gift   				&91 &17 &6th\\
18th &+18/+13/+8/+3 	&+11 &+6 &+6 	&-    					&103 &18 &6th\\
19th &+19/+14/+9/+4 	&+11 &+6 &+6 	&-    					&115 &19 &6th\\
20th &+20/+15/+10/+5	&+12 &+6 &+6 	&Holy Gift   				&127 &20 &6th\\
\bottomrule
\end{tabular}}
\end{table*}

\paragraph{Alignment:} Any good. A chaotic good Paladin is often referred to as a Paladin of Freedom. Other Paladins are rarely referred to by a specific title, but lawful good Paladins are sometimes called Paladins of Honor.
\paragraph{Hit Die:} d10
\paragraph{Class skills:}
The Paladin's class skills (and the key ability for each skill) are Concentration (Con), Craft (Int), Diplomacy (Cha), Gather Information (Cha), Handle Animal (Cha), Heal (Wis), Intimidate (Cha), Knowledge (local) (Int), Knowledge (nobility and royalty) (Int), Knowledge (religion) (Int), Knowledge (the planes) (Int), Profession (Wis), Ride (Dex), Sense Motive (Wis), and Spellcraft (Int).

\paragraph{Skill Points at 1st Level:} (4 + Int modifier) $\times$ 4.
\paragraph{Skill Points at each additional Level:} 4 + Int modifier.

\subsubsection{Class Features}
All the following are class features of the Paladin.

\paragraph{Weapon and Armor Proficiency:} 
Paladins are proficient with all simple and martial weapons, 
with all types of armor (heavy, medium, and light, but not exotic armors),
and with shields (except tower shields and exotic shields).

\paragraph{Spell Points/Day:} A Paladin's ability to cast spells is limited by the spell points he has available. 
His base daily allotment of spell points is given on \nameref{tab:Paladin} table. 
In addition, he receives bonus spell points per day if he has a high Charisma score.
His race may also provide bonus spell points per day, as may certain feats and items.

\paragraph{Spells Known:} A Paladin begins play knowing the \nameref{Spell:TouchOfVitality} spell, and one other Paladin spell of your choice. 
Each time he achieves a new level, he unlocks the knowledge of new spells.
Choose the spells known from the Paladin spell list
(Exception: The feats Expanded Knowledge and Epic Expanded Knowledge do allow a Paladin to learn spells of other classes, even specialist Wizard spells.).
A Paladin can cast any spell he knows that has a spell point cost equal to or lower than his caster level.
The number of times a Paladin can cast spells in a day is limited only by his daily spell points. 
A Paladin simply knows his spells; they are ingrained in his mind, though he must get a good night's sleep each day to regain all his spent spell points.
The Difficulty Class for saving throws against Paladin spells is 10 + one-half the number of spell points spent on the spell (round up) + the Paladin's Charisma modifier. 

Spells learned via the Paladin class are divine spells.
\paragraph{Maximum Spell Level Known:} A Paladin begins play with the ability to learn 1st-level spells. 
As he attains higher levels, a Paladin may gain the ability to master more complex spells, as shown on the \nameref{tab:Paladin} table.
To learn or cast a spell, a Paladin must have a Charisma score of at least 10 + the spell's level.

\paragraph{Aura of Good: (Su)} 
At will, as a free action, a Paladin can project a holy aura.
While the aura is active, the Paladin gains a +4 sacred bonus on Diplomacy checks versus Good creatures, and a +4 sacred bonus on Intimidate checks versus Evil creatures (Neutral creatures are not influenced either way).
All sapient creatures become instinctively aware of the Paladin's good alignment while the aura is active.
The Paladin can project this aura indefinitely, or until he dismisses it (another free action). 

\paragraph[Smite]{Smite: (Su)}
\label{sec:Smite}
You can infuse your attacks with supernatural determination and righteous fury.

In order to perform a Smite, you must expend your magical focus as part of making an attack.
The attack then gains a bonus on the attack roll equal to your Charisma modifier, and a bonus on the damage roll equal to your Paladin level.
You must decide whether or not to perform a Smite before making the attack. 
If the attack misses, you still expend your magical focus.
This is a Supernatural ability, activated as part of making an attack.

\paragraph{Divine Grace: (Su)} A Paladin gains a bonus on all saving throws equal to his Charisma modifier or his Paladin level, whichever is lower.
This is a Supernatural ability that functions continuously, requiring no activation.

\paragraph{Holy Gift:}
At 2nd level, the celestial powers the Paladin serves reward him with a gift of new abilities. He gains an additional gift at Paladin levels 5th, 8th, 11th, 17th, and 20th. Unless otherwise noted, a gift can only be selected once. Some gifts have specific prerequisites.

\subparagraph{Additional Feat:}
You gain a bonus feat from the list of feats noted as Fighter bonus feats. You may also select the \nameref{Feat:TurnUndead} feat.
This gift may be selected more than once.

\subparagraph{Aura of Courage (Su):}
You are a fearless champion who inspires his companions to bravery.
You gain immunity to fear, and each ally within 10 feet of you gains a +4 morale bonus on saving throws against fear effects.
This is a Supernatural ability that functions continuously while you are conscious, but not if you are unconscious or dead.

\subparagraph[Celestial Mount]{Celestial Mount:}
\label{sec:CelestialMountListing}
You gain the service of a blessed animal. See the \nameref{sec:CelestialMount} creature description. You must be a 5th-level Paladin to select this gift. If you have the \nameref{Feat:AnimalCompanion} feat, the \nameref{Feat:Familiar} feat, the \nameref{sec:FiendishMountListing} ability, or the \nameref{Feat:SpellstaffUser} feat, you may not select this gift.

Having this Gift counts as having a \hyperref[sec:LeadershipFeats]{Leadership feat}.
\subparagraph{Charging Smite (Ex):}
Your righteous charges are fearsome to behold. When you use your Smite class feature on a charge attack, you add twice your Paladin level to your damage roll instead of simply your Paladin level. You must be a 5th-level Paladin to select this gift.

\subparagraph{Detect Opposition (Su):}
You are an expert in foiling the machinations of others. 
You gain a bonus on all Sense Motive checks and Spot checks to see through disguises equal to your Paladin level.
This is a Supernatural ability that functions continuously.

\subparagraph{Divine Health (Su):}
Your connection to the divine fortifies you against ailments of the body. You gain immunity to all diseases, including supernatural and magical diseases. This is a Supernatural ability that functions continuously.

\subparagraph{Lay on Hands (Su):}
You are blessed with a supernatural ability to effectively heal wounds. Whenever a you cast a \nameref{Spell:TouchOfVitality} spell, you may expend your magical focus.
This infuses the touch with the blessing of the your holy patron, augmenting the spell as if the you had spent
an additional number of spell points on the spell equal to your Paladin level. 
These virtual spell points are supplied by the gift, rather than your own spellcasting ability, 
and thus do not count against the limit imposed by the fundamental rule of magic. 
If you also use your own spell points to augment the spell, they stack with these virtual spell points.
(Usually, this simply simply means that the Paladin may expend 
his magical focus to have his \nameref{Spell:TouchOfVitality} heal a number of additional points equal to twice his Paladin level.)
This is a Supernatural ability, activated as part of casting a \nameref{Spell:TouchOfVitality} spell.

\subparagraph{Miracle (Sp):}
Your calls to the greater powers rarely go unnoticed. Once per day, you can use \nameref{Spell:Miracle} as a spell-like ability. Caster level equals your Paladin level.
You must be a 17th-level Paladin to select this gift.

\subparagraph{Remove Disease (Sp):}
You are blessed with a supernatural ability to cure diseases. You can use \nameref{Spell:RemoveDisease} as a spell-like ability at will. Caster level equals your Paladin level.
You must be a 5th-level Paladin to select this gift.

\subparagraph{Shield Guardian (Su):}
Your defensive efforts benefit your entire party. All allies within 10' of you gain a shield bonus to AC equal to your shield bonus to AC (if any).
This is a Supernatural ability that functions continuously while you are conscious, but not if you are unconscious or dead.

\subparagraph{Stunning Smite (Su):}
Your righteous wrath towards Evil charges your smites against them with holy power, overwhelming their senses.
When you perform a smite attack against an evil creature, it must succeed on a Will save (DC 10 + $1/2$ your HD + your Charisma modifier) or be \emph{stunned} by for 1 round. You must be an 8th-level Paladin to select this gift.

\subparagraph{Vigil (Ex):}
It is no small feat to slip past a Paladin in combat. Your opponents must treat all squares you threaten as difficult terrain, slowing their movement and preventing charges. You must be aware of your opponent for this ability to take effect.

\subparagraph{Vigor (Ex):}
The holy energies that permeate your body make you more resilient than most. You gain damage reduction equal to one-half your Paladin level, minimum 1. This damage reduction is overcome by evil-aligned weapons.

\subsubsection{Code of Conduct:}
Sometimes, the divine patrons that grant Paladins their powers instigate a formal code of conduct to make sure their mortal servants are and remain true paragons of good.

Paladin players and GMs should work together to create a code of conduct appropriate for the campaign and character.

Examples of rules a Paladin has to abide by could be
\begin{itemize}
 \item A Paladin must maintain not only a good alignment, but a lawful good alignment.
 \item A Paladin must never willingly commit an evil act.
 \item A Paladin must respect legitimate authority.
 \item A Paladin must act with honor - he must never lie, cheat, or use poison.
 \item A Paladin must help those in need, provided the help is not used for evil ends.
 \item A Paladin must punish those who harm or threaten innocents.
 \item A Paladin may never knowingly associate with evil characters.
\end{itemize}
\subsubsection{Ex-Paladins:}
A Paladin who changes to a nongood alignment or grossly violates his code of conduct (if any) loses his ability to cast Paladin spells, the Aura of Good class feature, the Divine Grace class feature, the Smite Class feature, and all supernatural Holy Gifts.
The spellcasting and other abilities remain dormant until he atones (see the \nameref{Spell:Atonement} spell description).

\subsubsection[Antipaladin]{Variant: Antipaladin}
While the champions of good are well known, the forces of evil are no less willing to accept cruel mortals into their fold. Such a villain, an Antipaladin, is similar to a \nameref{sec:Blackguard}, but there is no fall or shift to evil involved - an Antipaladin is rotten from the start. 

To make an Antipaladin, replace Paladin class features with Blackguard class features in the following ways:

\paragraph{Alignment:} Any evil. Lawful evil Antipaladins are referred to as Paladins of Tyranny, while chaotic evil Antipaladins are called Paladins of Slaughter.
\paragraph{Class Skills:} As \nameref{sec:Blackguard}.

\paragraph{Class Features:}
The Antipaladin has all the standard Paladin class features, except as noted below.

\subparagraph{Spells:} Use the \nameref{sec:BlackguardSpellList}.
\subparagraph{Aura of Good:} Replace with \nameref{sec:AuroOfEvil}.
\subparagraph{Holy Gift:} Antipaladins select \nameref{sec:UnholyGift}s rather than Holy Gifts. For any Unholy Gift that requires a specific number $n$ of Blackguard levels, replace the level requirement with a requirement $n+6$ of Antipaladin levels.


