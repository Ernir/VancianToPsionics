\subsection[Cleric]{The Cleric}
\begin{quote}
\emph{The powers of the outer planes are real, and the work I do is proof of that.}
- \'Imel\'ia, halfling Cleric
\end{quote}
When a mortal places his faith in a higher power, sometimes power is invested in the mortal in turn.
These mortals are known as Clerics.

\begin{table*}
\centering
\caption{The Cleric}
\label{tab:Cleric}
\makebox[\textwidth]{
\begin{tabular}{llccclcc}
\toprule
	&	&	&	&	&		&\multicolumn{2}{c}{Spellcasting}\\ \cmidrule(r){7-8}
Level	&BAB	&Fort 	&Ref 	&Will 	&Special	&SP/day	&Known\\
\midrule
1st	&+0		&+2	&+0	&+2	&Domains (2)		&3	&2+CMW\\
2nd	&+1		&+3	&+0	&+3	&-			&6	&4\\
3rd	&+2		&+3	&+1	&+3	&-			&10	&6\\
4th	&+3		&+4	&+1	&+4	&-			&16	&7\\
5th	&+3		&+4	&+1	&+4	&Domain			&24	&8\\
6th	&+4		&+5	&+2	&+5	&-			&33	&10\\
7th	&+5		&+5	&+2	&+5	&-			&43	&11\\
8th	&+6/+1		&+6	&+2	&+6	&-			&55	&12\\
9th	&+6/+1		&+6	&+3	&+6	&-			&69	&14\\
10th	&+7/+1		&+7	&+3	&+7	&Domain			&84	&15\\
11th	&+8/+3		&+7	&+3	&+7	&-			&100	&16\\
12th	&+9/+4		&+8	&+4	&+8	&-			&118	&18\\
13th	&+9/+4		&+8	&+4	&+8	&-			&138	&19\\
14th	&+10/+5		&+9	&+4	&+9	&-			&159	&20\\
15th	&+11/+6/+1	&+9	&+5	&+9	&Domain			&181	&22\\
16th	&+12/+7/+2	&+10	&+5	&+10	&-			&205	&23\\
17th	&+12/+7/+2	&+10	&+5	&+10	&-			&231	&24\\
18th	&+13/+8/+3	&+11	&+6	&+11	&-			&258	&26\\
19th	&+14/+9/+4	&+11	&+6	&+11	&-			&286	&27\\
20th	&+15/+10/+5	&+12	&+6	&+12	&-			&316	&28\\
\bottomrule
\end{tabular}}
\end{table*}

\paragraph{Alignment:}
The alignment of a Cleric who worships a deity must be within one step of that of his deity  (that is, it may be one step away on either the lawful-chaotic axis or the good-evil axis, but not both). A deity-devoted Cleric may not be neutral unless his deity's alignment is also neutral.
\paragraph{Hit Die:} d8
\paragraph{Class skills:}
The Cleric's class skills (and the key ability for each skill) are 
Concentration (Con), Craft (Int), Diplomacy (Cha), Heal (Wis), Knowledge (arcana) (Int), Knowledge (history) (Int), Knowledge (religion) (Int), Knowledge (the planes) (Int), Profession (Wis), and Spellcraft (Int).

A Cleric's domains may grant him additional class skills.
\paragraph{Skill Points at 1st Level:} (4 + Int modifier) $\times$ 4.
\paragraph{Skill Points at each additional Level:} 4 + Int modifier.

\subsubsection{Class Features}
All the following are class features of the Cleric.

\paragraph{Weapon and Armor Proficiency:} 
Clerics are proficient with all simple weapons, as well as the favored weapon of their deity.
They are proficient with light and medium armor, and with shields (except tower shields and exotic shields).

\paragraph{Spell Points/Day:} 
A Cleric's ability to cast spells is limited by the spell points he has available. 
His base daily allotment of spell points is given on \nameref{tab:Cleric} table. 
In addition, he receives bonus spell points per day if he has a high Wisdom score.
His race may also provide bonus spell points per day, as may certain feats and items.

\paragraph{Spells Known:} A Cleric begins play knowing two Cleric spells of your choice, 
as well as the \nameref{Spell:TouchOfVitality} spell. 
Each time he achieves a new level, he unlocks the knowledge of new spells.
A Cleric's class spell list is the set of all spells that appear on the general \nameref{Domain:General} spell list or on the spell lists of one or more of the domains belonging to his deity. However, he must choose his spells known from only the generic spell list or from the lists of the domains he has available (see Domains, below).
(Exception: The feats Expanded Knowledge and Epic Expanded Knowledge do allow a Cleric to learn spells of other classes, 
including spells restricted to specialist Wizards.) 

Unlike most spellcasting classes, Clerics do not have a set maximum spell level known.
Instead, they can learn any spell on their domain lists as long as they can pay the spell's minimum spell point cost.
If a Cleric already knows all generic Cleric spells and all spells on his domain lists for which he qualifies when he is entitled to a new spell known, he does not gain one, but can (and must) select one immediately if he ever gains access to a new domain, or if his caster level increases to the point where he unlocks new spells on his pre-existing spell lists.

A Cleric can cast any spell he knows that has a spell point cost equal to or lower than his caster level.
The number of times a Cleric can cast spells in a day is limited only by his daily spell points. 
A Cleric simply knows his spells; they are ingrained in his mind, though he must get a good night's sleep each day to regain all his spent spell points.
The Difficulty Class for saving throws against Cleric spells is 10 + one-half the number of spell points spent on the spell (round up) + the Cleric's Wisdom modifier. 

Spells learned via the Cleric class are divine spells.
\paragraph{Domains:}
A Cleric's deity influences his alignment, what magic he can perform, his values, and how others see him. 
At first level, a Cleric chooses two domains from among those belonging to his deity.
He may select an additional domain on the levels indicated on the \nameref{tab:Cleric} table.

A Cleric who is not devoted to a particular deity selects domains that match his personal spiritual inclinations.

The domains a Cleric selects form the backbone of his abilities.
Each domain is divided into two parts, a spell list and a collection of granted powers.
See the \nameref{sec:Spells} chapter for information on individual \nameref{sec:ClericDomains}.
\subsubsection{Ex-Clerics:}
A Cleric who changes to an inappropriate alignment or grossly violates the code of conduct required by his god loses the ability to cast Cleric spells and all domain granted abilities.
The spellcasting and granted abilities remain dormant until he atones (see the \nameref{Spell:Atonement} spell description).
\subsubsection[Druid]{Variant: Druids}\footnote{For ease of referencing, this document ignores the distinction between Alternate Class Features and Variant Classes.}
\label{sec:Druid}
To some Clerics, revering nature and its awesome, intrinsic power is more important than the worship of deities or what they represent.
These Clerics are known as Druids, and are different from the standard Cleric class in several ways, as outlined below:

\paragraph{Alignment:} A Druid must be Neutral good, lawful neutral, neutral, chaotic neutral, or neutral evil.

\paragraph{Class Skills:} A Druid's class skills (and the key ability for each skill) are Concentration (Con), Craft (Int), Handle Animal (Cha), Heal (Wis), Knowledge (nature) (Int), Knowledge (religion) (Int), Profession (Wis), Ride (Dex), Spellcraft (Int), Survival (Wis), and Swim (Str).

A Druid's domains may grant him additional class skills, as for normal Clerics.

\paragraph{Language:} A Druid can learn a special language, known only to Druids. It is referred to as simply \emph{Druidic}.

\paragraph{Class Features:}
The Druid has all the standard Cleric class features, except as noted below.
\subparagraph{Weapon and Armor Proficiency:} A Druid does not gain proficiency with his deity's favored weapon, even if he selects a deity (making him proficient with simple weapons only).
A Druid does not have the standard Cleric's armor proficiency, instead, he is proficient only with padded armor, leather armor, hide armor, light wooden shields, and heavy wooden shields.

\paragraph{Deity and Domains:} A Druid does not gain his powers from a deity. 
He may have a patron deity as any other character can, but this deity is not the source of the Druid's power.
Instead of selecting from a deity's list of available domains, a Druid may select the domains of \emph{Air, Animal, Earth, Fire, Healing, Moon*, Plant, Sun, Travel, Vermin}, and \emph{Water}.

\subsubsection{Variant: Cloistered Cleric}
The Cloistered Cleric spends more time in study and prayer than other Clerics do, and less in martial training. 
He gives up some of the Cleric's combat prowess in exchange for greater skill access.

\paragraph{Hit Die:}
The Cloistered Cleric uses a d6 for his Hit Die (and has hit points at 1st level equal to 6 + Con modifier).

\paragraph{Base Attack Bonus:}
The Cloistered Cleric's lack of martial training means that he uses the poor base attack bonus.

\paragraph{Class Skills:}
The Cloistered Cleric's class skill list includes Decipher Script, Speak Language, and all Knowledge skills. The Cloistered Cleric gains skill points per level equal to 6 + Int modifier (and has this number x4 at 1st level). If he has the \nameref{Domain:Knowledge}, he gains a +2 bonus on all trained knowledge checks.

\paragraph{Class Features:}
The Cloistered Cleric has all the standard Cleric class features, except as noted below.

\subparagraph{Weapon and Armor Proficiency:}
Cloistered Clerics are proficient with only simple weapons and with light armor.

\subparagraph{Lore (Ex):}
You gain the Lore granted power of the \nameref{Domain:Knowledge}.
If you have the Knowledge Domain, you gain a +2 bonus on all Lore checks.

\subparagraph{Deity and Domains:}
Most Cloistered Clerics worship deities associated with knowledge and learning.

In addition to any domains selected from his deity's list, a Cloistered Cleric may select the \nameref{Domain:Knowledge}, even if that  domain is not normally available to Clerics of that deity.
\subsubsection{Variant: Alternate Approaches to Domains}
Clerics' default method of selecting domains becomes problematic if each deity in the GM's setting has very few domains assigned to it, especially if a deity knows fewer than 5 domains.

These alternate approaches are designed to give deity-devoted Clerics back the flexibility they by design should have.

If the options of Clerics are expanded in this way, Druids (see above) should receive similar benefits.
\paragraph{Variant: Domains First}
A different interpretation on domain access is the one of 
Clerics not selecting domains because they are offered by their patron deity, but rather that they choose to worship a patron deity over all others because he closely matches the domains he has selected.

This variant is appropriate in settings where abstract forces (such as good and evil) are constants higher than the deities themselves, or in settings where the deities are dependent upon their worshippers for power.
\paragraph{Variant: Pantheon Worship}
If a Cleric's patron deity belongs to a pantheon of gods, the Cleric will recognize the portfolio and powers of deities other than his patron deity.
Even though the Cleric will see his patron deity and his portfolio as the most important aspects of the faith, he will take up domains other than those offered by his patron deity if the situation demands.

For example, a Cleric of Thor (a god primarily associated with strength and war) might take up the Water domain (a domain which has nothing to do with Thor, but is offered by the god Aegir) if he routinely finds himself fighting campaigns at sea, or a Cleric of Baldr (a god of beauty) who has been charged with protecting the god's temples against molesters might assume the War domain so he might better fulfil his duty.

Alternatively, a Cleric might not be devoted to any particular deity, but rather worships the pantheon as a whole.

This variant is appropriate where the gods form pantheons.
\paragraph{Variant: Sects and Cults}
If a Cleric wants to worship one deity and one deity only, it is an indicator of the deity's portfolio being broad, and encompassing multiple aspects of life. 
The followers of such a multifaceted deity are likely to differ on some aspects of the faith, breaking the body of worshippers into sects and cults. 
The Clerics of each individual sect or cult would then take different domains to represent their interpretations of the deity.

For example, some Clerics of Thor might emphasize his role as a warrior, taking up the domains of Strength and War.
Others would emphasize his role as a god of thunder, taking up the Air domain, or his role as the protector of his extended family, taking up the domains of Good and Protection. 
Yet another group of worshippers might emphasize how Thor was never afraid to use any means necessary to crush his enemies,
their Clerics taking up the domains of Destruction and Evil.

This variant is appropriate where the gods are distant or ill understood by mortals.