\subsection[Bard]{The Bard}
\label{sec:Bard}
\begin{quote}
\emph{``I don't mean to brag, but when I sing, the angels themselves come down from their heavens to admire the brilliance.''}
- Cassius, human Bard
\end{quote}
A Bard is an arcane spellcaster whose magic manifests in the form of music or other creative or inspiring ways.

\begin{table*}
\centering
\caption{The Bard}
\label{tab:Bard}
\makebox[\textwidth]{
\begin{tabular}{llccclccc}
\toprule
	&	&	&	&	&		&\multicolumn{3}{c}{Spellcasting}\\ \cmidrule(r){7-9}
Level	&BAB	&Fort 	&Ref 	&Will 	&Special	&SP/day	&Known&Max level\\
\midrule
1st	&+0		&+0	&+2	&+2	&Bardic Performance, Bardic		&0	&2	&1st\\
	&		&	&	&	&Knowledge, Cantrips			&	&	&\\
2nd	&+1		&+0	&+3	&+3	&-					&2	&3	&1st\\
3rd	&+2		&+1	&+3	&+3	&Bardic Performance			&4	&4	&1st\\
4th	&+3		&+1	&+4	&+4	&-					&8	&5	&2nd\\
5th	&+3		&+1	&+4	&+4	&-					&12	&6	&2nd\\
6th	&+4		&+2	&+5	&+5	&Bardic Performance			&18	&7	&2nd\\
7th	&+5		&+2	&+5	&+5	&-					&24	&8	&3rd\\
8th	&+6/+1		&+2	&+6	&+6	&-					&32	&9	&3rd\\
9th	&+6/+1		&+3	&+6	&+6	&-					&40	&10	&3rd\\
10th	&+7/+2		&+3	&+7	&+7	&Bardic Performance			&50	&11	&4th\\
11th	&+8/+3		&+3	&+7	&+7	&-					&60	&12	&4th\\
12th	&+9/+4		&+4	&+8	&+8	&-					&72	&13	&4th\\
13th	&+9/+4		&+4	&+8	&+8	&-					&84	&14	&5th\\
14th	&+10/+5		&+4	&+9	&+9	&Bardic Performance			&98	&15	&5th\\
15th	&+11/+6/+1	&+5	&+9	&+9	&-					&112	&16	&5th\\
16th	&+12/+7/+2	&+5	&+10	&+10	&-					&128	&17	&6th\\
17th	&+12/+7/+2	&+5	&+10	&+10	&-					&144	&18	&6th\\
18th	&+13/+8/+3	&+6	&+11	&+11	&Bardic Performance			&162	&19	&6th\\
19th	&+14/+9/+4	&+6	&+11	&+11	&-					&180	&20	&6th\\
20th	&+15/+10/+5	&+6	&+12	&+12	&Bardic Performance			&200	&21	&6th\\
\bottomrule
\end{tabular}}
\end{table*}

\paragraph{Alignment:} Any nonlawful
\paragraph{Hit Die:} d6
\paragraph{Class skills:}
The Bard's class skills (and the key ability for each skill) are 
class skills (and the key ability for each skill) are Appraise (Int), Balance (Dex), Bluff (Cha), Climb (Str), Concentration (Con), Craft (Int), Decipher Script (Int), Diplomacy (Cha), Disguise (Cha), Escape Artist (Dex), Gather Information (Cha), Hide (Dex), Jump (Str), Knowledge (all skills, taken individually) (Int), Listen (Wis), Move Silently (Dex), Perform (Cha), Profession (Wis), Sense Motive (Wis), Sleight of Hand (Dex), Speak Language (N/A), Spellcraft (Int), Swim (Str), Tumble (Dex), and Use Magic Device (Cha).
\paragraph{Skill Points at 1st Level:} (6 + Int modifier) $\times$ 4.
\paragraph{Skill Points at each additional Level:} 6 + Int modifier.

\subsubsection{Class Features}
All the following are class features of the Bard.

\paragraph{Weapon and Armor Proficiency:} 
A Bard is proficient with all simple weapons, plus the longsword, rapier, sap, short sword, shortbow, and whip. 
Bards are proficient with light armor and shields (except tower shields).
Armor does not interfere with the casting of spells.

\paragraph{Spell Points/Day:} 
A Bard's ability to cast spells is limited by the spell points he has available. 
His base daily allotment of spell points is given on \nameref{tab:Bard} table. 
In addition, he receives bonus spell points per day if he has a high Charisma score.
His race may also provide bonus spell points per day, as may certain feats and items.

\paragraph{Spells Known:} A Bard begins play knowing two Bard spells of your choice. 
Each time he achieves a new level, he unlocks the knowledge of new spells.
Choose the spells known from the full Bard spell list.
(Exception: The feats Expanded Knowledge and Epic Expanded Knowledge 
do allow a Bard to learn spells of other classes, 
including spells restricted to specialist Wizards.) 

A Bard can cast any spell he knows that has a spell point cost equal to or lower than his caster level.
The number of times a Bard can cast spells in a day is limited only by his daily spell points. 
A Bard simply knows his spells; they are ingrained in his mind, 
though he must get a good night's sleep each day to regain all his spent spell points.
The Difficulty Class for saving throws against Bard spells is 10 + one-half the number of spell points spent on the spell (round up) + the Bard's Charisma modifier. 

Spells learned via the Bard class are arcane spells.
\paragraph{Maximum Spell Level Known:} A Bard begins play with the ability to learn 1st-level spells. 
As he attains higher levels, a Bard may gain the ability to master more complex spells, as shown on \nameref{tab:Bard} table.
To learn or cast a spell, a Bard must have a Charisma score of at least 10 + the spell's level.

\paragraph[Cantrips]{Cantrips (Su):} 
A Bard can use \nameref{sec:Cantrips} as a \nameref{sec:Wizard} can.

\paragraph{Bardic Performance:}
A Bard is trained to use the Perform skill to create magical effects on himself and those around him. At first level, he knows one type of Bardic Performance. At Bard levels 3, 5, 10, 14, 18, and 20, he learns an additional type of Bardic Performance if he has enough ranks in a Perform skill to do so. If he does not have enough ranks in a Perform skill to qualify for a new type of Bardic Performance when he is entitled to one, he does not gain one, but can (and must) select one immediately if he ever gains the required number of ranks in a Perform skill.

% He can use this ability for a number of rounds per day equal to 4 + his Charisma modifier. At each level after 1st a Bard can use Bardic Performance for 2 additional rounds per day. Each round, The Bard can produce any one of the types of Bardic Performance that he has mastered, as indicated by his level.

Starting a Bardic Performance is a standard action and requires the expenditure of the Bard's magical focus, but it can be maintained each round as a free action. 
Changing a Bardic Performance from one effect to another requires the Bard to stop the previous Performance and start a new one. 
A Bardic Performance ends immediately if the Bard is killed, paralyzed, stunned, knocked unconscious, or otherwise prevented from taking a free action to perform it each round. 
A Bard's Bardic Performance does not expire, nor does it need to be dismissed, it lasts until the Bard stops maintaining the Performance, and for five rounds thereafter. 
A Bard cannot maintain more than one Bardic Performance in a round (but previous performances still remain in effect until they expire if a Bard ceases to maintain one Bardic Performance in favor of another). 
A Bard cannot cast spells in any round in which he maintains a Bardic Performance, but his actions are not otherwise restricted (unless the use of his Perform skill physically requires it).

Each Bardic Performance has audible components, visual components, or both, as appropriate to the Perform skill being applied.

If a Bardic Performance has audible components, the targets must be able to hear the Bard for the Performance to have any effect, and such Performances are language dependent. A deaf Bard has a 20\% change to fail when attempting to use a Bardic Performance with an audible component. If he fails this check, the action is wasted, and the Bard's magical focus is still expended. Deaf creatures are immune to Bardic Performances with audible components.

If a Bardic Performance has a visual component, the targets must have line of sight to the Bard for the Performance to have any effect. A blind Bard has a 50\% chance to fail when attempting to use a Bardic Performance with a visual component. If he fails this check, the action is wasted, and the Bard's magical focus is still expended. Blind creatures are immune to Bardic Performances with visual components.

The types of Bardic Performance are as follows:
\subparagraph{Counterperformance (Su):}
The Bard can counter magic effects that depend on sound (but not spells that were merely cast with verbal components). 
This can take one of two forms:
 
\begin{itemize}
 \item \emph{Augment saves:} The Bard makes a Perform skill check each round he maintains this form of Counterperformance.
Any creature that can perceive the Bard's Performance that is affected by a sonic or language-dependent magical attack may use the Bard's Perform check result in place of its saving throw if, after the saving throw is rolled, the Perform check result proves to be higher. 
If a creature that perceives a counterperformance is already under the effect of a non-instantaneous sonic or language-dependent magical attack, it gains another saving throw against the effect each round it hears the counterperformance, but it must use the Bard's Perform skill check result for the save. 
This form of Counterperformance does nothing against effects that don't allow saves.
 \item \emph{Performance duel:} To use this ability, the Bard makes a Perform check, opposed by the Perform check of another Bard who is starting or maintaining a Bardic Performance. Both Bards must be able to perceive each others' performances. On a successful check, the Bard using Counterperformance immediately causes the Bardic Performance started or being maintained by the other Bard to end as if the other Bard had failed to maintain the Bardic Performance and the effect then expired.
 Alternatively, this use of Counterperformance can cause one Bardic Performance started and then abandoned (no longer maintained) by another Bard to immediately expire, with no opposed check required. The Bard must have perceived the Performance before it was abandoned for this use to be possible.
 Unlike most uses of Bardic Performance, this kind of Counterperformance is instantaneous in nature
\end{itemize}

A Bard must have 4 ranks in a Perform skill to learn or use Counterperformance.
\subparagraph{Fascinate (Su):}
The Bard can use Bardic Performance to cause one or more creatures to become fascinated with him. % Each creature to be fascinated must be perceive the Performance, and able to pay attention. The Bard must also be able to perceive the creature. 
The distraction of a nearby combat or other dangers prevents the ability from working. There is no limit to the number of creatures a Bard can fascinate at a time.

To use the ability, the Bard makes a Perform check. His check result is the DC for each affected creature's Will save against the effect. If a creature's saving throw succeeds, the Bard cannot attempt to fascinate that creature again for 24 hours. If its saving throw fails, the creature sits quietly and enjoys the Performance, taking no other actions for as long as the Bard maintains the Bardic Performance. While fascinated, a target takes a -4 penalty on skill checks made as reactions, such as Listen and Spot checks. Any potential threat requires the Bard to make another Perform check and allows the creature a new saving throw against a DC equal to the new Perform check result. Any obvious threat, such as someone drawing a weapon, casting a spell, or aiming a ranged weapon at the target, automatically breaks the effect. 
Fascinate is an enchantment (compulsion), mind-affecting ability.

A Bard must have 4 ranks in a Perform skill to learn or use Fascinate.
\subparagraph{Inspire Competence (Su):}
The Bard can use his Bardic Performance to help an ally succeed at a task. 
% The ally must be able to perceive the Performance. The Bard must also be able to perceive the ally.

The ally gets a +2 competence bonus on skill checks with a particular skill as long as the Bardic Performance continues. \href{http://www.giantitp.com/comics/oots0004.html}{Certain uses of this ability are infeasible.} A Bard can't inspire competence in himself. Inspire competence is a mind-affecting ability.

A Bard must have 6 ranks in a Perform skill to learn or use Inspire Competence.
\subparagraph{Inspire Courage (Su):}
The Bard can use Bardic Performance to inspire courage, bolstering himself and his allies.

An affected ally receives a +1 morale bonus on saving throws against charm and fear effects and a +1 morale bonus on attack and weapon damage rolls. At 5th level, and every four Bard levels thereafter, this bonus increases by 1 (+2 at 5th, +3 at 9th, +4 at 13th and +5 at 17th). Inspire Courage is a mind-affecting ability.

A Bard must have 4 ranks in a Perform skill to learn or use Inspire Courage.
\subparagraph{Inspire Dread (Su):}
The Bard can use Bardic Performance to inspire doubts and fear in his enemies.

An affected enemy receives a -1 penalty on saving throws against charm and fear effects and a -1 penalty on attack and weapon damage rolls, with no saving throw allowed. At 5th level, and every four Bard levels thereafter, this penalty increases by -1 (-2 at 5th, -3 at 9th, -4 at 13th and -5 at 17th). Inspire Dread is a mind-affecting ability.

A Bard must have 4 ranks in a Perform skill to learn or use Inspire Dread.
\subparagraph{Inspire Greatness (Su):}
The Bard can use Bardic Performance to inspire greatness in himself or a single willing ally within 30 feet, granting him or her extra fighting capability. 
For every three levels a Bard attains beyond 9th, he can target one additional ally with a single use of this ability (two at 12th level, three at 15th, four at 18th).
A creature inspired with greatness gains temporary hit points equal to 10 + twice its constitution modifier, a +2 competence bonus on attack rolls, a +2 competence bonus on Fortitude saves. Inspire greatness is a mind-affecting ability.

A Bard must have 12 ranks in a Perform skill to learn or use Inspire Greatness.
\subparagraph{Inspire Heroics (Su):}
The Bard can use Bardic Performance to inspire tremendous heroism in himself or a single willing ally.
For every three Bard levels the character attains beyond 15th, he can inspire heroics in one additional creature.
A creature so inspired gains a +4 morale bonus on saving throws and a +4 dodge bonus to AC. Inspire Heroics is a mind-affecting ability.

A Bard must have 18 ranks in a Perform skill to learn or use Inspire Heroics.
\subparagraph{Lullaby (Su):}
The Bard can use Bardic Performance to lull a single creature to sleep.
The creature must succeed on a Will save (DC 10 + $1/2$ the Bard's class level + Bard's Charisma modifier) or fall asleep for the effect's duration.
Creatures whose number of HD is higher than the number of ranks the Bard has in the Perform skill being applied are immune to this effect.

Sleeping creatures are helpless. Slapping or wounding awakens an affected creature, but normal noise does not. 
Awakening a creature is a standard action (an application of the aid another action).

Lullaby is a mind-affecting compulsion.

A Bard must have 6 ranks in a Perform skill to learn or use Inspire Heroics.
\subparagraph{Satire (Su):}
By employing tools like rude gestures and offensive lyrics, the Bard can use Bardic Performance to taunt creatures into attacking him.
All enemies who can perceive the Satire must succeed on a Will save (DC 10 + $1/2$ the Bard's class level + Bard's Charisma modifier) or be forced to attack the Bard in preference over all other available targets.
Affected enemies are not thrown into mindless rage, but must concentrate their efforts on doing the Bard direct harm. When making ranged attacks, they must attempt to strike the Bard. When using offensive spells or supernatural abilities, they must target the Bard with the attack or include him in its area. They may attack the Bard in melee, but they do not have to do so if doing so would cause them obvious harm, such as due to walking into a chasm or provoking an attack of opportunity. They can attack the Bard's allies if that is the best way to get to the Bard - for example, if the Bard's ally provides him with cover against ranged attacks or is standing in the way of a charge.

A Bard must have 6 ranks in a Perform skill to learn or use Satire.
\subparagraph{Song of Freedom (Sp):}
The Bard can use Bardic Performance to cast \nameref{Spell:RemoveCurse} as a spell-like ability. This functions as normal for spell-like abilities, except that it requires the expenditure of the Bard's magical focus (unlike most uses of Bardic Performance, Song of Freedom is instantaneous in nature). The spell-like ability has a caster level equal to the Bard's Bard level.
A Bard can't use song of freedom on himself.

A Bard must have 12 ranks in a Perform skill to learn or use Song of Freedom.
\subparagraph{Song of the Clouded Mind (Sp):}
The Bard can use Bardic Performance to cast \nameref{Spell:Confusion} as a spell-like ability.

This functions as normal for spell-like abilities except that it requires the expenditure of the Bard's magical focus, and its duration is determined as for other uses of Bardic Performance rather than the spell's duration (requiring a free action to maintain on each round, but lasts until the Bard ceases to perform. This overrides the spell's augmentation option.).
The saving throw DC against the Confusion is Charisma-based, and its caster level is equal to the Bard's Bard level.

A Bard must have 12 ranks in a Perform skill to learn or use Song of the Clouded Mind.
\subparagraph{Subliminal Intrusion (Su):}
The Bard can use Bardic Performance to weaken his enemies against mental intrusions.

An affected enemy takes a -1 penalty on will saving throws. At 5th level, and every four Bard levels thereafter, this penalty increases by -1 (-2 at 5th, -3 at 9th, -4 at 13th and -5 at 17th). Subliminal Intrusion is a mind-affecting ability.

A Bard must have 9 ranks in a Perform skill to learn or use Subliminal Intrusion.
\subparagraph{Suggestion (Sp):}
The Bard can use Bardic Performance to cast \nameref{Spell:Suggestion} as a spell-like ability. 
This functions as normal for spell-like abilities except that it requires the expenditure of the Bard's magical focus, and its duration is determined as for other uses of Bardic Performance rather than the spell's duration (requiring a free action to maintain on each round, but lasts until the Bard ceases to Perform and for five rounds thereafter).
The saving throw DC against the Suggestion is Charisma-based, and its caster level is equal to the Bard's Bard level.

A Bard must have 9 ranks in a Perform skill to learn or use Suggestion.
\paragraph{Bardic Knowledge:}
A Bard may make a special Bardic Knowledge check with a bonus equal to his Bard level + his Intelligence modifier to see whether he knows some relevant information about local notable people, 
legendary items, or noteworthy places. (If the Bard has 5 or more ranks in Knowledge (history), he gains a +2 bonus on this check.)

A successful Bardic knowledge check will not reveal the powers of a magic item but may give a hint as to its general function. 
A Bard may not take 10 or take 20 on this check; this sort of knowledge is essentially random.
\begin{tableonecolumn}
\caption{Bardic Knowledge}
\label{tab:BardicKnowledge}
\begin{tabular}{p{0.4cm}p{6cm}}
\toprule
DC &Type of Knowledge\\
\midrule
10 &Common, known by at least a substantial minority of the local population.\\
20 &Uncommon but available, known by only a few people legends.\\
25 &Obscure, known by few, hard to come by.\\
30 &Extremely obscure, known by very few, possibly forgotten by most who once knew it, possibly known only by those who don't understand the significance of the knowledge.\\
\bottomrule
\end{tabular}
\end{tableonecolumn}
\subsubsection{Ex-Bards}
A Bard who becomes lawful in alignment cannot progress in levels as a Bard, though he retains all his Bard abilities.