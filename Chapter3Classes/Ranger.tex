\subsection[Ranger]{The Ranger}
\label{sec:Ranger}
\begin{quote}
\emph{``Leave them to me. I'm an expert on humans.''}
- K\"argon, elven Ranger
\end{quote}

A Ranger is a divine champion of the woodlands and the natural world, who specializes in a distinct fighting style and the destruction of his sworn enemies.
\paragraph{Alignment:} Any.
\paragraph{Hit Die:} d8
\paragraph{Class skills:}
The Ranger's class skills (and the key ability for each skill) are Climb (Str), Concentration (Con), Craft (Int), Handle Animal (Cha), Heal (Wis), Hide (Dex), Jump (Str), Knowledge (dungeoneering) (Int), Knowledge (geography) (Int), Knowledge (nature) (Int), Listen (Wis), Move Silently (Dex), Profession (Wis), Ride (Dex), Search (Int), Spot (Wis), Survival (Wis), Swim (Str), Tumble (Dex) and Use Rope (Dex).

\paragraph{Skill Points at 1st Level:} (6 + Int modifier) $\times$ 4.
\paragraph{Skill Points at each additional Level:} 6 + Int modifier.
\begin{table*}
\caption{The Ranger}
\label{tab:Ranger}
\makebox[\textwidth]{
\begin{tabular}{llccclccc}
\toprule
	&	&	&	&	&					&\multicolumn{3}{c}{Spellcasting}\\ \cmidrule(r){7-9}
Level	&BAB	&Fort 	&Ref 	&Will 	&Special				&SP/day	&Known&Max level\\
\midrule
1st &+1			&+2 &+2 &+0	&Combat Style, Favored			&0&1&1st\\
    &			&&&		&Enemy, Wild Empathy			&&&\\
2nd &+2 		&+3 &+2 &+0 	&Bonus Feat				&1 &2 &1st\\
3rd &+3 		&+3 &+3 &+1 	&Intuition, Pass without		&3 &3 &1st\\ 
    &			&   &	&	&Trace					&  &  &\\
4th &+4 		&+4 &+4 &+1 	&-					&5 &4 &2nd\\
5th &+5 		&+4 &+4 &+1 	&Favored Enemy				&7 &5 &2nd\\
6th &+6/+1 		&+5 &+5 &+2 	&Bonus Feat, Improved			&11 &6 &2nd\\
    &			&   &	&	&Combat Style				&   &  &\\
7th &+7/+2 		&+5 &+5 &+2 	&-					&15 &7 &3rd\\
8th &+8/+3 		&+6 &+6 &+2 	&Camouflage				&19 &8 &3rd\\
9th &+9/+4 		&+6 &+6 &+3 	&Favored Enemy				&23 &9 &3rd\\
10th &+10/+5		&+7 &+7 &+3 	&Bonus Feat				&27 &10 &4th\\
11th &+11/+6/+1		&+7 &+7 &+3 	&Combat Style Mastery			&35 &11 &4th\\
12th &+12/+7/+2 	&+8 &+8 &+4 	&-					&43 &12 &4th\\
13th &+13/+8/+3 	&+8 &+8 &+4 	&Favored Enemy				&51 &13 &5th\\
14th &+14/+9/+4 	&+9 &+9 &+4 	&Bonus Feat				&59 &14 &5th\\
15th &+15/+10/+5	&+9 &+9 &+5 	&Hide in Plain Sight			&67 &15 &5th\\
16th &+16/+11/+6/+1 	&+10 &+10 &+5 	&-					&79 &16 &6th\\
17th &+17/+12/+7/+2 	&+10 &+10 &+5 	&Favored Enemy				&91 &17 &6th\\
18th &+18/+13/+8/+3 	&+11 &+11 &+6 	&Bonus Feat				&103 &18 &6th\\
19th &+19/+14/+9/+4 	&+11 &+11 &+6 	&-					&115 &19 &6th\\
20th &+20/+15/+10/+5	&+12 &+12 &+6 	&Bonus Feat				&127 &20 &6th\\
\hline
\end{tabular}}
\end{table*}

\subsubsection{Class Features}
All the following are class features of the Ranger.

\paragraph{Weapon and Armor Proficiency:} 
A Ranger is proficient with all simple and martial weapons, and with light armor and shields (except tower shields).

\paragraph{Spell Points/Day:} A Ranger's ability to cast spells is limited by the spell points he has available. 
His base daily allotment of spell points is given on \nameref{tab:Ranger} table. 
In addition, he receives bonus spell points per day if he has a high Wisdom score.
His race may also provide bonus spell points per day, as may certain feats and items.

\paragraph{Spells Known:} A Ranger begins play knowing one Ranger spell of your choice. 
Each time he achieves a new level, he unlocks the knowledge of a new spell.
Choose the spells known from the Ranger spell list
(Exception: The feats Expanded Knowledge and Epic Expanded Knowledge do allow a Ranger to learn spells of other classes, even specialist Wizard spells.).
A Ranger can cast any spell he knows that has a spell point cost equal to or lower than his caster level.
The number of times a Ranger can cast spells in a day is limited only by his daily spell points. 
A Ranger simply knows his spells; they are ingrained in his mind, though he must get a good night's sleep each day to regain all his spent spell points.
The Difficulty Class for saving throws against Ranger spells is 10 + one-half the number of spell points spent on the spell (round up) + the Ranger's Wisdom modifier. 

Spells learned via the Ranger class are divine spells.
\paragraph{Maximum Spell Level Known:} A Ranger begins play with the ability to learn 1st-level spells. 
As he attains higher levels, a Ranger may gain the ability to master more complex spells, as shown on the \nameref{tab:Ranger} table.
To learn or cast a spell, a Ranger must have a Wisdom score of at least 10 + the spell's level.

\paragraph{Combat Style (Ex):}
A Ranger must select one of two combat styles to pursue: archery or two-weapon combat. This choice affects the character's class features but does not restrict his selection of feats or special abilities in any way.

\subparagraph{Archery:} If the Ranger selects archery, he can add the lower of his Ranger level and his Dexterity modifier to ranged weapon damage rolls (but not damage caused by ranged spells, spell-like abilities, or supernatural abilities) in place of his strength modifier, if doing so is advantageous to him. 

\subparagraph{Two-weapon Combat:} If the Ranger selects two-weapon combat, he may ignore the dexterity requirement of the Two-Weapon Fighting Feat and any feat with that feat as a prerequisite, as well as the Dexterity requirement of the Multiweapon Fighting feat. He must still meet the feats' other prerequisites, if any.

\paragraph{Favored Enemy (Ex):}
At 1st level, a Ranger may select a type of creature from among those given on the \nameref{tab:RangerFE} table.

The Ranger gains a +2 bonus on Bluff, Listen, Sense Motive, Spot, and Survival checks when using these skills against creatures of this type. 
Likewise, he deals an extra 1d6 points of damage against such creatures. 
This extra damage applies only to weapon damage rolls, not to damage caused by spells, spell-like abilities, or supernatural abilities.
It applies fully even against creatures that are immune to critical hits or have concealment.

At 5th level and again at Ranger levels 9th, 13th and 17th, the Ranger may select an additional favored enemy from those given on the table. 
In addition, at each such interval, the bonus on skill checks against all your favored enemies (including the one just selected) increases by 2, and the bonus damage increases by one die (d6).

\begin{tableonecolumn}
\caption{Ranger Favored Enemies}
\label{tab:RangerFE}
\begin{tabular}{ll}
\toprule
Type (Subtype)&Type (Subtype)\\
\midrule
Aberration&Humanoid(reptilian)\\
Animal&Magical Beast\\
Construct&Monstrous Humanoid\\
Dragon&Ooze\\
Elemental&Outsider (air)\\
Fey&Outsider (chaotic)\\
Giant&Outsider (earth)\\
Humanoid (aquatic)&Outsider (evil)\\
Humanoid (dwarf)&Outsider (fire)\\
Humanoid (elf)&Outsider (good)\\
Humanoid (goblinoid)&Outsider (lawful)\\
Humanoid (gnoll)&Outsider (native)\\
Humanoid (gnome)&Outsider (water)\\
Humanoid (halfling)&Plant\\
Humanoid (human)&Undead\\
Humanoid (orc)&Vermin\\
\bottomrule
\end{tabular}
\end{tableonecolumn}

If the Ranger chooses humanoids or outsiders as a favored enemy, he must also choose an associated subtype, as indicated on the table. 
If a specific creature falls into more than one category of favored enemy, the Ranger's bonuses do not stack; he simply uses whichever bonus is higher.

\paragraph[Wild Empathy]{Wild Empathy (Ex):}
\label{sec:WildEmpathy}
You can improve the attitude of an animal.
This ability functions just like a Diplomacy check made to improve the attitude of a person. 
You roll 1d20 and add your Ranger level and your Charisma modifier to determine the wild empathy check result.
The typical domestic animal has a starting attitude of indifferent, while wild animals are usually unfriendly.
To use wild empathy, the animal and you must be able to study each other, 
which means that you must be within 30 feet of each other under normal conditions. 
Generally, influencing an animal in this way takes 1 minute but, as with influencing people, it might take more or less time.
You can also use this ability to influence a magical beast with an Intelligence score of 1 or 2, but she takes a -4 penalty on the check.
\paragraph{Bonus Feats:}
At 2nd level, a Ranger gets a bonus feat.
He gains an additional bonus feat at Ranger levels 6th, 10th, 14th, 18th, and 20th. 
These bonus feats must be drawn from the feats noted as fighter bonus feats, or from the following list: \emph{Endurance}, \emph{Track}. 
The Ranger must still meet all prerequisites for the bonus feat, including ability score and base attack bonus minimums as well as class requirements. 
A Ranger cannot choose feats that specifically require levels in the fighter class unless he is a multiclass character with the requisite levels in the fighter class.

These bonus feats are in addition to the feats that a character of any class gains every three levels. 
A Ranger is not limited to fighter bonus feats and the aforementioned list of feats when choosing these other feats.
\paragraph{Intuition (Ex):}
A Ranger more commonly relies on personal insight than on book-learning or social techniques.
Starting at 3rd level, a Ranger may apply his Wisdom modifier to any of the following skills in place of the normally associated ability modifier, if doing so is advantageous to him:
Gather Information, Handle Animal, Knowledge (dungeoneering), Knowledge (geography), Knowledge (local), Knowledge (nature), and Search.

\paragraph{Pass without Trace (Ex):}
You can move through any type of terrain and leave neither footprints nor scent. Tracking you is impossible. You may choose to leave a trail if so desired.

\paragraph{Improved Combat Style (Ex):}
At 6th level, a Ranger's aptitude in the combat style chosen at first level improves.

\subparagraph{Archery:} The Ranger gains the ability to snipe targets far away with unusual proficiency.
To do so, he makes a single ranged weapon attack against a target 30' or more away as a full-round action.
This attack, if it hits, is then automatically a critical threat, and it gains a +4 bonus on the damage roll for every additional attack the Ranger would have been entitled to due to a high Base Attack Bonus. If the critical threat is confirmed, this bonus damage is multiplied as normal. 
For example, an 11th level Ranger making use of this option would gain a +8 bonus on the damage roll due to the two additional attacks he gives up.

\subparagraph{Two-weapon Combat:} The Ranger he gains the ability to strike with both weapons as fast as he could strike with just one.
Whenever he makes an attack of opportunity, or attacks as a standard action, he may choose to attack with both of his weapons instead of just one.
Make the choice whether you also want to use your off hand before making the first attack roll. If you do, both attacks take the standard penalties for fighting with two weapons.
Attacking this way as a standard action is a specific action of its own, it is incompatible with other special attacks that are activated as a standard action, such as the Awesome Blow feat.

\paragraph{Camouflage (Ex):}
A Ranger of 8th level or higher can use the Hide skill in any sort of terrain, even if the terrain doesn't grant cover or concealment.

\paragraph{Combat Style Mastery (Ex):}
At 11th level, a Ranger's aptitude in the combat style chosen at first level improves again.

\subparagraph{Archery:} If he selected archery at 1st level, he gains the ability to ignore the effects of any kind of severe wind on his ranged attacks, including magical wind such as the one generated by the \nameref{Spell:WindWall} spell.

\subparagraph{Two-weapon Combat:} If the ranger selected two-weapon combat at 1st level, he gains the ability to rend with his weapons.
If he hits an opponent with a weapon in each hand in the same round, he automatically deals and additional points of damage equal to 2d6+1 $1/2$ times your Strength modifier. You can only rend once per round, regardless of how many successful attacks you make.

\paragraph{Hide in Plain Sight (Ex):}
A ranger of 15th level or higher can use the Hide skill even while being observed.

% \paragraph{Combat Style Supremacy (Ex):}
% At 19th level, a Ranger's aptitude in his chosen combat style (archery or two-weapon combat) reaches its pinnacle. 
% 
% If he selected archery at 1st level, STUFF
% 
% If the ranger selected two-weapon combat at 1st level, STUFF

\subsubsection{Variant: Wild Ranger}

A Ranger might forgo training in weapon combat in exchange for the ability to take animal form and move swiftly through the woodlands.

\paragraph{Class Features:}
The Wild Ranger has all the standard Ranger class features, except as noted below.
\subparagraph{Weapon and Armor Proficiency:} 
A Wild Ranger is not proficient with martial weapons.

\subparagraph{Combat Style, Improved Combat Style, Combat Style Mastery:}
A Wild Ranger does not gain Combat Style or any improvements thereof.

\subparagraph{Fast Movement (Ex):}
A Wild Ranger gains Fast Movement, as a Barbarian.

\subparagraph{Grow Claws (Su):}
As a swift action, a Wild Ranger can partially transmute himself, replacing his hands with sharp claws. He can dismiss the claws with another swift action.

You gain two natural attacks with your claws, each dealing 1d6 points of damage (assuming a medium-sized Wild Ranger. Smaller and larger Wild Rangers adjust the damage die accordingly). Your claws are primary natural weapons, and use the normal rules associated with such weapons.

Wild Rangers without hands (or comparable appendages) do not enjoy this class feature. Wild Rangers with only one hand gain one claw. Wild Rangers who already have claw attacks on their hand-like appendages can replace their preexisting claws with those granted by this class feature if doing so would be advantageous for them.

\subparagraph{Bonus Feats:} A Wild Ranger does not draw his bonus feats from the list of Fighter bonus feat. He uses the following list instead: \emph{Endurance}, \emph{Flyby Attack}, \emph{Hover}, \emph{Improved Multiattack}, \emph{Improved Natural Attack}, \emph{Multiattack}, \emph{Snatch}, \emph{Track}, \emph{Wingover}. As an additional benefit, he qualifies for feats as if he were a magical beast as well as a member of his real type.

\subparagraph{Wild Shape (Su):} At 6th level, a Wild Ranger gains the \nameref{sec:WildShape} ability of a Cleric with the Animal domain.

\subparagraph{Wild Senses (Ex):} At 6th level, a Wild Ranger's senses improve. He may choose to gain either Darkvision out to 60' or Scent out to a standard 30'.

\subparagraph{Savage Attacks (Ex): } At 11th level, a Wild Ranger's natural attacks improve. He may choose to gain one of the two following abilities:
\begin{itemize}
 \item The \emph{Constrict} special attack, dealing 1d8 + 1.5$\times$ the Ranger's strength modifier.
 \item A rend attack, dealing Rend 1d8+1.5$\times$ the Ranger's strength modifier points of damage if the Ranger hits the same opponent with at least two claw attacks in the same round.
\end{itemize}
