\subsection[Spellbreaker]{The Spellbreaker}
\label{sec:Spellbreaker}
\begin{quote}
\emph{``I thirst... for magic!''.}
-Kael, elven Spellbreaker
\end{quote}

\paragraph{Hit Die:} d10
\paragraph{Class skills:}
The Spellbreaker's class skills (and the key ability for each skill) are Balance (Dex), Bluff (Cha), Climb (Str), Concentration (Con), Craft (Int), Diplomacy (Cha), Disguise (Cha), Gather Information (Cha), Intimidate (Cha), Jump (Str), Knowledge (all skills, taken individually) (Int), Listen (Wis), Profession (Wis), Sense Motive (Wis), Spellcraft (Int), Spot (Wis), Swim (Str), Tumble (Dex).

\paragraph{Skill Points at 1st Level:} (4 + Int modifier) $\times$ 4.
\paragraph{Skill Points at each additional Level:} 4 + Int modifier.

%[01:11:51 GMT] Ërnir: Langaði til að sjá ýmislegt frá Spellbreakerunum í WC3. Magic immunity/resistance thingy, feedback/mana burn (lemur caster, hann missir galdramojo og tekur auka skaða), stela buffum/tennisa debuffs til baka, stela summons.
%[01:13:04 GMT] Ërnir: Það sem mig langaði til að bæta við er stuff til að leech-a frá göldrum sem er castað nálægt honum (ó, castaðirðu transmutation galdri? Þá læri ég að fljúga í smá stund), og það að stela concentration göldrum eins og Incantatrix.

\begin{table*}
\caption{The Spellbreaker}
\label{tab:Spellbreaker}
\makebox[\textwidth]{
\begin{tabular}{llccclccc}
\toprule
	&	&	&	&	&					&\multicolumn{3}{c}{Spellcasting}\\ \cmidrule(r){7-9}
Level	&BAB	&Fort 	&Ref 	&Will 	&Special				&SP/day	&Known&Max level\\
\midrule
1st  &+1 		&+2 &+0 &+2	&Siphon Energy, Spell Resistance	&0   &2 &1st \\
2nd  &+2 		&+3 &+0 &+3 	&Bonus Feat				&1   &3 &1st \\
3rd  &+3 		&+3 &+1 &+3 	&Leech Magic +1				&3   &3 &1st \\
4th  &+4 		&+4 &+1 &+4 	&Selective Resistance			&5   &4 &1st \\
5th  &+5 		&+4 &+1 &+4 	&Dispel Magic (standard)		&7   &5 &2nd \\
6th  &+6/+1 		&+5 &+2 &+5 	&Bonus Feat				&11  &6 &2nd \\
7th  &+7/+2 		&+5 &+2 &+5 	&Manipulate Magic			&15  &6 &2nd \\
8th  &+8/+3 		&+6 &+2 &+6 	&Leech Magic +2				&19  &7 &2nd \\
9th  &+9/+4 		&+6 &+3 &+6 	&Steal Control				&23  &8 &3rd \\
10th &+10/+5		&+7 &+3 &+7 	&Bonus Feat				&27  &8 &3rd \\
11th &+11/+6/+1		&+7 &+3 &+7 	&Antimagic Affinity			&35  &9 &3rd \\
12th &+12/+7/+2		&+8 &+4 &+8 	&-					&43  &10&3rd \\
13th &+13/+8/+3		&+8 &+4 &+8 	&Dispel Magic (swift)			&51  &11&4th \\
14th &+14/+9/+4		&+9 &+4 &+9 	&Bonus Feat				&59  &11&4th \\
15th &+15/+10/+5	&+9 &+5 &+9 	&Arcane Vigor				&67  &12&4th \\
16th &+16/+11/+6/+1	&+10&+5 &+10 	&Leech Magic +3				&79  &13&4th \\
17th &+17/+12/+7/+2	&+10&+5 &+10 	&Disjunction				&91  &14&5th \\
18th &+18/+13/+8/+3	&+11&+6 &+11 	&Bonus Feat				&103 &14&5th \\
19th &+19/+14/+9/+4	&+11&+6 &+11 	&Dispel Magic (immediate)		&115 &16&5th \\
20th &+20/+15/+10/+5	&+12&+6 &+12 	&Arcanist's Doom			&127 &16&5th \\
\bottomrule
\end{tabular}}
\end{table*}

\subsubsection{Class Features}
All the following are class features of the Spellbreaker.

\paragraph{Weapon and Armor Proficiency:} 
Spellbreakers are proficient with all simple and martial weapons, with light and medium armor, and with shields (including tower shields, but not exotic shields).

\paragraph{Spell Resistance (Ex):} A Spellbreaker has spell resistance equal to 12 + his class level. If he already has spell resistance from another source (such as from his race), use the highest value with a +2 bonus.

\paragraph{Spells Known:} A Spellbreaker begins play knowing two Spellbreaker spells of your choice.
At the levels indicated on \nameref{tab:Spellbreaker} table, he unlocks the knowledge of a new spell.
Choose the spells known from the full Spellbreaker spell list.
(Exception: The feats Expanded Knowledge and Epic Expanded Knowledge do allow a Spellbreaker to learn spells of other classes, including spells restricted to specialist Wizards.) 

A Spellbreaker can cast any spell he knows that has a spell point cost equal to or lower than his caster level.
The number of times a Spellbreaker can cast spells in a day is limited only by his daily spell points. 
A Spellbreaker simply knows his spells; they are ingrained in his mind, though he must obtain spell points from other spellcasters in order to take advantage of them (see Spell Points, below).
The Difficulty Class for saving throws against Spellbreaker spells is 10 + one-half the number of spell points spent on the spell (round up) + the Spellbreaker's Intelligence modifier.

Spells learned via the Spellbreaker class are arcane spells.
\paragraph{Spell Points:} A Spellbreaker's ability to cast spells is limited by the spell points he has available. 
Unlike most spellcasters, a Spellbreaker can \emph{not} regain spell points simply by concentrating at the start of the day - the only way a Spellbreaker can regain spent spell points is using his class features to leech them from other spellcasters.
His base spell point capacity is given on the \nameref{tab:Spellbreaker} table. In addition, he receives bonus spell points per day if he has a high Intelligence score.
Provided he can obtain the required spell points, a Spellbreaker has no daily limit on the number of spells he can cast.

A multiclass Spellbreaker with levels in other spellcasting classes faces this same restriction - he can never (again) regain spell points by concentration.
\paragraph{Siphon Energy (Su):} Whenever a Spellbreaker successfully deals damage to a spellcasting opponent with a weapon (not a spell, even if the spell is a ray or otherwise requires a touch attack), the Spellbreaker steals some of his spell energy for himself. The opponent loses a number of spell points equal to the Spellbreaker's class level, the Spellbreaker gains a number of spell points equal to the number the opponent lost, and the opponent takes a number of points of damage equal to twice the number of spell points lost. The struck opponent loses spell points and takes damage even if the Spellbreaker is already at his maximum number of spell points.

Alternatively, a Spellbreaker can siphon any number of spell points from a willing spellcaster by touching him as a standard action (the spellcaster knows the number of spell points being siphoned). This use does not deal damage to the ''donating`` spellcaster.

A spellcaster's spell point reserve can not be reduced below zero by the use of this ability.

Using this ability does not require an action, it is activated as part of making the attack (or standard action) in question.
\paragraph{Bonus Feat:} 
At Spellbreaker levels 2nd, 6th, 10th, 14th and 18th, the Spellbreaker may select any Magical feat as a bonus feat. He must meet the feat's prerequisites, if any.
\paragraph{Leech Magic (Su):} Starting at Spellbreaker level 3rd, you emit an aura that takes its toll on all spellcasting in your vicinity. Each spell cast within 30' has its spell point cost increased by 1. This additional spell point does not count towards the spell's augmentation or other potential benefits of increased spell point expenditure (such as the calculation of the spell's save DC), but it does count towards the limit of it being impossible to spend more spell points on a spell than the caster's caster level. This may make it impossible for a spellcaster to cast his highest level spells. You add the additional spell point to your Spell Point reserve, 

Any spellcaster is aware of this dampening effect as soon as he enters the aura. Should he choose to cast a spell anyway, the additional spell point spent is added to your spell point pool.

As a free action, you can prevent the aura from affecting specified spellcasters within range for one round. You must be aware of the spellcaster and his location to do so.

At Spellbreaker level 8th, the spell point cost is instead increased by 2, and at Spellbreaker level 16th, by 3.
\paragraph{Selective Resistance (Ex):} Starting at Spellbreaker level 4th, you can lower your spell resistance as an immediate action. You retain the ability to lower your spell resistance as a standard action, should you desire to do so.

\paragraph{Dispel Magic (Sp):} Starting at Spellbreaker level 5th, a Spellbreaker can use \nameref{Spell:DispelMagic} at will as a standard action. Caster level equals your number of Hit Dice.

At Spellbreaker level 14th, you can use this ability as a swift action three times per day.

At Spellbreaker level 19th, you can use this ability as an immediate action. You can use the ability as a swift action or as an immediate action a combined number of three times per day.
\paragraph{Manipulate Magic (Su):} Starting at Spellbreaker level 7th, whenever you succeed on a dispel check to dispel an ongoing spell effect affecting a creature, you can instead move the spell to a different creature. The spell continues to affect its new recipient, as if it had been cast on that creature originally.

The spell's recipient must be within 30' of both you and the creature it affected originally (the target of the dispelling effect).
The recipient must be a legal target for the spell in question. You can not move spells with a range of Personal. If the spell being moved allows saving throws and/or spell resistance, the recipient is allowed to apply those defenses as normal. In the case of spells that establish a special connection between the spell's target and its caster (for example: \nameref{Spell:Charm} and \nameref{Spell:ShieldOther}), you are thereafter considered to be the spell's caster.

The spell's duration, caster level, augmentations, metamagic effects, and other variable factors are not reset or changed when the spell is moved.
\paragraph{Steal Control (Su):} Starting at Spellbreaker level 9th, whenever you succeed on a dispel check to dispel an ongoing Conjuration (Summoning) spell that summoned a creature, you can instead assume control of the spell. The spell is then not dispelled, continuing to function as normal, with the exception that it now serves you and obeys your commands as it obeyed the spell's original caster.

\paragraph{Antimagic Affinity (Ex):} Starting at Spellbreaker level 11th, you can use the Supernatural abilities granted by the Spellbreaker class inside the area of an \nameref{Spell:AntimagicField} (or in similar locations, such as planes with dead magic). 

You also gain the ability to shield one of your magic items from the effects of an Antimagic Field as a swift action. The shield lasts for one round. While shielded, the item functions normally. If the item leaves your possession for the duration of the effect, the shield instantly ends.

You continue to lose access to your spellcasting, spell-like abilities, and other supernatural abilities while within an Antimagic Field, as normal.
\paragraph{Arcane Vigor (Su):} Starting at Spellbreaker level 15th, whenever a caster fails to affect you due to your spell resistance class feature, you gain temporary hit points equal to twice the number of spell points spent on the spell. These temporary hit points last for a number of rounds equal to your Spellbreaker class level.

\paragraph{Disjunction (Sp):} Starting at Spellbreaker level 17th, a Spellbreaker can use \nameref{Spell:Disjunction} three times per day. Caster level equals your number of Hit Dice.

\paragraph{Arcanist's Doom (Su):} At Spellbreaker level 20th, you learn to forever destroy a spellcaster's ability to do magic.
Using this ability requires touching the spellcaster throughout one minute of concentration. If carried out to completion, the spellcaster instantly loses all of his spell points, and permanently loses his ability to regain them. Only a \nameref{Spell:Wish} or \nameref{Spell:Miracle} spell can restore the spellcaster's ability to regain spell points.