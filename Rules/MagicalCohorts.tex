\section{Magical Cohorts}
\subsection{Animal Companion}
\label{sec:AnimalCompanion}
An animal companion is a normal animal that gains increased strength and power due to a bond it has formed with a character
who has the \nameref{Feat:AnimalCompanion} feat. 
This animal can be one of the animals on the following list (or another ordinary animal of similar power, at the GM's discretion):
badger, camel, dire rat, dog, riding dog, eagle, hawk, horse (light or heavy), owl, pony, snake (Small or Medium viper), or wolf.
Only a normal, unmodified animal may become an animal companion.

See below for details on how animal companions work.
\subsubsection{Animal Companion Statistics}
An animal that becomes an animal companion retains the appearance, type, speeds, base natural armor, natural attacks, space, reach, special attacks, special qualities, 
physical ability scores, racial bonus feats (but \emph{not} other feats) and alignment of the normal animal it once was. 
Other statistics change, or are replaced entirely, as outlined below.
\begin{itemize}
 \item \textbf{Hit Dice:} An animal companion becomes a creature with a number of hit dice equal to the master's number of levels in spellcasting classes 
 (levels of different spellcasting classes stack), 
 regardless of how many hit dice the original creature had. 
 These hit dice are animal hit dice, with its base attack bonus and base saving throws being modified accordingly.
 When the master gains additional levels in a spellcasting class, the animal companion gains additional animal hit dice. 
 \subitem \textbf{Hit Points:} Animal companions gain hit points as characters do, gaining the maximum possible number of hit points at first HD, 
 and rolling thereafter.
 \item \textbf{Size:} An animal companion grows in size according to its advancement table.
 \item \textbf{Base Attack Bonus and Base Saving Throws:} Recalculate with respect to the animal companion's new number of hit dice (see ``hit dice`` above). 
 Animals use the moderate Base Attack Bonus progression, as Clerics do. They have good fortitude and reflex saving throws.
 \item \textbf{Special Qualities:} An animal companion gains the Evasion ability, as the Rogue class feature.
 \item \textbf{Ability Scores:} An animal companion retains its own Strength, Dexterity, and Constitution scores. 
 Its Intelligence score changes to 2, and its Wisdom and Charisma both change to 10.
 It gains ability score increases as its number of HD increases as any other creature does.
 \item \textbf{Skills:} The animal companion's master may rearrange the base creature's skill ranks when the animal companion is bonded. 
 The animal companion's ''class'' skills are balance, climb, escape artist, hide, intimidate, jump, listen, move silently, spot, survival, swim, tumble. 
 The animal companion retains any racial skill bonuses it may have.
 Note that the animal companion's intelligence, body shape, and tricks known may place severe restrictions on its use of skills.
 \item \textbf{Feats:} The animal companion's master may rearrange the creature's feats (other than racial bonus feats) when the companion is bonded.
 The animal companion gains additional feats as it gains extra hit dice, as other creatures do. 
 The animal companion may choose any feat for which it qualifies, including special \nameref{sec:CompanionFeats}.
\end{itemize}
An animal companion can neither speak nor understand languages, as normal for animals.

The master of an animal companion automatically succeeds on all handle animal checks to handle or push its animal companion, 
and can perform them as a free action.
All others automatically fail these checks.
\subsubsection{Bonding an Animal Companion}
Bonding an animal companion requires finding the creature in question, and succeeding on a Wild Empathy check to render it friendly or helpful.
Bonding it requires an informal ritual that takes 1 hour to complete.
\subsubsection{Animal Companions and Death}
An animal companion can be resurrected or otherwise raised from the dead as a character can. 
Being raised from the dead fully restores it, it does not have experience to lose as characters do, its abilities are a function of its master.

If a character who has an animal companion dies, 
the animal companion does not lose the statistics it has gained due to being an animal companion (unless it so wishes), 
it retains the statistics it had at the time of its master's death.

An orphaned animal companion usually renounces its status as a familiar and becomes a normal animal once again some time after the death of its master.
\subsubsection{Dismissing an Animal Companion}
A master may dismiss his animal companion at any time. 
The animal companion then immediately becomes a normal creature of its type, and returns to the wild 
(although its friendly attitude towards the old master remains).
This allows the master to bond a new companion, following the same rules as bonding one in the first place.
The bonding ritual can also be used to restore the status of a previously dismissed animal companion, if it is found again.

The animal companion itself may also choose to abandon its master, 
but an event dramatic enough to cause a creature of an animal's intelligence to make such a decision is extremely rare
(an example might be the master being slain and resurrected as an undead creature).
In any case, the result is still that the companion loses all of its animal companion abilities, and becomes a normal creature again.
\subsection{Celestial Mount}
\label{sec:CelestialMount}
A celestial mount is a creature from the upper planes, sent to aid a Paladin in his quests. 
This creature can be one of the animals on the following list 
(or another ordinary animal suitable as a mount for the Paladin in question, subject to GM approval.
Any creature so selected should not be more powerful than those that follow):
heavy warhorse, riding dog, warpony, or a large shark (appropriate for aquatic campaigns).

Most celestial mounts are animals (turned into magical beasts due to their celestial nature), 
but some might be magical beasts to begin with, or even dragons or creatures not natively found on the material plane.
They must always be normal, unmodified creatures of their type before taking on the changes that follow when a creature becomes a celestial mount.

See below for details on how celestial mounts work.
\subsubsection{Celestial Mount Statistics}
A creature that becomes a celestial mount retains the appearance, size, speeds, base natural armor, natural attacks, space, reach, special attacks, special qualities, 
physical ability scores and racial bonus feats (but \emph{not} other feats) of the normal animal it once was.
Other statistics change, or are replaced entirely, as outlined below.

These changes include the benefits of the (modified) celestial template that is applied to the creature.
\begin{itemize}
 \item \textbf{Type and subtype:} If the base creature is an animal, it becomes a magical beast.
 Celestial mounts hail from an appropriate celestial plane, and thus gain the extraplanar subtype when on the material plane.
 \item \textbf{Hit Dice:} A celestial mount becomes a creature with a number of hit dice equal to the master's number of levels in spellcasting classes 
 (levels of different spellcasting classes stack), 
 regardless of how many hit dice the original creature had. 
 These hit dice are hit dice corresponding to its type (usually magical beast hit dice), with its base attack bonus and base saving throws being modified accordingly.
 When the master gains additional levels in a spellcasting class, the celestial mount gains additional hit dice. 
 \subitem \textbf{Hit Points:} Celestial mounts gain hit points as characters do, gaining the maximum possible number of hit points at first HD, 
 and rolling thereafter.
 \item \textbf{Size:} A celestial mount's size never changes as a result of gaining additional hit dice, even if the base creature's advancement table would indicate otherwise.
 \item \textbf{Base Attack Bonus and Base Saving Throws:} Recalculate with respect to the celestial mount's new number of hit dice (see ``hit dice`` above). 
 Magical beasts use the best Base Attack Bonus progression, as Fighters do. They have good fortitude and reflex saving throws.
 \item \textbf{Special Attacks:} Once per day, a celestial mount can smite a creature.
 It gains a bonus on an attack roll equal to its charisma modifier, and a bonus on the following damage roll equal to its HD.
 This is a supernatural ability activated as part of making the attack.
 \item \textbf{Special Qualities:} A celestial mount gains the following special qualities:
 \subitem The Evasion ability, as the Rogue class feature.
 \subitem Damage reduction, as normal for a celestial creature.
 \subitem Resistance to acid, cold and electricity as normal for a celestial creature.
 \item \textbf{Ability Scores:} A celestial companion retains its own Strength, Dexterity, and Constitution scores. 
 Those of its mental ability scores (Intelligence, Wisdom, and Charisma) that are below 10 are increased to 10.
 It gains ability score increases as its number of HD increases as any other creature does.
 \item \textbf{Skills:} The celestial mount's master may rearrange the base creature's skill ranks when the mount is called. 
 The celestial mount's ''class'' skills are balance, climb, escape artist, hide, intimidate, jump, listen, move silently, spot, survival, swim, tumble. 
 The celestial mount retains any racial skill bonuses it may have.
 Note that the celestial mount's body shape may place severe restrictions on its use of skills.
 \item \textbf{Feats:} The celestial mount's master may rearrange the creature's feats (other than racial bonus feats) when the mount is called.
 The celestial mount gains additional feats as it gains extra hit dice, as other creatures do. 
 The celestial mount may choose any feat for which it qualifies, including special \nameref{sec:CompanionFeats}.
 \item \textbf{Alignment:} Becomes the same as the alignment of the Paladin at the time of summoning. 
 It typically remains at that alignment, even if its master's alignment changes.
\end{itemize}
A celestial companion knows (can understand, speak, and read, and write if its body allows) 
celestial and one language of its master's choice (so long as it is a language the master knows). 
\subsubsection{Calling a Celestial Mount}
Initially calling a celestial mount from the upper planes requires performing a ritual that takes 1 hour to complete.
The Paladin chooses what kind of mount he receives.
\subsubsection{Celestial Mounts and Death}
Should the celestial mount die, it immediately disappears, leaving behind any equipment it was carrying. 
The Paladin may not call it or another mount for thirty days or until he gains a Paladin level, whichever comes first.
At the end of the period, the Paladin can call it again, using the same ritual used to call it in the first place.
Being called back from the dead fully restores it, it does not have experience to lose as characters do, its abilities are a function of its master.

If a character who has a celestial mount dies, 
it does not lose the statistics it has gained due to being a celestial mount (unless it so wishes), 
it retains the statistics it had at the time of its master's death.

A celestial mount typically does the best it can to facilitate the resurrection of its master, 
continuing to cooperate with its master's allies if that is the best way to bring it back to life.
A celestial mount who deems that quest hopeless usually renounces its status as a mount and returns to its celestial home.
\subsubsection{Dismissing a Celestial Mount}
A Paladin may dismiss his celestial mount at any time, although rarely done without good reason. 
The mount then immediately returns to its celestial home.

This allows the Paladin to call a new mount, following the same rules as calling one in the first place.
However, he must wait thirty days or until he gains a Paladin level before calling another, whichever comes first.
The calling ritual can also be used to restore the status of a previously dismissed celestial mount, if it is found again.

The celestial mount itself may also choose to abandon its master.
This is a rare event, but when it occurs, it is usually due to its master straying from the path of good.
In any case, the result is still that the companion immediately returns to the celestial realms, 
and the Paladin cannot gain the services of a celestial mount for thirty days or until he gains a Paladin level.
\subsection{Familiar}
\label{sec:Familiar}
A familiar is a normal animal that gains new powers and becomes a magical beast when summoned to service by a spellcaster
who has the \nameref{Feat:Familiar} feat. 
This animal can be one of the animals on the following list (or another small, ordinary animal, at the GM's discretion):
Bat, cat, hawk, lizard, owl, rat, raven, snake, toad, or weasel.
Only a normal, unmodified animal may become a familiar.

See below for details on how familiars work.
\subsubsection{Familiar Statistics}
An animal that becomes a familiar retains the appearance, size, speeds, base natural armor, natural attacks, space, reach, special attacks, special qualities, physical ability scores, and racial bonus feats
(but \emph{not} other feats) of the normal animal it once was. Other statistics change, or are replaced entirely, as outlined below.

\begin{itemize}
 \item \textbf{Hit Dice:} A familiar becomes a creature with a number of hit dice equal to the master's number of levels in spellcasting classes (levels of different spellcasting classes stack), 
 regardless of how many hit dice the original creature had. These hit dice are magical beast hit dice, with its base attack bonus and base saving throws being modified accordingly.
 When the master gains additional levels in a spellcasting class, the familiar gains additional magical beast hit dice.
 \subitem \textbf{Hit Points:} Familiars are more fragile than most creatures. They gain only one hit point per hit dice (as if they always roll a one on their HP roll), including at 1st. 
 A familiar still adds its full Constitution modifier to its hit points for each HD it has or gains, as other creatures do.
 \item \textbf{Type:} A familiar becomes a magical beast, losing its previous type and subtypes and all benefits associated with them, and gaining all benefits of the magical beast type, with the specific exception
 of a magical beast's 60' Darkvision. This may change what kind of spells can affect the creature.
 \item \textbf{Size:} A familiar's size never changes as a result of gaining additional hit dice, even if the base creature's advancement table would indicate otherwise.
 \item \textbf{Base Attack Bonus and Base Saving Throws:} Recalculate with respect to the familiar's new number of hit dice (see ``hit dice`` above). 
 Magical beasts use the best Base Attack Bonus progression, as Fighters do. They have good fortitude and reflex saving throws.
 \item \textbf{Special Qualities:} A familiar gains the Evasion ability, as the Rogue class feature.
 \item \textbf{Ability Scores:} A familiar retains its own Strength, Dexterity, and Constitution scores. Its Intelligence changes to 6, and its Wisdom and Charisma both change to 10.
 It gains ability score increases as its number of HD increases as any other creature does.
 \item \textbf{Skills:} The familiar's master may rearrange the base creature's skill ranks when the familiar is summoned. 
 The familiar's ''class'' skills are the same as that of its master. The familiar retains any racial skill bonuses it may have.
 \item \textbf{Feats:} The familiar's master may rearrange the creature's feats (other than racial bonus feats) when the familiar is summoned.
 The familiar gains additional feats as it gains extra hit dice, as other creatures do. The familiar may choose any feat for which it qualifies, including special \nameref{sec:CompanionFeats}.
 \item \textbf{Alignment:} Becomes the same as the alignment of the master at the time of summoning. It typically remains at that alignment, even if its master's alignment changes.
\end{itemize}

A familiar can speak one language of its master's choice (so long as it is a language the owner knows). 
A familiar can understand all other languages known by its master, but cannot speak them. This is a supernatural ability. 
\subsubsection{Summoning a Familiar}
Summoning a familiar requires the expenditure of magical components costing 100 GP, 
and performing a ritual that takes 1 hour to complete.
The spellcaster chooses the kind of familiar he receives.

It is assumed that the familiar comes to the master via magical or mundane means at the end of the summoning ritual, 
rather than appearing out of nowhere.
The GM may place restrictions on what kind of familiars are available (or whether they are available at all), 
based on the locale in which the ritual is cast.
\subsubsection{Familiars and Death}
Resurrecting or replacing a slain Familiar requires this same ritual as summoning one (including the material cost).
This ``resurrection`` or replacement fully restores it, it does not have experience to lose as characters do, its abilities are a function of its master.

If a spellcaster who is the master of a familiar dies, the familiar does not wink out of existance or lose the statistics it has gained due to being a familiar (unless it so wishes), 
it retains the statistics it had at the time of its owner's death.

An orphaned familiar typically does the best it can to facilitate the resurrection of its master, continuing to cooperate with its master's allies if that is the best way to bring it back to life.
A familiar who deems that quest hopeless usually renounces its status as a familiar and becomes a normal animal once again.
\subsubsection{Dismissing a Familiar}
A master may dismiss his familiar at any time. The familiar then immediately becomes a normal creature of its type, and returns to the wild.
This allows the master to summon a new familiar, following the same rules as summoning a familiar in the first place.
The ritual can also be used to restore the status of a previously dismissed familiar.

The familiar itself, being sentient, may also choose to abandon its master.
This very rarely happens, usually as a result of a radical alignment change, or events such as the death (or onset of undeath) of the master.
In any case, the result is still that the familiar loses all of its familiar abilities, and becomes a normal creature again.
\subsection{Fiendish Mount}
\label{sec:FiendishMount}
A fiendish mount is a creature from the lower planes, sent to aid a Blackguard in his atrocities. 

A fiendish mount functions precisely as a \nameref{sec:CelestialMount} does, with the following exceptions:
\begin{itemize}
 \item Replace all references to Paladin with Blackguard.
 \item \textbf{Special Qualities:} A fiendish mount gains the following special qualities:
 \subitem The Evasion ability, as the Rogue class feature.
 \subitem Damage reduction, as normal for a fiendish creature.
 \subitem Resistance to cold and fire, as normal for a fiendish creature.
\end{itemize}

\subsection{Spellstaff}
\label{sec:SpellStaff}
A spellstaff is, as the name suggests, a staff relating to spellcasters.
While usually seen as a simple item belonging to a character, is technically a creature in it own right, much like a \nameref{sec:Familiar}.

Spellstaffs work exactly like Familiars, with the following exceptions:
\begin{itemize}
 \item \textbf{Feat:} In order to obtain a spellstaff, the master must select the Spellstaff User feat (see \nameref{Feat:SpellstaffUser}) rather than the Familiar feat.
 \item \textbf{Body:} Rather than those of an animal, a spellstaff uses the base statistics outlined in the \nameref{sec:SpellStaffMonsterEntry}.
 \item \textbf{Hit Dice and Type:} Rather than becoming a Magical Beast, a spellstaff remains a creature of the construct type. 
 All of its HD are and remain Construct HD, with the appropriate changes in statistics and immunities.
 \subitem \textbf{Hit Points:} Spellstaffs gain hit points as characters do, gaining the maximum possible number of hit points at first HD, 
 and rolling thereafter.
 \item Spellstaffs are created or replaced, rather than summoned or resurrected (see \nameref{sec:CraftingASpellstaff}, below).
\end{itemize}
In all other ways, a spellstaff is considered a familiar.

Those spellcasters who choose to take on a spellstaff rather than the more physically capable companion that is a familiar 
usually do so because of the increased capacity a trained spellstaff user has to focus himself
- represented by the Spellstaff Containment feat (see the \nameref{Feat:SpellstaffContainment} feat description).
\subsubsection{Crafting/Repairing a Spellstaff}
\label{sec:CraftingASpellstaff}
Obtaining a spellstaff requires the expenditure of materials costing 300 GP in a crafting process that takes 1 hour to complete.
Replacing a spellstaff that has been destroyed requires this same amount of time and materials. 
Replacing a spellstaff like that fully restores its abilities.
\subsubsection{Spellstaff Monster Entry}
\label{sec:SpellStaffMonsterEntry}
A Spellstaff is never encountered alone - it is a creature whose fate is inexorably tied to that of a master.
These are the basic statistics of a Spellstaff, which are then heavily modified by its link to its master.
\begin{table*}
\caption{Spellstaff}
\makebox[\textwidth]{
\begin{tabular}{|p{0.3\textwidth}|p{0.7\textwidth}|}
\hline
\textbf{Size/Type:}&Tiny Construct\\
\textbf{Hit Dice}&1d10 (5 HP)\\
\textbf{Initiative}&-5\\
\textbf{Speed}&-\\
\textbf{Armor Class:}& 7 (+2 size, -5 Dex), touch 7, flatfooted 7\\
\textbf{Base Attack/Grapple:}&+0/-13\\
\textbf{Attack:}& -\\
\textbf{Full Attack:}& -\\
\textbf{Space/Reach:}& 2 $1/2$ ft./0 ft.\\
\textbf{Special Attacks:}&-\\
\textbf{Special Qualities:}&Blindsense 20', construct traits, just a walking stick, hardness 8\\
\textbf{Saves:}&Fort +0, Ref +0, Will +0\\
\textbf{Abilities:}&Str -, Dex -, Con -, Int 6, Wis 10, Cha 10\\
\textbf{Skills:}&Unassigned\\
\textbf{Feats:}&Unassigned\\
% \textbf{Environment:}&N/A\\
% \textbf{Organization:}&N/A\\
% \textbf{Treasure:}&None\\
% \textbf{Advancement:}&-\\
% \textbf{Level Adjustment:}&-\\
\hline \end{tabular}}
\end{table*}

\paragraph{Blindsense (Ex):} A spellstaff notices and locates creatures within 20 feet. Opponents still have 100\% concealment against a creature with blindsense. 
 
\paragraph{Construct Traits:} A Spellstaff has immunity to poison, sleep, paralysis, stunning, disease, death effects, 
necromancy effects, mind-affecting effects (charms, compulsions, phantasms, patterns, and morale effects), 
and any effect that requires a Fortitude save unless it also works on objects or is harmless. 
It is not subject to critical hits, nonlethal damage, ability damage, ability drain, fatigue, exhaustion, or energy drain. 
It cannot heal damage, but it can be repaired. 
Spellstaffs do not have the usual construct trait of darkvision.

\paragraph{Just a walking stick:} When held by its owner, a Spellstaff is in many ways considered an item rather than a creature.
For the purposes of saving throws, a held Spellstaff is considered an attended magical item.
As such, it is immune to many spells, and it uses its owner's saving throw modifiers if they are better than its own.
A Spellstaff cannot be attacked directly when held - a sunder attempt must be made against it
(its hit points and hardness are not recalculated even if used as a magic item, use its creature statistics).
It can be used as a weapon, serving as a masterwork quality quarterstaff (one end masterwork).
They can even receive enhancements as such, although this only affects their use as a weapon, not their normal statistics.

\newpage
\subsection{Variant: Generic Summoned Monsters}
\label{sec:SummonedMonsters}
A GM may decide that the original psionic approach\footnote{That is, collapsing all forms of ''summoning`` into the Astral Construct power} to summons is superior one. In such a case, remove all specific summoning spells (such as \nameref{Spell:SummonFireElemental} and \nameref{Spell:SummonVermin}) from all spell lists, replace them with an instance of the generic \nameref{Spell:SummonMonster} spell, and refer to the rules below.

These generic summoned Monsters (referred to hereafter as simply ``Monsters``) are not ``real'' creatures
in most senses of the word - they are conjured beings that exist only for a short time, summoned out
of the malleable material that makes up the outer planes.

A Monster can be any kind of creature the caster wishes it to (within size limitations), 
appearing as a generic version of that kind of creature.
Good summoners championing a cause of good might summon angels or celestial animals, while a priest of nature
summons wild beasts or plants.
The summoner's involuntary preconceptions about what each summoned creature ``should`` look like color their magic, 
the result being that virtually all summoners stick to a particular theme.

Regardless of the type of Monster summoned, the spell points spent by the summoner during the casting
of the spell determine the level of the Monster created, and thereby its strength, abilities, and power.
\subsubsection{Combat Statistics}
Monsters act as directed by their creators. They act faithfully, and do not fear battle or worry for their lives.
As a free action, a Monster's summoner can direct the Monster to attack particular enemies, 
use specific tactics, perform other actions, or do nothing at all. 
The Monster does exactly what its creator directs it to do.

\paragraph{Natural Attack:} Every Monster has one or two natural attacks, referred to simply as such in the statistics blocks.
What kind of Natural Attack this is (bite, claw, slam, tentacle, hoof, gore, manufactured weapon, and so on) is left up to the summoner.
This affects the Monster's damage type (piercing, slashing or bludgeoning), but not its reach, base damage, or any other variable.
If the Monster has only one natural attack, the natural attack adds the Monster's Strength modifier x 1-1/2 to damage, otherwise it adds only
its strength modifier.

\paragraph{Items:} The summoner may have the Monster appear wearing armor and using a weapon. (Appropriate for Devils and similar creatures.)
These items are considered part of the monster and cannot be removed from it - making this is for virtually all purposes only a cosmetic change. 
It does not give the monster options or statistics beyond those given by its stat blocks and menu ablities.

\paragraph{Outsider Traits:} Monsters, being summoned out of extraplanar material, always have the outsider type. 
This gives them Darkvision out to 60 feet, along with other outsider traits.

\paragraph{Mindless:} Monsters are not ''real'' creatures, and do not think for themselves.
They have no Intelligence score, and complete immunity to all mind-affecting effects (charms, compulsions, phantasms, patterns, and morale effects).

\paragraph{Skills and Feats:} Being Mindless, Monsters do not naturally come with any skills or feats.

\paragraph{Alignment:} A monster is considered to have the same alignment as its summoner for all purposes.

\paragraph{Other:} Other statistics generally given in monster stat blocks (Environment, Organization, Challenge Rating, Treasure, Advancement, and Level Adjustment)
are omitted for Monsters, due to them being a function of the caster that summons them, rather than a monster in their own right.
\subsubsection{Special Abilities} 
Every summoned Monster has a special ability of the summoner's choosing. 
When the caster begins to cast the Summon Monster spell, 
he chooses these special abilities from a menu of abilities appropriate to that level of Monster.

A caster can always substitute two choices from a lesser menu for one of its given abilities. 
Multiple selections of the same menu choice do not stack unless the ability specifically notes that stacking is allowed.
A Monster does not need to meet the prerequisites for a feat granted by a menu choice. 

\paragraph{Monster Menu A:}

A caster summoning a 1st-level, 2nd-level, or 3rd-level Monster can choose one 
special ability from this menu.
\begin{itemize}
\item Buff (Ex): The Monster has an extra 5 hit points.
\item Quick (Ex): The Monster's land speed is increased by 10 feet.
\item Cleave (Ex): The Monster has the Cleave feat. 
\item Deflection (Ex): The Monster has a +1 deflection bonus to Armor Class.
\item Fly (Ex): The Monster has physical wings and a fly speed of 20 feet (average).
\item Improved Bull Rush (Ex): The Monster has the Improved Bull Rush feat.
\item Improved Natural Attack (Ex): The Monster has the Improved Natural Attack feat.
\item Mobility (Ex): The Monster has the Mobility feat.
\item Power Attack (Ex): The Monster has the Power Attack feat.
\item Resistance (Ex): Choose one of the following energy types: fire, cold, acid, electricity, or sonic. 
The Monster has resistance 5 against that energy type.
\item Swim (Ex): The Monster is streamlined and shark like, and has a swim speed of 30 feet.
\item Trip (Ex): If the Monster hits with a natural attack, 
it can attempt to trip the opponent as a free action without 
making a touch attack or provoking attacks of opportunity. 
If the attempt fails, the opponent cannot react to trip the Monster.
\end{itemize}

\paragraph{Monster Menu B:}

A caster creating a 4th-level, 5th-level, or 6th-level Monster 
can choose one special ability from this menu. 
Alternatively, the monster can have two special abilities from Menu A.
\begin{itemize}
\item Energy Touch (Ex): The Monster's physical attacks are wreathed in energy of a type you choose 
(fire, cold, acid, or electricity) when you summon the Monster, dealing an extra 1d6 points of damage.
\item Extra Attack: If the Monster is Medium or smaller, 
it has two natural attacks instead of one when it makes a full attack. 
Its bonus on damage rolls for each attack is equal to its Strength modifier, 
not its Strength modifier x 1-1/2. If the Monster is Large or larger, 
it has three natural attacks instead of two when it makes a full attack. Its attacks are otherwise unchanged. 
\item Fast Healing (Ex): The Monster heals 2 hit points each round. 
\item Heavy Deflection (Ex): The Monster has a +4 deflection bonus to Armor Class.
\item Improved Buff (Ex): The Monster has an extra 15 hit points.
\item Improved Critical (Ex): The Monster has the Improved Critical feat with its natural attacks.
\item Improved Damage Reduction (Ex): The Monster's surface forms a 
hard carapace and provides an additional 3 points of damage reduction 
(or damage reduction 3/magic if it does not already have damage reduction).
\item Improved Fly (Ex): The Monster has physical wings and a fly speed of 40 feet (average).
\item Improved Grab (Ex): To use this ability, the Monster must hit with its natural attack. 
A Monster can use this ability only on a target that is at least one size smaller than itself. 
\item Improved Swim: The Monster is streamlined and sharklike, and has a swim speed of 60 feet.
\item Muscle (Ex): The Monster has a +4 bonus to its Strength score.
\item Poison Touch (Ex): If the Monster hits with a natural attack, 
the target must make an initial Fortitude save (DC 10 + 1/2 Monster's HD + Monster's Cha modifier) 
or take 1 point of Constitution damage. 
One minute later, the target must save again or take 1d2 points of Constitution damage.
\item Pounce (Ex): If the Monster charges a foe, it can make a full attack. 
\item Smite (Su): Once per day the Monster can make one attack that deals extra damage equal to its Hit Dice.
\item Trample (Ex): As a standard action during its turn each round, 
a Large or larger Monster can literally run over an opponent at least one size smaller than itself. 
It merely has to move over the opponent to deal bludgeoning damage equal to 1d8 + its Str modifier. 
The target can attempt a Reflex save (DC 10 + 1/2 Monster's Hit Dice + Monster's Str modifier) 
to negate the damage, or it can instead choose to make an attack of opportunity at a –4 penalty.
\end{itemize}

\paragraph{Monster Menu C:}

A caster creating a 7th-level, 8th-level, or 9th-level Monster 
can choose one special ability from this menu. 
Alternatively, the Monster can have two special abilities from Menu B. 
(One or both of the Menu B choices can be swapped for two choices from Menu A.)
\begin{itemize}
\item Blindsight (Ex): The Monster has blindsight out to 60 feet.
\item Constrict (Ex): The Monster has the improved grab ability with its natural attack. 
In addition, on a successful grapple check, the Monster deals damage equal to its natural attack damage.
\item Extra Buff (Ex): The Monster has an extra 30 hit points.
\item Extreme Damage Reduction (Ex): The Monster's skin has hard, 
armor-like plates (or appears to wear actual armor) and provides an additional 6 points of damage reduction.
\item Extreme Deflection (Ex): The Monster has a +8 deflection bonus to Armor Class.
\item Natural Invisibility (Su): The Monster is constantly invisible, even when attacking. 
\item Spell Resistance (Ex): The Monster has spell resistance equal to 10 + its Hit Dice.
\item Rend (Ex): The Monster must make claw attacks in order to select this special ability. If a Monster that hits the same opponent with two claw attacks in the same round rends its foe, which deals extra damage equal to 2d6 + 1-1/2 times its Str modifier.
\item Spring Attack (Ex): The Monster has the Spring Attack feat.
\item Whirlwind Attack (Ex): The Monster has the Whirlwind Attack feat.
\end{itemize}
\subsubsection[Summon Monster]{Summon Monster Spell}
\label{Spell:SummonMonster}
Conjuration (Summoning)
\\ \textbf{Level:} Animal 1, Conjurer 1, Planes 1, Ranger 1
\\ \textbf{Components:} V, S
\\ \textbf{Casting Time:} 1 round
\\ \textbf{Range:} Close (25 ft. + 5 ft./2 levels)
\\ \textbf{Effect:} One summoned monster
\\ \textbf{Duration:} 1 round/level (D)
\\ \textbf{Saving Throw:} None
\\ \textbf{Spell Resistance:} No
\\ \textbf{Spell Points:} 1

\emph{You summon forth construction blocks from the outer planes, and assemble them to form a creature to aid you in battle.}

This spell summons one 1st-level monster (see the \nameref{sec:SummonedMonsters} section) from another plane of existence to attack your enemies. 
It appears where you designate and acts immediately, on your turn. It attacks your opponents to the best of its ability. 
As a free action, you can mentally direct it not to attack, to attack particular enemies, or to perform other actions. 
The monster acts normally on the last round of the spell's duration and dissipates at the end of its turn.

\paragraph{Augment:} You can augment the spell in one or both of the following ways: 
\begin{enumerate}
 \item For every 2 additional spell points you spend, the level of the summoned monster increases by one (to a maximum of 9).
 \item If you spend 4 additional spell points, you can cast this spell as a standard action.
\end{enumerate}

\begin{table*}%[h]
\caption{1st-level Summoned Monster}
\centering
%\rowcolors{1}{white}{lightgray}
\makebox[\textwidth]{
\begin{tabular}{|p{0.3\textwidth}|p{0.7\textwidth}|}
\hline
\textbf{Size/Type:}&{Small Outsider}\\
\textbf{Hit Dice}&$1\text{d8}+2$ (6 HP)\\
\textbf{Initiative}&+2\\
\textbf{Speed}&30 ft. (6 squares)\\
\textbf{Armor Class:}& 18 (+2 Dex, +5 natural, +1 size), touch 13, flatfooted 16\\
\textbf{Base Attack/Grapple:}&+1/-1\\
\textbf{Attack:}& Natural Attack +4 melee (1d4+3)\\
\textbf{Full Attack:}& Natural Attack +4 melee (1d4+3)\\
\textbf{Space/Reach:}& 5 ft./5 ft.\\
\textbf{Special Attacks:}&-\\
\textbf{Special Qualities:}&One ability from Menu A, outsider traits\\
\textbf{Saves:}&Fort +4, Ref +4, Will +2\\
\textbf{Abilities:}&Str 15, Dex 15, Con 15, Int —, Wis 11, Cha 10\\
\hline
\end{tabular}}
\end{table*}
\begin{table*}
\caption{2nd-level Summoned Monster}
\centering
\makebox[\textwidth]{
\begin{tabular}{|p{0.3\textwidth}|p{0.7\textwidth}|}
\hline
\textbf{Size/Type:}&{Medium Outsider}\\
\textbf{Hit Dice}&$2\text{d8}+6$ (15 HP)\\
\textbf{Initiative}&+2\\
\textbf{Speed}&40 ft. (8 squares)\\
\textbf{Armor Class:}& 18 (+2 Dex, +6 natural), touch 12, flatfooted 16\\
\textbf{Base Attack/Grapple:}&+2/+5\\
\textbf{Attack:}& Natural Attack +4 melee (1d6+4)\\
\textbf{Full Attack:}& Natural Attack +4 melee (1d6+4)\\
\textbf{Space/Reach:}& 5 ft./5 ft.\\
\textbf{Special Attacks:}&-\\
\textbf{Special Qualities:}&One ability from Menu A, outsider traits\\
\textbf{Saves:}&Fort +6, Ref +5, Will +3\\
\textbf{Abilities:}&Str 17, Dex 15, Con 16, Int —, Wis 11, Cha 10\\
\hline
\end{tabular}}
\end{table*}
\begin{table*}
\caption{3rd-level Summoned Monster}
\centering
\makebox[\textwidth]{
\begin{tabular}{|p{0.3\textwidth}|p{0.7\textwidth}|}
\hline
\textbf{Size/Type:}&{Medium Outsider}\\
\textbf{Hit Dice}&$3\text{d8}+12$ (25 HP)\\
\textbf{Initiative}&+2\\
\textbf{Speed}&40 ft. (8 squares)\\
\textbf{Armor Class:}& 20 (+2 Dex, +8 natural), touch 12, flatfooted 18\\
\textbf{Base Attack/Grapple:}&+3/+8\\
\textbf{Attack:}& Natural Attack +8 melee (1d6+7)\\
\textbf{Full Attack:}& Natural Attack +8 melee (1d6+7)\\
\textbf{Space/Reach:}& 5 ft./5 ft.\\
\textbf{Special Attacks:}&-\\
\textbf{Special Qualities:}&One ability from Menu A, outsider traits\\
\textbf{Saves:}&Fort +7, Ref +5, Will +3\\
\textbf{Abilities:}&Str 21, Dex 15, Con 18, Int —, Wis 11, Cha 10\\
\hline
\end{tabular}}
\end{table*}
\begin{table*}
\caption{4th-level Summoned Monster}
\centering
\makebox[\textwidth]{
\begin{tabular}{|p{0.3\textwidth}|p{0.7\textwidth}|}
\hline
\textbf{Size/Type:}&{Medium Outsider}\\
\textbf{Hit Dice}&$5\text{d8}+25$ (47 HP)\\
\textbf{Initiative}&+2\\
\textbf{Speed}&40 ft. (8 squares)\\
\textbf{Armor Class:}& 22 (+2 Dex, +10 natural), touch 12, flatfooted 20\\
\textbf{Base Attack/Grapple:}&+5/+12\\
\textbf{Attack:}& Natural Attack +12 melee (1d6+10)\\
\textbf{Full Attack:}& Natural Attack +12 melee (1d6+10)\\
\textbf{Space/Reach:}& 5 ft./5 ft.\\
\textbf{Special Attacks:}&-\\
\textbf{Special Qualities:}&One ability from Menu B, outsider traits\\
\textbf{Saves:}&Fort +9, Ref +6, Will +4\\
\textbf{Abilities:}&Str 25, Dex 15, Con 20, Int —, Wis 11, Cha 10\\
\hline
\end{tabular}}
\end{table*}
\begin{table*}
\caption{5th-level Summoned Monster}
\centering
\makebox[\textwidth]{
\begin{tabular}{|p{0.3\textwidth}|p{0.7\textwidth}|}
\hline
\textbf{Size/Type:}&{Large Outsider}\\
\textbf{Hit Dice}&$7\text{d8} + 35$ (66 HP)\\
\textbf{Initiative}&+1\\
\textbf{Speed}&40 ft. (8 squares)\\
\textbf{Armor Class:}& 23 (+1 Dex, +13 natural, -1 size), touch 10, flatfooted 22\\
\textbf{Base Attack/Grapple:}&+7/+20\\
\textbf{Attack:}& Natural Attack +15 melee (1d8+9)\\
\textbf{Full Attack:}& 2 Natural Attacks +15 melee (1d8+9)\\
\textbf{Space/Reach:}& 10 ft./10 ft.\\
\textbf{Special Attacks:}&-\\
\textbf{Special Qualities:}&One ability from Menu B, outsider traits, damage reduction 5/magic\\
\textbf{Saves:}&Fort +10, Ref +6, Will +5\\
\textbf{Abilities:}&Str 29, Dex 13, Con 21, Int —, Wis 11, Cha 10\\
\hline
\end{tabular}}
\end{table*}
\begin{table*}
\caption{6th-level Summoned Monster}
\centering
\makebox[\textwidth]{
\begin{tabular}{|p{0.3\textwidth}|p{0.7\textwidth}|}
\hline
\textbf{Size/Type:}&{Large Outsider}\\
\textbf{Hit Dice}&$10\text{d8} + 60$ (105 HP)\\
\textbf{Initiative}&+1\\
\textbf{Speed}&40 ft. (8 squares)\\
\textbf{Armor Class:}& 25 (+1 Dex, +15 natural, -1 size), touch 10, flatfooted 24\\
\textbf{Base Attack/Grapple:}&+10/+25\\
\textbf{Attack:}& Natural Attack +20 melee (1d8+11)\\
\textbf{Full Attack:}& 2 Natural Attacks +20 melee (1d8+11)\\
\textbf{Space/Reach:}& 10 ft./10 ft.\\
\textbf{Special Attacks:}&-\\
\textbf{Special Qualities:}&One ability from Menu B, outsider traits, damage reduction 10/magic\\
\textbf{Saves:}&Fort +14, Ref +8, Will +7\\
\textbf{Abilities:}&Str 33, Dex 13, Con 23, Int —, Wis 11, Cha 10\\
\hline
\end{tabular}}
\end{table*}
\begin{table*}
\caption{7th-level Summoned Monster}
\centering
\makebox[\textwidth]{
\begin{tabular}{|p{0.3\textwidth}|p{0.7\textwidth}|}
\hline
\textbf{Size/Type:}&{Large Outsider}\\
\textbf{Hit Dice}&$12\text{d8} + 84$ (138 HP)\\
\textbf{Initiative}&+1\\
\textbf{Speed}&40 ft. (8 squares)\\
\textbf{Armor Class:}& 27 (+1 Dex, +17 natural, -1 size), touch 10, flatfooted 26\\
\textbf{Base Attack/Grapple:}&+12/+28\\
\textbf{Attack:}& Natural Attack +23 melee (1d8+12)\\
\textbf{Full Attack:}& 2 Natural Attacks +23 melee (1d8+12)\\
\textbf{Space/Reach:}& 10 ft./10 ft.\\
\textbf{Special Attacks:}&-\\
\textbf{Special Qualities:}&One ability from Menu C, outsider traits, damage reduction 10/magic\\
\textbf{Saves:}&Fort +15, Ref +9, Will +8\\
\textbf{Abilities:}&Str 35, Dex 13, Con 24, Int —, Wis 11, Cha 10\\
\hline
\end{tabular}}
\end{table*}
\begin{table*}
\caption{8th-level Summoned Monster}
\centering
\makebox[\textwidth]{
\begin{tabular}{|p{0.3\textwidth}|p{0.7\textwidth}|}
\hline
\textbf{Size/Type:}&{Large Outsider}\\
\textbf{Hit Dice}&$14\text{d8} + 112$ (175 HP)\\
\textbf{Initiative}&+1\\
\textbf{Speed}&40 ft. (8 squares)\\
\textbf{Armor Class:}& 29 (+1 Dex, +19 natural, -1 size), touch 10, flatfooted 28\\
\textbf{Base Attack/Grapple:}&+14/+32\\
\textbf{Attack:}& Natural Attack +27 melee (1d8+14)\\
\textbf{Full Attack:}& 2 Natural Attacks +27 melee (1d8+14)\\
\textbf{Space/Reach:}& 10 ft./10 ft.\\
\textbf{Special Attacks:}&-\\
\textbf{Special Qualities:}&One ability from Menu C, outsider traits, damage reduction 15/magic\\
\textbf{Saves:}&Fort +17, Ref +10, Will +9\\
\textbf{Abilities:}&Str 39, Dex 13, Con 26, Int —, Wis 11, Cha 10\\
\hline
\end{tabular}}
\end{table*}
\begin{table*}
\caption{9th-level Summoned Monster}
\centering
\makebox[\textwidth]{
\begin{tabular}{|p{0.3\textwidth}|p{0.7\textwidth}|}
\hline
\textbf{Size/Type:}&{Large Outsider}\\
\textbf{Hit Dice}&$17\text{d8}+136$ (212 HP)\\
\textbf{Initiative}&+0\\
\textbf{Speed}&50 ft. (10 squares)\\
\textbf{Armor Class:}& 33 (+25 natural, -2 size), touch 8, flatfooted 33\\
\textbf{Base Attack/Grapple:}&+14/+41\\
\textbf{Attack:}& Natural Attack +31 melee (2d6+16)\\
\textbf{Full Attack:}& 2 Natural Attacks +31 melee (2d6+16)\\
\textbf{Space/Reach:}& 15 ft./15 ft.\\
\textbf{Special Attacks:}&-\\
\textbf{Special Qualities:}&Two abilities from Menu C, outsider traits, damage reduction 15/magic\\
\textbf{Saves:}&Fort +18, Ref +10, Will +10\\
\textbf{Abilities:}&Str 43, Dex 11, Con 27, Int —, Wis 11, Cha 10\\
\hline
\end{tabular}}
\end{table*}
