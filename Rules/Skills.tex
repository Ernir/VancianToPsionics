\section{Skills}
Unless noted otherwise in this chapter, use the rules text presented in the
\href{http://www.wizards.com/default.asp?x=d20/article/srd35}{d20 srd}.
\subsection[Concentration]{Concentration (Con)}
\label{sec:Concentration}
You are particularly good at focusing your mind.
\subsubsection{Check}
You must make a Concentration check whenever you might potentially be distracted (by taking damage, by harsh weather, and so on) 
while engaged in some action that requires your full attention. 
Such actions include casting a spell, concentrating on an active spell, directing a spell, or using a spell-like ability.

If the Concentration check succeeds, you can continue with the action as normal. 
If the check fails, the action automatically fails and is wasted. 
If you were in the process of casting a spell, the spell points are lost. 
If you were concentrating on an active spell, the spell ends as if you had ceased concentrating on it. 
If you were directing a spell, the direction fails but the spell remains active. 
If you were using a spell-like ability, that use of the ability is lost.

The \nameref{tab:Concentration} table summarizes various types of distractions that cause you to make a Concentration check. 
If the distraction occurs while you are trying to cast a spell, 
you must add the level of the spell you are trying to cast to the appropriate Concentration DC. 
\begin{table*}
\centering
\caption{Concentration}
\label{tab:Concentration}
\begin{tabular}{p{0.3\textwidth}p{0.7\textwidth}}
\hline
\textbf{Concentration DC$^1$}&\textbf{Distraction}\\
\hline
10 + damage dealt&Damaged during the action.$^2$\\
10 + half of continuous damage last dealt&Taking continuous damage during the action.$^3$\\
15&Entangled.\\
Distracting spell's save DC&Distracted by nondamaging spell.$^4$\\
20&Gain magical focus.\\
20&Grappling or pinned. (You can cast spells normally unless you fail your Concentration check.)\\
Distracting spell's save DC&Weather caused by spell$^4$\\
DC 20 + twice the level of the spell&Attempting to cast a spell without its components.\\
\hline
\end{tabular}
\begin{enumerate}
\item If you are trying to cast, concentrate on, or direct a spell when the distraction occurs, add the level of the spell to the indicated DC.
\item Such as during the casting of a spell with a casting time of 1 round or more. 
Also from an attack of opportunity or readied attack made in response to the spell being cast 
(for spells with a casting time of 1 standard action) or the action being taken (for activities requiring no more than a full-round action).
\item Such as from standing in natural fire or lava.
\item If the spell allows no save, use the save DC it would have if it did allow a save.
\end{enumerate}
\end{table*}

\paragraph{Gain Magical Focus}
The concentration skill is used to gain magical focus. See \nameref{sec:MagicFocus}.
\subsubsection{Action}
Usually none. Making a Concentration check doesn't take an action; it is either a free action (when attempted reactively) or part of another action (when attempted actively).
Meditating to gain magical focus is a full-round action. 
\subsubsection{Try Again}
Yes, though a success doesn't cancel the effects of a previous failure, 
such as the loss of the spell points for a spell being cast or the disruption of a spell being concentrated on. 
\subsubsection{Special}
You can use Concentration to cast a spell, use a spell-like ability, or use a skill defensively, so as to avoid attacks of opportunity altogether. 
This doesn't apply to other actions that might provoke attacks of opportunity.

The DC of the check is 15 (plus the spell's level, if casting a spell or using a spell-like ability defensively). 
If the Concentration check succeeds, you may attempt the action normally without provoking any attacks of opportunity. 
A successful Concentration check still doesn't allow you to take 10 on another check if you are in a stressful situation; 
you must make the check normally. If the Concentration check fails, the related action also automatically fails (with any appropriate ramifications), 
and the action is wasted, just as if your concentration had been disrupted by a distraction.

A character with the Combat Casting feat gets a +4 bonus on Concentration checks made to cast a spell or use a spell-like ability while on the defensive or while grappling or pinned.
\subsection[Spellcraft]{Spellcraft (Int; Trained only)}
\label{sec:Spellcraft}
This is the skill representing the practical understanding of magic.
Use this skill to identify spells as they are being cast, 
or those spells that are already in place and have noticeable effects.

\subsubsection{Check}
You can identify spells and magic effects. The DCs for Spellcraft checks relating to various tasks are summarized on the \nameref{tab:Spellcraft} table.
\begin{table*}
\centering
\caption{Spellcraft}
\label{tab:Spellcraft}
\begin{tabular}{p{0.3\textwidth}p{0.7\textwidth}}
\toprule
\textbf{DC}&\textbf{Task}\\
\midrule
10 + spell points spent on spell&Identify a spell being cast. (You must see or hear the spell's verbal or somatic components.) No action required. No retry.\\
10 + spell level&Research original spell.\\
10 + hardness + thickness in feet&Bypass barrier with \nameref{Feat:BurrowingSpell} feat.\\
13 &Identify a \nameref{Spell:GlyphOfWarding}. Requires a full round of study. No retry.\\
15 + spell level&Decipher scroll. One try per day. Requires one minute of study.\\
15 + spell level&Read scroll in preparation for casting an unknown spell. Requires one full round. Retries allowed.\\
20 + spell level&Identify a spell that's already in place and in effect. You must be able to see or detect the effects of the spell. No action required. No retry.\\
20 + spell level&Identify a spell from its aura using \nameref{Spell:DetectMagic}\\
20 + spell level&Identify materials created or shaped by magic, such as noting that an iron wall is the result of a \nameref{Spell:WallOfIron} spell. No action required. No retry.\\
25 + spell level&After succeeding on a saving throw against a spell targeted on you, determine what that spell was. No action required. No retry.\\
25 &Identify a potion. Requires 1 minute. No retry.\\
30 &Identify the effects of a \nameref{Spell:PlanarBinding} spell when subjected to it. No action required. No retry.\\
30 or higher&Understand a strange or unique magical effect, such as the effects of a magic stream. Time required varies. No retry.\\
\bottomrule
\end{tabular}
\normalsize
\end{table*}
\subsubsection{Action}
Varies, as noted above.
\subsubsection{Try Again}
See above.
\subsubsection{Special}
If you are a wizard, you get a +2 bonus on Spellcraft checks when dealing with a spell or effect from your specialty school. 
If you have the Magical Aptitude feat, you get a +2 bonus on Spellcraft checks.
\subsubsection{Synergy}
\begin{itemize}
 \item If you have 5 or more ranks in Knowledge (arcana), you get a +2 bonus on Spellcraft checks.
 \item If you have 5 or more ranks in Use Magic Device, you get a +2 bonus on Spellcraft checks to decipher spells on scrolls.
 \item If you have 5 or more ranks in Spellcraft, you get a +2 bonus on Use Magic Device checks related to matrices and scrolls.
\end{itemize}
% Additionally, certain spells allow you to gain information about magic, 
% provided that you make a successful Spellcraft check as detailed in the spell description.

\subsection[Use Magic Device]{Use Magic Device (Cha; Trained Only)}
\label{sec:UseMagicDevice}
Use this skill to activate magic devices you should normally not be able to activate.
\subsubsection{Check}
You can use this skill to read a scroll or to activate a magic item. Use Magic Device lets you use a magic item as if you had the spell ability or class features of another class, as if you were a different race, or as if you were of a different alignment.

You make a Use Magic Device check each time you activate a device such as a wand. If you are using the check to emulate an alignment or some other quality in an ongoing manner, you need to make the relevant Use Magic Device check once per hour.

You must consciously choose which requirement to emulate. That is, you must know what you are trying to emulate when you make a Use Magic Device check for that purpose. The DCs for various tasks involving Use Magic Device checks are summarized on the \nameref{tab:UseMagicDevice} table.
\begin{table*}
\centering
\caption{Use Magic Device}
\small
\label{tab:UseMagicDevice}
\begin{tabular}{ll}
\toprule
\textbf{DC}&\textbf{Task}\\
\midrule
25&Activate blindly\\
25+spell level&Decipher a written spell\\
20+caster level&Use a matrix\\
20&Use a wand\\
20&Emulate a class feature\\
25&Emulate an ability score\\
30&Emulate an alignment\\
\bottomrule
\end{tabular}
\normalsize
\end{table*}
\paragraph{Activate Blindly:}
Some magic items are activated by special words, thoughts, or actions. 
You can activate such an item as if you were using the activation word, thought, or action, even when you're not and even if you don't know it.
You do have to perform some equivalent activity in order to make the check. 
That is, you must speak, wave the item around, or otherwise attempt to get it to activate. 
You get a special +2 bonus on your Use Magic Device check if you've activated the item in question at least once before. 
If you fail by 9 or less, you can't activate the device. If you fail by 10 or more, you suffer a mishap. 
A mishap means that magical energy gets released but it doesn't do what you wanted it to do. 
The default mishaps are that the item affects the wrong target or that uncontrolled magical energy is released, dealing 2d6 points of damage to you. 
This mishap is in addition to the chance for a mishap that you normally run when you cast a spell from a matrix that you could not otherwise cast yourself.

\paragraph{Decipher a Written Spell:}
This usage works just like deciphering a written spell with the \nameref{sec:Spellcraft} skill, except that the DC is 5 points higher. 

\paragraph{Emulate an Ability Score:}
To cast a spell from a matrix, you need a high score in the appropriate ability (Intelligence for Wizard spells, Wisdom for Cleric or Ranger spells, or Charisma for Sorcerer, Paladin, or Bard spells). 
Your effective ability score (appropriate to the class you're emulating when you try to cast the spell from the matrix) is your Use Magic Device check result minus 15. If you already have a high enough score in the appropriate ability, you don't need to make this check.
The ability score emulated is only used for purposes of whether or not you can activate the item, the item still uses its normal method to assign save DCs and other effects that may depend on an ability score.

\paragraph{Emulate an Alignment:}
Some magic items have positive or negative effects based on the user's alignment. 
Use Magic Device lets you use these items as if you were of an alignment of your choice. 
You can emulate only one alignment at a time.

\paragraph{Emulate a Class Feature:}
Sometimes you need to use a class feature to activate a magic item. 
In this case, your effective level in the emulated class equals your Use Magic Device check result minus 20. 
This skill does not let you actually use the class feature of another class. 
It just lets you activate items as if you had that class feature. 
If the class whose feature you are emulating has an alignment requirement, you must meet it, either honestly or by emulating an appropriate alignment with a separate Use Magic Device check (see above).

\paragraph{Emulate a Race:}
Some magic items work only for members of certain races, or work better for members of those races. 
You can use such an item as if you were a race of your choice. 
You can emulate only one race at a time.
Certain items might remain anatomically incompatible. For example, it is not possible to use Horseshoes of Speed without having four hooves.

\paragraph{Use a Matrix:}
If you are casting a spell from a matrix, you have to identify it first. 
Normally, to cast a spell from a matrix, you must have the matrix's spell on your class spell list. 
Use Magic Device allows you to use a matrix as if you had a particular spell on your class spell list. 
The DC is equal to 20 + the caster level of the spell you are trying to cast from the matrix.
In addition, casting a spell from a matrix requires a minimum score (10 + spell level) in the appropriate ability. 
If you don't have a sufficient score in that ability, you must emulate the ability score with a separate Use Magic Device check (see above).

This use of the skill also applies to other spell completion magic items.
\paragraph{Use a Wand:}
Normally, to use a wand, you must have the wand's spell on your class spell list. 
This use of the skill allows you to use a wand as if you had a particular spell on your class spell list. 

This use of the skill also applies to other spell trigger magic items.%, such as staffs.
\subsubsection{Action}
Usually none. The Use Magic Device check is made as part of the action (if any) required to activate the magic item.

Deciphering a written spell takes the same action as deciphering it with the use of Spellcraft, that is, it requires one minute of study, and can be retried once per day.
\subsubsection{Try Again}
Yes, but if you ever roll a natural 1 while attempting to activate an item and you fail, then you can't try to activate that item again for 24 hours.
\subsubsection{Special}
You cannot take 10 with this skill.

You can't aid another on Use Magic Device checks. Only the user of the item may attempt such a check.

If you have the Magical Aptitude feat, you get a +2 bonus on Use Magic Device checks.
\subsubsection{Synergy}
\begin{itemize}
\item If you have 5 or more ranks in Spellcraft, you get a +2 bonus on Use Magic Device checks related to matrices and scrolls.
\item If you have 5 or more ranks in Decipher Script, you get a +2 bonus on Use Magic Device checks related to scrolls.
\item If you have 5 or more ranks in Use Magic Device, you get a +2 bonus to Spellcraft checks made to decipher spells on scrolls.
\end{itemize}
