\section{Magical feats}

Magical feats are available only to characters and creatures with the ability to cast spells. 
(In other words, they either have a spell point reserve or have spell-like abilities.)
Because magical feats are supernatural abilities - a departure from the general rule that feats do not grant supernatural abilities - 
they cannot be disrupted in combat (as spells can be) and generally do not provoke attacks of opportunity (except as noted in their descriptions). 
Supernatural abilities are not subject to spell resistance and cannot be dispelled; however, they do not function in areas where magic is suppressed, 
such as in an antimagic field. Leaving such an area immediately allows magical feats to be used.
Many magical feats can be used only when you are magically focused; others require you to expend your magical focus to gain their benefit. 
Expending your magical focus does not require an action; it is part of another action (such as using a feat). 
When you expend your magical focus, it applies only to the action for which you expended it.

\subsection{Metamagic Feats}
As a caster's knowledge of magic grows, he can learn to cast spells in ways slightly different from how the spells were originally designed or learned. 
Of course, casting a spell while using a metamagic feat is more expensive than casting the spell normally.
\begin{list}{\labelitemi}{\leftmargin=1em}
\item Casting Time:
Spells cast using metamagic feats take the same time as casting the spells normally unless the feat description specifically says otherwise.
\item Casting Cost:
To use a metamagic feat, a caster must both expend his magical focus (see \nameref{sec:MagicFocus}) 
and pay an increased spell point cost as given in the feat description.
\item Limits on Use: 
As with all spells, you cannot spend more spell points on a spell than your caster level. 
Metamagic feats merely let you cast spells in different ways; they do not let you violate this rule. 
\item Effects of Metamagic Feats on a Spell: 
In all ways, a metamagic spell operates at its original spell level, even though it costs additional spell points. 
The modifications to a spell made by a metamagic feat have only their noted effect on the spell. 
A caster can't use a metamagic feat to alter a spell being cast from a potion, wand, or other device.
\end{list}
Casting a spell modified by the Quicken Spell feat does not provoke attacks of opportunity.
Some metamagic feats apply only to certain spells, as described in each specific feat entry.

\emph{Magical Items and Metamagic Spells:} With the right item creation feat, 
you can store a metamagic spell in a potion or wand. 
Level limits for potions apply as if the spell point increase actually raised the level of the spell. 
A character doesn't need the appropriate metamagic feat to activate an item in which a metamagic spell is stored, 
but does need the metamagic feat to create such an item.
\subsection{Companion Feats}
\label{sec:CompanionFeats}
The companions of spellcasters have focuses different from those of most adventurers, and a unique magical link to their master that grants them special powers.
Some of these are represented by companion feats. Only companions (familiars, spellstaffs, animal companions, and celestial mounts, but not their hosts) can take these feats.

When the description of a companion feat refers to the master's level, it refers to his number of levels in spellcasting classes.
\subsection{Feat Descriptions}
\begin{table*}
\label{tab:Feats}
\caption{Feats}
\makebox[\textwidth]{%\resizebox{\textwidth}{!}{
\small
\begin{tabular}{|p{0.28\textwidth}|p{0.3\textwidth}|p{0.42\textwidth}|}
\hline
\textbf{General Feats}&\textbf{Prerequisites}&\textbf{Benefit}\\
\hline
\nameref{Feat:AnimalCompanion}&Access to the Animal Domain or Ranger level 1st&Gain Animal Companion\\
\nameref{Feat:AugmentSummoning}&-&Creatures you summon are stronger and healthier.\\
\nameref{Feat:Familiar}&Caster level 1st&Gain Familiar\\
\nameref{Feat:MagicalSpark}&1st level only&Gain limited spellcasting ability\\
\nameref{Feat:SpellstaffUser}&Caster level 1st&Gain Spellstaff\\
\hline
\textbf{Magical Feats}&\textbf{Prerequisites}&\textbf{Benefit}\\
\hline
\nameref{Feat:BondedWeapon}&Spellstaff User, character level 3rd&You can enchant your spellstaff as a weapon\\
\nameref{Feat:CarefulPush}&Pushing the Limits, caster level 5th&Negate damage from Pushing the Limits\\
\nameref{Feat:CelestialSummons}&Knowledge (the planes) 4 ranks, good&Apply celestial template to your summoned monsters\\
\nameref{Feat:CircleMagicLeader}&Caster level 3rd&You can lead a Circle of spellcasters\\
\nameref{Feat:ContinuedTraining}&Spellcraft 4 ranks&Caster level increases by 4, to a maximum of your character level\\
\nameref{Feat:DeadlyDuelist}&BAB +1&Can forcefully begin lethal duels.\\
\nameref{Feat:DomainAttunement}&Access to a Domain&Access the spells of a new Domain\\
\nameref{Feat:FiendishSummons}&Knowledge (the planes) 4 ranks, evil&Apply fiendish template to your summoned monsters\\
\nameref{Feat:ExpandedKnowledge}&Caster level 3rd&Learn additional spell from any list\\
\nameref{Feat:GreaterMagicalEndowment}&Magical Endowment&Add +2 to save DC of spell\\
\nameref{Feat:MagicalEndowment}&-&Add +1 to save DC of spell\\
\nameref{Feat:MagicallyGifted}&A spell point reserve&Gain more spell points\\
\nameref{Feat:MentalFortress}&No Man's Fool&Gain immunity to mind-affecting effects\\
\nameref{Feat:NoMansFool}&-&Gain protection from mental control\\
\nameref{Feat:PushingTheLimits}&-&Take HP damage to increase your caster level\\
\nameref{Feat:RapidRefocusing}&Wis 13, Concentration 7 ranks&Regain magical focus as move action\\
\nameref{Feat:SpellstaffContainment}&Spellstaff User, caster level 3rd&Store second magical focus in your spellstaff\\
\nameref{Feat:TurnUndead}&Cleric or Paladin level 1st, non-evil&Damage undead with power of your faith\\
\hline
\textbf{Metamagic Feats}&\textbf{Prerequisites}&\textbf{Benefit}\\
\hline
\nameref{Feat:BurrowingSpell}&-&Spells bypass barriers\\
\nameref{Feat:ChainSpell}&-&Spell arcs to secondary targets\\
\nameref{Feat:DelaySpell}&-&Delay the activation of spells\\
\nameref{Feat:EmpowerSpell}&-&Variable numeric benefits of spells are increased\\
\nameref{Feat:EnlargeSpell}&-&Your spells reach farther\\
\nameref{Feat:ExtendSpell}&-&Your spells last longer than normal\\
\nameref{Feat:MaximizeSpell}&-&Maximize variable numeric benefits of spells\\
\nameref{Feat:OpportunitySpell}&-&Cast spell as attack of opportunity\\
\nameref{Feat:QuickenSpell}&-&Cast spell as a swift action\\
\nameref{Feat:ScryAndDie}&Scrying spell&Cast spell through scrying spell\\
\nameref{Feat:SilentSpell}&-&Omit verbal component from spell\\
\nameref{Feat:SplitRay}&A metamagic feat&Fire two rays at once\\
\nameref{Feat:StillSpell}&-&Omit somatic component from spell\\
\nameref{Feat:TwinSpell}&-&Cast spell twice\\
\nameref{Feat:UnconditionalSpell}&-&Cast spell regardless of circumstances\\
\hline
\textbf{Item Creation Feats}&\textbf{Prerequisites}&\textbf{Benefit}\\
\hline
\nameref{Feat:CraftMagicArmsAndArmor}&Caster level 5th&You can create magic arms and armor\\
\nameref{Feat:CraftRod}&Caster level 9th&You can create rods\\
\nameref{Feat:CraftWand}&Caster level 5th&You can create wands\\
\nameref{Feat:CraftWondrousItem}&Caster level 3rd&You can create various wondrous items\\
\nameref{Feat:ForgeRing}&Caster level 7th&You can create magic rings\\
\nameref{Feat:ImbueMatrix}&Caster level 1st&You can create matrices\\
\nameref{Feat:ScribeScroll}&Caster level 1st&You can create scrolls\\
\hline
\textbf{Companion Feats}&\textbf{Prerequisites}&\textbf{Benefit}\\
\hline
\nameref{Feat:CerebralCompanion}&-&Companion gains intelligence close to your own\\
\nameref{Feat:CompanionAlertness}&-&Companion grants and gains awareness bonuses\\
\nameref{Feat:CompanionCommunication}&Master level 6th&Familiar can communicate with similar creatures\\
\nameref{Feat:CompanionEvasion}&-&Companion takes no damage on successful reflex saves\\
\nameref{Feat:CompanionHardiness}&-&Companion gains natural armor bonus\\
\nameref{Feat:CompanionSightLink}&Master level 12th&Share senses with companion\\
\nameref{Feat:CompanionSpellLink}&-&Share spells with Companion\\
\nameref{Feat:CompanionSpellResistance}&Master level 12th&Companion gains spell resistance\\
\nameref{Feat:FamiliarSpellDelivery}&Master level 3rd&Familiar can deliver spells\\
\nameref{Feat:FamiliarToughness}&-&Familiar gains more hit points\\
\hline
\end{tabular}}
\end{table*}

\subsubsection{Animal Companion}
\label{Feat:AnimalCompanion}
You form a bond with an animal that aids you in your quests.

\paragraph{Prerequisites:} Access to the \nameref{Domain:Animal} or \nameref{sec:Ranger} level 1st

\paragraph{Benefit:} This feat allows you to gain a \nameref{sec:AnimalCompanion}.

\paragraph{Special:} If you have the \nameref{sec:CelestialMountListing} ability, the \nameref{Feat:Familiar} feat, the \nameref{sec:FiendishMountListing} ability, or the \nameref{Feat:SpellstaffUser} feat, you may not take this feat.

\subsubsection{Augment Summoning}
\label{Feat:AugmentSummoning}
\paragraph{Benefit:}
Each creature you conjure with any summon spell gains a +4 enhancement bonus to Strength and Constitution for the duration of the spell that summoned it.

% % \subsubsection{Arcane Shot [Magical]}
% % You can charge your ranged attacks with additional damage potential.
% % 
% % \paragraph{Prerequisite:} Point Blank Shot.
% % 
% % \paragraph{Benefit:} To use this feat, you must expend your magical focus. 
% % Your ranged attack deals +2d6 points of damage. 
% % You must decide whether or not to use this feat prior to making an attack. 
% % If your attack misses, you still expend your magical focus.
% % 
\subsubsection[Bonded Weapon]{Bonded Weapon [Magical]}
\label{Feat:BondedWeapon}
Your bonded item is a powerful arcane weapon.
 
\paragraph{Prerequisite:} Spellstaff User, character level 3rd.
 
\paragraph{Benefit:} You gain the ability to bestow your Spellstaff with magical enhancements, with respect to its use as a weapon. Its statistics as a creature are not affected. 
In addition, your spellstaff does not use the statistics of a quarterstaff when used as a weapon, 
but rather the statistics of any one weapon with which you are proficient, chosen at the time you take this feat.
This weapon can be an exotic weapon or a weapon made out of unusual material, but it must be a masterwork weapon. 
In any case, the cost of the base weapon is replaces the
normal (300 gp) cost involved with summoning or replacing a Spellstaff, as described by the Spellstaff User feat.

In order to enhance your Spellstaff, you must retreat to a peaceful location and disassemble magical items worth the difference between the market price of the magic item you are upgrading your Spellstaff
into becoming and the market price of your Spellstaff as it was previously (which is more efficient than selling your lesser magical items at half value).
You can also spend gold to cover the difference, although that requires access to a merchant capable of providing you with the necessary magical components.
The upgrade process takes 8 hours.% for each 1000 gp of the difference.

For example, an elven Wizard with this feat could upgrade his spell``staff'' from being a 
\emph{masterwork adamantine longsword} into being a \emph{+1 Spell Storing adamantine longsword} by spending 8 hours and %days and
disassembling a \emph{+2 Greataxe}, which was useless to him.
Alternatively, he could have bought components worth 7000 gp, and disassembled one of his Pearls of Power capable of storing one spell point.
\subsubsection[Burrowing Spell]{Burrowing Spell [Metamagic]}
\label{Feat:BurrowingSpell}
Your spells sometimes bypass barriers.

\paragraph{Benefit:} To use this feat, you must expend your magical focus. 
You can attempt to cast your spells against targets that are sheltered behind a wall or force effect. 
Your spell briefly skips through the Astral Plane to bypass the barrier.
The strength and thickness of the barrier determine your chance of success. 
To successfully bypass the barrier with your spell, 
you make a Spellcraft check against a DC equal to 10 + the hardness of the barrier + 1 per foot of thickness (minimum 1). 
Assign a hardness of 20 to barriers without a hardness rating, such as force effects. 
Force walls are assumed to have less than 1 foot of thickness unless noted otherwise.
If a spell requires line of sight (which includes most spells that affect a target or targets instead of an area), 
you cannot cast it as a burrowing spell unless you can somehow see the target.%, such as with clairvoyant sense.
Using this feat increases the spell point cost of the spell by 2. The spell's total cost cannot exceed your caster level.

% \subsubsection[Brew Potion]{Brew Potion [Item Creation]}
% 
% \paragraph{Prerequisite:}
% Caster level 3rd.
% 
% \paragraph{Benefit:}
% You can create a potion of any 3rd-level or lower spell that you know and that targets one or more creatures. 
% Brewing a potion takes one day. 
% When you create a potion, you set the caster level, 
% which must be sufficient to cast the spell in question and no higher than your own level. 
% The base price of a potion is its spell level $\times$ its caster level $\times$ 50 gp. 
% To brew a potion, you must spend 1/25 of this base price in XP and use up raw materials costing one half this base price.
% 
% When you create a potion, you make any choices that you would normally make when casting the spell. 
% Whoever drinks the potion is the target of the spell.
% 
% Any potion that stores a spell with a costly material component or an XP cost also carries a commensurate cost. 
% In addition to the costs derived from the base price, 
% you must expend the material component or pay the XP when creating the potion.
\subsubsection[Careful Push]{Careful Push [Magical]}
\label{Feat:CarefulPush}
You can push spells with less cost to yourself.

\paragraph{Prerequisite:} Pushing the Limits, caster level 5th.

\paragraph{Benefit:} To use this feat, you must expend your magical focus. When casting a spell of 3rd level or lower, you do not take damage from \nameref{Feat:PushingTheLimits}.
\subsubsection[Celestial Summons]{Celestial Summons [Magical]}
\label{Feat:CelestialSummons}
Your summoned creatures are tinged with the power of the heavens.

\paragraph{Prerequisites:} Knowledge (the planes) 4 ranks, good alignment.

\paragraph{Benefit:} Whenever you use a Conjuration (Summoning) spell to summon an animal, you can apply the Celestial template to it as part of the summoning process. This adds the [Good] descriptor to the spell.
\subsubsection[Cerebral Companion]{Cerebral Companion [Companion]}
\label{Feat:CerebralCompanion}
Your companion is not just more clever than animals, it is as clever as some clever people.

\paragraph{Benefit:} Your companion's intelligence score becomes equal to your own intelligence score - 2, or its normal intelligence score, whichever is higher. If your intelligence score later changes (temporarily or permanently), the companion's intelligence is affected immediately.

\emph{Special:} An animal companion cannot take this feat.
\subsubsection[Chain Spell]{Chain Spell [Metamagic]}
\label{Feat:ChainSpell}
You can cast spells that arc to hit other targets in addition to the primary target. 

\paragraph{Benefit:} To use this feat, you must expend your magical focus. 
You can chain any spell that specifies a single target and has a range greater than touch. 
After the primary target is struck, the spell can arc to a number of secondary targets equal to your caster level (maximum twenty).
You choose secondary targets as you like, but they must all be within 30 feet of the primary target, and no target can be struck more than once. 
You can choose to affect fewer secondary targets than the maximum (to avoid allies in the area, for example).
If the chained spell deals damage, the secondary targets each take half as much damage. 
Each target gets to make a saving throw as normal, if one is allowed by the spell. 
For spells that don't deal damage, the save DCs against arcing effects are reduced by 4.
Using this feat increases the spell point cost of the spell by 6. The spell's total cost cannot exceed your caster level.
\subsubsection[Circle Magic Leader]{Circle Magic Leader [Magical]}
\label{Feat:CircleMagicLeader}
You know how to contact the minds of other spellcasters to form a circle.
\paragraph{Prerequisite:} Caster level 3rd.
\paragraph{Benefit:} You can serve as a Circle Leader in \nameref{sec:CircleMagic}. Members of your Circle must remain with 20' of your location at all times, or lose the connection. Your Circle can have up to 7 members, including yourself.
\paragraph{Normal:} You can only join Circles led by others, you cannot lead them yourself.
\subsubsection[Circle Magic Master]{Circle Magic Master [Magical]}
\label{Feat:CircleMagicMaster}
You have great experience in leading the mental concert that is a magic Circle.
\paragraph{Prerequisite:} Caster level 7th, \nameref{Feat:CircleMagicLeader}.
\paragraph{Benefit:} Members of your Circle can stray out to 60' from your location without losing the connection. Your Circle can have up to 13 members, including yourself. You automatically succeed on all Concentration checks to maintain your circle.
\subsubsection[Companion Alertness]{Companion Alertness [Companion]}
\label{Feat:CompanionAlertness}
Your companion acts as your constant watchdog.

\paragraph{Benefit:} Whenever the master and companion are within 5' of one another, the master gains the benefit of the Alertness feat.
In the case of a Familiar, the companion additionally gains Spot as a ``class'' skill, even if it is not a class skill for the master.
\subsubsection[Companion Communication]{Companion Communication [Companion]}
\label{Feat:CompanionCommunication}
Your companion can communicate with creatures similar to itself.

\paragraph{Prerequisite:} Master must be 6th level.

\paragraph{Benefit:} Your companion can communicate with creatures of approximately the same kind as itself (including dire varieties): 
bats with bats, rats with rodents, cats with felines, hawks and owls and ravens with birds, 
lizards and snakes with reptiles, toads with amphibians, weasels with similar creatures (weasels, minks, polecats, ermines, skunks, wolverines, and badgers), and Spellstaffs with constructs. 
Such communication is limited by the intelligence of the conversing creatures. 

This is a Supernatural ability.
\subsubsection[Companion Evasion]{Companion Evasion [Companion]}
\label{Feat:CompanionEvasion}
Your companion isn't easy to blast down.

\paragraph{Benefit:} When subjected to an attack that normally allows a Reflex saving throw for half damage, 
Your companion takes no damage if it makes a successful saving throw and half damage even if the saving throw fails. 

\subsubsection[Companion Hardiness]{Companion Hardiness [Companion]}
\label{Feat:CompanionHardiness}
Your companion is tougher than it seems.

\paragraph{Benefit:} The companion gains a bonus to its natural armor equal to one-half the number of levels its master has in spellcasting classes.

\subsubsection[Companion Sight Link]{Companion Sight Link [Companion]}
\label{Feat:CompanionSightLink}
You and your companion can see through each others' eyes.

\paragraph{Prerequisite:} Master must be 12th level.

\paragraph{Benefit:} As a free action, you and your companion can share your senses, each experiencing everything that the other does for the duration of the effect.
Either the companion or the master can initiate the link, but the master may refuse the link, while the companion may not.
You can maintain the connection for up to one minute per level of the master per day, divided up between rounds as you wish. 

This is a Supernatural ability.
\subsubsection[Companion Spell Link]{Companion Spell Link [Companion]}
\label{Feat:CompanionSpellLink}
You and your companion have a special link when it comes to spells.

\paragraph{Benefit:} At the master's option, he may have any spell (but not any spell-like ability) he casts on himself also affect his companion. 
The companion must be within 5 feet at the time of casting to receive the benefit.

If the spell or effect has a duration other than instantaneous, 
it stops affecting the companion if it moves farther than 5 feet away and will not affect the companion again even if it returns to the master before the duration expires. 
Additionally, the master may cast a spell with a target of ``You`` on his companion (as a touch range spell) instead of on himself.

A master and his companion can share spells even if the spells normally do not affect creatures of the companion's type. 

This is a Supernatural ability.
\subsubsection[Companion Spell Resistance]{Companion Spell Resistance [Companion]}
\label{Feat:CompanionSpellResistance}
Your companion is highly resistant to spells.

\paragraph{Prerequisite:} Master must be 12th level.

\paragraph{Benefit:} The companion gains spell resistance equal to the master's level + 10.

\paragraph{Special:} An animal companion may not select this feat (but a celestial mount can, as can all other kinds of companions).
\subsubsection[Continued Training]{Continued Training [Magical]}
\label{Feat:ContinuedTraining}
Although you have strayed from the path of magic, the abilities you have already gained continue to improve.

\paragraph{Prerequisite:} Spellcraft 4 ranks.

\paragraph{Benefit:} Choose a spellcasting class that you possess.
Your caster level for the chosen casting class increases by four. 
This benefit can't increase your caster level higher than your Hit Dice. 
Even if you can't benefit from the full bonus immediately, if you later gain levels of noncasting classes or levels in other spellcasting classes, you might be able to apply the rest of the bonus.

This feat does not affect your spells known or spell points gained from class levels. 
It only increases your caster level, which helps you overcome spell resistance, increases the duration and other effects of your spells, and determines how many spell points you may spend on a single spell and how many bonus spell points you receive for your key ability score.
\subsubsection[Craft Magic Arms And Armor]{Craft Magic Arms And Armor [Item Creation]}
\label{Feat:CraftMagicArmsAndArmor}
\paragraph{Prerequisite:}
Caster level 5th.

\paragraph{Benefit:} You can create any magic weapon, armor, or shield whose prerequisites you meet. Enhancing a weapon, suit of armor, or shield takes one day for each 1,000 gp in the price of its magical features. To enhance a weapon, suit of armor, or shield, you must spend 1/25 of its features’ total price in XP and use up raw materials costing one-half of this total price.

The weapon, armor, or shield to be enhanced must be a masterwork item that you provide. Its cost is not included in the above cost.

You can also mend a broken magic weapon, suit of armor, or shield if it is one that you could make. Doing so costs half the XP, half the raw materials, and half the time it would take to craft that item in the first place.

\subsubsection[Craft Rod]{Craft Rod [Item Creation]}
\label{Feat:CraftRod}
You can create scepterlike devices called rods, which have a variety of magical powers.

\paragraph{Prerequisite:}
Caster level 9th.

\paragraph{Benefit}
You can create any rod whose prerequisites you meet. 
Crafting a rod takes one day for each 1,000 gp in its base price. 
To craft a rod, you must spend 1/25 of its base price in XP and use up raw materials costing one-half of its base price.

Some rods incur extra costs in material components or XP, as noted in their descriptions. 
These costs are in addition to those derived from the rod’s base price.

\subsubsection[Craft Wand]{Craft Wand [Item Creation]}
\label{Feat:CraftWand}
You can create slender sticks called wands than cast spells when charges are expended.

\paragraph{Prerequisite:}
Caster level 5th.

\paragraph{Benefit:}
You can create a wand of any spell you know (barring exceptions, as noted in a spell's description). 
Crafting a wand takes one day for each 1,000 gp in its base price. 
The base price of a wand is its caster level $\times$ the power level $\times$ 750 gp. 
To craft a wand, you must spend 1/25 of this base price in XP and use up raw materials costing one-half of this base price.

A newly created wand has 50 charges.

Any wand that stores a spell with an XP cost also carries a commensurate cost. 
In addition to the XP cost derived from the base price, you must pay fifty times the XP cost.

\subsubsection[Craft Wondrous Item]{Craft Wondrous Item [Item Creation]}
\label{Feat:CraftWondrousItem}
You can create various wondrous items.

\paragraph{Prerequisite:} Caster level 3rd.

\paragraph{Benefit:}
You can create any wondrous item whose prerequisites you meet. 
Enchanting a wondrous item takes one day for each 1,000 gp in its price. 
To enchant a wondrous item, you must spend 1/25 of the item's price in XP and use up raw materials costing half of this price.

You can also mend a broken wondrous item if it is one that you could make. 
Doing so costs half the XP, half the raw materials, and half the time it would take to craft that item in the first place.

Some wondrous items incur extra costs in material components or XP, as noted in their descriptions. 
These costs are in addition to those derived from the item’s base price. 
You must pay such a cost to create an item or to mend a broken one.

% \subsubsection[]{Deep Impact [Magical]}
% You can strike your foe with a melee weapon as if making a touch attack.
% 
% \paragraph{Prerequisite:} Str 13, Arcane Weapon, base attack bonus +5.
% 
% \paragraph{Benefit:} To use this feat, you must expend your magical focus. 
% You can resolve your attack with a melee weapon as a touch attack. 
% You must decide whether or not to use this feat prior to making an attack. 
% If your attack misses, you still expend your magical focus.
\subsubsection[Deadly Duelist]{Deadly Duelist [Magical]}
\label{Feat:DeadlyDuelist}
To you, \nameref{sec:MageDuels} are not just harmless sparring.

\paragraph{Benefit:} You can expend your magical focus to start a mage duel with unwilling spellcasters. To do so, start the duel normally. However, the challenged must make a Will save (DC 10 + 1/2 your caster level + your key ability modifier) to be allowed to turn down the challenge.

In addition, you can choose to deal lethal damage when you win or your opponent loses a Test of Power.
\paragraph{Normal:} Mage duels are only performed between willing participants, and Tests of Power deal only nonlethal damage.

\subsubsection[Delay Spell]{Delay Spell [Metamagic]}
\label{Feat:DelaySpell}
You can cast spells that go off up to 5 rounds later.

\paragraph{Benefit:} To use this feat, you must expend your magical focus. 
You can cast a spell as a delayed spell. A delayed spell doesn't activate immediately. 
When you cast the spell, you choose one of three trigger mechanisms: 
\begin{enumerate}
 \item The spell activates when you take a standard action to activate it;
 \item It activates when a creature enters the area that the spell will affect (only spells that affect areas can use this trigger condition); or
 \item It activates on your turn after 5 rounds pass.
\end{enumerate}   
If you choose one of the first two triggers and the conditions are not met within 5 rounds, the spell activates automatically on the fifth round.
Only area and personal spells can be delayed.
Any decisions you would make about the delayed spell, including attack rolls, designating targets, or determining or shaping an area, are decided when the spell is cast. 
Any effects resolved by those affected by the spell, including saving throws, are decided when the delay period ends.
A delayed spell can be dispelled normally during the delay, 
and can be detected normally in the area or on the target by the use of spells that can detect magical effects. 
Using this feat increases the spell point cost of the spell by 2. The spell's total cost cannot exceed your caster level.
\subsubsection[Domain Attunement]{Domain Attunement [Magical]}
\label{Feat:DomainAttunement}
You gain the ability to access the spells of a previously inaccessible Cleric Domain.
\paragraph{Prerequisites:} Access to a Domain.

\paragraph{Benefit:} Select a Cleric Domain compatible with your creed and ethos. You add the spells from that domain to your class spell list.

\paragraph{Special:} You can select this feat multiple times. Each time you do, you choose a different Domain to attune yourself to.
\subsubsection[Empower Spell]{Empower Spell [Metamagic]}
\label{Feat:EmpowerSpell}
You can cast spells to greater effect.

\paragraph{Benefit:} To use this feat, you must expend your magical focus.
You can empower a spell. All variable, numeric effects of an empowered spell are increased by one-half. 
An empowered spell deals half again as much damage as normal, 
cures half again as many hit points, affects half again as many targets, and so forth, as appropriate. 
Augmented spells can also be empowered (multiply 1-1/2 times the damage total of the augmented spell). 
For example, a Scorching Ray [Fire] spell augmented to cost 5 spell points would deal
$1.5 \times$ 5d6+5 points of damage.
Saving throws and opposed checks (such as the one you make when you cast dispel magic) are not affected, 
nor are spells without random variables.
Using this feat increases the spell point cost of the spell by 2. The spell's total cost cannot exceed your caster level.

\subsubsection[Enlarge Spell]{Enlarge Spell [Metamagic]}
\label{Feat:EnlargeSpell}
You can cast spells farther than normal.

\paragraph{Benefit:} To use this feat, you must expend your magical focus. 
You can alter a spell with a range of close, medium, or long to increase its range by 100\%. 
An enlarged spell with a range of close has a range of 50 feet + 5 feet per level, 
a medium-range spell has a range of 200 feet + 20 feet per level, 
and a long-range spell has a range of 800 feet + 80 feet per level.
Spells whose ranges are not defined by distance, as well as spells whose ranges are not close, medium, or long, are not affected.
Using this feat does not increase the spell point cost of the spell.

\subsubsection[Expanded Knowledge]{Expanded Knowledge [Magical]}
\label{Feat:ExpandedKnowledge}
You learn another spell.

\paragraph{Prerequisites:} Caster level 3rd.

\paragraph{Benefit:} Add to your spells known one additional spell of any level up to one level lower than the highest-level spell you can cast. 
You can choose any spell, including spells normally restricted to specialist Wizards, or even from another class's list.

\paragraph{Special:} You can gain this feat multiple times. Each time, you learn one new spell at any level up to one less than the highest-level spell you can cast.

\subsubsection[Extend Spell]{Extend Spell [Metamagic]}
\label{Feat:ExtendSpell}
You can cast spells that last longer than normal.

\paragraph{Benefit:} To use this feat, you must expend your magical focus.
You can cast an extended spell. An extended spell lasts twice as long as normal. 
A spell with a duration of concentration, instantaneous, or permanent is not affected by this feat.
Using this feat increases the spell point cost of the spell by 2. The spell's total cost cannot exceed your caster level.

% \subsubsection{Fell Shot [Magical]}
% You can strike your foe with a ranged weapon as if making a touch attack.
% 
% \paragraph{Prerequisite:} Dex 13, Point Blank Shot, Arcane Shot, base attack bonus +5.
% 
% \paragraph{Benefit:} To use this feat, you must expend your magical focus. You can resolve your ranged attack as a ranged touch attack.
% You must decide whether or not to use this feat prior to making an attack. If your attack misses, you still expend your magical focus.
\subsubsection[Forge Ring]{Forge Ring [Item Creation]}
\label{Feat:ForgeRing}
\paragraph{Prerequisite:} 
Caster level 7th.
\paragraph{Benefit:}
You can create any ring whose prerequisites you meet. 
Crafting a ring takes one day for each 1,000 gp in its base price. 
To craft a ring, you must spend 1/25 of its base price in XP and use up raw materials costing one-half of its base price.
 
You can also mend a broken ring if it is one that you could make. 
Doing so costs half the XP, half the raw materials, and half the time it would take to forge that ring in the first place.

Some magic rings incur extra costs in material components or XP, as noted in their descriptions. 
You must pay such a cost to forge such a ring or to mend a broken one.

% \subsubsection{Greater Arcane Shot [Magical]}
% You can charge your ranged attacks with additional damage potential.
% 
% \paragraph{Prerequisite:} Point Blank Shot, Arcane Shot, base attack bonus +5.
% 
% \paragraph{Benefit:} When you use the Arcane Shot feat, your ranged attack deals an extra 4d6 points of damage instead of an extra 2d6 points.
% 
% \subsubsection{Greater Arcane Weapon [Magical]}
% You can charge your melee weapon with additional damage potential.
% 
% \paragraph{Prerequisite:} Str 13, Arcane Weapon, base attack bonus +5.
% 
% \paragraph{Benefit:} When you use the Arcane Weapon feat, your attack with a melee weapon deals an extra 4d6 points of damage instead of an extra 2d6 points.
\subsubsection{Familiar}
\label{Feat:Familiar}
You are the master of a small, intelligent animal that does your bidding.

\paragraph{Prerequisites:} Caster level 1st.

\paragraph{Benefit:} This feat allows you to gain a \nameref{sec:Familiar}.

\paragraph{Special:} If you have the \nameref{Feat:AnimalCompanion} feat, the \nameref{sec:CelestialMountListing} ability, the \nameref{sec:FiendishMountListing} ability, or the \nameref{Feat:SpellstaffUser} feat, you may not take this feat.

\subsubsection[Familiar Spell Delivery]{Familiar Spell Delivery [Companion]}
\label{Feat:FamiliarSpellDelivery}
Your familiar can act as an extension of your own body with respect to touch spells.

\paragraph{Prerequisite:} Master must be 3rd level.

\paragraph{Benefit:} If the master and the familiar are in contact at the time the master casts a touch spell, he can designate his familiar as the ''toucher.'' 
The familiar can then deliver the touch spell just as the master could. 
As usual, if the master casts another spell before the touch is delivered, the touch spell dissipates. 

This is a Supernatural ability.

\paragraph{Special:} Only a familiar or spellstaff can take this feat, not any other kind of companion.
\subsubsection[Familiar Toughness]{Familiar Toughness [Companion]}
\label{Feat:FamiliarToughness}
Your familiar is as hard to kill as any other creature.

\paragraph{Benefit:} Your familiar gains hit points for its HD as any other creature does, receiving the maximum possible result at 1st HD, and rolling thereafter.

\paragraph{Normal:} Your familiar receives the minimum possible number of hit points each hit die, including the first.
\subsubsection[Fiendish Summons]{Fiendish Summons [Magical]}
\label{Feat:FiendishSummons}
Your summoned creatures have a fiendish bent.

\paragraph{Prerequisites:} Knowledge (the planes) 4 ranks, evil alignment.

\paragraph{Benefit:} Whenever you use a Conjuration (Summoning) spell to summon an animal, you can apply the Fiendish template to it as part of the summoning process. This adds the [Evil] descriptor to the spell.
\subsubsection[Greater Magical Endowment]{Greater Magical Endowment [Magical]}
\label{Feat:GreaterMagicalEndowment}
You can endow your spells with more concentrated focus.

\paragraph{Prerequisite:} Magical Endowment.

\paragraph{Benefit:} When you use the Magical Endowment feat, you add +2 to the save DC of a spell you cast instead of +1.
\subsubsection[Imbue Matrix]{Imbue Matrix [Item Creation]}
\label{Feat:ImbueMatrix}
\paragraph{Prerequisite:}
Caster level 1st.

\paragraph{Benefit:}
You can create \nameref{Item:Matrices} of any spell that you know. 
Imbuing a matrix takes one day for each 1,000 gp in its base price. 
The base price of a matrix is its spell level $\times$ its caster level $\times$ 25 gp, see the \nameref{tab:Matrices} table.
To infuse a matrix, you must spend 1/25 of this base price in XP and use up raw materials costing one-half of this base price.

Any matrix that stores a spell with an XP cost also carries a commensurate cost. 
In addition to the costs derived from the base price, you must pay the XP when infusing the matrix.

Ordinarily, matrices are spell completion items. 
However, by doubling the base cost of the matrix (and all derived factors), you may instead create \nameref{Item:FoolproofMatrices}, usable by anyone.
\subsubsection[Magical Endowment]{Magical Endowment [Magical]}
\label{Feat:MagicalEndowment}
You can endow your spells with a little bit of extra focus.

\paragraph{Benefit:} To use this feat, you must expend your magical focus. You add 1 to the save DC of a spell you cast.

\subsubsection[Magical Spark]{Magical Spark [General]}
\label{Feat:MagicalSpark}
You were born with the spark of magic in your blood.

\paragraph{Prerequisite:} This feat can only be taken at 1st level.

\paragraph{Benefit:}
You become a Magical character. You gain a reserve of 2 spell points, and you can take
magical feats, metamagic feats, and item creation feats.
If you have or take a class that grants spell points, the spell
points gained from Magical Spark are added to your total spell
point reserve.
When you take this feat, choose one 1st-level spell from any
magical class list (but not a Specialist only Wizard spell).
You know this spell (it becomes one of your
spells known). You can cast this spell with the spell
points provided by this feat if you have a Charisma score of 11
or higher. If you have no levels in a spellcasting class, you are considered
a 1st-level spellcaster when casting this spell. If you have
spellcasting class levels, you can cast the spell at the highest
caster level you have attained, and use the key ability modifier of that class to determine this spell's saving throw DC.
If you have no spellcasting class levels, use Charisma to determine how hard your spell is to resist.

\subsubsection[Magically Gifted]{Magically Gifted [Magical]}
\label{Feat:MagicallyGifted}
You gain additional spell points to supplement those you already had.

\paragraph{Prerequisite:} Having a spell point reserve.

\paragraph{Benefit:} You gain a number of spell points equal to 1 + your number of levels in spellcasting classes (levels of different spellcasting classes stack). If you later gain more levels in spellcasting classes, the number of spell points you gain from this feat increases accordingly. If you lose a spellcasting level (such as from energy drain), you lose a spell point.

\subsubsection[Maximize Spell]{Maximize Spell [Metamagic]}
\label{Feat:MaximizeSpell}
You can cast spells to maximum effect.

\paragraph{Benefit:} To use this feat, you must expend your magical focus.
You can maximize a spell. All variable, numeric effects of a spell modified by this feat are maximized. 
A maximized spell deals maximum damage, cures the maximum number of hit points, affects the maximum number of targets, and so on, as appropriate. 
Saving throws and opposed checks are not affected, nor are spells without random variables.
Augmented spells can be maximized; a maximized augmented spell deals the maximum damage (or cures the maximum hit points, and so on) of the augmented spell.
An empowered and maximized spell gains the separate benefits of each feat: the maximum result plus one-half the normally rolled result.
Using this feat increases the spell point cost of the spell by 4. The spell's total cost cannot exceed your caster level.
\footnote{\textbf{When to maximize?}

The canny reader may have realized that simply slapping a damage-enhancing metamagic feat like Empower Spell or Maximize Spell
on the spell does not always result in more damage, as the metamagic costs prevent you from fully augmenting your spells.
See table \ref{tab:metamagicmath} for an example comparison.}
\begin{table*}
\label{tab:metamagicmath}
\caption{A metamagic comparison}
\begin{center}
\begin{tabular}{|c|c|c|c|c|c|}
\hline
SP  &Augment&Augment&Empowered&Maximized&Emp. \& Max.\\
cost&dice   &average&average  &average  &average\\
\hline
 1 &  1d6 &  3.5 &  -   &  -   &   -   \\ 
 2 &  2d6 &  7.0 &  -   &  -   &   -   \\ 
 3 &  3d6 & 10.5 &  5.2 &  -   &   -   \\ 
 4 &  4d6 & 14.0 & 10.5 &  -   &   -   \\ 
 5 &  5d6 & 17.5 & 15.8 &  6.0 &   -   \\ 
 6 &  6d6 & 21.0 & 21.0 & 12.0 &   -   \\ 
 7 &  7d6 & 24.5 & 26.2 & 18.0 &   7.8 \\ 
 8 &  8d6 & 28.0 & 31.5 & 24.0 &  15.5 \\ 
 9 &  9d6 & 31.5 & 36.8 & 30.0 &  23.2 \\ 
10 & 10d6 & 35.0 & 42.0 & 36.0 &  31.0 \\ 
11 & 11d6 & 38.5 & 47.2 & 42.0 &  38.8 \\ 
12 & 12d6 & 42.0 & 52.5 & 48.0 &  46.5 \\ 
13 & 13d6 & 45.5 & 57.8 & 54.0 &  54.2 \\ 
14 & 14d6 & 49.0 & 63.0 & 60.0 &  62.0 \\ 
15 & 15d6 & 52.5 & 68.2 & 66.0 &  69.8 \\ 
16 & 16d6 & 56.0 & 73.5 & 72.0 &  77.5 \\ 
17 & 17d6 & 59.5 & 78.8 & 78.0 &  85.2 \\ 
18 & 18d6 & 63.0 & 84.0 & 84.0 &  93.0 \\ 
19 & 19d6 & 66.5 & 89.2 & 90.0 & 100.8 \\ 
20 & 20d6 & 70.0 & 94.5 & 96.0 & 108.5 \\ 
\hline
\end{tabular}
\end{center}

{\small This uses a hypothetical first level spell that deals 1d6 points of damage,
and has an augment that increases its damage by 1d6 per additional SP spent.
This does not take into account any external factors, such as metamagic-enhanced
spells having lower saving throw DCs than equivalent augmented spells.}
\end{table*}
\subsubsection[Mental Fortress]{Mental Fortress [Magical]}
\label{Feat:MentalFortress}
Sticks and stones may break your bones, but words will never hurt you.
\paragraph{Prerequisite:} \nameref{Feat:NoMansFool}
\paragraph{Benefit:} While magically focused, you are immune to all mind-affecting spells and effects.
\subsubsection[No Man's Fool]{No Man's Fool [Magical]}
\label{Feat:NoMansFool}
You are not the kind of person to fall prey to cheap parlor tricks.
\paragraph{Benefit:} While magically focused, you are protected from all forms of possession and mental control (including enchantment (charm) effects and enchantment (compulsion) effects that would grant the caster ongoing control over you, such as \nameref{Spell:Dominate}). 
The protection does not prevent such effects from targeting you, but it suppresses the effect for as long as you are magically focused. 
If you lose your magical focus before the effect granting mental control does, the would-be controller would then be able to mentally command you. 
Gaining magical focus does not expel a possesing life force if it is in place before the focus is gained. 
\subsubsection[Opportunity Spell]{Opportunity Spell [Metamagic]}
\label{Feat:OpportunitySpell}
You can make spell-enhanced attacks of opportunity.

\paragraph{Benefit:} To use this feat, you must expend your magical focus. 
When you make an attack of opportunity, you can use any spell you know with a range of touch, if you have at least one hand free.
Note that this metamagic feat does not increase your natural reach.
Casting this spell is an immediate action.
You cannot use this feat with a touch spell whose casting time is longer than 1 full-round action.
Using this feat increases the spell point cost of the spell by 6. The spell's total cost cannot exceed your caster level.

\paragraph{Normal:} Attacks of opportunity can be made only with melee weapons.

\footnote{A note on the \emph{Persistent Spell} Metamagic feat that appears in the \href{http://www.wizards.com/default.asp?x=d20/article/srd35}{d20 srd}:
This feat is intentionally omitted. Spells can now have augments that drastically alter their durations - spells that don't have such augments
are usually spells that shouldn't have their durations greatly fiddled with in the first place. Adding this feat is not recommended.}
\subsubsection[Pushing the Limits]{Pushing the Limits [Magical]}
\label{Feat:PushingTheLimits}
You can burn your life force to strengthen your spells.

\paragraph{Benefit:} While casting a spell, you can increase your effective caster level by one, but in so doing you take 1d8 points of damage. 
At 8th level, you can choose to increase your effective caster level by two, but you take 3d8 points of damage. 
At 15th level, you can increase your effective caster level by three, but you take 5d8 points of damage.
The effective increase in caster level increases all caster level-dependent effects, such as range, duration, and 
your ability to overcome spell resistance. \emph{However}, unlike most other effects that increase your caster level, this does \emph{not}
increase the number of spell points you can spend on a single spell. 
This is an exception from the fundamental rule of magic (see \nameref{sec:MagicOverview}).

\paragraph{Normal:} Your caster level is equal to your total levels in classes that cast spells.

\subsubsection[Quicken Spell]{Quicken Spell [Metamagic]}
\label{Feat:QuickenSpell}
You can cast a spell with a moment's thought.

\paragraph{Benefit:} To use this feat, you must expend your magical focus. 
You can quicken a spell, reducing the spell's casting time to 1 swift action. 
A spell whose casting time is longer than 1 full round action cannot be quickened.
Using this feat increases the spell point cost of the spell by 6. The spell's total cost cannot exceed your caster level.
Casting a quickened spell does not provoke attacks of opportunity.
\subsubsection[Rapid Refocusing]{Rapid Refocusing [Magical]}
\label{Feat:RapidRefocusing}
You can focus your mind faster than normal, even under duress.

\paragraph{Prerequisite:} Wis 13, Concentration 7 ranks.

\paragraph{Benefit:} You can take a move action to become magically focused.

\paragraph{Normal:} A character without this feat must take a full-round action to become magically focused.

\subsubsection[Scribe Scroll]{Scribe Scroll [Item Creation]}
\label{Feat:ScribeScroll}
\paragraph{Prerequisite:}
Caster level 1st.

\paragraph{Benefit:}
You can create \nameref{Item:Scrolls} of any spell that you know. 
Scribing a scroll takes one day for each 1,000 gp in its base price. 
The base price of a scroll is its spell level squared times 1000 gp, see the \nameref{tab:Scrolls} table.
%The base price of a scroll is its spell level $\times$ its caster level $\times$ 25 gp.
To scribe a scroll, you must spend 1/25 of this base price in XP and use up raw materials costing one-half of this base price.

Any scroll that stores a spell with an XP cost also carries a commensurate cost. 
In addition to the costs derived from the base price, you must pay the XP when scribing the scroll.
\subsubsection[Scry and Die]{Scry and Die [Metamagic]}
\label{Feat:ScryAndDie}
\paragraph{Prerequisite:} 
Must know the \nameref{Spell:Scrying} spell.

\paragraph{Benefit:} To use this feat, you must expend your magical focus. 
After successfully finding a creature with the \nameref{Spell:Scrying} spell, 
you can cast any targeted spell you know with a range of 10' or more through the sensor
on the scryed subject, as if you were located near the creature yourself.
You cannot cast spells on any other creature than the one you were scrying on, 
even if it is within the sensor's range of vision. You cannot cast any spell
on the creature you wouldn't be able to cast if you were physically present.
Using this feat doubles the spell's spell point cost.
The spell's total cost cannot exceed your caster level.
\subsubsection[Silent Spell]{Silent Spell [Metamagic]}
\label{Feat:SilentSpell}
You are an expert in casting spells without making a sound.

\paragraph{Benefit:} To use this feat, you must expend your magical focus.
This allows you to automatically succeed on any Concentration check required to remove a spell's Verbal component.
Using this feat does not increase the spell's spell point cost.

\paragraph{Special:} If you also have the \nameref{Feat:StillSpell} feat, you can apply the benefits of that feat along with this one, only expending your magical focus once 
(effectively, making the combination possible without Spellstaff Containment).
\subsubsection[Spellstaff Containment]{Spellstaff Containment [Magical]}
\label{Feat:SpellstaffContainment}
Your Spellstaff has advanced enough that it can hold a magical focus that you store within it.

\paragraph{Prerequisites:} Spellstaff User, caster level 3rd.

\paragraph{Benefit:} You can magically focus your Spellstaff. 
At any time when you need to expend your magical focus, you can expend your Spellstaff's magical focus instead, as long as the staff is within 5 feet of you. 
Magically focusing your Spellstaff works just like focusing yourself (normally a full-round action).
The Spellstaff cannot focus itself - only the owner can spend the time to focus the crystal.

\subsubsection{Spellstaff User}
\label{Feat:SpellstaffUser}
You have created a Spellstaff.

\paragraph{Prerequisites:} Caster level 1st.
 
\paragraph{Benefit:} This feat allows you to gain a \nameref{sec:SpellStaff}. 

\paragraph{Special:} If you have the \nameref{Feat:AnimalCompanion} feat, the \nameref{sec:CelestialMountListing} ability, the \nameref{Feat:Familiar} feat, or the \nameref{sec:FiendishMountListing} ability, you may not take this feat.
\subsubsection[Split Ray]{Split Ray [Metamagic]}
\label{Feat:SplitRay}
You can affect two targets with a single ray.

\paragraph{Prerequisite:} Any other metamagic feat.

\paragraph{Benefit:} To use this feat, you must expend your magical focus. You can split rays you cast. 
The split ray affects any two targets that are both within the spell's range and within 30 feet of each other. 
If the ray deals damage, each target takes as much damage as a single target would take.
Using this feat increases the spell point cost of the spell by 2.

\subsubsection[Still Spell]{Still Spell [Metamagic]}
\label{Feat:StillSpell}
You are an expert in casting spells without moving a muscle.

\paragraph{Benefit:} To use this feat, you must expend your magical focus.
This allows you to automatically succeed on any Concentration check required to remove a spell's Somatic component.
Using this feat does not increase the spell's spell point cost.

\paragraph{Special:} If you also have the \nameref{Feat:SilentSpell} feat, you can apply the benefits of that feat along with this one, only expending your magical focus once 
(effectively, making the combination possible without \nameref{Feat:SpellstaffContainment}).
\subsubsection[Turn Undead]{Turn Undead [Magical]}
\label{Feat:TurnUndead}
You can project the power of your faith to harm the living dead.

\paragraph{Prerequisites:} Cleric or Paladin level 1st, non-evil alignment.

\paragraph{Benefit:} In order to Turn Undead, you must expend your magical focus as a standard action.
All undead creatures within 60' take 1d6 points of damage per hit die you have. 
A successful Will saving throw (DC 10 + $1/2$ your character level + your charisma modifier) halves the damage\footnote{
Some creatures have Turn Resistance. This means that the creature gains a profane bonus on the Will save equal to that amount.}.
You must have line of sight and line of effect to each individual undead creature to be affected. This is a Supernatural ability.
\subsubsection[Twin Spell]{Twin Spell [Metamagic]}
\label{Feat:TwinSpell}
You can cast a spell simultaneously with another spell just like it.

\paragraph{Benefit:} To use this feat, you must expend your magical focus. 
You can twin a spell. Casting a spell altered by this feat causes the spell to take effect twice on the area or target, 
as if you were simultaneously casting the same spell two times on the same location or target. 
Any variables in the spell (such as duration, number of targets, and so on) are the same for both of the resulting spells. 
The target experiences all the effects of both spells individually and receives a saving throw (if applicable) for each. 
In some cases, such as a twinned \nameref{Spell:Charm}, failing both saving throws results in redundant effects 
(although, in this example, any ally of the target would have to succeed on two dispel attempts to free the target from the charm effect).
Using this feat increases the spell point cost of the spell by 6. The spell's total cost cannot exceed your caster level.

\subsubsection[Unconditional Spell]{Unconditional Spell [Metamagic]}
\label{Feat:UnconditionalSpell}
Disabling conditions do not hold you back.

\paragraph{Benefit:} To use this feat, you must expend your magical focus. Your mental strength is enough to overcome some otherwise disabling conditions. You can cast an unconditional spell when you are dazed, confused, nauseated, shaken, or stunned.
Only personal spells and spells that affect your person can be cast as unconditional spells.
Using this feat increases the spell point cost of the spell by 8. The spell's total cost cannot exceed your caster level.