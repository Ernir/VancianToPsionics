\documentclass[a4paper,10pt]{article}
\usepackage[T1]{fontenc}
\usepackage[utf8]{inputenc}
\usepackage{scalefnt}
\usepackage[top=0.9in, bottom=1in, left=0.9in, right=0.9in]{geometry}
\usepackage{multirow}
\usepackage[table]{xcolor}
\usepackage{amsmath}
\usepackage{textcomp}
\usepackage{multicol}
\usepackage{graphicx}
\usepackage{tabularx}
\usepackage{titlesec}
%\usepackage{stfloats}
%\usepackage{parskip}
%\usepackage{helvetica}
\usepackage{times} %Times font
\usepackage{tikz}
\usepackage[pdftex,bookmarks=true,colorlinks=true,linkcolor=blue,pdfauthor={Eirikur Ernir Thorsteinsson},pdftitle={From Vancian to Psionic}]{hyperref}
\usepackage[all]{hypcap}

\setlength{\parskip}{1ex plus 0.2ex minus 0.2ex}
\titlespacing*{\subsubsection}{0pt}{3.25ex plus 0.2ex minus .2ex}{1.5ex plus .2ex}
\titlespacing*{\paragraph} {0pt}{1ex plus 0.2ex minus 0.2ex}{1em}
\titlespacing*{\subparagraph} {\parindent}{1ex plus 0.2ex minus 0.2ex}{1em}

\setcounter{topnumber}{2}
\setcounter{bottomnumber}{1}
\setcounter{totalnumber}{2}
\setcounter{dbltopnumber}{2}
\renewcommand{\topfraction}{0.85}
\renewcommand{\dbltopfraction}{0.85}
\renewcommand{\bottomfraction}{0.85}
\renewcommand{\textfraction}{0.15}
\renewcommand{\floatpagefraction}{0.7}
\renewcommand{\dblfloatpagefraction}{0.7}

\setcounter{secnumdepth}{-1} 
\hyphenpenalty=5000
\interfootnotelinepenalty=10000
\frenchspacing
\title{From Vancian to Psionic}
\date{}
\author{}

\begin{document}
\begin{titlepage}
\maketitle
\thispagestyle{empty}
\includegraphics[width=\textwidth]{Pics/TripleSpiral.png}
\vfill
\begin{flushright} \large
\emph{Version:} \\
Beta 1.11
\end{flushright}
\end{titlepage}
\section{Table of Contents}
\setlength{\columnsep}{0.5in}
\begin{multicols}{2}
\scriptsize
\tableofcontents
\listoftables
\listoffigures
\normalsize
\end{multicols}
\newpage
\begin{multicols}{2}

\part{The Spellcasting System}
The most fundamental part of converting a spellcasting system must be to change the base mechanics themselves. The following chapter deals with translating the underlying \emph{Psionic system} to use magical terminology.

\section{Magic Overview}
\label{sec:MagicOverview}
Magical powers spring from the rotations of the planes, the souls of living creatures, the will of the deities,
and more mysterious sources still.
It is a virtually omnipresent power, as real as muscle and steel.

Even an undead creature or a being that has no physical form can have the reserve of inner strength necessary to cast spells, 
as long as it has an Intelligence score of at least 1. %Vermin possessed of a hive mind ability are an exception to this rule.

A spell is a one-time magical effect. 

Spellcasting characters and creatures need not prepare their spells for use ahead of time. 
They either have sufficient spell points to cast a spell, or they do not.
A spell is cast when a spellcasting character pays its spell point cost. 
Some innately magical creatures automatically cast spells, called spell-like abilities, without paying a spell point cost. 
Other creatures pay spell points to cast their spells, just as characters do.

Each spell has a specific effect. A spell known to a spellcasting character can be used whenever he or she can spend the spell points to pay for it.

Magic has one fundamental rule. This most fundamental rule of magic is as follows:
\begin{quote}
\Large{The maximum number of spell points you can spend on a spell is equal to your caster level.}
\end{quote}
Spell points and caster levels are explained in detail below.
\subsection{Casting Spells}
Spellcasting characters and magical creatures cast spells. 
Whether they cost spell points when cast by a spellcasting character, or are cast as spell-like abilities, spells' effects remain the same.
\subsubsection{Choosing a Spell}
First you must choose which spell to cast. 
You can select any spell you know, provided you are capable of casting spells of that level or higher. 
To cast a spell, you must pay spell points, which count against your daily total. 
You can cast the same spell multiple times if you have points left to pay for it.

\subsubsection{Concentration}
To cast a spell, you must concentrate. If something threatens to interrupt your concentration while you're casting a spell, you must succeed on a \nameref{sec:Concentration} check or lose the spell points without casting the spell. 
The more distracting the interruption and the higher the level of the spell that you are trying to cast, the higher the DC. 
(Higher-level spells require more mental effort.)

\paragraph{Injury:} Getting hurt or being affected by hostile magic while trying to cast a spell can break your concentration and ruin a spell. 
If you take damage while trying to cast a spell, you must make a \nameref{sec:Concentration} check (DC 10 + points of damage taken + the level of the spell you're casting). 
The interrupting event strikes during casting if it occurs between when you start and when you complete casting a spell (for a spell with a casting time of 1 round or longer) or if it comes in response to your casting the spell (such as an attack of opportunity provoked by the casting of the spell or a contingent attack from a readied action).
If you are taking continuous damage half the damage is considered to take place while you are casting a spell. 
You must make a Concentration check (DC 10 + 1/2 the damage that the continuous source last dealt + the level of the spell you're casting).
If the last damage dealt was the last damage that the effect could deal then the damage is over, and it does not distract you.
Repeated damage does not count as continuous damage.

\paragraph{Spell:} If you are affected by a spell while attempting to cast a spell of your own, you must make a \nameref{sec:Concentration} check or lose the spell you are casting.
If the spell affecting you deals damage, the Concentration DC is 10 + points of damage + the level of the spell you're casting. 
If the spell interferes with you or distracts you in some other way, the Concentration DC is the spell's save DC + the level of the spell you're casting. 
For a spell with no saving throw, it's the DC that the spell's saving throw would have if a save were allowed.

\paragraph{Grappling or Pinned:} To cast a spell while grappling or pinned, you must make a \nameref{sec:Concentration} check (DC 20 + the level of the spell you're casting) or lose the spell. 
You cannot provide a somatic component (see components, below) while grappling, and if pinned, you may not be able to provide a verbal component, at the option of the creature that has you pinned.

\paragraph{Vigorous Motion:} If you are riding on a moving mount, taking a bouncy ride in a wagon, on a small boat in rough water, belowdecks in a storm-tossed ship, or simply being jostled in a similar fashion, you must make a \nameref{sec:Concentration} check (DC 10 + the level of the spell you're casting) or lose the spell.

\paragraph{Violent Motion:} If you are on a galloping horse, taking a very rough ride in a wagon, on a small boat in rapids or in a storm, on deck in a storm-tossed ship, or being tossed roughly about in a similar fashion, you must make a \nameref{sec:Concentration} check (DC 15 + the level of the spell you're casting) or lose the spell.

\paragraph{Violent Weather:} If you are in a high wind carrying blinding rain or sleet, the DC is 5 + the level of the spell you're casting. If you are in wind-driven hail, dust, or debris, the DC is 10 + the level of the spell you're casting. In either case, you lose the spell if you fail the \nameref{sec:Concentration} check. If the weather is caused by a spell, use the rules in the spell subsection above.

\paragraph{Casting spells on the Defensive:} If you want to cast a spell without provoking attacks of opportunity, you need to dodge and weave. You must make a \nameref{sec:Concentration} check (DC 15 + the level of the spell you're casting) to succeed. You lose the spell points without successful casting it if you fail.

\paragraph{Entangled:} If you want to cast a spell while entangled in a net or while affected by a spell with similar effects you must make a DC 15 \nameref{sec:Concentration} check to cast the spell. You lose the spell if you fail.

\subsubsection{Counterspells}
\label{sec:Counterspells}
It is possible to cast any spell as a counterspell. 
By doing so, you are using the spell's energy to disrupt the casting of the same spell by another character. 
Counterspelling works even if one spell is divine and the other arcane.

\paragraph{How Counterspells Work}
To use a counterspell, you must select an opponent as the target of the counterspell. 
You do this by choosing the ready action. 
In doing so, you elect to wait to complete your action until your opponent tries to cast a spell. 
(You may still move your speed, since ready is a standard action.)

If the target of your counterspell tries to cast a spell, make a \nameref{sec:Spellcraft} check (DC 15 + the spell's level). 
This check is a free action. 
If the check succeeds, you correctly identify the opponent's spell and can attempt to counter it. 
If the check fails, you can't do either of these things.

To complete the action, you must then cast the correct spell.
As a general rule, a spell can only counter itself. 
If you are able to cast the same spell you cast it, altering it slightly to create a counterspell effect. 
If the target is within range of the spell, both spells automatically negate each other with no other results.

\paragraph{Counterspelling Metamagic Spells and Augmented Spells}
Augments and metamagic feats are not taken into account when determining whether a spell can be countered.
You do not need to match the opponent's spell augments or metamagic applications.

\paragraph{Dispel Magic as a Counterspell}
You can use \nameref{Spell:DispelMagic} to counterspell another spellcaster, and you don't need to identify the spell he or she is casting. 
However, \nameref{Spell:DispelMagic} doesn't always work as a counterspell.

\subsubsection{Caster Level}
The variables of a spell's effect often depend on its caster level, which is (usually) equal to your spellcasting class level. 
A spell that can be augmented for additional effect is also limited by your caster level (you can't spend more spell points on a spell than your caster level). 
See \nameref{sec:Augment}, below.
You can cast a spell at a lower caster level than normal, but the caster level must be high enough for you to cast the spell in question, and all level-dependent features must be based on the same caster level.
In the event that a class feature or other special ability provides an adjustment to your caster level, this adjustment applies not only to all effects based on caster level (such as range, duration, and augmentation potential) but also to your caster level check to overcome your target's spell resistance and to the caster level used in dispel checks (both the dispel check and the DC of the check).

\subsubsection{Spell Failure}
If you try to cast a spell in conditions where the characteristics of the spell (range, area, and so on) cannot be made to conform, the spell fails and the spell points are wasted. 
Spells also fail if your concentration is broken (see \nameref{sec:Concentration}, above).

\subsubsection{The Spell's Result}
Once you know which creatures (or objects or areas) are affected, and whether those creatures have made successful saving throws (if any were allowed), you can apply whatever results a spell entails.

\subsubsection{Special Spell Effects}
Certain special features apply to all spells.

\paragraph{Attacks:} Some spells refer to attacking. 
All offensive combat actions, even those that don't damage opponents, such as disarm and bull rush, are considered attacks. 
All spells that opponents can resist with saving throws, that deal damage, or that otherwise harm or hamper subjects are considered attacks. 
\nameref{Spell:SummonMonster} and similar spells are not considered attacks because the spells themselves don't harm anyone.

\paragraph{Bonus Types:} Many spells give creatures bonuses to ability scores, Armor Class, attacks, and other attributes. 
Each bonus has a type that indicates how the spell grants the bonus. 
The important aspect of bonus types is that two bonuses of the same type don't generally stack. 
With the exception of dodge bonuses, most circumstance bonuses, and racial bonuses, only the better bonus works (see \nameref{sec:CombiningMagicalEffects}).
The same principle applies to penalties - a character taking two or more penalties of the same type applies only the worst one.
If the type of a bonus is not specified, it is an ``untyped'' bonus, which stacks with everything but another instance of what granted the untyped bonus.

\paragraph{Bringing Back the Dead:} Some powerful spells have the ability to restore slain characters to life. 
When a living creature dies, its soul departs the body, leaves the Material Plane, travels through the Astral Plane, and goes to abide on the plane where the creature's deity resides. 
If the creature did not worship a deity, its soul departs to the plane corresponding to its alignment. 
Bringing someone back from the dead means retrieving his or her soul and returning it to his or her body.

\subparagraph{Level Loss:} The passage from life to death and back again is a wrenching journey for a being's soul. 
Consequently, any creature brought back to life usually loses one level of experience.
The character's new experience point total is midway between the minimum needed for his or her new (reduced) level and the minimum needed for the next one. 
If the character was 1st level at the time of death, he or she loses 2 points of Constitution instead of losing a level. 
This level loss or Constitution loss cannot be repaired by any mortal means, even the spells \nameref{Spell:Wish} or \nameref{Spell:Miracle}. 
A revived character can regain a lost level by earning XP through further adventuring. 
A revived character who was 1st level at the time of death can regain lost points of Constitution by improving his or her Constitution score when he or she attains a level that allows an ability score increase.

\subparagraph{Memory Loss:} A character brought back from the dead has barely any memories relating to its afterlife. At most, a vague sensation of what the deity's plane felt like remains.

\subparagraph{Preventing Revivification:} 
Enemies can take steps to make it more difficult for a character to be returned from the dead. 
Keeping the body being the most elementary, though the most powerful of spellcasters can bypass this limitation. 
See individual spell descriptions.

\subparagraph{Revivification Against One's Will:} 
A soul knows the name, alignment, and patron deity (if any) of the character attempting to revive it and may refuse to return on that basis.
Only the foulest magic can return a soul to life if it does not wish to be.
\subsubsection{Combining Magical Effects}
\label{sec:CombiningMagicalEffects}
\paragraph{Psionics-Magic Transparency:} The default rule for the interaction of psionics and magic is simple: 
Powers interact with spells and spells interact with powers in the same way a spell or normal spell-like ability interacts with another spell or spell-like ability. 
This is known as psionics-magic transparency.
Though not explicitly called out in the spell descriptions or magic item descriptions, spells, spell-like abilities, and magic items that could potentially affect psionics do affect psionics. 
When the rule about psionics-magic transparency is in effect, it has the following ramifications:
\begin{list}{\labelitemi}{\leftmargin=1em}
\item Spell resistance is effective against powers, using the same mechanics. 
Likewise, power resistance is effective against spells, using the same mechanics as spell resistance. 
If a creature has one kind of resistance, it is assumed to have the other. (The effects have similar ends despite having been brought about by different means.)
\item All spells that dispel magic (such as \nameref{Spell:DispelMagic}) have equal effect against powers of the same level using the same mechanics, and vice versa.
\item The spell \nameref{Spell:DetectMagic} detects powers as if they were spells.
\item Dead magic areas are also dead psionics areas.
\end{list}
\paragraph{Multiple Effects:} Spells or magical effects usually work as described no matter how many other spells or magical effects happen to be operating in the same area or on the same recipient. 
Except in special cases, a spell does not affect the way another spell operates. 
Whenever a spell has a specific effect on other spells, the spell description explains the effect (and vice versa for spells that affect spells). 
Several other general rules apply when spells or magical effects operate in the same place.

\paragraph{Stacking Effects:} Spells that provide bonuses or penalties on attack rolls, damage rolls, saving throws, and other attributes usually do not stack with themselves. 
More generally, two bonuses of the same type don't stack even if they come from different spells. 
You use whichever bonus gives you the better result. 

\paragraph{Different Bonus Types:} The bonuses or penalties from two different spells stack if the effects are of different types. 
A bonus that isn't named (just a ``+2 bonus`` rather than a ``+2 insight bonus'') stacks with any bonus but another instance of the same effect that granted the bonus.

\paragraph{Same Effect More than Once in Different Strengths:} 
In cases when two or more similar or identical effects are operating in the same area or on the same target, but at different strengths, only the best one applies. 
If one spell is dispelled or its duration runs out, the other spell remains in effect (assuming its duration has not yet expired).

\paragraph{Same Effect with Differing Results:} 
The same spell can sometimes produce varying effects if applied to the same recipient more than once. 
The last effect in a series trumps the others. 
None of the previous spells are actually removed or dispelled, but their effects become irrelevant while the final spell in the series lasts.

\paragraph{One Effect Makes Another Irrelevant:} Sometimes, a spell can render another spell irrelevant.

\subparagraph{Multiple Mental Control Effects:} Sometimes magical effects that establish mental control render one another irrelevant. 
Mental controls that don't remove the recipient's ability to act usually do not interfere with one another, though one may modify another. If a creature is under the control of two or more creatures, it tends to obey each to the best of its ability, and to the extent of the control each effect allows. 
If the controlled creature receives conflicting orders simultaneously, the competing controllers must make opposed Charisma checks to determine which one the creature obeys.

\subparagraph{Spells with Opposite Effects:} Spells with opposite effects apply normally, with all bonuses, penalties, or changes accruing in the order that they apply.
Some spells negate or counter each other. This is a special effect that is noted in a spell's description.

\paragraph{Instantaneous Effects:} Two or more magical effects with instantaneous durations work cumulatively when they affect the same object, place, or creature.

\subsection{The Spell Point Reserve}
Spellcasting characters fuel their abilities through a pool, or reserve, of spell points. 
Your spell point reserve is equal to your base spell points gained from your class\footnote{
Individual class tables contain information on how many spell points are granted by that class. However, the mathematically inclined may be interested in the formulas behind the tables:
\begin{list}{\labelitemi}{\leftmargin=1em}
 \item The Bard has $\lfloor\frac{\text{level}^2}{2}\rfloor$ base SP/day.
 \item The Cleric and Wizard have $\lceil (\text{level}^2+\text{level}+1) \cdot \frac{3}{4}\rceil$ base SP/day.
 \item The Sorcerer has $\lceil \text{level}^2+\text{level}+1 \rceil$ base SP/day.
 \item The Paladin and Ranger progressions are not known to me, as they emulate the Psychic Warrior progression. Polynomial fitting has not been fruitful.
\end{list}
In all the formulas, ``level'' refers to the class level of the spellcasting class in question.}, bonus spell points from a high key ability score (see Abilities and Spellcasters, below), and any additional bonus spell points from sources such as your character race and feat selections.
\subsubsection{Multiclass Spellcasting Characters}
If you have levels in more than one spellcasting class, you combine your spell points from each class to make up your reserve. 
You can use these spell points to cast spells from any spellcasting class you have. 
While you maintain a single reserve of spell points from your class, race, and feat selections, you are still limited by the caster level you have achieved with each spell you know. 
\subsubsection{Abilities and spellcasters}
The ability that your spells depend on - your key ability score as a spellcaster - is related to what spellcasting class (or classes) you have levels in, as detailed for each individual spellcasting class.
The modifier for this ability is referred to as your key ability modifier. 
If your character's key ability score is 9 or lower, you can't cast spells from that spellcasting class.

\textbf{How To Determine Bonus Spell Points:} 
Your key ability score grants you additional spell points equal to 
\begin{quote}
\centering
\large 
 {your key ability modifier $\times$ your caster level $\times \frac{1}{2}$.}
\end{quote}
The results of this equation are summarized on the \nameref{tab:BonusSpellPoints} table.
\begin{table*}
\label{tab:BonusSpellPoints}
\caption{Ability Modifiers and Bonus Spell Points}
\makebox[\textwidth]{\resizebox{\textwidth}{!}{
\begin{tabular}{|p{0.06\textwidth}|*{19}{p{0.025\textwidth}}p{0.03\textwidth}|}
\hline
Ability&\multicolumn{20}{c|}{Bonus Spell Points (by Caster Level)}\\
Score&	1st&	2nd&	3rd&	4th&	5th&	6th&	7th&	8th&	9th&	10th&	11th&	12th&	13th&	14th&	15th&	16th&	17th&	18th&	19th&	20th\\
\hline
10-11&	0&	0&	0&	0&	0&	0&	0&	0&	0&	0&	0&	0&	0&	0&	0&	0&	0&	0&	0&	0\\
12-13&	0&	1&	1&	2&	2&	3&	3&	4&	4&	5&	5&	6&	6&	7&	7&	8&	8&	9&	9&	10\\
14-15&	1&	2&	3&	4&	5&	6&	7&	8&	9&	10&	11&	12&	13&	14&	15&	16&	17&	18&	19&	20\\
16-17&	1&	3&	4&	6&	7&	9&	10&	12&	13&	15&	16&	18&	19&	21&	22&	24&	25&	27&	28&	30\\
18-19&	2&	4&	6&	8&	10&	12&	14&	16&	18&	20&	22&	24&	26&	28&	30&	32&	34&	36&	38&	40\\
20-21&	2&	5&	7&	10&	12&	15&	17&	20&	22&	25&	27&	30&	32&	35&	37&	40&	42&	45&	47&	50\\
22-23&	3&	6&	9&	12&	15&	18&	21&	24&	27&	30&	33&	36&	39&	42&	45&	48&	51&	54&	57&	60\\
24-25&	3&	7&	10&	14&	17&	21&	24&	28&	31&	35&	38&	42&	45&	49&	52&	56&	59&	63&	66&	70\\
26-27&	4&	8&	12&	16&	20&	24&	28&	32&	36&	40&	44&	48&	52&	56&	60&	64&	68&	72&	76&	80\\
28-29&	4&	9&	13&	18&	22&	27&	31&	36&	40&	45&	49&	54&	58&	63&	67&	72&	76&	81&	85&	90\\
30-31&	5&	10&	15&	20&	25&	30&	35&	40&	45&	50&	55&	60&	65&	70&	75&	80&	85&	90&	95&	100\\
32-33&	5&	11&	16&	22&	27&	33&	38&	44&	49&	55&	60&	66&	71&	77&	82&	88&	93&	99&	104&	110\\
34-35&	6&	12&	18&	24&	30&	36&	42&	48&	54&	60&	66&	72&	78&	84&	90&	96&	102&	108&	114&	120\\
36-37&	6&	13&	19&	26&	32&	39&	45&	52&	58&	65&	71&	78&	84&	91&	97&	104&	110&	117&	123&	130\\
38-39&	7&	14&	21&	28&	35&	42&	49&	56&	63&	70&	77&	84&	91&	98&	105&	112&	119&	126&	133&	140\\
40-41&	7&	15&	22&	30&	37&	45&	52&	60&	67&	75&	82&	90&	97&	105&	112&	120&	127&	135&	142&	150\\
\hline
\end{tabular}}}
\end{table*}
\subsubsection{Daily Spell Point Acquisition:}
\label{sec:DailySpellPointAcquisition}
To regain used daily spell points, a spellcasting character must have a clear mind. 
To clear his mind, he \footnote{A number of lines in this documents tend to assume that the character is male and if not human, at least humanoid-shaped.
This is because the document's original author is a human male who really didn't give gender-neutral language enough thought from the beginning.} must first sleep for 8 hours.
The character does not have to slumber for every minute of the time, but he must refrain from movement, combat, casting spells, skill use, conversation, or any other demanding physical or mental task during the rest period. 
If his rest is interrupted, each interruption adds 1 hour to the total amount of time he has to rest to clear his mind, and he must have at least 1 hour of rest immediately prior to regaining lost spell points. 
If the character does not need to sleep for some reason, he still must have 8 hours of restful calm before regaining spell points.

\paragraph{Recent Casting Limit/Rest Interruptions:} If a spellcasting character has cast spells recently, the drain on his resources reduces his capacity to regain spell points. 
When he regains spell points for the coming day, all spell points he has used within the last 8 hours count against his daily limit.

\paragraph{Peaceful Environment:} To regain spell points, a spellcasting character must have enough peace, quiet, and comfort to allow for proper concentration. 
The spellcasting character's surroundings need not be luxurious, but they must be free from overt distractions, such as combat raging nearby or other loud noises. 
Exposure to inclement weather prevents the necessary concentration, as does any injury or failed saving throw the character might incur while concentrating on regaining spell points.

\paragraph{Regaining Spell Points:} Once the character has rested in a suitable environment, it takes an act of concentration spanning 1 full round to regain all power points of the spellcasting character's daily limit. 
This can be an instant's meditation, a prayer to the character's deity, or any other minor ritual the character performs at the start of each day.

\paragraph{Death and Spell Points:} If a character dies, all daily spell points stored in his mind are wiped away. 
A potent effect (such as \nameref{Spell:Wish}) can recover the lost spell points when it recovers the character.
\subsubsection[Magical Focus]{Gain Magical Focus}

\begin{figure*}
  \caption{Wizard, Cleric and Sorcerer attempt to regain their Magical Focus.}
  \centering
    \includegraphics{Pics/SpellPoints.png}
\end{figure*}

\label{sec:MagicFocus}
Merely holding a reservoir of magical spell points in mind gives spellcasting characters a special energy. 
Spellcasting characters can put that energy to work without actually paying a spell point cost - they can become magically focused as a special use of the \nameref{sec:Concentration} skill.

If you have 1 or more spell points available, you can meditate to attempt to become magically focused. 
The DC to become magically focused is 20. Meditating is a full-round action that provokes attacks of opportunity. 
When you are magically focused, you can expend your focus on any single Concentration check you make thereafter. 
When you expend your focus in this manner, your Concentration check is treated as if you rolled a 15. 
It's like taking 10, except that the number you add to your Concentration modifier is 15. 
You can also expend your focus to gain the benefit of a magical feat - many magical feats are activated in this way.

Once you are magically focused, you remain focused until you expend your focus, become unconscious, or go to sleep (or enter a meditative trance, in the case of elves), or until your spell point reserve drops to 0.
\subsubsection{Using Stored Spell Points}
\label{sec:UsingStoredSpellPoints}
A variety of magical items exist to store spell points for later use, in particular a storage device called a \nameref{Item:PearlOfPower}. 
Regardless of what sort of item stores the spell points, all spellcasting characters must follow strict rules when tapping stored spell points.

\paragraph{A Single Source:} When using spell points from a storage item to cast a spell, a spellcasting character may not pay the spell's cost with spell points from more than one source. 
He must either use an item, his own spell point reserve, or some other discrete spell point source to pay the casting cost. 

\paragraph{Recharging:} Most spell point storage devices allow spellcasting characters to ``recharge`` the item with their own spell points. 
Doing this depletes the character's spell point reserve on a 1-for-1 basis as if he had casted a spell; however, those spell points remain indefinitely stored. 
The opposite is not true - spellcasting characters may not use spell points stored in a storage item to replenish their own spell point reserves.
\subsection{Adding Spells}
Spellcasting characters can learn new spells when they attain certain levels, as indicated on their individual class tables. %A Wizard can learn any spell from the Wizard list, including spells only available to members of his school of specialization. A Cleric can learn any spell from a domain he knows. 

\subsubsection{Spells Gained at a New Level:} Spellcasting characters perform a certain amount of personal research, prayer or meditation between adventures in an attempt to unlock latent magical abilities.
Each time a spellcasting character attains a new level, he or she learns additional spells according to his class description. 
These spells represent abilities unlocked from latency. The spells must be of levels the characters can cast (see the class table for each class).

\subsubsection{Independent Research:} 
A spellcaster also can research a spell independently, duplicating an existing spell or creating an entirely new one. 
If characters are allowed to develop new spells, use these guidelines to handle the situation.
Any kind of caster can create a new spell. 
The research involved requires access to a retreat conducive to uninterrupted research, prayer, or meditation. 
Research involves an expenditure of 200 XP per week and takes one week per level of the spell. 
At the end of that time, the character makes a \nameref{sec:Spellcraft} check (DC 10 + spell level). 
If that check succeeds, the character learns the new spell if her research produced a viable spell. 
If the check fails, the character must go through the research process again if she wants to keep trying.

Spells learned through independent research still count against the spellcaster's number of spells known.
\subsubsection{Cast an Unknown Spell from a Scroll:}
A spellcasting character can attempt to cast a spell from a source other than his own knowledge (usually a scroll, although other means of storing magical knowledge may exist, such as a magical stone tablet). 
See \nameref{Item:Scrolls} for information on how to do so.

%To do so, the character must first decipher the scroll, as described under \nameref{Item:Scrolls}.

% Next, the spellcasting character must choose one of the spells available on the scroll and read it.
% As part of reading the spell, make a \nameref{sec:Spellcraft} check 
% (DC 15 + the spell's level) to see if the spell will be correctly cast. 
% If the spell is not on the caster's class list, he automatically fails this check.
% This check requires one full round, which provokes attacks of opportunity.
% 
% Upon successfully making the check, the character can immediately attempt to cast that spell even if he doesn't know it 
% (assuming he has spell points left for the day). 
% He can attempt to cast the spell normally on his next turn.
% He retains the ability to cast the selected spell for only 1 round. 
% If he doesn't cast the spell, fails the Spellcraft check, or casts a different spell, 
% he loses his chance to cast that spell unless the source is read again.
\subsection{Special Abilities}
Magical creatures can create magical effects without having levels in a spellcasting class (although they can take a spellcasting class to further enhance their abilities); such creatures have the magical subtype.
Characters using \nameref{Item:Wands} and other magical items can also create magical effects.

\subsubsection{Spell-like Abilities}
The casting of spells by creatures without a spellcasting class (creatures with the magical subtype, also simply called magical creatures) is considered a spell-like ability (Sp). 

Usually, a magical creature's spell-like ability works just like the spell of that name. 
A few spell-like abilities are unique; these are explained in the text where they are described.
Spell-like abilities have no verbal or somatic components, but do they require an XP cost if the equivalent spell has an XP cost. 
The user activates them mentally.

A spell-like ability has a casting time of 1 standard action unless noted otherwise in the ability description. 
In all other ways, a spell-like ability functions just like a spell, notably including that using a spell-like ability provokes attacks of opportunity and is subject to interruption, and the save DCs of spell-like abilities are calculated as normal (if no key ability modifier for calculating the save DC is given, default to Charisma). 
However, a magical creature does not have to pay a spell-like ability's spell point cost.

Spell-like abilities are subject to spell resistance and to being dispelled by dispel magic. 
They do not function in areas where magic is suppressed or negated.

All creatures with spell-like abilities are assigned a caster level, which which indicates how difficult it is to dispel their spell-like effects and determines all level-dependent variables (such as range or duration) the abilities might have (like spellcasters, creatures with spell-like abilities may voluntarily lower their caster level). 
When a creature uses a spell-like ability, the spell is cast as if the creature had spent a number of spell points equal to its caster level.
If the spell has augments, it may choose those if its caster level is sufficient.
\subsubsection{Supernatural Abilities}
Some creatures have magical abilities that are considered supernatural (Su). 
Magical feats are also supernatural abilities. 

These abilities cannot be disrupted in combat, as spells can be, and do not provoke attacks of opportunity (except as noted in their descriptions). 

Supernatural abilities are not subject to spell resistance and cannot be negated or dispelled; however, they do not function in areas where magic is suppressed.
\subsection{Magical Maladies}
\subsubsection{Ability Burn}
This is a special form of ability damage that cannot be magically healed. 
It is caused by the use of certain magical feats and spells. It returns only through natural healing.

\subsubsection{Disease, Cascade Flu}
Spread by brain moles and other vermin; injury; DC 13; incubation one day; damage magical cascade.

A magical cascade is a loss of control over magical abilities. Using spell points becomes dangerous for a character infected by cascade flu, once the incubation period has run its course. 
Every time an afflicted character casts a spell, she must make a DC 16 Concentration check. 
On a failed check, a magical cascade is triggered. 
The spell operates normally, but during the following round, without the character's volition, two additional spells she knows are cast randomly, and their spell cost is deducted from the character's reserve. 
During the next round, three additional spells are cast, and so on, until all the magical character's spell points are drained. 
Spells with a range of personal or touch always affect the diseased character. 
For other spells that affect targets, roll d\%: On a 01-50 result, the spell affects the diseased character, and 51-00 indicates that the spell targets other creatures in the vicinity. 
Magical creatures (those that cast their spells without paying points) cascade until all the spells they know have been cast at least twice.
As with any disease, a spellcasting character who is injured or attacked by a creature carrying a disease or parasite, or who otherwise has contact with contaminated material, must make an immediate Fortitude save. 
On a success, the disease fails to gain a foothold. 
On a failure, the character takes damage (or incurs the specified effect) after the incubation period. 
Once per day afterward, the afflicted character must make a successful Fortitude save to avoid repeating the damage. 
Two successful saving throws in a row indicate she has fought off the disease.
\subsubsection{Disease, Cerebral Parasites}
Spread by contact with infected magical or spellcasting creatures; contact; DC 15; incubation 1d4 days; damage 1d8 spell points. 

Cerebral parasites are tiny organisms, undetectable to normal sight. 
An afflicted character may not even know he carries the parasites - until he discovers he has fewer spell points for the day than expected. 
Magical creatures with cerebral parasites are limited to using each of their known spells only once per day (instead of freely casting them). 
See the note about diseases under Cascade Flu, above.

\subsubsection{Negative Levels}
\label{sec:NegativeLevels}
Spellcasting characters can gain negative levels just like members of other character classes. 
A spellcasting character loses access to one spell per negative level from the highest level of spell he can cast; 
he also loses a number of spell points equal to the cost of that spell. 
If two or more spells fit these criteria, the caster decides which one becomes inaccessible. 
The lost spell becomes available again as soon the negative level is removed, providing the caster is capable of using it at that time. 
Lost spell points also return.
\subsection{Spell Descriptions}
The description of each spell is presented in a standard format. Each category of information is explained and defined below.
\subsubsection{Name}
The first line of every spell description gives the name by which the spell is generally known. 
A spell might be known by other names in some locales, and specific casters might have names of their own for their spells.

\subsubsection{School (Subschool)}
\label{sec:MagicalSchools}
Beneath the spell name is a line giving the school of magic (and the subschool, if appropriate) that the spell belongs to.
Every spell belongs to one of eight schools of magic. A school of magic is a group of related spells that work in similar ways.
\paragraph{Abjuration}
Abjurations are protective spells. They create physical or magical barriers, negate magical or physical abilities, harm trespassers, or even banish the subject of the spell to another plane of existence.
If an abjuration creates a barrier that keeps certain types of creatures at bay, that barrier cannot be used to push away those creatures. 
If you force the barrier against such a creature, you feel a discernible pressure against the barrier. 
If you continue to apply pressure, you end the spell.
\paragraph{Conjuration}
Each conjuration spell belongs to one of four subschools. 
Conjurations bring manifestations of objects, creatures, or some form of energy to you (the summoning subschool), actually transport creatures from another plane of existence to your plane (calling), transport creatures or objects over great distances (teleportation), or create objects or effects on the spot (creation). 
Creatures you conjure usually, but not always, obey your commands.
A creature or object brought into being or transported to your location by a conjuration spell cannot appear inside another creature or object, nor can it appear floating in an empty space. It must arrive in an open location on a surface capable of supporting it.
The creature or object must appear within the spell's range, but it does not have to remain within the range.

\subparagraph{Calling:}
A calling spell transports a creature from another plane to the plane you are on. 
% The spell grants the creature the one-time ability to return to its plane of origin, 
% although the spell may limit the circumstances under which this is possible. 
Unless otherwise noted, the spell does not grant the creature the ability to return to its plane of origin,a second spell has to be cast in order to send it home.
Creatures who are called actually die when they are killed.
The duration of a calling spell is instantaneous, which means that the called creature can't be dispelled.
A called creature cannot use any summoning or calling abilities it may have, or any spell or other ability with an XP cost.

\subparagraph{Creation:}
A creation spell manipulates matter to create an object or creature in the place the spellcaster designates (subject to the limits noted above). 
If the spell has a duration other than instantaneous, magic holds the creation together, and when the spell ends, the conjured creature or object vanishes without a trace. 
If the spell has an instantaneous duration, the created object or creature is merely assembled through magic. 
It lasts indefinitely and does not depend on magic for its existence.

\subparagraph{Summoning:}
A summoning spell instantly conjures a creature or object in a place you designate. 
When the spell ends or is dispelled, a summoned creature disappears, but a summoned object is not sent back unless the spell description specifically indicates this. 
A summoned creature also disappears if it is killed or if its hit points drop to 0 or lower.
When the spell that summoned a creature ends and the creature disappears, all the spells it has cast expire. 
A summoned creature cannot use any innate summoning or calling abilities it may have.
A summoned creature always refuses to use any spell or other ability with an XP cost.

\subparagraph{Teleportation:}
A teleportation spell transports one or more creatures or objects a great distance. 
The most powerful of these spells can cross planar boundaries. 
The transportation is (unless otherwise noted) one-way and not dispellable.
Teleportation is instantaneous travel through the Astral Plane. Anything that blocks astral travel also blocks teleportation.
\paragraph{Divination}
Divination spells enable you to learn secrets long forgotten, to predict the future, to find hidden things, and to foil deceptive spells.
Many divination spells have cone-shaped areas. These move with you and extend in the direction you look. 
The cone defines the area that you can sweep each round. 
If you study the same area for multiple rounds, you can often gain additional information, as noted in the descriptive text for the spell.

\subparagraph{Scrying:}

A scrying spell creates an invisible magical sensor that sends you information. 
Unless noted otherwise, the sensor has the same powers of sensory acuity that you possess. 
This level of acuity includes any spells or effects that target you, but not spells or effects that emanate from you. 
However, the sensor is treated as a separate, independent sensory organ of yours, and thus it functions normally even if you have been blinded, deafened, or otherwise suffered sensory impairment.
Any creature with an Intelligence score of 12 or higher can notice the sensor by making a DC 20 Intelligence check. 
The sensor can be dispelled as if it were an active spell.
Lead sheeting or magical protection blocks a scrying spell, and you sense that the spell is so blocked.
\paragraph{Enchantment}
Enchantment spells affect the minds of others, influencing or controlling their behavior.
All enchantments are mind-affecting spells. Two types of enchantment spells grant you influence over a subject creature.

\subparagraph{Charm:}
A charm spell changes how the subject views you, typically making it see you as a good friend.

\subparagraph{Compulsion:}
A compulsion spell forces the subject to act in some manner or changes the way her mind works. 
Some compulsion spells determine the subject's actions or the effects on the subject, some compulsion spells allow you to determine the subject's actions when you cast the spell, and others give you ongoing control over the subject.
\paragraph{Evocation}
Evocation spells manipulate energy or tap an unseen source of power to produce a desired end. 
In effect, they create something out of nothing. 
Many of these spells produce spectacular effects, and evocation spells can deal large amounts of damage.
\paragraph{Illusion}
Illusion spells deceive the senses or minds of others. 
They cause people to see things that are not there, not see things that are there, hear phantom noises, or remember things that never happened.

\subparagraph{Figment:}
A figment spell creates a false sensation. Those who perceive the figment perceive the same thing, not their own slightly different versions of the figment. (It is not a personalized mental impression.) 
Figments cannot make something seem to be something else. 
A figment that includes audible effects cannot duplicate intelligible speech unless the spell description specifically says it can. 
If intelligible speech is possible, it must be in a language you can speak. 
If you try to duplicate a language you cannot speak, the image produces gibberish. 
Likewise, you cannot make a visual copy of something unless you know what it looks like.
Because figments and glamers (see below) are unreal, they cannot produce real effects the way that other types of illusions can. 
They cannot cause damage to objects or creatures, support weight, provide nutrition, or provide protection from the elements. 
Consequently, these spells are useful for confounding or delaying foes, but useless for attacking them directly.
A figment's AC is equal to 10 + its size modifier.

\subparagraph{Glamer:}
A glamer spell changes a subject's sensory qualities, making it look, feel, taste, smell, or sound like something else, or even seem to disappear.

\subparagraph{Pattern:}
Like a figment, a pattern spell creates an image that others can see, but a pattern also affects the minds of those who see it or are caught in it. 
All patterns are mind-affecting spells.

\subparagraph{Phantasm:}
A phantasm spell creates a mental image that usually only the caster and the subject (or subjects) of the spell can perceive. 
This impression is totally in the minds of the subjects. It is a personalized mental impression. (It's all in their heads and not a fake picture or something that they actually see.) 
Third parties viewing or studying the scene don't notice the phantasm. All phantasms are mind-affecting spells.

\subparagraph{Shadow:}
A shadow spell creates something that is partially real from extradimensional energy. 
Such illusions can have real effects. Damage dealt by a shadow illusion is real.

\subparagraph{Saving Throws and Illusions (Disbelief):}
Creatures encountering an illusion usually do not receive saving throws to recognize 
it as illusory until they study it carefully or interact with it in some fashion.
A successful saving throw against an illusion reveals it to be false, 
but a figment or phantasm remains as a translucent outline.
A failed saving throw indicates that a character fails to notice something is amiss. 
A character faced with proof that an illusion isn't real needs no saving throw. 
If any viewer successfully disbelieves an illusion and communicates this fact to others, 
each such viewer gains a saving throw with a +4 bonus.

\paragraph{Necromancy}
Necromancy spells manipulate the power of death, unlife, and the life force. 
Spells involving undead creatures make up a large part of this school.

\subparagraph{Healing:}
Certain necromancy spells heal creatures or even bring them back to life.
\paragraph{Transmutation}
Transmutation spells change the properties of some creature, thing, or condition.

% \subparagraph{Polymorph:}  Pre-1.10 beta version
% Some Transmutation spells change the subject's form into that of another creature entirely.
% When under a Polymorph subschool spell, the subject loses some class and most racial features.
% Of your class features, you retain all but your ability to cast spells, use spell-like or supernatural abilities that require activation, and your ability to manifest psionic powers. 
% Your hit point total never changes as a result of a Polymorph subschool spell, even if your new form has a Constitution score different from your own.
% Of your racial features, you retain your bonus feats, bonus skill points, skill bonuses, your racial bonus feats, and racial weapon proficiencies. All other racial features are lost.
% You retain your own type and subtypes, and all your feats.
% Unless otherwise noted, your ability scores and natural armor bonus are unchanged from that of your natural form. You retain your ability to speak unless your new form has no organs capable of supporting speech.
% Magic items and articles of clothing not feasibly capable of being worn, held or carried by your new form meld into your body, continuing to provide their benefits.
% A creature can never be the subject of more than one Polymorph spell simultaneously. If multiple Polymorph spells are cast on a creature in succession, the older spells are suppressed while the newest is in effect.
% Recognizing that a creature is under a Polymorph spell (rather than being a normal, average member of the creature type the subject morphed into) is generally a DC 20 spot check, or DC
% 15 for members of the creature type that the subject morphed into.

\subparagraph{Polymorph:}
Some Transmutation spells change the subject's form into that of another creature entirely.
When under a Polymorph subschool spell, the subject loses most of its racial features.
Of its racial features, the subject retains its bonus feats, bonus skill points, skill bonuses, and racial weapon proficiencies. All other racial features are lost.
It retains its own type and subtypes.
Unless otherwise noted, its ability scores and natural armor bonus are unchanged from that of its natural form. It retains its ability to speak unless the new form has no organs capable of supporting speech (assume that animalistic mouths are sufficient for providing speech).
Magic items and articles of clothing not feasibly capable of being worn, held or carried by the new form meld into the subject's body, continuing to provide their benefits if applicable.
A creature can never be the subject of more than one Polymorph spell simultaneously. If multiple Polymorph spells are cast on a creature in succession, the older spells are suppressed while the newest is in effect.
Recognizing that a creature is under a Polymorph spell (rather than being a normal, average member of the creature type the subject morphed into) is generally a DC 20 spot check, or DC
15 for members of the creature type that the subject morphed into.
Your hit point total never changes as a result of a Polymorph subschool spell, even if your new form has a Constitution score different from your own.
\subsubsection{Descriptor}
Appearing on the same line as the school and subschool (when applicable) is a descriptor that further categorizes the spell in some way. 
Some spells have more than one descriptor, some have none. Descriptors are shown in brackets.

The descriptors that apply to spells are 
\emph{acid, air, chaotic, cold, darkness, death, earth, electricity, evil, fear, fire, force, good, language-dependent, lawful, light, mind-affecting, minion, sonic,} and \emph{water.} 

Most of these descriptors have no game effect by themselves, but they govern how the spell interacts with other spells, with special abilities, with unusual creatures, with alignment, and so on.

\paragraph{Language-dependent spells} use intelligible language as a medium.

\paragraph{Mind-affecting spells} work only against creatures with an Intelligence score of 1 or higher.

\paragraph[Minion]{Minion Spells} 
\label{sec:MinionSpells}
are spells that place minions of one kind or another under your control for an extended period of 
time (often permanently).
Regardless of the number of different spells that give you minions, you can control only (2 + your charisma modifier) HD worth of creatures per character level (minimum 1 HD worth of creatures per level, if your charisma is 8 or lower). 
If you exceed this number, you must immediately release enough creatures from your control to bring you beneath the limit again.
In the case of active spells that give you ongoing control, the spells immediately expire.
In the case of creatures you have created being under your control, the creatures immediately become uncontrolled.
\subsubsection{Level}
The next line of the spell description gives a spell's level, a number between 1 and 9 that defines the spell's relative strength. 
This number is preceded by the name of the class whose members can cast the spell.
\subsubsection{Components}
\label{sec:Components}
When a spell is cast, a component may be needed to facilitate the spell. This component may be somatic or verbal.

\paragraph{Verbal components (V)} A verbal component is a spoken incantation. 
To provide a verbal component, you must be able to speak in a strong voice. 
A \nameref{Spell:Silence} spell or a gag spoils the incantation (and thus the spell). 
A spellcaster who has been deafened has a 20\% chance to spoil any spell he tries to cast with a verbal component.

\paragraph{Somatic components (S)} A somatic component is a measured and precise movement of the hand. 
You must have at least one hand free to provide a somatic component.

\paragraph{Dispense with Components:} Despite the fact that almost every spell has a component, a spellcasting character can always choose to attempt to cast the spell without the flashy accompaniment of magical words and hand gestures, usually to avoid attention or to circumvent a condition that prevents him from using components (see above). 
To cast a spell without any components (no matter how many components it might have), a caster must make a Concentration check (DC 15 + the level of the spell).
This check is part of the action of casting the spell. If the check is unsuccessful, the components are needed if the spell is to go off.
Even if a caster casts a spell without a component, he is still subject to attacks of opportunity in appropriate circumstances. 
(Of course, another Concentration check can be made as normal to either cast defensively or maintain the spell if attacked.)

\subsubsection{Casting Time}
Most spells have a casting time of 1 standard action. Others take 1 round or more, while a few require only a free action.
A spell that takes 1 round to cast requires a full-round action. 
It comes into effect just before the beginning of your turn in the round after you began casting the spell. 
You then act normally after the spell is completed. 
A spell that takes 1 minute to cast comes into effect just before your turn 1 minute later (and for each of those 10 rounds, you are casting a spell as a full-round action, as noted above for 1-round casting times). 
These actions must be consecutive and uninterrupted, or the spell points are lost and the spell fails.
When you use a spell that takes 1 round or longer to cast, you must continue the concentration from the current round to just before your turn in the next round (at least). 
If you lose concentration before the casting time is complete, the spell points are lost and the spell fails.
You make all pertinent decisions about a spell (range, target, area, effect, version, and so forth) when the spell comes into effect.

\subsubsection{Range}
A spell's range indicates how far from you it can reach, as defined in the Range entry of the spell description. 
A spell's range is the maximum distance from you that the spell's effect can occur, as well as the maximum distance at which you can designate the spell's point of origin. 
If any portion of the area would extend beyond the range, that area is wasted. Standard ranges include the following:

\paragraph{Personal:} The spell affects only you.

\paragraph{Touch:} You must touch a creature or object to affect it. A touch spell that deals damage can score a critical hit just as a weapon can. 
A touch spell threatens a critical hit on a natural roll of 20 and deals double damage on a successful critical hit. 
Some touch spells allow you to touch multiple targets. 
You can touch as many willing targets as you can reach, but all targets of the spell must be touched in the same round that you cast the spell.

\paragraph{Close:} The spell reaches as far as 25 feet away from you. The maximum range increases 5 feet for every two caster levels you have.

\paragraph{Medium:} The spell reaches as far as 100 feet + 10 feet per caster level.

\paragraph{Long:} The spell reaches as far as 400 feet + 40 feet per caster level.

\paragraph{Range Expressed in Feet:} Some spells have no standard range category, just a range expressed in feet.

\subsubsection{Aiming a Spell}
You must make some choice about whom the spell is to affect or where the spell's effect is to originate, 
depending on the type of spell. The next entry in a spell description defines the spell's target (or targets), its effect, or its area, as appropriate.

\paragraph{Target or Targets:} Some spells have a target or targets. You cast these spells on creatures or objects, as defined by the spell itself. 
You must be able to see or touch the target, and you must specifically choose that target. 
However, you do not have to select your target until you finish casting the spell.
If you cast a targeted spell on the wrong type of target the spell has no effect. 
If the target of a spell is yourself (the spell description has a line that reads ''Target: You``), 
you do not receive a saving throw and spell resistance does not apply. 
The Saving Throw and Spell Resistance lines are omitted from such spells.
Some spells can be cast only on willing targets. 
Declaring yourself as a willing target is something that can be done at any time (even if you're flat-footed or it isn't your turn). 
Unconscious creatures are automatically considered willing, but a character who is conscious but immobile or helpless (such as one who is bound, cowering, grappling, paralyzed, pinned, or stunned) is not automatically willing. 
The Saving Throw and spell Resistance lines are usually omitted from such spells, since only willing subjects can be targeted.

\paragraph{Effect:} Some spells, such as most conjuration spells, create things rather than affect things that are already present. 
Unless otherwise noted in the spell description, you must designate the location where these things are to appear, either by seeing it or defining it. Range determines how far away an effect can appear, but if the effect is mobile, it can move regardless of the spell's range once created.

\paragraph{Ray:} Some effects are rays. You aim a ray as if using a ranged weapon, though typically you make a ranged touch attack rather than a normal ranged attack. 
As with a ranged weapon, you can fire into the dark or at an invisible creature and hope you hit something. 
You don't have to see the creature you're trying to hit, as you do with a targeted spell. 
Intervening creatures and obstacles, however, can block your line of sight or provide cover for the creature you're aiming at.
If a ray spell has a duration, it's the duration of the effect that the ray causes, not the length of time the ray itself persists.
If a ray spell deals damage, you can score a critical hit just as if it were a weapon. 
A ray spell threatens a critical hit on a natural roll of 20 and deals double damage on a successful critical hit.

\paragraph{Spread:} Some effects spread out from a point of origin (which may be a grid intersection, or may be the caster) to a distance described in the spell. The effect can extend around corners and into areas that you can't see. 
Figure distance by actual distance traveled, taking into account turns the effect may take. 
When determining distance for spread effects, count around walls, not through them. 
As with movement, do not trace diagonals across corners. 
You must designate the point of origin for such an effect (unless the effect is centered on you), but you need not have line of effect (see below) to all portions of the effect.

\paragraph{(S) Shapeable:} If an Effect line ends with ''(S)`` you can shape the spell. 
A shaped effect can have no dimension smaller than 10 feet.

\paragraph{Area:} Some spells affect an area. Sometimes a spell description specifies a specially defined area, but usually an area falls into one of the categories defined below.
Regardless of the shape of the area, you select the point where the spell originates, but otherwise you usually don't control which creatures or objects the spell affects. 
The point of origin of a spell that affects an area is always a grid intersection. 
When determining whether a given creature is within the area of a spell, count out the distance from the point of origin in squares just as you do when moving a character or when determining the range for a ranged attack. 
The only difference is that instead of counting from the center of one square to the center of the next, you count from intersection to intersection.
You can count diagonally across a square, but every second diagonal counts as 2 squares of distance. 
If the far edge of a square is within the spell's area, anything within that square is within the spell's area. 
If the spell's area touches only the near edge of a square, however, anything within that square is unaffected by the spell.

\subparagraph{Burst, Emanation, or Spread:} \label{sec:AreaShapes} Most spells that affect an area function as a burst, an emanation, or a spread. 
In each case, you select the spell's point of origin and measure its effect from that point. 

A burst spell affects whatever it catches in its area, even including creatures that you can't see. 
It can't affect creatures with total cover from its point of origin (in other words, its effects don't extend around corners). 
The default shape for a burst effect is a sphere, but some burst spells are specifically described as cone-shaped.

A burst's area defines how far from the point of origin the spell's effect extends.

An emanation spell functions like a burst spell, except that the effect continues to radiate from the point of origin for the duration of the spell.

A spread spell spreads out like a burst but can turn corners. You select the point of origin, and the spell spreads out a given distance in all directions. 
Figure the area the spell effect fills by taking into account any turns the effect takes.

\subparagraph{Cone, Line, or Sphere:} Most spells that affect an area have a particular shape, such as a cone, line, or sphere. 

A cone-shaped spell shoots away from you in a quarter-circle in the direction you designate. 
It starts from any corner of your square and widens out as it goes. 
Most cones are either bursts or emanations (see above), and thus won't go around corners.

A line-shaped spell shoots away from you in a line in the direction you designate. 
It starts from any corner of your square and extends to the limit of its range or until it strikes a barrier that blocks line of effect. 
A line-shaped spell affects all creatures in squares that the line passes through or touches.

A sphere-shaped spell expands from its point of origin to fill a spherical area. Spheres may be bursts, emanations, or spreads.

\subparagraph{Other:} A spell can have a unique area, as defined in its description.

\paragraph{Line of Effect:} A line of effect is a straight, unblocked path that indicates what a spell can affect. 
A solid barrier cancels a line of effect, but it is not blocked by fog, darkness, and other factors that limit normal sight. 
You must have a clear line of effect to any target that you cast a spell on or to any space in which you wish to create an effect. 
You must have a clear line of effect to the point of origin of any spell you cast.
A burst, cone, or emanation spell affects only an area, creatures, or objects to which it has line of effect from its origin (a spherical burst's center point, a cone-shaped burst's starting point, or an emanation's point of origin). 
An otherwise solid barrier with a hole of at least 1 square foot through it does not block a spell's line of effect. 
Such an opening means that the 5-foot length of wall containing the hole is no longer considered a barrier for the purpose of determining a spell's line of effect.

\subsubsection{Duration}
A spell's Duration line tells you how long the magical energy of the spell lasts.

\paragraph{Timed Durations:} Many durations are measured in rounds, minutes, hours, or some other increment. When the time is up, the magical energy sustaining the effect fades, and the spell ends. If a spell's duration is variable it is rolled secretly.

\paragraph{Instantaneous:} The magical energy comes and goes the instant the spell is cast, though the consequences might be long-lasting or permanent.

\paragraph{Permanent:} The energy remains as long as the effect does. This means the spell is vulnerable to dispel magic.

\paragraph{Concentration:} The spell lasts as long as you concentrate on it. Concentrating to maintain a spell is a standard action that does not provoke attacks of opportunity. Anything that could break your concentration when casting a spell can also break your concentration while you're maintaining one, causing the spell to end. You can't cast a spell while concentrating on another one. Some spells may last for a short time after you cease concentrating. In such a case, the spell keeps going for the given length of time after you stop concentrating, but no longer. Otherwise, you must concentrate to maintain the spell, but you can't maintain it for more than a stated duration in any event. If a target moves out of range, the spell reacts as if your concentration had been broken.

\paragraph{Subjects, Effects, and Areas:} If the spell affects creatures directly the result travels with the subjects for the spell's duration. If the spell creates an effect, the effect lasts for the duration. The effect might move or remain still. Such an effect can be destroyed prior to when its duration ends. If the spell affects an area then the spell stays with that area for its duration. Creatures become subject to the spell when they enter the area and are no longer subject to it when they leave.

\paragraph{Touch Spells and Holding the Charge:} In most cases, if you don't discharge a touch spell on the round you cast it, you can hold the charge (postpone the discharge of the spell) indefinitely. You can make touch attacks round after round. If you touch anything with your hand while holding a charge, the spell discharges. If you cast another spell, the touch spell dissipates.
Some touch spells allow you to touch multiple targets as part of the spell. You can't hold the charge of such a spell; you must touch all the targets of the spell in the same round that you finish casting the spell. You can touch one friend (or yourself) as a standard action or as many as six friends as a full round action.

\paragraph{Discharge:} Occasionally a spell lasts for a set duration or until triggered or discharged.

\paragraph{(D) Dismissible:} If the Duration line ends with ''(D),`` you can dismiss the spell at will. You must be within range of the spell's effect and must mentally will the dismissal, which uses the same components as when you first cast the spell. Dismissing a spell is a standard action that does not provoke attacks of opportunity. A spell that depends on concentration is dismissible by its very nature, and dismissing it does not take an action or require a component, since all you have to do to end the spell is to stop concentrating on your turn.

\subsubsection{Saving Throw}
\label{sec:SavingThrow}
Usually a harmful spell allows a target to make a saving throw to avoid some or all of the effect. 
The Saving Throw line in a spell description defines which type of saving throw the spell allows and describes how saving throws against the spell work.

\paragraph{Negates:} The spell has no effect on a subject that makes a successful saving throw.

\paragraph{Partial:} The spell causes an effect on its subject, such as death. A successful saving throw means that some lesser effect occurs (such as being dealt damage rather than being killed).

\paragraph{Half:} The spell deals damage, and a successful saving throw halves the damage taken (round down). 

\paragraph{None:} No saving throw is allowed.

\paragraph{(object):} The spell can be cast on objects, which receive saving throws only if they are magical or if they are attended (held, worn, grasped, or the like) by a creature resisting the spell, in which case the object uses the creature's saving throw bonus unless its own bonus is greater. (This notation does not mean that a spell can be cast only on objects. Some spells of this sort can be cast on creatures or objects.) A magic item's saving throw bonuses are each equal to 2 + one-half the item's caster level.

\paragraph{(harmless):} The spell is usually beneficial, not harmful, but a targeted creature can attempt a saving throw if it desires.

\paragraph{Saving Throw Difficulty Class:} 

A saving throw against your spell has a DC of
\begin{quote}
\centering
\large 
 {10 + one-half the number of spell points spent on the spell (round up) + your key ability modifier.}
%\normalsize
\end{quote}
Count all spell points spent on augmenting a spell in order to determine its spell point cost for this purpose, 
but do not count the additional spell point cost incurred by adding a metamagic feat to a spell.\footnote{This is a new, and most fundamental rule.}

\paragraph{Succeeding on a Saving Throw:} A creature that successfully saves against a spell that has no obvious physical effects feels a hostile force or a tingle, but cannot deduce the exact nature of the attack unless it succeeds on the appropriate \nameref{sec:Spellcraft} check. 
Likewise, if a creature's saving throw succeeds against a targeted spell you sense that the spell has failed. 
You do not sense when creatures succeed on saves against effect and area spells.

\paragraph{Failing a Saving Throw against Mind-Affecting Spells:} If you fail your save, you are unaware that you have been affected by a spell.

\paragraph{Automatic Failures and Successes:} A natural 1 (the d20 comes up 1) on a saving throw is always a failure, and the spell may deal damage to exposed items (see Items Surviving after a Saving Throw, below). A natural 20 (the d20 comes up 20) is always a success.

\paragraph{Voluntarily Giving up a Saving Throw:} A creature can voluntarily forego a saving throw and willingly accept a spell's result. Even a character with a special resistance to magic can suppress this quality.
A creature  can under no conditions whatsoever be directly forced to give up its saving throw, even with Enchantment spells or the control granted over a Called creature.

\paragraph{Items Surviving after a Saving Throw:} Unless the descriptive text for the spell specifies otherwise, all items carried or worn by a creature are assumed to survive a magical attack.

\subsubsection{Spell Resistance}
Spell resistance is a special defensive ability. If your spell is being resisted by a creature with spell resistance, you must make a caster level check (d20 + caster level) at least equal to the creature's spell resistance for the spell to affect that creature. The defender's spell resistance functions like an Armor Class against magical attacks.\footnote{Power resistance is equivalent to spell resistance unless the Psionics Is Different option is in use.} Include any adjustments to your caster level on this caster level check.
The Spell Resistance line and the descriptive text of a spell description tell you whether spell resistance protects creatures from the spell. In many cases, spell resistance applies only when a resistant creature is targeted by the spell, not when a resistant creature encounters a spell that is already in place.
The terms “object” and “harmless” mean the same thing for spell resistance as they do for saving throws. A creature with spell resistance must voluntarily lower the resistance (a standard action) to be affected by a spell noted as harm less. In such a case, you do not need to make the caster level check described above.

\subsubsection{Spell Points}
All spells have a Spell Points line, indicating the spell's cost. This is the minimum number of spell points that must be paid in order to cast the spell.
The spellcasting character class tables show how many spell points a character has access to each day, depending on level.
A spell's cost is determined by its spell level, as shown on the Spell Points by Spell Level table. %table \ref{tab:SpellPointsBySpellLevel}. 
Every spell's cost is noted in its description for ease of reference.
\begin{center}
\resizebox{7cm}{!}{
\begin{tabular}{|l|*{9}{c|}}
\multicolumn{10}{c}{\textbf{Spell Points by Spell Level}}\\
\hline
\textbf{Level}&1&2&3&4&5&6&7&8&9\\
\hline
\textbf{Cost}&1&3&5&7&9&11&13&15&17\\
\hline
\end{tabular}}
\end{center}

\paragraph{Spell Point Limit:} 
The spell point cost mentioned in each spell's description is the minimum number of spell points needed to cast the spell. 
You can, if you wish, spend more than this minimum number on a spell, usually to increase the spell's saving throwDC, or to use an augment the spell may have.
The maximum number of points you can spend on a spell (for any reason) is equal to your caster level (the fundamental rule of magic).

\paragraph{XP Cost (XP):} On the same line that the spell point cost of a spell is indicated, the spell's experience point cost, if any, is noted. Particularly powerful effects entail an experience point cost to you. No spell or power can restore XP lost in this manner. You cannot spend so much XP that you lose a level, so you cannot cast a spell with an XP cost unless you have enough XP to spare. However, you can, on gaining enough XP to attain a new level, use those XP for casting a spell rather than keeping them and advancing a level. The XP are expended when you cast the spell, whether or not the casting succeeds.

\subsubsection{Flavor Text}
Most spells have a sentence or two of ''flavor text`` - text that has no immutable meaning, but provides an example of how characters might understand or perceive the spell, or an explanation of how the spell might fit into a campaign.
If this flavor text does not fit the way you imagine the spell, this flavor text should be discarded and replaced with a description of your own making.

\subsubsection{Descriptive Text}
This portion of a spell description details what the spell does and how it works. If one of the previous lines in the description included ''see text,`` this is where the explanation is found. If the spell you're reading about is based on another spell you might have to refer to a different spell for the “see text” information. If a spell is the equivalent of a spell an entry of “see spell text” directs you to the appropriate spell description.

\paragraph[Augment]{Augment:} 
\label{sec:Augment}
Many spells have variable effects based on the number of spell points you spend when you cast them. The more points spent, the more powerful the spell. How this extra expenditure affects a spell is specific to the spell. Some augmentations allow you to increase the number of damage dice, while others extend a spell's duration or modify a spell in unique ways. Each spell that can be augmented includes an entry giving how many spell points it costs to augment and the effects of doing so. However, you can spend only a total number of points on a spell equal to your caster level.
Augmenting a spell takes place as part of another action (casting a spell). Unless otherwise noted in the Augment section of an individual spell description, you can augment a spell only at the time you cast it. Some Augments radically alter the spell's characteristics.
\subsection{Epic Magic}
Epic magic is very cool and will definitely be implemented one day. \newpage

\part{Races and Classes}
As the base system is changed, races and classes must be adapted as well. The following three chapters deal with updating player races that deal with magic, the base classes, as well as the prestige classes presented in the \href{http://www.wizards.com/default.asp?x=d20/article/srd35}{d20 srd} to consistently work with the updated underlying system.
\subsection{Spiritwalker Orc}
\subsubsection{Description}
Spiritwalker orcs are physically similar to their more common brethren, although they tend to be slightly less thick of chest. Most favor unobtrusive, earth-colored clothing when not dressed for war. % Those who brave interacting with them will find that they possess, if not keen minds, at least surprisingly honed senses and trust in intuition.

Like other orcs, spiritwalker orcs speak Orc. If given the opportunity, they are likely to learn Common, Sylvan, or Terran.
\subsubsection{Magic}
Spiritwalker orcs have a great affinity for many forms of divine magic thanks to their high Wisdom. In addition, each spiritwalker orc has a natural ability to communicate with a certain type of animal. According to the spiritwalker orcs themselves, this is due to them not finding it very hard to talk to their ancestors, even if said ancestors have been reborn in beastial form.
\subsubsection{Combat}
Spiritwalker orcs are a proud and stubborn lot - if insulted or mistreated, they usually attempt to regain face by humiliating the offender in a duel or contest of strength. 

If forced to full-out battle, they prefer overwhelming assaults, seeking to end the fight quickly. They rarely retreat or flee once committed to a fight.
\subsubsection{Racial Traits}
Spiritwalker orcs possess all orc racial traits, with additions and exceptions as noted below.
\begin{itemize}
 \item +2 Strength, -2 Intelligence, +2 Wisdom, -2 Charisma: Spiritwalker orcs are less brutish than some others of their kind, and more in tune with their surroundings.
 \item No Light Sensitivity: Spiritwalker Orcs have adapted to life outside underground dungeons.
 \item Spell-Like Ability: 1/day - \nameref{Spell:ConverseWithNature}. Caster level 1st (regardless of character level). This spell-like ability only allows the spiritwalker orc to communicate with one certain kind of animal (selected at character creation), usually an animal that serves as the totem animal of the orc's tribe or family.
\end{itemize} \newpage
\section{Spellcasting Classes}
\subsection[Bard]{The Bard}
\label{sec:Bard}
\begin{quote}
\emph{``I don't mean to brag, but when I sing, the angels themselves come down from their heavens to admire the brilliance.''}
- Cassius, human Bard
\end{quote}
A Bard is an arcane spellcaster whose magic manifests in the form of music or other creative or inspiring ways.

\begin{table*}
\centering
\caption{The Bard}
\label{tab:Bard}
\makebox[\textwidth]{
\begin{tabular}{llccclccc}
\toprule
	&	&	&	&	&		&\multicolumn{3}{c}{Spellcasting}\\ \cmidrule(r){7-9}
Level	&BAB	&Fort 	&Ref 	&Will 	&Special	&SP/day	&Known&Max level\\
\midrule
1st	&+0		&+0	&+2	&+2	&Bardic Performance, Bardic		&0	&2	&1st\\
	&		&	&	&	&Knowledge, Cantrips			&	&	&\\
2nd	&+1		&+0	&+3	&+3	&-					&2	&3	&1st\\
3rd	&+2		&+1	&+3	&+3	&Bardic Performance			&4	&4	&1st\\
4th	&+3		&+1	&+4	&+4	&-					&8	&5	&2nd\\
5th	&+3		&+1	&+4	&+4	&-					&12	&6	&2nd\\
6th	&+4		&+2	&+5	&+5	&Bardic Performance			&18	&7	&2nd\\
7th	&+5		&+2	&+5	&+5	&-					&24	&8	&3rd\\
8th	&+6/+1		&+2	&+6	&+6	&-					&32	&9	&3rd\\
9th	&+6/+1		&+3	&+6	&+6	&-					&40	&10	&3rd\\
10th	&+7/+2		&+3	&+7	&+7	&Bardic Performance			&50	&11	&4th\\
11th	&+8/+3		&+3	&+7	&+7	&-					&60	&12	&4th\\
12th	&+9/+4		&+4	&+8	&+8	&-					&72	&13	&4th\\
13th	&+9/+4		&+4	&+8	&+8	&-					&84	&14	&5th\\
14th	&+10/+5		&+4	&+9	&+9	&Bardic Performance			&98	&15	&5th\\
15th	&+11/+6/+1	&+5	&+9	&+9	&-					&112	&16	&5th\\
16th	&+12/+7/+2	&+5	&+10	&+10	&-					&128	&17	&6th\\
17th	&+12/+7/+2	&+5	&+10	&+10	&-					&144	&18	&6th\\
18th	&+13/+8/+3	&+6	&+11	&+11	&Bardic Performance			&162	&19	&6th\\
19th	&+14/+9/+4	&+6	&+11	&+11	&-					&180	&20	&6th\\
20th	&+15/+10/+5	&+6	&+12	&+12	&Bardic Performance			&200	&21	&6th\\
\bottomrule
\end{tabular}}
\end{table*}

\paragraph{Alignment:} Any nonlawful
\paragraph{Hit Die:} d6
\paragraph{Class skills:}
The Bard's class skills (and the key ability for each skill) are 
class skills (and the key ability for each skill) are Appraise (Int), Balance (Dex), Bluff (Cha), Climb (Str), Concentration (Con), Craft (Int), Decipher Script (Int), Diplomacy (Cha), Disguise (Cha), Escape Artist (Dex), Gather Information (Cha), Hide (Dex), Jump (Str), Knowledge (all skills, taken individually) (Int), Listen (Wis), Move Silently (Dex), Perform (Cha), Profession (Wis), Sense Motive (Wis), Sleight of Hand (Dex), Speak Language (N/A), Spellcraft (Int), Swim (Str), Tumble (Dex), and Use Magic Device (Cha).
\paragraph{Skill Points at 1st Level:} (6 + Int modifier) $\times$ 4.
\paragraph{Skill Points at each additional Level:} 6 + Int modifier.

\subsubsection{Class Features}
All the following are class features of the Bard.

\paragraph{Weapon and Armor Proficiency:} 
A Bard is proficient with all simple weapons, plus the longsword, rapier, sap, short sword, shortbow, and whip. 
Bards are proficient with light armor and shields (except tower shields).
Armor does not interfere with the casting of spells.

\paragraph{Spell Points/Day:} 
A Bard's ability to cast spells is limited by the spell points he has available. 
His base daily allotment of spell points is given on \nameref{tab:Bard} table. 
In addition, he receives bonus spell points per day if he has a high Charisma score.
His race may also provide bonus spell points per day, as may certain feats and items.

\paragraph{Spells Known:} A Bard begins play knowing two Bard spells of your choice. 
Each time he achieves a new level, he unlocks the knowledge of new spells.
Choose the spells known from the full Bard spell list.
(Exception: The feats Expanded Knowledge and Epic Expanded Knowledge 
do allow a Bard to learn spells of other classes, 
including spells restricted to specialist Wizards.) 

A Bard can cast any spell he knows that has a spell point cost equal to or lower than his caster level.
The number of times a Bard can cast spells in a day is limited only by his daily spell points. 
A Bard simply knows his spells; they are ingrained in his mind, 
though he must get a good night's sleep each day to regain all his spent spell points.
The Difficulty Class for saving throws against Bard spells is 10 + one-half the number of spell points spent on the spell (round up) + the Bard's Charisma modifier. 

Spells learned via the Bard class are arcane spells.
\paragraph{Maximum Spell Level Known:} A Bard begins play with the ability to learn 1st-level spells. 
As he attains higher levels, a Bard may gain the ability to master more complex spells, as shown on \nameref{tab:Bard} table.
To learn or cast a spell, a Bard must have a Charisma score of at least 10 + the spell's level.

\paragraph[Cantrips]{Cantrips (Su):} 
A Bard can use \nameref{sec:Cantrips} as a \nameref{sec:Wizard} can.

\paragraph{Bardic Performance:}
A Bard is trained to use the Perform skill to create magical effects on himself and those around him. At first level, he knows one type of Bardic Performance. At Bard levels 3, 5, 10, 14, 18, and 20, he learns an additional type of Bardic Performance if he has enough ranks in a Perform skill to do so. If he does not have enough ranks in a Perform skill to qualify for a new type of Bardic Performance when he is entitled to one, he does not gain one, but can (and must) select one immediately if he ever gains the required number of ranks in a Perform skill.

% He can use this ability for a number of rounds per day equal to 4 + his Charisma modifier. At each level after 1st a Bard can use Bardic Performance for 2 additional rounds per day. Each round, The Bard can produce any one of the types of Bardic Performance that he has mastered, as indicated by his level.

Starting a Bardic Performance is a standard action and requires the expenditure of the Bard's magical focus, but it can be maintained each round as a free action. 
Changing a Bardic Performance from one effect to another requires the Bard to stop the previous Performance and start a new one. 
A Bardic Performance ends immediately if the Bard is killed, paralyzed, stunned, knocked unconscious, or otherwise prevented from taking a free action to perform it each round. 
A Bard's Bardic Performance does not expire, nor does it need to be dismissed, it lasts until the Bard stops maintaining the Performance, and for five rounds thereafter. 
A Bard cannot maintain more than one Bardic Performance in a round (but previous performances still remain in effect until they expire if a Bard ceases to maintain one Bardic Performance in favor of another). 
A Bard cannot cast spells in any round in which he maintains a Bardic Performance, but his actions are not otherwise restricted (unless the use of his Perform skill physically requires it).

Each Bardic Performance has audible components, visual components, or both, as appropriate to the Perform skill being applied.

If a Bardic Performance has audible components, the targets must be able to hear the Bard for the Performance to have any effect, and such Performances are language dependent. A deaf Bard has a 20\% change to fail when attempting to use a Bardic Performance with an audible component. If he fails this check, the action is wasted, and the Bard's magical focus is still expended. Deaf creatures are immune to Bardic Performances with audible components.

If a Bardic Performance has a visual component, the targets must have line of sight to the Bard for the Performance to have any effect. A blind Bard has a 50\% chance to fail when attempting to use a Bardic Performance with a visual component. If he fails this check, the action is wasted, and the Bard's magical focus is still expended. Blind creatures are immune to Bardic Performances with visual components.

The types of Bardic Performance are as follows:
\subparagraph{Counterperformance (Su):}
The Bard can counter magic effects that depend on sound (but not spells that were merely cast with verbal components). 
This can take one of two forms:
 
\begin{itemize}
 \item \emph{Augment saves:} The Bard makes a Perform skill check each round he maintains this form of Counterperformance.
Any creature that can perceive the Bard's Performance that is affected by a sonic or language-dependent magical attack may use the Bard's Perform check result in place of its saving throw if, after the saving throw is rolled, the Perform check result proves to be higher. 
If a creature that perceives a counterperformance is already under the effect of a non-instantaneous sonic or language-dependent magical attack, it gains another saving throw against the effect each round it hears the counterperformance, but it must use the Bard's Perform skill check result for the save. 
This form of Counterperformance does nothing against effects that don't allow saves.
 \item \emph{Performance duel:} To use this ability, the Bard makes a Perform check, opposed by the Perform check of another Bard who is starting or maintaining a Bardic Performance. Both Bards must be able to perceive each others' performances. On a successful check, the Bard using Counterperformance immediately causes the Bardic Performance started or being maintained by the other Bard to end as if the other Bard had failed to maintain the Bardic Performance and the effect then expired.
 Alternatively, this use of Counterperformance can cause one Bardic Performance started and then abandoned (no longer maintained) by another Bard to immediately expire, with no opposed check required. The Bard must have perceived the Performance before it was abandoned for this use to be possible.
 Unlike most uses of Bardic Performance, this kind of Counterperformance is instantaneous in nature
\end{itemize}

A Bard must have 4 ranks in a Perform skill to learn or use Counterperformance.
\subparagraph{Fascinate (Su):}
The Bard can use Bardic Performance to cause one or more creatures to become fascinated with him. % Each creature to be fascinated must be perceive the Performance, and able to pay attention. The Bard must also be able to perceive the creature. 
The distraction of a nearby combat or other dangers prevents the ability from working. There is no limit to the number of creatures a Bard can fascinate at a time.

To use the ability, the Bard makes a Perform check. His check result is the DC for each affected creature's Will save against the effect. If a creature's saving throw succeeds, the Bard cannot attempt to fascinate that creature again for 24 hours. If its saving throw fails, the creature sits quietly and enjoys the Performance, taking no other actions for as long as the Bard maintains the Bardic Performance. While fascinated, a target takes a -4 penalty on skill checks made as reactions, such as Listen and Spot checks. Any potential threat requires the Bard to make another Perform check and allows the creature a new saving throw against a DC equal to the new Perform check result. Any obvious threat, such as someone drawing a weapon, casting a spell, or aiming a ranged weapon at the target, automatically breaks the effect. 
Fascinate is an enchantment (compulsion), mind-affecting ability.

A Bard must have 4 ranks in a Perform skill to learn or use Fascinate.
\subparagraph{Inspire Competence (Su):}
The Bard can use his Bardic Performance to help an ally succeed at a task. 
% The ally must be able to perceive the Performance. The Bard must also be able to perceive the ally.

The ally gets a +2 competence bonus on skill checks with a particular skill as long as the Bardic Performance continues. \href{http://www.giantitp.com/comics/oots0004.html}{Certain uses of this ability are infeasible.} A Bard can't inspire competence in himself. Inspire competence is a mind-affecting ability.

A Bard must have 6 ranks in a Perform skill to learn or use Inspire Competence.
\subparagraph{Inspire Courage (Su):}
The Bard can use Bardic Performance to inspire courage, bolstering himself and his allies.

An affected ally receives a +1 morale bonus on saving throws against charm and fear effects and a +1 morale bonus on attack and weapon damage rolls. At 5th level, and every four Bard levels thereafter, this bonus increases by 1 (+2 at 5th, +3 at 9th, +4 at 13th and +5 at 17th). Inspire Courage is a mind-affecting ability.

A Bard must have 4 ranks in a Perform skill to learn or use Inspire Courage.
\subparagraph{Inspire Dread (Su):}
The Bard can use Bardic Performance to inspire doubts and fear in his enemies.

An affected enemy receives a -1 penalty on saving throws against charm and fear effects and a -1 penalty on attack and weapon damage rolls, with no saving throw allowed. At 5th level, and every four Bard levels thereafter, this penalty increases by -1 (-2 at 5th, -3 at 9th, -4 at 13th and -5 at 17th). Inspire Dread is a mind-affecting ability.

A Bard must have 4 ranks in a Perform skill to learn or use Inspire Dread.
\subparagraph{Inspire Greatness (Su):}
The Bard can use Bardic Performance to inspire greatness in himself or a single willing ally within 30 feet, granting him or her extra fighting capability. 
For every three levels a Bard attains beyond 9th, he can target one additional ally with a single use of this ability (two at 12th level, three at 15th, four at 18th).
A creature inspired with greatness gains temporary hit points equal to 10 + twice its constitution modifier, a +2 competence bonus on attack rolls, a +2 competence bonus on Fortitude saves. Inspire greatness is a mind-affecting ability.

A Bard must have 12 ranks in a Perform skill to learn or use Inspire Greatness.
\subparagraph{Inspire Heroics (Su):}
The Bard can use Bardic Performance to inspire tremendous heroism in himself or a single willing ally.
For every three Bard levels the character attains beyond 15th, he can inspire heroics in one additional creature.
A creature so inspired gains a +4 morale bonus on saving throws and a +4 dodge bonus to AC. Inspire Heroics is a mind-affecting ability.

A Bard must have 18 ranks in a Perform skill to learn or use Inspire Heroics.
\subparagraph{Lullaby (Su):}
The Bard can use Bardic Performance to lull a single creature to sleep.
The creature must succeed on a Will save (DC 10 + $1/2$ the Bard's class level + Bard's Charisma modifier) or fall asleep for the effect's duration.
Creatures whose number of HD is higher than the number of ranks the Bard has in the Perform skill being applied are immune to this effect.

Sleeping creatures are helpless. Slapping or wounding awakens an affected creature, but normal noise does not. 
Awakening a creature is a standard action (an application of the aid another action).

Lullaby is a mind-affecting compulsion.

A Bard must have 6 ranks in a Perform skill to learn or use Inspire Heroics.
\subparagraph{Satire (Su):}
By employing tools like rude gestures and offensive lyrics, the Bard can use Bardic Performance to taunt creatures into attacking him.
All enemies who can perceive the Satire must succeed on a Will save (DC 10 + $1/2$ the Bard's class level + Bard's Charisma modifier) or be forced to attack the Bard in preference over all other available targets.
Affected enemies are not thrown into mindless rage, but must concentrate their efforts on doing the Bard direct harm. When making ranged attacks, they must attempt to strike the Bard. When using offensive spells or supernatural abilities, they must target the Bard with the attack or include him in its area. They may attack the Bard in melee, but they do not have to do so if doing so would cause them obvious harm, such as due to walking into a chasm or provoking an attack of opportunity. They can attack the Bard's allies if that is the best way to get to the Bard - for example, if the Bard's ally provides him with cover against ranged attacks or is standing in the way of a charge.

A Bard must have 6 ranks in a Perform skill to learn or use Satire.
\subparagraph{Song of Freedom (Sp):}
The Bard can use Bardic Performance to cast \nameref{Spell:RemoveCurse} as a spell-like ability. This functions as normal for spell-like abilities, except that it requires the expenditure of the Bard's magical focus (unlike most uses of Bardic Performance, Song of Freedom is instantaneous in nature). The spell-like ability has a caster level equal to the Bard's Bard level.
A Bard can't use song of freedom on himself.

A Bard must have 12 ranks in a Perform skill to learn or use Song of Freedom.
\subparagraph{Song of the Clouded Mind (Sp):}
The Bard can use Bardic Performance to cast \nameref{Spell:Confusion} as a spell-like ability.

This functions as normal for spell-like abilities except that it requires the expenditure of the Bard's magical focus, and its duration is determined as for other uses of Bardic Performance rather than the spell's duration (requiring a free action to maintain on each round, but lasts until the Bard ceases to perform. This overrides the spell's augmentation option.).
The saving throw DC against the Confusion is Charisma-based, and its caster level is equal to the Bard's Bard level.

A Bard must have 12 ranks in a Perform skill to learn or use Song of the Clouded Mind.
\subparagraph{Subliminal Intrusion (Su):}
The Bard can use Bardic Performance to weaken his enemies against mental intrusions.

An affected enemy takes a -1 penalty on will saving throws. At 5th level, and every four Bard levels thereafter, this penalty increases by -1 (-2 at 5th, -3 at 9th, -4 at 13th and -5 at 17th). Subliminal Intrusion is a mind-affecting ability.

A Bard must have 9 ranks in a Perform skill to learn or use Subliminal Intrusion.
\subparagraph{Suggestion (Sp):}
The Bard can use Bardic Performance to cast \nameref{Spell:Suggestion} as a spell-like ability. 
This functions as normal for spell-like abilities except that it requires the expenditure of the Bard's magical focus, and its duration is determined as for other uses of Bardic Performance rather than the spell's duration (requiring a free action to maintain on each round, but lasts until the Bard ceases to Perform and for five rounds thereafter).
The saving throw DC against the Suggestion is Charisma-based, and its caster level is equal to the Bard's Bard level.

A Bard must have 9 ranks in a Perform skill to learn or use Suggestion.
\paragraph{Bardic Knowledge:}
A Bard may make a special Bardic Knowledge check with a bonus equal to his Bard level + his Intelligence modifier to see whether he knows some relevant information about local notable people, 
legendary items, or noteworthy places. (If the Bard has 5 or more ranks in Knowledge (history), he gains a +2 bonus on this check.)

A successful Bardic knowledge check will not reveal the powers of a magic item but may give a hint as to its general function. 
A Bard may not take 10 or take 20 on this check; this sort of knowledge is essentially random.
\begin{tableonecolumn}
\caption{Bardic Knowledge}
\label{tab:BardicKnowledge}
\begin{tabular}{p{0.4cm}p{6cm}}
\toprule
DC &Type of Knowledge\\
\midrule
10 &Common, known by at least a substantial minority of the local population.\\
20 &Uncommon but available, known by only a few people legends.\\
25 &Obscure, known by few, hard to come by.\\
30 &Extremely obscure, known by very few, possibly forgotten by most who once knew it, possibly known only by those who don't understand the significance of the knowledge.\\
\bottomrule
\end{tabular}
\end{tableonecolumn}
\subsubsection{Ex-Bards}
A Bard who becomes lawful in alignment cannot progress in levels as a Bard, though he retains all his Bard abilities. 
\subsection[Cleric]{The Cleric}
\begin{quote}
\emph{The powers of the outer planes are real, and the work I do is proof of that.}
- \'Imel\'ia, halfling Cleric
\end{quote}
When a mortal places his faith in a higher power, sometimes power is invested in the mortal in turn.
These mortals are known as Clerics.
\paragraph{Alignment:}
The alignment of a Cleric who worships a deity must be within one step of that of his deity  (that is, it may be one step away on either the lawful-chaotic axis or the good-evil axis, but not both). A deity-devoted Cleric may not be neutral unless his deity's alignment is also neutral.
\paragraph{Hit Die:} d8
\paragraph{Class skills:}
The Cleric's class skills (and the key ability for each skill) are 
Concentration (Con), Craft (Int), Diplomacy (Cha), Heal (Wis), Knowledge (arcana) (Int), Knowledge (history) (Int), Knowledge (religion) (Int), Knowledge (the planes) (Int), Profession (Wis), and Spellcraft (Int).

A Cleric's domains may grant him additional class skills.
\paragraph{Skill Points at 1st Level:} (4 + Int modifier) $\times$ 4.
\paragraph{Skill Points at each additional Level:} 4 + Int modifier.
\begin{table*}
\centering
\caption{The Cleric}
\label{tab:Cleric}
\makebox[\textwidth]{
\begin{tabular}{|l|l|c|c|c|l|c|c|}
\hline
\multirow{2}{*}{\textbf{Level}}&\multirow{2}{*}{\textbf{BAB}}&\textbf{Fort}&\textbf{Ref}&\textbf{Will}&\multirow{2}{*}{\textbf{Special}}&\multirow{2}{*}{\textbf{SP/day}}&\textbf{Spells}\\
&&\textbf{save}&\textbf{save}&\textbf{save}&&&\textbf{known}\\
\hline
1st	&+0		&+2	&+0	&+2	&Domains (2)		&3	&2+CMW\\
2nd	&+1		&+3	&+0	&+3	&-			&6	&4\\
3rd	&+2		&+3	&+1	&+3	&-			&10	&6\\
4th	&+3		&+4	&+1	&+4	&-			&16	&7\\
5th	&+3		&+4	&+1	&+4	&Domain			&24	&8\\
6th	&+4		&+5	&+2	&+5	&-			&33	&10\\
7th	&+5		&+5	&+2	&+5	&-			&43	&11\\
8th	&+6/+1		&+6	&+2	&+6	&-			&55	&12\\
9th	&+6/+1		&+6	&+3	&+6	&-			&69	&14\\
10th	&+7/+1		&+7	&+3	&+7	&Domain			&84	&15\\
11th	&+8/+3		&+7	&+3	&+7	&-			&100	&16\\
12th	&+9/+4		&+8	&+4	&+8	&-			&118	&18\\
13th	&+9/+4		&+8	&+4	&+8	&-			&138	&19\\
14th	&+10/+5		&+9	&+4	&+9	&-			&159	&20\\
15th	&+11/+6/+1	&+9	&+5	&+9	&Domain			&181	&22\\
16th	&+12/+7/+2	&+10	&+5	&+10	&-			&205	&23\\
17th	&+12/+7/+2	&+10	&+5	&+10	&-			&231	&24\\
18th	&+13/+8/+3	&+11	&+6	&+11	&-			&258	&26\\
19th	&+14/+9/+4	&+11	&+6	&+11	&-			&286	&27\\
20th	&+15/+10/+5	&+12	&+6	&+12	&-			&316	&28\\
\hline
\end{tabular}}
\end{table*}
\subsubsection{Class Features}
All the following are class features of the Cleric.

\paragraph{Weapon and Armor Proficiency:} 
Clerics are proficient with all simple weapons, as well as the favored weapon of their deity.
They are proficient with light and medium armor, and with shields (except tower shields and exotic shields).

\paragraph{Spell Points/Day:} 
A Cleric's ability to cast spells is limited by the spell points he has available. 
His base daily allotment of spell points is given on \nameref{tab:Cleric} table. 
In addition, he receives bonus spell points per day if he has a high Wisdom score.
His race may also provide bonus spell points per day, as may certain feats and items.

\paragraph{Spells Known:} A Cleric begins play knowing two Cleric spells of your choice, 
as well as the \nameref{Spell:TouchOfVitality} spell. 
Each time he achieves a new level, he unlocks the knowledge of new spells.
A Cleric's class spell list is the set of all spells that appear on the general \nameref{Domain:General} spell list or on the spell lists of one or more of the domains belonging to his deity. However, he must choose his spells known from only the generic spell list or from the lists of the domains he has available (see Domains, below).
(Exception: The feats Expanded Knowledge and Epic Expanded Knowledge do allow a Cleric to learn spells of other classes, 
including spells restricted to specialist Wizards.) 

Unlike most spellcasting classes, Clerics do not have a set maximum spell level known.
Instead, they can learn any spell on their domain lists as long as they can pay the spell's minimum spell point cost.
If a Cleric already knows all generic Cleric spells and all spells on his domain lists for which he qualifies when he is entitled to a new spell known, he does not gain one, but can (and must) select one immediately if he ever gains access to a new domain, or if his caster level increases to the point where he unlocks new spells on his pre-existing spell lists.

A Cleric can cast any spell he knows that has a spell point cost equal to or lower than his caster level.
The number of times a Cleric can cast spells in a day is limited only by his daily spell points. 
A Cleric simply knows his spells; they are ingrained in his mind, though he must get a good night's sleep each day to regain all his spent spell points.
The Difficulty Class for saving throws against Cleric spells is 10 + one-half the number of spell points spent on the spell (round up) + the Cleric's Wisdom modifier. 

Spells learned via the Cleric class are divine spells.
\paragraph{Domains:}
A Cleric's deity influences his alignment, what magic he can perform, his values, and how others see him. 
At first level, a Cleric chooses two domains from among those belonging to his deity.
He may select an additional domain on the levels indicated on the \nameref{tab:Cleric} table.

A Cleric who is not devoted to a particular deity selects domains that match his personal spiritual inclinations.

The domains a Cleric selects form the backbone of his abilities.
Each domain is divided into two parts, a spell list and a collection of granted powers.
See the \nameref{sec:Spells} chapter for information on individual \nameref{sec:ClericDomains}.
\subsubsection{Ex-Clerics:}
A Cleric who changes to an inappropriate alignment or grossly violates the code of conduct required by his god loses the ability to cast Cleric spells and all domain granted abilities.
The spellcasting and granted abilities remain dormant until he atones (see the \nameref{Spell:Atonement} spell description).
\subsubsection[Druid]{Variant: Druids}
\label{sec:Druid}
To some Clerics, revering nature and its awesome, intrinsic power is more important than the worship of deities or what they represent.
These Clerics are known as Druids, and are different from the standard Cleric class in several ways, as outlined below:

\paragraph{Alignment:} A Druid must be Neutral good, lawful neutral, neutral, chaotic neutral, or neutral evil.

\paragraph{Class Skills:} A Druid's class skills (and the key ability for each skill) are Concentration (Con), Craft (Int), Handle Animal (Cha), Heal (Wis), Knowledge (nature) (Int), Knowledge (religion) (Int), Profession (Wis), Ride (Dex), Spellcraft (Int), Survival (Wis), and Swim (Str).

A Druid's domains may grant him additional class skills, as for normal Clerics.

\paragraph{Language:} A Druid can learn a special language, known only to Druids. It is referred to as simply \emph{Druidic}.

\paragraph{Class Features:}
The Druid has all the standard Cleric class features, except as noted below.
\subparagraph{Weapon and Armor Proficiency:} A Druid does not gain proficiency with his deity's favored weapon, even if he selects a deity (making him proficient with simple weapons only).
A Druid does not have the standard Cleric's armor proficiency, instead, he is proficient only with padded armor, leather armor, hide armor, light wooden shields, and heavy wooden shields.

\paragraph{Deity and Domains:} A Druid does not gain his powers from a deity. 
He may have a patron deity as any other character can, but this deity is not the source of the Druid's power.
Instead of selecting from a deity's list of available domains, a Druid may select the domains of \emph{Air, Animal, Earth, Fire, Healing, Moon*, Plant, Sun, Travel, Vermin}, and \emph{Water}.

\subsubsection{Variant: Cloistered Cleric}
The Cloistered Cleric spends more time in study and prayer than other Clerics do, and less in martial training. 
He gives up some of the Cleric's combat prowess in exchange for greater skill access.

\paragraph{Hit Die:}
The Cloistered Cleric uses a d6 for his Hit Die (and has hit points at 1st level equal to 6 + Con modifier).

\paragraph{Base Attack Bonus:}
The Cloistered Cleric's lack of martial training means that he uses the poor base attack bonus.

\paragraph{Class Skills:}
The Cloistered Cleric's class skill list includes Decipher Script, Speak Language, and all Knowledge skills. The Cloistered Cleric gains skill points per level equal to 6 + Int modifier (and has this number x4 at 1st level). If he has the \nameref{Domain:Knowledge}, he gains a +2 bonus on all trained knowledge checks.

\paragraph{Class Features:}
The Cloistered Cleric has all the standard Cleric class features, except as noted below.

\subparagraph{Weapon and Armor Proficiency:}
Cloistered Clerics are proficient with only simple weapons and with light armor.

\subparagraph{Lore (Ex):}
You gain the Lore granted power of the \nameref{Domain:Knowledge}.
If you have the Knowledge Domain, you gain a +2 bonus on all Lore checks.

\subparagraph{Deity and Domains:}
Most Cloistered Clerics worship deities associated with knowledge and learning.

In addition to any domains selected from his deity's list, a Cloistered Cleric may select the \nameref{Domain:Knowledge}, even if that  domain is not normally available to Clerics of that deity.
\subsubsection{Variant: Alternate Approaches to Domains}
Clerics' default method of selecting domains becomes problematic if each deity in the GM's setting has very few domains assigned to it, especially if a deity knows fewer than 5 domains.

These alternate approaches are designed to give deity-devoted Clerics back the flexibility they by design should have.

If the options of Clerics are expanded in this way, Druids (see above) should receive similar benefits.
\paragraph{Variant: Domains First}
A different interpretation on domain access is the one of 
Clerics not selecting domains because they are offered by their patron deity, but rather that they choose to worship a patron deity over all others because he closely matches the domains he has selected.

This variant is appropriate in settings where abstract forces (such as good and evil) are constants higher than the deities themselves, or in settings where the deities are dependent upon their worshippers for power.
\paragraph{Variant: Pantheon Worship}
If a Cleric's patron deity belongs to a pantheon of gods, the Cleric will recognize the portfolio and powers of deities other than his patron deity.
Even though the Cleric will see his patron deity and his portfolio as the most important aspects of the faith, he will take up domains other than those offered by his patron deity if the situation demands.

For example, a Cleric of Thor (a god primarily associated with strength and war) might take up the Water domain (a domain which has nothing to do with Thor, but is offered by the god Aegir) if he routinely finds himself fighting campaigns at sea, or a Cleric of Baldr (a god of beauty) who has been charged with protecting the god's temples against molesters might assume the War domain so he might better fulfil his duty.

Alternatively, a Cleric might not be devoted to any particular deity, but rather worships the pantheon as a whole.

This variant is appropriate where the gods form pantheons.
\paragraph{Variant: Sects and Cults}
If a Cleric wants to worship one deity and one deity only, it is an indicator of the deity's portfolio being broad, and encompassing multiple aspects of life. 
The followers of such a multifaceted deity are likely to differ on some aspects of the faith, breaking the body of worshippers into sects and cults. 
The Clerics of each individual sect or cult would then take different domains to represent their interpretations of the deity.

For example, some Clerics of Thor might emphasize his role as a warrior, taking up the domains of Strength and War.
Others would emphasize his role as a god of thunder, taking up the Air domain, or his role as the protector of his extended family, taking up the domains of Good and Protection. 
Yet another group of worshippers might emphasize how Thor was never afraid to use any means necessary to crush his enemies,
their Clerics taking up the domains of Destruction and Evil.

This variant is appropriate where the gods are distant or ill understood by mortals. 
\subsection[Paladin]{The Paladin}
\label{sec:Paladin}
\begin{quote}
\emph{Where evil lurks, that is where I stand vigilant.}
-Tulkas, half-giant Paladin
\end{quote}

A Paladin is a hero who has dedicated his life and soul to the promotion of good and destruction of evil, and gained divine powers in return.
\paragraph{Alignment:} Any good. A chaotic good Paladin is often referred to as a Paladin of Freedom. Other Paladins are rarely referred to by a specific title, but lawful good Paladins are sometimes called Paladins of Honor.
\paragraph{Hit Die:} d10
\paragraph{Class skills:}
The Paladin's class skills (and the key ability for each skill) are Concentration (Con), Craft (Int), Diplomacy (Cha), Gather Information (Cha), Handle Animal (Cha), Heal (Wis), Intimidate (Cha), Knowledge (local) (Int), Knowledge (nobility and royalty) (Int), Knowledge (religion) (Int), Knowledge (the planes) (Int), Profession (Wis), Ride (Dex), Sense Motive (Wis), and Spellcraft (Int).

\paragraph{Skill Points at 1st Level:} (4 + Int modifier) $\times$ 4.
\paragraph{Skill Points at each additional Level:} 4 + Int modifier.
\begin{table*}
\caption{The Paladin}
\label{tab:Paladin}
\makebox[\textwidth]{
\begin{tabular}{|l|l|c|c|c|l|c|c|c|}
\hline
&&\textbf{Fort}&\textbf{Ref}&\textbf{Will}&&&\textbf{Spells}&\textbf{Max}\\
\textbf{Level}&\textbf{BAB}&\textbf{save}&\textbf{save}&\textbf{save}&\textbf{Special}&\textbf{SP/day}&\textbf{known}&\textbf{level}\\
\hline
1st &+1			&+2 &+0 &+0	&Aura of Good, Divine			&0&1+CMW&1st\\
    &			&&&		&Grace, Smite				&&&\\
2nd &+2 		&+3 &+0 &+0 	&Holy Gift 				&1 &2 &1st\\
3rd &+3 		&+3 &+1 &+1 	&-    					&3 &3 &1st\\
4th &+4 		&+4 &+1 &+1 	&-    					&5 &4 &2nd\\
5th &+5 		&+4 &+1 &+1 	&Holy Gift   				&7 &5 &2nd\\
6th &+6/+1 		&+5 &+2 &+2 	&-    					&11 &6 &2nd\\
7th &+7/+2 		&+5 &+2 &+2 	&-    					&15 &7 &3rd\\
8th &+8/+3 		&+6 &+2 &+2 	&Holy Gift   				&19 &8 &3rd\\
9th &+9/+4 		&+6 &+3 &+3 	&-    					&23 &9 &3rd\\
10th &+10/+5		&+7 &+3 &+3 	&-    					&27 &10 &4th\\
11th &+11/+6/+1		&+7 &+3 &+3 	&Holy Gift   				&35 &11 &4th\\
12th &+12/+7/+2 	&+8 &+4 &+4 	&-    					&43 &12 &4th\\
13th &+13/+8/+3 	&+8 &+4 &+4 	&-    					&51 &13 &5th\\
14th &+14/+9/+4 	&+9 &+4 &+4 	&Holy Gift   				&59 &14 &5th\\
15th &+15/+10/+5	&+9 &+5 &+5 	&-    					&67 &15 &5th\\
16th &+16/+11/+6/+1 	&+10 &+5 &+5 	&-    					&79 &16 &6th\\
17th &+17/+12/+7/+2 	&+10 &+5 &+5 	&Holy Gift   				&91 &17 &6th\\
18th &+18/+13/+8/+3 	&+11 &+6 &+6 	&-    					&103 &18 &6th\\
19th &+19/+14/+9/+4 	&+11 &+6 &+6 	&-    					&115 &19 &6th\\
20th &+20/+15/+10/+5	&+12 &+6 &+6 	&Holy Gift   				&127 &20 &6th\\
\hline
\end{tabular}}
\end{table*}

% \begin{table*}
% \caption{The Paladin}
% \label{tab:Paladin}
% \makebox[\textwidth]{
% \rowcolors{6}{}{lightgray}
% \begin{tabular}{llccclccc}
% &&\textbf{Fort}&\textbf{Ref}&\textbf{Will}&&&\textbf{Spells}&\textbf{Max}\\
% \textbf{Level}&\textbf{BAB}&\textbf{save}&\textbf{save}&\textbf{save}&\textbf{Special}&\textbf{SP/day}&\textbf{known}&\textbf{level}\\
% \rowcolor{lightgray} 1st &+1			&+2 &+0 &+2	&Aura of Good, Divine			&0&1+CMW&1st\\
% \rowcolor{lightgray}&			&&&		&Grace, Smite				&&&\\
% 2nd &+2 		&+3 &+0 &+3 	&Holy Gift 				&1 &2 &1st\\
% 3rd &+3 		&+3 &+1 &+3 	&-    					&3 &3 &1st\\
% 4th &+4 		&+4 &+1 &+4 	&-    					&5 &4 &2nd\\
% 5th &+5 		&+4 &+1 &+4 	&Holy Gift   				&7 &5 &2nd\\
% 6th &+6/+1 		&+5 &+2 &+5 	&-    					&11 &6 &2nd\\
% 7th &+7/+2 		&+5 &+2 &+5 	&-    					&15 &7 &3rd\\
% 8th &+8/+3 		&+6 &+2 &+6 	&Holy Gift   				&19 &8 &3rd\\
% 9th &+9/+4 		&+6 &+3 &+6 	&-    					&23 &9 &3rd\\
% 10th &+10/+5		&+7 &+3 &+7 	&-    					&27 &10 &4th\\
% 11th &+11/+6/+1		&+7 &+3 &+7 	&Holy Gift   				&35 &11 &4th\\
% 12th &+12/+7/+2 	&+8 &+4 &+8 	&-    					&43 &12 &4th\\
% 13th &+13/+8/+3 	&+8 &+4 &+8 	&-    					&51 &13 &5th\\
% 14th &+14/+9/+4 	&+9 &+4 &+9 	&Holy Gift   				&59 &14 &5th\\
% 15th &+15/+10/+5	&+9 &+5 &+9 	&-    					&67 &15 &5th\\
% 16th &+16/+11/+6/+1 	&+10 &+5 &+10 	&-    					&79 &16 &6th\\
% 17th &+17/+12/+7/+2 	&+10 &+5 &+10 	&Holy Gift   				&91 &17 &6th\\
% 18th &+18/+13/+8/+3 	&+11 &+6 &+11 	&-    					&103 &18 &6th\\
% 19th &+19/+14/+9/+4 	&+11 &+6 &+11 	&-    					&115 &19 &6th\\
% 20th &+20/+15/+10/+5	&+12 &+6 &+12 	&Holy Gift   				&127 &20 &6th\\
% \end{tabular}}
% \end{table*}
\subsubsection{Class Features}
All the following are class features of the Paladin.

\paragraph{Weapon and Armor Proficiency:} 
Paladins are proficient with all simple and martial weapons, 
with all types of armor (heavy, medium, and light, but not exotic armors),
and with shields (except tower shields and exotic shields).

\paragraph{Spell Points/Day:} A Paladin's ability to cast spells is limited by the spell points he has available. 
His base daily allotment of spell points is given on \nameref{tab:Paladin} table. 
In addition, he receives bonus spell points per day if he has a high Charisma score.
His race may also provide bonus spell points per day, as may certain feats and items.

\paragraph{Spells Known:} A Paladin begins play knowing the \nameref{Spell:TouchOfVitality} spell, and one other Paladin spell of your choice. 
Each time he achieves a new level, he unlocks the knowledge of new spells.
Choose the spells known from the Paladin spell list
(Exception: The feats Expanded Knowledge and Epic Expanded Knowledge do allow a Paladin to learn spells of other classes, even specialist Wizard spells.).
A Paladin can cast any spell he knows that has a spell point cost equal to or lower than his caster level.
The number of times a Paladin can cast spells in a day is limited only by his daily spell points. 
A Paladin simply knows his spells; they are ingrained in his mind, though he must get a good night's sleep each day to regain all his spent spell points.
The Difficulty Class for saving throws against Paladin spells is 10 + one-half the number of spell points spent on the spell (round up) + the Paladin's Charisma modifier. 

Spells learned via the Paladin class are divine spells.
\paragraph{Maximum Spell Level Known:} A Paladin begins play with the ability to learn 1st-level spells. 
As he attains higher levels, a Paladin may gain the ability to master more complex spells, as shown on the \nameref{tab:Paladin} table.
To learn or cast a spell, a Paladin must have a Charisma score of at least 10 + the spell's level.

\paragraph{Aura of Good: (Su)} 
At will, as a free action, a Paladin can project a holy aura.
While the aura is active, the Paladin gains a +4 sacred bonus on Diplomacy checks versus Good creatures, and a +4 sacred bonus on Intimidate checks versus Evil creatures (Neutral creatures are not influenced either way).
All sapient creatures become instinctively aware of the Paladin's good alignment while the aura is active.
The Paladin can project this aura indefinitely, or until he dismisses it (another free action). 

\paragraph[Smite]{Smite: (Su)}
\label{sec:Smite}
You can infuse your attacks with supernatural determination and righteous fury.

In order to perform a Smite, you must expend your magical focus as part of making an attack.
The attack then gains a bonus on the attack roll equal to your Charisma modifier, and a bonus on the damage roll equal to your Paladin level.
You must decide whether or not to perform a Smite before making the attack. 
If the attack misses, you still expend your magical focus.
This is a Supernatural ability, activated as part of making an attack.

\paragraph{Divine Grace: (Su)} A Paladin gains a bonus on all saving throws equal to his Charisma modifier or his Paladin level, whichever is lower.
This is a Supernatural ability that functions continuously, requiring no activation.

% \paragraph{Bonus Feats:}
% At 2nd level, a Paladin gets a bonus feat.
% He gains an additional bonus feat at Paladin levels 5th, 8th, 11th, 14th, 17th, and 20th. 
% These bonus feats must be drawn from the feats noted as fighter bonus feats, or those feats which have one or more Paladin levels as a prerequisite\footnote{
% The feats that require Paladin levels to take are \nameref{Feat:AuraOfCourage}, \nameref{Feat:CelestialMount}, \nameref{Feat:ChargingSmite}, 
% \nameref{Feat:DetectOpposition}, \nameref{Feat:DefensiveBastion}, \nameref{Feat:DivineHealth}, \nameref{Feat:LayOnHands}, 
% \nameref{Feat:RemoveDisease}, \nameref{Feat:ShieldGuardian}, and \nameref{Feat:TurnUndead}.}. 
% The Paladin must still meet all prerequisites for the bonus feat, 
% including ability score and base attack bonus minimums as well as class requirements. 
% A Paladin cannot choose feats that specifically require levels in the fighter class 
% unless he is a multiclass character with the requisite levels in the fighter class.
% 
% These bonus feats are in addition to the feats that a character of any class gains every three levels. 
% A Paladin is not limited to fighter bonus feats and feats with Paladin levels as a requirement when choosing these other feats.

\paragraph{Holy Gift:}
At 2nd level, the celestial powers the Paladin serves reward him with a gift of new abilities. He gains an additional gift at Paladin levels 5th, 8th, 11th, 17th, and 20th. Unless otherwise noted, a gift can only be selected once. Some gifts have specific prerequisites.

\subparagraph{Additional Feat:}
You gain a bonus feat from the list of feats noted as Fighter bonus feats. You may also select the \nameref{Feat:TurnUndead} feat.
This gift may be selected more than once.

\subparagraph{Aura of Courage (Su):}
You are a fearless champion who inspires his companions to bravery.
You gain immunity to fear, and each ally within 10 feet of you gains a +4 morale bonus on saving throws against fear effects.
This is a Supernatural ability that functions continuously while you are conscious, but not if you are unconscious or dead.

\subparagraph[Celestial Mount]{Celestial Mount:}
\label{sec:CelestialMountListing}
You gain the service of a blessed animal. See the \nameref{sec:CelestialMount} creature description. You must be a 5th-level Paladin to select this gift. If you have the \nameref{Feat:AnimalCompanion} feat, the \nameref{Feat:Familiar} feat, the \nameref{sec:FiendishMountListing} ability, or the \nameref{Feat:SpellstaffUser} feat, you may not select this gift.

\subparagraph{Charging Smite (Ex):}
Your righteous charges are fearsome to behold. When you use your Smite class feature on a charge attack, you add twice your Paladin level to your damage roll instead of simply your Paladin level. You must be a 5th-level Paladin to select this gift.

\subparagraph{Detect Opposition (Su):}
You are an expert in foiling the machinations of others. 
You gain a bonus on all Sense Motive checks and Spot checks to see through disguises equal to your Paladin level.
This is a Supernatural ability that functions continuously.

\subparagraph{Divine Health (Su):}
Your connection to the divine fortifies you against ailments of the body. You gain immunity to all diseases, including supernatural and magical diseases. This is a Supernatural ability that functions continuously.

\subparagraph{Lay on Hands (Su):}
You are blessed with a supernatural ability to effectively heal wounds. Whenever a you cast a \nameref{Spell:TouchOfVitality} spell, you may expend your magical focus.
This infuses the touch with the blessing of the your holy patron, augmenting the spell as if the you had spent
an additional number of spell points on the spell equal to your Paladin level. 
These virtual spell points are supplied by the gift, rather than your own spellcasting ability, 
and thus do not count against the limit imposed by the fundamental rule of magic. 
If you also use your own spell points to augment the spell, they stack with these virtual spell points.
(Usually, this simply simply means that the Paladin may expend 
his magical focus to have his \nameref{Spell:TouchOfVitality} heal a number of additional points equal to twice his Paladin level.)
This is a Supernatural ability, activated as part of casting a \nameref{Spell:TouchOfVitality} spell.

\subparagraph{Remove Disease (Sp):}
You are blessed with a supernatural ability to cure diseases. You can use \nameref{Spell:RemoveDisease} as a spell-like ability at will.
You must be a 5th-level Paladin to select this gift.

\subparagraph{Shield Guardian (Su):}
Your defensive efforts benefit your entire party. All allies within 10' of you gain a shield bonus to AC equal to your shield bonus to AC (if any).
This is a Supernatural ability that functions continuously while you are conscious, but not if you are unconscious or dead.

\subparagraph{Stunning Smite (Su):}
Your righteous wrath towards Evil charges your smites against them with holy power, overwhelming their senses.
When you perform a smite attack against an evil creature, it must succeed on a Will save (DC 10 + $1/2$ your HD + your Charisma modifier) or be \emph{stunned} by for 1 round. You must be an 8th-level Paladin to select this gift.

\subparagraph{Vigil (Ex):}
It is no small feat to slip past a Paladin in combat. Your opponents must treat all squares you threaten as difficult terrain, slowing their movement and preventing charges. You must be aware of your opponent for this ability to take effect.

\subparagraph{Vigor (Ex):}
The holy energies that permeate your body make you more resilient than most. You gain damage reduction equal to one-half your Paladin level, minimum 1. This damage reduction is overcome by evil-aligned weapons.

\subsubsection{Code of Conduct:}
Sometimes, the divine patrons that grant Paladins their powers instigate a formal code of conduct to make sure their mortal servants are and remain true paragons of good.

Paladin players and GMs should work together to create a code of conduct appropriate for the campaign and character.

Examples of rules a Paladin has to abide by could be
\begin{list}{\labelitemi}{\leftmargin=1em}
 \item A Paladin must maintain not only a good alignment, but a lawful good alignment.
 \item A Paladin must never willingly commit an evil act.
 \item A Paladin must respect legitimate authority.
 \item A Paladin must act with honor - he must never lie, cheat, or use poison.
 \item A Paladin must help those in need, provided the help is not used for evil ends.
 \item A Paladin must punish those who harm or threaten innocents.
 \item A Paladin may never knowingly associate with evil characters.
\end{list}
\subsubsection{Ex-Paladins:}
A Paladin who changes to a nongood alignment or grossly violates his code of conduct (if any) loses his ability to cast Paladin spells, the Aura of Good class feature, the Divine Grace class feature, the Smite Class feature, and all supernatural feats that have one or more Paladin levels as a prerequisite.
The spellcasting and other abilities remain dormant until he atones (see the \nameref{Spell:Atonement} spell description).

\subsubsection[Antipaladin]{Variant: Antipaladin}
While the champions of good are well known, the forces of evil are no less willing to accept cruel mortals into their fold. Such a villain, an Antipaladin, is similar to a \nameref{sec:Blackguard}, but there is no fall or shift to evil involved - an Antipaladin is rotten from the start. 

To make an Antipaladin, replace Paladin class features with Blackguard class features in the following ways:

\paragraph{Alignment:} Any evil. Lawful evil Antipaladins are referred to as Paladins of Tyranny, while chaotic evil Antipaladins are called Paladins of Slaughter.
\paragraph{Class Skills:} As \nameref{sec:Blackguard}.

\paragraph{Class Features:}
The Antipaladin has all the standard Paladin class features, except as noted below.

\subparagraph{Spells:} Use the \nameref{sec:BlackguardSpellList}.
\subparagraph{Aura of Good:} Replace with \nameref{sec:AuroOfEvil}.
\subparagraph{Holy Gift:} Antipaladins select \nameref{sec:UnholyGift}s rather than Holy Gifts. For any Unholy Gift that requires a specific number $n$ of Blackguard levels, replace the level requirement with a requirement $n+6$ of Antipaladin levels.


 
\subsection[Ranger]{The Ranger}
\label{sec:Ranger}
\begin{quote}
\emph{``Leave them to me. I'm an expert on humans.''}
- K\"argon, elven Ranger
\end{quote}

A Ranger is a divine champion of the woodlands and the natural world, who specializes in a distinct fighting style and the destruction of his sworn enemies.
\paragraph{Alignment:} Any.
\paragraph{Hit Die:} d8
\paragraph{Class skills:}
The Ranger's class skills (and the key ability for each skill) are Climb (Str), Concentration (Con), Craft (Int), Handle Animal (Cha), Heal (Wis), Hide (Dex), Jump (Str), Knowledge (dungeoneering) (Int), Knowledge (geography) (Int), Knowledge (nature) (Int), Listen (Wis), Move Silently (Dex), Profession (Wis), Ride (Dex), Search (Int), Spot (Wis), Survival (Wis), Swim (Str), Tumble (Dex) and Use Rope (Dex).

\paragraph{Skill Points at 1st Level:} (6 + Int modifier) $\times$ 4.
\paragraph{Skill Points at each additional Level:} 6 + Int modifier.
\begin{table*}
%\centering
\caption{The Ranger}
\label{tab:Ranger}
%\rowcolors{1}{white}{lightgray}
\makebox[\textwidth]{
\begin{tabular}{|l|l|c|c|c|l|c|c|c|}
\hline
\multirow{2}{*}{\textbf{Level}}&\multirow{2}{*}{\textbf{BAB}}&\textbf{Fort}&\textbf{Ref}&\textbf{Will}&\multirow{2}{*}{\textbf{Special}}&\multirow{2}{*}{\textbf{SP/day}}&\textbf{Spells}&\textbf{Max}\\
&&\textbf{save}&\textbf{save}&\textbf{save}&&&\textbf{known}&\textbf{level}\\
\hline
\multirow{2}{*}{1st} &\multirow{2}{*}{+1}			&\multirow{2}{*}{+2} &\multirow{2}{*}{+2} &\multirow{2}{*}{+0} 	&Combat Style, Favored 			&\multirow{2}{*}{0} &\multirow{2}{*}{1} &\multirow{2}{*}{1st}\\
    &			&   &	&	&Enemy, Wild Empathy			&  &  &\\
2nd &+2 		&+3 &+2 &+0 	&Bonus Feat				&1 &2 &1st\\
\multirow{2}{*}{3rd} &\multirow{2}{*}{+3} 		&\multirow{2}{*}{+3} &\multirow{2}{*}{+3} &\multirow{2}{*}{+1} 	&Intuition, Pass without		&\multirow{2}{*}{3} &\multirow{2}{*}{3} &\multirow{2}{*}{1st}\\ 
    &			&   &	&	&Trace					&  &  &\\
4th &+4 		&+4 &+4 &+1 	&-					&5 &4 &2nd\\
5th &+5 		&+4 &+4 &+1 	&Favored Enemy				&7 &5 &2nd\\
\multirow{2}{*}{6th} &\multirow{2}{*}{+6/+1} 		&\multirow{2}{*}{+5} &\multirow{2}{*}{+5} &\multirow{2}{*}{+2} 	&Bonus Feat, Improved	&\multirow{2}{*}{11} &\multirow{2}{*}{6} &\multirow{2}{*}{2nd}\\
    &			&   &	&	&Combat Style				&   &  &\\
7th &+7/+2 		&+5 &+5 &+2 	&-					&15 &7 &3rd\\
8th &+8/+3 		&+6 &+6 &+2 	&Camouflage				&19 &8 &3rd\\
9th &+9/+4 		&+6 &+6 &+3 	&Favored Enemy				&23 &9 &3rd\\
10th &+10/+5		&+7 &+7 &+3 	&Bonus Feat				&27 &10 &4th\\
11th &+11/+6/+1		&+7 &+7 &+3 	&Combat Style Mastery			&35 &11 &4th\\
12th &+12/+7/+2 	&+8 &+8 &+4 	&-					&43 &12 &4th\\
13th &+13/+8/+3 	&+8 &+8 &+4 	&Favored Enemy				&51 &13 &5th\\
14th &+14/+9/+4 	&+9 &+9 &+4 	&Bonus Feat				&59 &14 &5th\\
15th &+15/+10/+5	&+9 &+9 &+5 	&Hide in Plain Sight			&67 &15 &5th\\
16th &+16/+11/+6/+1 	&+10 &+10 &+5 	&-					&79 &16 &6th\\
17th &+17/+12/+7/+2 	&+10 &+10 &+5 	&Favored Enemy				&91 &17 &6th\\
18th &+18/+13/+8/+3 	&+11 &+11 &+6 	&Bonus Feat				&103 &18 &6th\\
19th &+19/+14/+9/+4 	&+11 &+11 &+6 	&-					&115 &19 &6th\\
20th &+20/+15/+10/+5	&+12 &+12 &+6 	&Bonus Feat				&127 &20 &6th\\
\hline
\end{tabular}}
\end{table*}
\subsubsection{Class Features}
All the following are class features of the Ranger.

\paragraph{Weapon and Armor Proficiency:} 
A Ranger is proficient with all simple and martial weapons, and with light armor and shields (except tower shields).

\paragraph{Spell Points/Day:} A Ranger's ability to cast spells is limited by the spell points he has available. 
His base daily allotment of spell points is given on \nameref{tab:Ranger} table. 
In addition, he receives bonus spell points per day if he has a high Wisdom score.
His race may also provide bonus spell points per day, as may certain feats and items.

\paragraph{Spells Known:} A Ranger begins play knowing one Ranger spell of your choice. 
Each time he achieves a new level, he unlocks the knowledge of a new spell.
Choose the spells known from the Ranger spell list
(Exception: The feats Expanded Knowledge and Epic Expanded Knowledge do allow a Ranger to learn spells of other classes, even specialist Wizard spells.).
A Ranger can cast any spell he knows that has a spell point cost equal to or lower than his caster level.
The number of times a Ranger can cast spells in a day is limited only by his daily spell points. 
A Ranger simply knows his spells; they are ingrained in his mind, though he must get a good night's sleep each day to regain all his spent spell points.
The Difficulty Class for saving throws against Ranger spells is 10 + one-half the number of spell points spent on the spell (round up) + the Ranger's Wisdom modifier. 

Spells learned via the Ranger class are divine spells.
\paragraph{Maximum Spell Level Known:} A Ranger begins play with the ability to learn 1st-level spells. 
As he attains higher levels, a Ranger may gain the ability to master more complex spells, as shown on the \nameref{tab:Ranger} table.
To learn or cast a spell, a Ranger must have a Wisdom score of at least 10 + the spell's level.

\paragraph{Combat Style (Ex):}
A Ranger must select one of two combat styles to pursue: archery or two-weapon combat. This choice affects the character's class features but does not restrict his selection of feats or special abilities in any way.

If the Ranger selects archery, he can add the lower of his Ranger level and his Dexterity modifier to ranged weapon damage rolls (but not damage caused by ranged spells, spell-like abilities, or supernatural abilities) in place of his strength modifier, if doing so is advantageous to him. 

If the Ranger selects two-weapon combat, he may ignore the dexterity requirement of the Two-Weapon Fighting Feat and any feat with that feat as a prerequisite, as well as the Dexterity requirement of the Multiweapon Fighting feat. He must still meet the feats' other prerequisites, if any.
\paragraph{Favored Enemy (Ex):}
At 1st level, a Ranger may select a type of creature from among those given on the Ranger Favored Enemies table. %\nameref{tab:RangerFE} table.

The Ranger gains a +2 bonus on Bluff, Listen, Sense Motive, Spot, and Survival checks when using these skills against creatures of this type. 
Likewise, he deals an extra 1d6 points of damage against such creatures. 
This extra damage applies only to weapon damage rolls, not to damage caused by spells, spell-like abilities, or supernatural abilities.
It applies fully even against creatures that are immune to critical hits or have concealment.

At 5th level and again at Ranger levels 9th, 13th and 17th, the Ranger may select an additional favored enemy from those given on the table. 
In addition, at each such interval, the bonus on skill checks against all your favored enemies (including the one just selected) increases by 2, and the bonus damage increases by one die (d6).

If the Ranger chooses humanoids or outsiders as a favored enemy, he must also choose an associated subtype, as indicated on the table. 
If a specific creature falls into more than one category of favored enemy, the Ranger's bonuses do not stack; he simply uses whichever bonus is higher.

\begin{center}
\small
\begin{tabular}{|p{3.2cm}|p{3.2cm}|}
\multicolumn{2}{c}{\textbf{\normalsize Ranger Favored Enemies}}\\
\hline
Type (Subtype)&Type (Subtype)\\
\hline
Aberration&Humanoid(reptilian)\\
Animal&Magical Beast\\
Construct&Monstrous Humanoid\\
Dragon&Ooze\\
Elemental&Outsider (air)\\
Fey&Outsider (chaotic)\\
Giant&Outsider (earth)\\
Humanoid (aquatic)&Outsider (evil)\\
Humanoid (dwarf)&Outsider (fire)\\
Humanoid (elf)&Outsider (good)\\
Humanoid (goblinoid)&Outsider (lawful)\\
Humanoid (gnoll)&Outsider (native)\\
Humanoid (gnome)&Outsider (water)\\
Humanoid (halfling)&Plant\\
Humanoid (human)&Undead\\
Humanoid (orc)&Vermin\\
\hline
\end{tabular}
\end{center}
\subparagraph[Wild Empathy]{Wild Empathy (Ex):}
\label{sec:WildEmpathy}
You can improve the attitude of an animal.
This ability functions just like a Diplomacy check made to improve the attitude of a person. 
You roll 1d20 and add your Ranger level and your Charisma modifier to determine the wild empathy check result.
The typical domestic animal has a starting attitude of indifferent, while wild animals are usually unfriendly.
To use wild empathy, the animal and you must be able to study each other, 
which means that you must be within 30 feet of each other under normal conditions. 
Generally, influencing an animal in this way takes 1 minute but, as with influencing people, it might take more or less time.
You can also use this ability to influence a magical beast with an Intelligence score of 1 or 2, but she takes a -4 penalty on the check.
\paragraph{Bonus Feats:}
At 2nd level, a Ranger gets a bonus feat.
He gains an additional bonus feat at Ranger levels 6th, 10th, 14th, 14th, 18th, and 20th. 
These bonus feats must be drawn from the feats noted as fighter bonus feats, or from the following list: \emph{Endurance}, \emph{Track}. 
The Ranger must still meet all prerequisites for the bonus feat, including ability score and base attack bonus minimums as well as class requirements. 
A Ranger cannot choose feats that specifically require levels in the fighter class unless he is a multiclass character with the requisite levels in the fighter class.

These bonus feats are in addition to the feats that a character of any class gains every three levels. 
A Ranger is not limited to fighter bonus feats and the aforementioned list of feats when choosing these other feats.
\paragraph{Intuition (Ex):}
A Ranger more commonly relies on personal insight than on book-learning or social techniques.
Starting at 3rd level, a Ranger may apply his Wisdom modifier to any of the following skills in place of the normally associated ability modifier, if doing so is advantageous to him:
Gather Information, Handle Animal, Knowledge (dungeoneering), Knowledge (geography), Knowledge (local), Knowledge (nature), and Search.

\paragraph{Pass without Trace (Ex):}
You can move through any type of terrain and leave neither footprints nor scent. Tracking you is impossible. You may choose to leave a trail if so desired.

\paragraph{Improved Combat Style (Ex):}
At 6th level, a Ranger's aptitude in his chosen combat style (archery or two-weapon combat) improves.

If he selected archery at 1st level, he gains the ability to snipe targets far away with unusual proficiency.
To do so, he makes a single ranged weapon attack against a target 30' or more away as a full-round action.
This attack, if it hits, is then automatically a critical threat, and it gains a +4 bonus on the damage roll for every additional attack the Ranger would have been entitled to due to a high Base Attack Bonus. If the critical threat is confirmed, this bonus damage is multiplied as normal. 
For example, an 11th level Ranger making use of this option would gain a +8 bonus on the damage roll due to the two additional attacks he gives up.

If he selected two-weapon combat at 1st level, he gains the ability to strike with both weapons as fast as he could strike with just one.
Whenever he makes an attack of opportunity, or attacks as a standard action, he may choose to attack with both of his weapons instead of just one.
Make the choice whether you also want to use your off hand before making the first attack roll. If you do, both attacks take the standard penalties for fighting with two weapons.
Attacking this way as a standard action is a specific action of its own, it is incompatible with other special attacks that are activated as a standard action, such as the Awesome Blow feat.

\paragraph{Camouflage (Ex):}
A Ranger of 8th level or higher can use the Hide skill in any sort of terrain, even if the terrain doesn't grant cover or concealment.

\paragraph{Combat Style Mastery (Ex):}
At 11th level, a Ranger's aptitude in his chosen combat style (archery or two-weapon combat) improves again. 

If he selected archery at 1st level, he gains the ability to ignore the effects of any kind of severe wind on his ranged attacks, including magical wind such as the one generated by the \nameref{Spell:WindWall} spell.

If the ranger selected two-weapon combat at 1st level, he gains the ability to rend with his weapons.
If he hits an opponent with a weapon in each hand in the same round, he automatically deals and additional points of damage equal to 2d6+1 $1/2$ times your Strength modifier. You can only rend once per round, regardless of how many successful attacks you make.
\paragraph{Hide in Plain Sight (Ex):}
A ranger of 15th level or higher can use the Hide skill even while being observed.

% \paragraph{Combat Style Supremacy (Ex):}
% At 19th level, a Ranger's aptitude in his chosen combat style (archery or two-weapon combat) reaches its pinnacle. 
% 
% If he selected archery at 1st level, STUFF
% 
% If the ranger selected two-weapon combat at 1st level, STUFF





 
\subsection[Sorcerer]{The Sorcerer}
\label{sec:Sorcerer}
\begin{quote}
\emph{It's quite simple, really. I say magic words, and magic happens.}
- Nora, human Sorceress
\end{quote}
A Sorcerer is an user of arcane magic that is born, not made.
\paragraph{Alignment:} Any
\paragraph{Hit Die:} d4
\paragraph{Class skills:}
The sorcerer's class skills (and the key ability for each skill) are 
Bluff (Cha), Concentration (Con), Craft (Int), Diplomacy (Cha), Disguise (Cha), Gather Information (Cha), Knowledge (arcana) (Int), Knowledge (the planes) (Int), Profession (Wis), Sense Motive (Wis), Speak Language (N/A) and Spellcraft (Int). 

\paragraph{Skill Points at 1st Level:} (4 + Int modifier) $\times$ 4.
\paragraph{Skill Points at each additional Level:} 4 + Int modifier.
\begin{table*}
\centering
\caption{The Sorcerer}
\label{tab:Sorcerer}
\makebox[\textwidth]{
\begin{tabular}{llccclccc}
\toprule
	&	&	&	&	&				&\multicolumn{3}{c}{Spellcasting}\\ \cmidrule(r){7-9}
Level	&BAB	&Fort 	&Ref 	&Will 	&Special			&SP/day	&Known&Max level\\
\midrule
1st	&+0	&+0	&+0	&+2	&Cantrips, Magical Intuition, 	&3	&2	&1st\\
	&	&	&	&	&Wild Magic +1			&	&	&\\
2nd	&+1	&+0	&+0	&+3	&-				&7	&3	&1st\\
3rd	&+1	&+1	&+1	&+3	&Wild Magic +2			&13	&4	&2nd\\
4th	&+2	&+1	&+1	&+4	&-				&21	&5	&2nd\\
5th	&+2	&+1	&+1	&+4	&Bonus feat			&31	&6	&3rd\\
6th	&+3	&+2	&+2	&+5	&-				&43	&7	&3rd\\
7th	&+3	&+2	&+2	&+5	&Wild Magic +3			&57	&8	&4th\\
8th	&+4	&+2	&+2	&+6	&-				&73	&9	&4th\\
9th	&+4	&+3	&+3	&+6	&-				&91	&10	&5th\\
10th	&+5	&+3	&+3	&+7	&Bonus feat			&111	&11	&5th\\
11th	&+5	&+3	&+3	&+7	&Wild Magic +4			&133	&12	&6th\\
12th	&+6/+1	&+4	&+4	&+8	&-				&157	&13	&6th\\
13th	&+6/+1	&+4	&+4	&+8	&-				&183	&14	&7th\\
14th	&+7/+2	&+4	&+4	&+9	&-				&211	&15	&7th\\
15th	&+7/+2	&+5	&+5	&+9	&Bonus feat, Wild Magic +5	&241	&16	&8th\\
16th	&+8/+3	&+5	&+5	&+10	&-				&273	&17	&8th\\
17th	&+8/+3	&+5	&+5	&+10	&-				&307	&18	&9th\\
18th	&+9/+4	&+6	&+6	&+11	&-				&343	&19	&9th\\
19th	&+9/+4	&+6	&+6	&+11	&Wild Magic +6			&381	&20	&9th\\
20th	&+10/+5	&+6	&+6	&+12	&Bonus feat			&421	&21	&9th\\
\bottomrule
\end{tabular}}
\end{table*}
\subsubsection{Class Features}
All the following are class features of the Sorcerer.

\paragraph{Weapon and Armor Proficiency:} Sorcerers are proficient with all simple weapons. 
They are not proficient with any type of armor or shield. 
Armor does not, however, interfere with the casting of spells.

\paragraph{Spell Points/Day:} 
A Sorcerer's ability to cast spells is limited by the spell points he has available. 
His base daily allotment of spell points is given on \nameref{tab:Sorcerer} table. 
In addition, he receives bonus spell points per day if he has a high Charisma score.
His race may also provide bonus spell points per day, as may certain feats and items.

\paragraph{Spells Known:} A Sorcerer begins play knowing two Sorcerer spells of your choice. 
Each time he achieves a new level, he unlocks the knowledge of new spells.
Choose the spells known from the full Sorcerer spell list.
(Exceptions: See the Magical Intuition class feature.
In addition, the feats Expanded Knowledge and Epic Expanded Knowledge 
do allow a Sorcerer to learn spells of other classes, 
including spells restricted to specialist Wizards.) 

A Sorcerer can cast any spell he knows that has a spell point cost equal to or lower than his caster level.
The number of times a Sorcerer can cast spells in a day is limited only by his daily spell points. 
A Sorcerer simply knows his spells; they are ingrained in his mind, 
though he must get a good night's sleep each day to regain all his spent spell points.
The Difficulty Class for saving throws against Sorcerer spells is 10 + one-half the number of spell points spent on the spell (round up) + the Sorcerer's Charisma modifier. 

Spells learned via the Sorcerer class are arcane spells.
\paragraph{Maximum Spell Level Known:} A Sorcerer begins play with the ability to learn 1st-level spells. 
As he attains higher levels, 
a Sorcerer may gain the ability to master more complex spells, as shown on \nameref{tab:Sorcerer} table.
To learn or cast a spell, a Sorcerer must have a Charisma score of at least 10 + the spell's level.

\paragraph{Bonus Feats:} 
A Sorcerer gains a bonus feat at 5th level, 10th level, 15th level, and 20th level. 
This feat must be a magical feat, a metamagic feat, or the Familiar feat.
These bonus feats are in addition to the feats that a character of any class gains every three levels. 

\paragraph[Cantrips]{Cantrips (Su):} 
A Sorcerer can use \nameref{sec:Cantrips} as a \nameref{sec:Wizard} can.

\paragraph{Magical Intuition:} 
A Sorcerer can select spells normally restricted to specialist Wizards as their spells known,
subject to the restriction that the number of specialist spells so selected may not exceed one-third of the Sorcerer's total number of spells known.

Example: A first level Sorcerer knows two spells. He could not select a spell restricted to specialist Wizards as one of those
two spells, since that would mean one-half of his spells consists of specialist Wizard spells.
However, he could select a specialist Wizard spell as his spell known when he reaches second level.
When he reaches 5th level (and thereby learns his 6th spell), he could learn a second specialist Wizard spell.

Spells known granted by the \nameref{Feat:ExpandedKnowledge} feat or from levels in other classes are not included - 
they do not count towards the limit of no more than one-third of the Sorcerer's spell repertoire being specialist Wizard spells,
and they do not count when determining the Sorcerer's number of spells known for this purpose.

This class feature is considered a part of the Sorcerer's Spells Known class feature. Sorcerers continue to benefit from it if their Spells Known are progressed via a prestige class.

\paragraph{Wild Magic:} A Sorcerer's intuitive understanding of magic allows him to make use of the universe's
own powerful magical eddies in ways that other spellcasters can't.
When casting a spell, a Sorcerer can choose to add +1 to the spell's save DC 
and gain a +1 bonus on the caster level check to overcome the subject's spell resistance.
(Spells that do not offer saving throws or spell resistance gain no benefit from Wild Magic.)

At 3rd level, a Sorcerer can choose to gain up to a +2 bonus on the spell's save DC and caster level check. 
At 7th level, he can gain a +3 bonus; at 11th level, a +4 bonus; at 15th level, a +5 bonus; and at 19th level, a +6 bonus.

Relinquishing control over a spell in this manner is dangerous, however.
For each +1 added to the spell's save DC and roll to overcome the subject's spell resistance, 
there is a cumulative 5\% chance of the Sorcerer suffering a magical backlash.
A Sorcerer hit by a magical backlash is \emph{staggered} for one round, and loses a number of spell points
equal to twice the bonus gained on the spell's save DC and caster level check.

For example, a 10th-level Sorcerer could choose to add +2 to the save DC and roll to overcome spell resistance of a
\nameref{Spell:Fireball} spell he is casting. He would then have a 10\% chance of suffering the magical backlash.
If the backlash strikes, he loses 4 spell points and is \emph{staggered} for one round. 
\subsection[Wizard]{The Wizard}
\label{sec:Wizard}
\begin{quote}
\emph{Behold your fate, creatures of darkness! 
Your demise is at hand, for I wield arcane power beyond your feeble goblin reasoning! 
The forces of the very cosmos are mine to command, and yet still you cannot comprehend the dark dismal end in store for you and your wicked compatriots. 
Nay! Your little brains can only leave you gasping in horror as I bend reality to my very will. 
The magic I wield is capable of rending asunder the universe--nay, the whole of the multiverse, and in fact is wasted on such pitiful creatures as yourself. 
But I shall bring it to bear nonetheless, and you shall rue the day I chose to wreak such unimaginable havoc on your lives with the sheer volume of my arcane works. 
And lo, in days and years to come, when the children come to play in the smoking crater that once held your den of evil, 
they shall know nothing of your wicked ways but all shall feel the echoes of the power spent here today.}
- \href{http://www.giantitp.com/comics/oots0010.html}{Vaarsuvius}, elven Wizard
\end{quote}
A Wizard is a learned user of arcane magic.
\paragraph{Alignment:} Any
\paragraph{Hit Die:} d4
\paragraph{Class skills:}
The Wizard's class skills (and the key ability for each skill) are Concentration (Con), 
Knowledge (all skills, taken individually) (Int), Profession (Wis), and Spellcraft (Int). 
In addition, a Wizard gains access to additional class skills based on his specialization (see below):
\subparagraph{Abjurer:} Diplomacy (Cha), Heal (Wis), Speak Language (N/A), and Survival (Wis)
\subparagraph{Conjurer:} Appraise (int), Craft (Int), Disable Device (Int), and Forgery (Int)
\subparagraph{Diviner:} Decipher Script (Int), Gather Information (Cha), Listen (Wis), and Spot (Wis).
\subparagraph{Enchanter:} Bluff (Cha), Diplomacy (Cha), Gather Information (Cha) and Sense Motive (Wis).
\subparagraph{Evoker:} Autohypnosis (Wis), Disable Device (Int), Intimidate (Cha) and Tumble (Dex).
\subparagraph{Illusionist:} Disguise (Cha), Forgery (Int), Hide (Dex), and Move Silently (Dex)
\subparagraph{Necromancer:} Bluff (Cha), Disguise (Cha), Heal (Wis), and Sense Motive (Wis).
\subparagraph{Transmuter:} Balance (Dex), Climb (Str), Jump (Str) and Swim (Str).

\paragraph{Skill Points at 1st Level:} (2 + Int modifier) $\times$ 4.
\paragraph{Skill Points at each additional Level:} 2 + Int modifier.
\begin{table*}
\centering
\caption{The Wizard}
\label{tab:Wizard}
\makebox[\textwidth]{
\begin{tabular}{|l|l|c|c|c|l|c|c|c|}
\hline
\multirow{2}{*}{\textbf{Level}}&\multirow{2}{*}{\textbf{BAB}}&\textbf{Fort}&\textbf{Ref}&\textbf{Will}&\multirow{2}{*}{\textbf{Special}}&\multirow{2}{*}{\textbf{SP/day}}&\textbf{Spells}&\textbf{Max}\\
&&\textbf{save}&\textbf{save}&\textbf{save}&&&\textbf{known}&\textbf{level}\\
\hline
\multirow{2}{*}{1st}	&\multirow{2}{*}{+0}	&\multirow{2}{*}{+0}	&\multirow{2}{*}{+0}	&\multirow{2}{*}{+2}	&Bonus feat, cantrips, 			&\multirow{2}{*}{3}	&\multirow{2}{*}{2}	&\multirow{2}{*}{1st}\\
	&	&	&	&	&specialization				&	&	&\\
2nd	&+1	&+0	&+0	&+3	&-					&6	&4	&1st\\
3rd	&+1	&+1	&+1	&+3	&-					&10	&6	&2nd\\
4th	&+2	&+1	&+1	&+4	&-					&16	&7	&2nd\\
5th	&+2	&+1	&+1	&+4	&Bonus feat				&24	&8	&3rd\\
6th	&+3	&+2	&+2	&+5	&-					&33	&10	&3rd\\
7th	&+3	&+2	&+2	&+5	&-					&43	&11	&4th\\
8th	&+4	&+2	&+2	&+6	&-					&55	&12	&4th\\
9th	&+4	&+3	&+3	&+6	&-					&69	&14	&5th\\
10th	&+5	&+3	&+3	&+7	&Bonus feat				&84	&15	&5th\\
11th	&+5	&+3	&+3	&+7	&-					&100	&16	&6th\\
12th	&+6/+1	&+4	&+4	&+8	&-					&118	&18	&6th\\
13th	&+6/+1	&+4	&+4	&+8	&-					&138	&19	&7th\\
14th	&+7/+2	&+4	&+4	&+9	&-					&159	&20	&7th\\
15th	&+7/+2	&+5	&+5	&+9	&Bonus feat				&181	&22	&8th\\
16th	&+8/+3	&+5	&+5	&+10	&-					&205	&23	&8th\\
17th	&+8/+3	&+5	&+5	&+10	&-					&231	&24	&9th\\
18th	&+9/+4	&+6	&+6	&+11	&-					&258	&26	&9th\\
19th	&+9/+4	&+6	&+6	&+11	&-					&286	&27	&9th\\
20th	&+10/+5	&+6	&+6	&+12	&Bonus feat				&316	&28	&9th\\
\hline
\end{tabular}}
\end{table*}
\subsubsection{Class Features}
All the following are class features of the Wizard.

\paragraph{Weapon and Armor Proficiency:} Wizards are proficient with the club, dagger, heavy crossbow, light crossbow, quarterstaff, and shortspear. 
They are not proficient with any type of armor or shield. Armor does not, however, interfere with the casting of spells.

\paragraph{Spell Points/Day:} A Wizard's ability to cast spells is limited by the spell points he has available. 
His base daily allotment of spell points is given on \nameref{tab:Wizard} table. 
 In addition, he receives bonus spell points per day if he has a high Intelligence score.
%\footnote{\textbf{Variant: Sorcerer} 
% 
% Some users of magic are not the careful, studious folk that populate most mage guilds - they are simply born.
% These mages are called Sorcerers. Their magic comes intuitively, and thus they use their Charisma score rather than their Intelligence score
% to determine their spells' saving throw DCs, their bonus spell points, and the maximum spell level they have access to.
% 
% Since Sorcerers can devote their time to study things other than magic, they receive (4 + Int modifier) $\times$ 4 skill points at first level, 
% and 4 + Int modifier skill points at each level thereafter, as well as proficiency with all simple weapons.}.
His race may also provide bonus spell points per day, as may certain feats and items.

\paragraph{Specialization:} Every Wizard must decide at 1st level which school of magic he will specialize in. 
Choosing a specialization provides a Wizard with access to the class skills associated with that school (see above), as well as the spells restricted to that school. 
However, choosing a discipline also means that the Wizard cannot learn spells that are restricted to other schools. 
He can't even use such spells by employing magical items.

\paragraph{Spells Known:} A Wizard begins play knowing two Wizard spells of your choice. 
Each time he achieves a new level, he unlocks the knowledge of new spells.
Choose the spells known from the Sorcerer/Wizard spell list, and/or from the list of spells only available to Wizards of his specialization. He may not learn spells from the spell list of another specialization. (Exception: The feats Expanded Knowledge and Epic Expanded Knowledge do allow a Wizard to learn spells of other spell lists, including spell lists of other specialist wizards, and even the lists belonging to other classes.) 
A Wizard can cast any spell he knows that has a spell point cost equal to or lower than his caster level. A Wizard whose spells known have been progressed by a prestige class can continue to select spells of his specialization.
The number of times a Wizard can cast spells in a day is limited only by his daily spell points. 
A Wizard simply knows his spells; they are ingrained in his mind, though he must get a good night's sleep each day to regain all his spent spell points.
The Difficulty Class for saving throws against Wizard spells is 10 + one-half the number of spell points spent on the spell (round up) + the Wizard's Intelligence modifier. 

Spells learned via the Wizard class are arcane spells.
\paragraph{Maximum Spell Level Known:} A Wizard begins play with the ability to learn 1st-level spells. 
As he attains higher levels, a Wizard may gain the ability to master more complex spells, as shown on \nameref{tab:Wizard} table.
To learn or cast a spell, a Wizard must have an Intelligence score of at least 10 + the spell's level.
\paragraph{Bonus Feats:} A Wizard gains a bonus feat at 1st level, 5th level, 10th level, 15th level, and 20th level. 
This feat must be a magical feat, a metamagic feat, an item creation feat or the Familiar feat.
These bonus feats are in addition to the feats that a character of any class gains every three levels. 

\paragraph[Cantrips]{Cantrips (Su):}
\label{sec:Cantrips}
Cantrips are minor tricks that novice spellcasters use for practice.
Using a cantrip requires no expenditure of spell points.
You can perform one as a standard action whenever you are magically focused.
This does not expend your focus.

The cantrips you can perform are:
\begin{list}{\labelitemi}{\leftmargin=1em}
 \item Lifting and moving up to 5 pounds of items from a distance.
This requires concentrating on the cantrip (a standard action).
You can manipulate the moved items as if you were using one hand for the task.
 \item Coloring, cleaning, or soiling items in a 1-foot cube.
 \item Chilling, warming, or flavoring 1 pound of nonliving material. 
 \item Lighting an unattended object within 30' on fire.
 \item Dimly illuminating a 5-foot radius around you, like a candle. 
The light emitted can be of any color, and usually appears as a small globe hovering near you. 
 \item Dealing 1d3 points of cold, electricity, fire or acid damage to a target within 30' by succeeding on a ranged touch attack against it.
 \item Creating small objects that look crude and artificial. 
The materials created by a cantrip are extremely fragile, and they cannot be used as tools or weapons. 
 \item Inscribing your personal rune or mark, which can consist of no more than six characters. 
The writing can be visible or invisible. 
You etch the rune upon any substance without harm to the material upon which it is placed. 
If an invisible mark is made, a \nameref{Spell:DetectMagic} spell causes it to glow and be visible.
\nameref{Spell:SeeInvisibility}, \nameref{Spell:TrueSeeing}, a gem of seeing, or a robe of eyes likewise allows the user to see an invisible mark of this kind. 
Unlike other cantrips, this one is permanent (but is removable by either a \nameref{Spell:DispelMagic} spell or a rigorous mundane cleaning).
 \item Magically deciphering magical inscriptions on objects - books, \nameref{Item:Scrolls}, weapons, and the like - that would otherwise be unintelligible.
 \item Increasing your reading speed to 250 pages per minute.
\end{list}

Cantrips lack the power to duplicate any other spell effects.
They cannot inflict status conditions, or affect the concentration of spellcasters.

Any actual change to an object (beyond just moving, cleaning, or soiling it, or creating a personal mark) persists only 1 hour.
\subsubsection{Magical Schools}
A school is one of eight groupings of spells, each defined by a common theme. 
The eight schools are Abjuration, Conjuration, Divination, Enchantment, Evocation, Illusion, Necromancy and Transmutation. 
The schools are described in detail in section \ref{sec:MagicalSchools}, and summarized below.

\paragraph{Abjuration:} 
Spells that protect, block, or banish. An abjuration specialist is called an abjurer.
\paragraph{Conjuration:}
Spells that bring creatures or materials to the caster. A conjuration specialist is called a conjurer.
\paragraph{Divination:}
Spells that reveal information. A divination specialist is called a diviner.
\paragraph{Enchantment:} 
Spells that imbue the recipient with some property or grant the caster power over another being. An enchantment specialist is called an enchanter.
\paragraph{Evocation:}
Spells that manipulate energy or create something from nothing. An evocation specialist is called an evoker.
\paragraph{Illusion:}
Spells that alter perception or create false images. An illusion specialist is called an illusionist.
\paragraph{Necromancy:} 
Spells that manipulate, create, or destroy life or life force. A necromancy specialist is called a necromancer.
\paragraph{Transmutation:}
Spells that transform the recipient physically or change its properties in a more subtle way. A transmutation specialist is called a transmuter. \newpage
\section{Prestige Classes}
\subsection{Arcane Archer}
\begin{quote}
\emph{``P'TwAnG!''}
- Sambo, drow Arcane Archer
\end{quote}
An Arcane Archer is an elf who has learned to combine two of the iconic elven disciplines - archery and arcane magic.
\paragraph{Hit Die:} d8
\paragraph{Requirements:}
To qualify to become an Arcane Archer, a character must fulfill all the following criteria.
\subparagraph{Race:}
Elf or half-elf.
\subparagraph{Feats:}
Point Blank Shot, Precise Shot, Weapon Focus (longbow or shortbow).
\subparagraph{Spells:}
Ability to cast \nameref{Spell:MagicWeapon} as an Arcane Spell.
\subparagraph{Special:}
Base Attack Bonus +6 OR Knowledge (arcana) 9 ranks.
\paragraph{Class Skills}
The Arcane Archer's class skills (and the key ability for each skill) are Concentration (Con), Craft (Int), Escape Artist (Dex), Hide (Dex), Jump (Str), Knowledge (arcana) (Int), Listen (Wis), Move Silently (Dex), Ride (Dex), Spot (Wis), Survival (Wis), and Use Rope (Dex).
\paragraph{Skill Points at Each Level}
4 + Int modifier.

\begin{table*}
\centering
\caption{The Arcane Archer}
\label{tab:ArcaneArcher}
\makebox[\textwidth]{
\begin{tabular}{|l|l|c|c|c|l|l|}
\hline
\multirow{2}{*}{\textbf{Level}}&\multirow{2}{*}{\textbf{BAB}}&\textbf{Fort}&\textbf{Ref}&\textbf{Will}&\multirow{2}{*}{\textbf{Special}}&\textbf{Spell points/day and}\\
&&\textbf{save}&\textbf{save}&\textbf{save}&&\textbf{max spell level}\\
\hline
1st	&+1	&+2	&+2	&+0	&Magic Arrows				&+1 level of existing class\\
2nd	&+2	&+3	&+3	&+0	&Advanced Learning, Imbue Arrow		&+1 level of existing class\\
3rd	&+3	&+3	&+3	&+1	&Enhance Arrows (Energy)		&+1 level of existing class\\
4th	&+4	&+4	&+4	&+1	&Advanced Learning			&+1 level of existing class\\
5th	&+5	&+4	&+4	&+1	&Enhance Arrows (Distance)		&+1 level of existing class\\
6th	&+6	&+5	&+5	&+2	&Advanced Learning			&+1 level of existing class\\
7th	&+7	&+5	&+5	&+2	&Enhance Arrows (Elemental Burst)	&+1 level of existing class\\
8th	&+8	&+6	&+6	&+2	&Advanced Learning			&+1 level of existing class\\
9th	&+9	&+6	&+6	&+3	&Enhance Arrows (Aligned)		&+1 level of existing class\\
10th	&+10	&+7	&+7	&+3	&Advanced Learning			&+1 level of existing class\\
\hline
\end{tabular}
}
\end{table*}

\subsubsection{Class Features}
All the following are Class Features of the Arcane Archer prestige class.

\paragraph{Weapon and Armor Proficiency:} Arcane Archers gain no proficiency with any weapon or armor.

\paragraph{Spellcasting:} When a new Arcane Archer level is gained, the character gains spell points per day, and an increase in caster level and maximum available spell level as if he had also gained a level in whatever arcane spellcasting class in which he could cast the \nameref{Spell:MagicWeapon} spell before he added the prestige class level. 
He does not, however, gain any other benefit a character of that class would have gained, nor does he gain spells known (but see the Advanced Learning class feature, below). 
If a character had more than one arcane spellcasting class in which he could cast \nameref{Spell:MagicWeapon} before he became a Arcane Archer, he must decide to which class he adds each level of Arcane Archer for the purpose of determining what spellcasting class gains the benefit of the spellcasting advancement.

\paragraph{Magic Arrows (Su):}
An arcane Archer never wants for arrows.
By making the motion of drawing a bowstring, an Arcane Archer can create a temporary arrow of pure magic.
These Magic Arrows cannot be stored or removed from the bow, they last only from the time the bow is drawn and until it is fired.
Magic Arrows bypass DR/magic, even if the bow does not have an enhancement bonus.
They otherwise function as normal arrows.

\paragraph{Advanced Learning:}
At 2nd level, and again at Arcane Archer levels 4th, 6th, 8th, and 10th, the Arcane Archer gains a bonus spell known, chosen from the following list: \nameref{Spell:ArrowOfDeath}, \nameref{Spell:BloodlettingArrow}, \nameref{Spell:HailOfArrows}, \nameref{Spell:ParalyzingArrow}, \nameref{Spell:PhaseArrow}, \nameref{Spell:SeekerArrow}, and \nameref{Spell:StunningArrow}.

When so learned, he adds the spell to his list of spells known from whatever arcane spellcasting class in which he could cast the \nameref{Spell:MagicWeapon} spell before he added the prestige class level. If more than one arcane spellcasting class would qualify, choose one. He must be able to cast spells of the appropriate level in the class before adding the spell. If he qualifies for none of the listed spells, he gains one immediately if he qualifies at some point in the future.

\paragraph{Imbue Arrow (Su):}
At 2nd level, an Arcane Archer gains the ability to place an area spell upon an arrow.
The spell may not have a casting time longer than one standard action, and may be of a level no higher than your Arcane Archer class level -1.
When the arrow is fired, the spell's area is centered on where the arrow lands, even if the spell could normally be centered only on the caster. If the arrow misses its intended target, the spell still goes off. Treat the arrow as a splash weapon that missed when determining the center of the spell's effect in such a case.

It takes a standard action to cast the spell and fire the arrow. 
The arrow must be fired in the round the spell is cast, or the spell is wasted.
Effects that allow you to fire more than one arrow as part of a single standard action work as normal, but only one of the arrows fired carries the spell, not all of them.
\paragraph{Enhance Arrows (Su):}
At 3rd level, every arrow fired by an Arcane Archer gains one of the following elemental themed weapon qualities: \emph{flaming}, \emph{frost}, or \emph{shock}. Unlike magic weapons created by normal means, the archer need not spend gold pieces to accomplish this task. However, an archer's magic arrows only function for him.

At 5th level, every arrow fired by an Arcane Archer gains the \emph{distance} weapon quality.

At 7th level, every arrow fired by an Arcane Archer gains one of the following elemental burst weapon qualities: \emph{flaming burst}, \emph{icy burst}, or \emph{shocking burst}. This ability replaces the ability gained at 3rd level.

At 9th level, every arrow fired by an Arcane Archer gains one of the following aligned weapon qualities: \emph{anarchic}, \emph{axiomatic}, \emph{holy} or \emph{unholy}. The Arcane Archer cannot choose an ability that is the opposite of his alignment (for example, a lawful good Arcane Archer could not choose anarchic or unholy as his weapon quality).

The bonuses granted by a magic bow (or magic arrows) apply as normal to arrows that have been enhanced with this ability. Duplicate abilities do not stack.





 
\subsection{Archmage}
\begin{quote}
\emph{I have survived many adventures, but my ongoing study of magic is truly the greatest of them all.}
- Thelonius, elven Archmage
\end{quote}
An Archmage is a powerful spellcaster who has begun to delve into the most fundamental workings of magic.
\paragraph{Hit Die:} d4
\paragraph{Requirements:}
To qualify to become an Archmage, a character must fulfill all the following criteria.
\subparagraph{Skills:} Knowledge (arcana) 15 ranks, Spellcraft 15 ranks.
\subparagraph{Feats:} Magical Endowment, Expanded Knowledge, Skill Focus (Spellcraft).
\subparagraph{Spells:} Ability to cast 7th-level arcane spells, knowledge of 4th-level or higher spells from at least five schools.
\paragraph{Class Skills}
The Archmage's class skills (and the key ability for each skill) are Concentration (Con), Craft (alchemy) (Int), 
Intimidate (Cha), Knowledge (all skills taken individually) (Int), Profession (Wis), Search (Int), and Spellcraft (Int). 
\paragraph{Skill Points at each level:} 2 + Int modifier.
\begin{table*}
\centering
\caption{The Archmage}
\label{tab:Archmage}
\makebox[\textwidth]{
%\makebox[\textwidth]{\resizebox{1.2\textwidth}{!}{
\begin{tabular}{|l|l|c|c|c|c|l|}
\hline
\textbf{Level}&\textbf{BAB}&\textbf{Fort}&\textbf{Ref}&\textbf{Will}&\textbf{Special}&\textbf{Spellcasting}\\
\hline
1st	&+0	&+0	&+0	&+2	&High Arcana	&+1 level of existing class\\
2nd	&+1	&+0	&+0	&+3	&High Arcana	&+1 level of existing class\\
3rd	&+1	&+1	&+1	&+3	&High Arcana	&+1 level of existing class\\
4th	&+2	&+1	&+1	&+4	&High Arcana	&+1 level of existing class\\
5th	&+2	&+1	&+1	&+4	&High Arcana	&+1 level of existing class\\
\hline
\end{tabular}
%}}
}
\end{table*}
\subsubsection{Class Features}
All the following are Class Features of the Archmage prestige class.

\paragraph{Weapon and Armor Proficiency:} Archmages gain no proficiency with any weapon or armor.

\paragraph{Spellcasting:} When a new Archmage level is gained, 
the character gains spell points per day, an increase in caster level, spells known and maximum available spell level
as if he had also gained a level in whatever arcane spellcasting class in which he could cast 7th-level spells before he added the prestige class level. 
He does not, however, gain any other benefit a character of that class would have gained. 
If a character had more than one arcane spellcasting class in which he could cast 7th-level spells before he became an Archmage, 
he must decide to which class he adds each level of Archmage for the purpose of determining what spellcasting class gains the benefit of the spellcasting advancement.

\paragraph{High Arcana:}
An Archmage gains the opportunity to select a special ability from among those described below by permanently eliminating a specific number of Spell Points.
These spell points are subtracted from the final number of spell points he Archmage would otherwise have.
Effectively, these spell points are spent on ``fueling'' the High Arcana.
When a High Arcana refers to an Archmage's ``caster level'', it means his highest caster level.

\subparagraph{Arcane Fire (Su):}
As a standard action, an Archmage may expend his magical focus to manifest a bolt of raw magical energy. 
The bolt is a ranged touch attack with long range (400 feet + 40 feet/caster level of the Archmage) 
that deals 1d6 points of damage per caster level of the Archmage, with no saving throw. 
Learning this High Arcana removes 15 spell points from the Archmage's pool.

\subparagraph{Arcane Reach (Su):}
The Archmage can use spells with a range of touch on a target up to 30 feet away. 
In the case of a spell that would ordinarily require a touch attack, the Archmage must make a ranged touch attack instead.
Arcane reach can be selected a second time as a High Arcana (paying the cost again), in which case the range increases to 60 feet. 
Learning this High Arcana removes 13 spell points from the Archmage's pool.

\subparagraph{Mastery of Counterspelling (Ex):}
When the Archmage counterspells a spell, it is turned back upon the caster as if it were fully affected by a \nameref{Spell:SpellTurning} spell. 
If the spell cannot be affected by spell turning, then it is merely counterspelled. 
Learning this High Arcana removes 13 spell points from the Archmage's pool.

\subparagraph{Mastery of Elements:}
The Archmage can alter an arcane spell when cast so that it utilizes a different element from the one it normally uses, 
even if the spell normally does not allow more than one element to be selected. 
This ability can only alter a spell with the acid, cold, fire, electricity, or sonic descriptor. 
All characteristics of the spell, except for its energy type descriptor and damage type remain unchanged.
The caster decides whether to alter the spell's energy type and chooses the new energy type when he begins casting. 
Learning this High Arcana removes 3 spell points from the Archmage's pool.

\subparagraph{Mastery of Shaping:}
The Archmage can alter area and effect spells that use one of the following shapes: burst, cone, cylinder, emanation, or spread. 
The alteration consists of creating spaces within the spell's area or effect that are not subject to the spell.
The minimum dimension for these spaces is a 5-foot cube. Furthermore, any shapeable spells have a minimum dimension of 5 feet instead of 10 feet. 
Learning this High Arcana removes 11 spell points from the Archmage's pool.

\subparagraph{Spell Power:}
This ability provides the Archmage with a +1 inherent bonus to one of his mental ability scores. 
Spell Power can be selected more than once as a High Arcana (paying the cost each time). 
Each additional time it is selected, he gains a +1 inherent bonus to another mental ability score, or one of his existing inherent bonuses to a mental ability score increases by one.
However, an inherent bonus may not exceed +5 for a single ability score.
Learning this High Arcana removes 9 spell points from the Archmage's pool.

\subparagraph{Spell-Like Ability (Sp):}
An Archmage who selects this High Arcana can use one of his spells known as a spell-like ability twice per day.
The caster level for this spell-like ability is equal to the Archmage's hit dice.
The activation of the spell-like ability requires the same action as casting the spell itself.
If this High Arcana is selected more than once, it can apply to the same spell chosen the first time (increasing the number of times per day it can be used) or to a different spell.
Learning this High Arcana removes a number of spell points equal to the minimum amount required to cast the spell in question from the Archmage's pool. 
%\subsection{Arcane Trickster}
\subsection{Assassin}
\begin{quote}
\emph{I am quite aware that the work I do isn't pretty. Take your moral preachings elsewhere.}
- Forbes, human Assassin
\end{quote}
An Assassin is a contract killer who uses stealth, spellcasting, and other supernatural abilities to make sure his victims die and stay dead.
\paragraph{Hit Die:} d6
\paragraph{Requirements:}
To qualify to become an Assassin, a character must fulfill all the following criteria.
\subparagraph{Alignment:} Any nongood.
\subparagraph{Skills:} Disguise 4 ranks, Hide 8 ranks, Move Silently 8 ranks.
% \item \textbf{Special:} The character must kill someone for no other reason than to join the Assassins.
\paragraph{Class Skills}
The Assassin's class skills (and the key ability for each skill) are Balance (Dex), Bluff (Cha), Climb (Str), Craft (Int), Decipher Script (Int), Diplomacy (Cha), Disable Device (Int), Disguise (Cha), Escape Artist (Dex), Forgery (Int), Gather Information (Cha), Hide (Dex), Intimidate (Cha), Jump (Str), Listen (Wis), Move Silently (Dex), Open Lock (Dex), Search (Int), Sense Motive (Wis), Sleight of Hand (Dex), Spot (Wis), Swim (Str), Tumble (Dex), Use Magic Device (Cha), and Use Rope (Dex).
\paragraph{Skill Points at each level:} 4 + Int modifier.
\begin{table*}
\centering
\caption{The Assassin}
\label{tab:Assassin}
\makebox[\textwidth]{
\begin{tabular}{|l|c|c|c|c|l|c|c|c|}
\hline
\multirow{2}{*}{\textbf{Level}}&\multirow{2}{*}{\textbf{BAB}}&\textbf{Fort}&\textbf{Ref}&\textbf{Will}&\multirow{2}{*}{\textbf{Special}}&\multirow{2}{*}{\textbf{SP/day}}&\textbf{Spells}&\textbf{Max}\\
&&\textbf{save}&\textbf{save}&\textbf{save}&&&\textbf{known}&\textbf{level}\\
\hline
\multirow{2}{*}{1st}	&\multirow{2}{*}{+0}	&\multirow{2}{*}{+0}	&\multirow{2}{*}{+2}	&\multirow{2}{*}{+0}	&Sneak attack +1d6, death 			&\multirow{2}{*}{1}	&\multirow{2}{*}{1}	&\multirow{2}{*}{1st}\\
	&	&	&	&	&attack, poison use				&	&	&\\
2nd	&+1	&+0	&+3	&+0	&Poison mastery					&3	&2	&1st\\
3rd	&+2	&+1	&+3	&+1	&Sneak attack +2d6				&6	&3	&2nd\\
4th	&+3	&+1	&+4	&+1	&Sure killer					&10	&4	&2nd\\
\multirow{2}{*}{5th}	&\multirow{2}{*}{+3}	&\multirow{2}{*}{+1}	&\multirow{2}{*}{+4}	&\multirow{2}{*}{+1}	&Improved uncanny dodge, 			&\multirow{2}{*}{15}	&\multirow{2}{*}{5}	&\multirow{2}{*}{3rd}\\
	&	&	&	&	&sneak attack +3d6				&	&	&\\
6th	&+4	&+2	&+5	&+2	&Venom immunity					&23	&6	&3rd\\
7th	&+5	&+2	&+5	&+2	&Sneak attack +4d6				&31	&7	&3rd\\
8th	&+6	&+2	&+6	&+2	&Hide in plain sight				&43	&8	&4th\\
9th	&+6	&+3	&+6	&+3	&Sneak attack +5d6				&55	&9	&4th\\
10th	&+7	&+3	&+7	&+3	&True death					&71	&10	&4th\\
\hline
\end{tabular}
}
\normalsize
\end{table*}
\subsubsection{Class Features}
All the following are Class Features of the Assassin prestige class.

\paragraph{Weapon and Armor Proficiency:}
Assassins are proficient with the crossbow (hand, light, or heavy), dagger (any type), dart, rapier, sap, shortbow (normal and composite), and short sword. 
Assassins are proficient with light armor but not with shields.

\paragraph{Spell Points/Day:} 
An Assassin's ability to cast spells is limited by the spell points he has available. 
His base daily allotment of spell points is given on \nameref{tab:Assassin} table. 
In addition, he receives bonus spell points per day if he has a high Intelligence score.
His race may also provide bonus spell points per day, as may certain feats and items.

\paragraph{Spells Known:} An Assassin begins play knowing an Assassin spell of your choice. 
Each time he achieves a new level, he unlocks the knowledge of new spells.
Choose the spells known from the full Assassin spell list.
(Exception: The feats Expanded Knowledge and Epic Expanded Knowledge 
do allow an Assassin to learn spells of other classes, 
including spells restricted to specialist Wizards.) 

An Assassin can cast any spell he knows that has a spell point cost equal to or lower than his caster level.
The number of times an Assassin can cast spells in a day is limited only by his daily spell points. 
An Assassin simply knows his spells; they are ingrained in his mind, 
though he must get a good night's sleep each day to regain all his spent spell points.
The Difficulty Class for saving throws against Assassin spells is 
10 + one-half the number of spell points spent on the spell (round up) + the Assassin's Intelligence modifier. 

Spells learned via the Assassin class are arcane spells.
\paragraph{Maximum Spell Level Known:} An Assassin begins play with the ability to learn 1st-level spells. 
As he attains higher levels, 
an Assassin may gain the ability to master more complex spells, as shown on \nameref{tab:Assassin} table.
To learn or cast a spell, an Assassin must have a Intelligence score of at least 10 + the spell's level.

\paragraph{Sneak Attack:}
This is exactly like the rogue ability of the same name. 
The extra damage dealt increases by +1d6 every other level (1st, 3rd, 5th, 7th, and 9th). 
If an Assassin gets a sneak attack bonus from another source the bonuses on damage stack.

\paragraph{Death Attack (Ex):}
If an Assassin studies his victim for 3 rounds and then makes a sneak attack with a weapon that successfully deals damage, 
the sneak attack has the additional effect of possibly either \emph{paralyzing} or killing the target (Assassin's choice). 
While studying the victim, the Assassin can undertake other actions so long as his attention stays focused on the target 
and the target does not detect the Assassin or recognize the Assassin as an enemy. 
If the victim of such an attack fails a Fortitude save (DC 10 + the Assassin's class level + the Assassin's Int modifier) 
against the kill effect, she dies. If the saving throw fails against the \emph{paralysis} effect, 
the victim is rendered \emph{helpless} and unable to act for 1d6 rounds plus 1 round per level of the Assassin. 
If the victim's saving throw succeeds, the attack is just a normal sneak attack. 
Once the Assassin has completed the 3 rounds of study, he must make the death attack within the next 3 rounds.

If a death attack is attempted and fails (the victim makes her save) 
or if the Assassin does not launch the attack within 3 rounds of completing the study, 
3 new rounds of study are required before he can attempt another death attack.

At the Assassin's option, he may expend his magical focus when he makes the death attack 
in order to make the attack count as a spell with the [death] descriptor for the purposes of spell interactions.
This makes it harder to raise the victim from the dead, but makes protections such as \nameref{Spell:DeathWard} effective against it.
\paragraph{Poison Use (Ex):}
Assassins are trained in the use of poison and never risk accidentally poisoning themselves when applying poison to a weapon or using a poisoned weapon.

\paragraph{Poison Mastery (Su):}
Starting at 2nd level, an Assassin can expend his magical focus when delivering poison in order to increase the poison's save DC by a number equal to $1/2$ his Assassin level.
The focus must be expended as part of the action that delivers the poison, for example when attacking with a poisoned weapon (but not when applying poison to the weapon),
throwing a vial of poisonous gas, or poisoning a drink.
\paragraph{Uncanny Dodge (Ex):}
Starting at 2nd level, an Assassin retains his Dexterity bonus to AC (if any) regardless of being caught \emph{flat-footed} or struck by an \emph{invisible} attacker. 
(He still loses any Dexterity bonus to AC if immobilized.)

If a character gains uncanny dodge from a second class the character automatically gains improved uncanny dodge (see below) instead.

\paragraph{Sure Killer (Ex):}
At 4th level, an Assassin can perform a coup de grace as a standard action that does not provoke attacks of opportunity.
His coup de graces ignore any immunity the target may have to sneak attacks or effects that require a fortitude save.
\paragraph{Improved Uncanny Dodge (Ex):}
At 5th level, an Assassin can no longer be flanked, 
since he can react to opponents on opposite sides of him as easily as he can react to a single attacker. 
This defense denies rogues the ability to use flank attacks to sneak attack the Assassin. 
The exception to this defense is that a rogue at least four levels higher than the Assassin can flank him (and thus sneak attack him).

If a character gains uncanny dodge (see above) from a second class the character automatically gains improved uncanny dodge, 
and the levels from those classes stack to determine the minimum rogue level required to flank the character.

\paragraph{Venom Immunity (Ex):}
At 6th level, an Assassin gains immunity to all poisons.

\paragraph{Hide in Plain Sight (Su):}
At 8th level, an Assassin can use the Hide skill even while being observed. 
As long as he is within 10 feet of some sort of shadow, an Assassin can hide himself from view in the open without having anything to actually hide behind. 
He cannot, however, hide in his own shadow.

\paragraph{True Death (Su):}
At 10th level, whenever an Assassin succeeds on a Death Attack, he may opt to disperse the subject's soul as part of making the attack.

Raising a creature from the dead whose soul has been dispersed requires casting a \nameref{Spell:Wish} or \nameref{Spell:Miracle} 
spell to retrieve the soul prior to casting the spell that should raise it.

\subsubsection{Assassin Spell list}
\paragraph{1st-Level Assassin Spells}
\begin{list}{\labelitemi}{\leftmargin=1em}
\item \nameref{Spell:AlignedProtection}: +2 to AC and saves, counter mind control, hedge out elementals and outsiders.
\item \nameref{Spell:Blackfire}*: Fires only shed light for you.
\item \nameref{Spell:ControlFall}: Objects or creatures fall slowly.
\item \nameref{Spell:DetectPoison}: Detects poison or disease in one creature or small object.
\item \nameref{Spell:DisguiseSelf}: Changes your appearance.
\item \nameref{Spell:Fog}: Fog surrounds you.
\item \nameref{Spell:PassWithoutTrace}: One subject/level leaves no tracks.
\item \nameref{Spell:TrueStrike}: +20 on your next attack roll.
\item \nameref{Spell:Ventriloquism}: Makes sounds appear out of nowhere.
\item \nameref{Spell:Sleep}: Puts 4 HD of creatures into magical slumber.
\end{list}
\paragraph{2nd-Level Assassin Spells}
\begin{list}{\labelitemi}{\leftmargin=1em}
\item \nameref{Spell:AlterSelf}: Perform minor physical changes on yourself.
\item \nameref{Spell:AnimalsMovement}: Grants additional movement capabilities.
\item \nameref{Spell:Darkness}: 20-ft. radius of supernatural shadow.
\item \nameref{Spell:Darkvision}: See 30 ft. in total darkness.
\item \nameref{Spell:Invisibility}: Subject is invisible for 1 min./level or until it attacks.
\item \nameref{Spell:MaskAlignment}: Protects subject's alignment from being revealed via divinations.
\item \nameref{Spell:Silence}: Negates sound in 20-ft. radius.
\item \nameref{Spell:WombatsBoost}: Subject gains +4 to an ability score for 1 min./level.
\end{list}
\paragraph{3rd-Level Assassin Spells}
\begin{list}{\labelitemi}{\leftmargin=1em}
\item \nameref{Spell:Blindsense}*: Subject can notice things it cannot see.
\item \nameref{Spell:Clairvoyance}: See and hear a distant location.
\item \nameref{Spell:FalseLife}: Gain 1d10 temporary hp.
\item \nameref{Spell:GaseousForm}: Subject becomes insubstantial and can fly slowly.
\item \nameref{Spell:Locate}: Senses direction toward object (specific or type).
\item \nameref{Spell:Nondetection}: Masks object or creature against scrying.
\end{list}
\paragraph{4th-Level Assassin Spells}
\begin{list}{\labelitemi}{\leftmargin=1em}
\item \nameref{Spell:ArcaneEye}: Invisible floating eye moves 30 ft./round.
\item \nameref{Spell:DimensionalAnchor}: Bars extradimensional movement.
\item \nameref{Spell:DimensionDoor}: Teleports you short distance.
\item \nameref{Spell:FreedomOfMovement}: Subject moves normally despite impediments.
\item \nameref{Spell:Glibness}: You gain a large bonus on Bluff checks.
\end{list}
 
\subsection{Blackguard}
\label{sec:Blackguard}
\begin{quote}
\emph{I was once like you. So full of hope. But the world doesn't reward hope, pup.}
- Torgar, human Blackguard
\end{quote}
A Blackguard is usually a Paladin who has not only fallen from grace, but become what he once fought. Whether they made a conscious choice to turn from the path of good or were slowly corrupted, Blackguards are some of the most vile foes a mortal will face.
\paragraph{Hit Die:} d10
\paragraph{Requirements:}
To qualify to become a Blackguard, a character must fulfill all the following criteria.
\subparagraph{Alignment:} Any evil.
\subparagraph{Base Attack Bonus:} +6
\subparagraph{Skills:} Intimidate 9 ranks, Knowledge (religion) 2 ranks.
\subparagraph{Feat:} Power Attack
\paragraph{Class Skills}
The Blackguard's class skills (and the key ability for each skill) are Bluff (Cha), Concentration (Con), Craft (Int), Disguise (Cha), Forgery (Cha), Gather Information (Cha), Intimidate (Cha), Hide (Dex), Knowledge (Arcana) (Int), Knowledge (local) (Int), Knowledge (religion) (Int), Knowledge (the planes) (Int), Move Silently (Dex), Profession (Wis), Ride (Dex), Sense Motive (Wis), and Spellcraft (Int).

\paragraph{Skill Points at each additional Level:} 4 + Int modifier. 
\begin{table*}
\centering
\caption{The Blackguard}
\label{tab:Blackguard}
\makebox[\textwidth]{
\begin{tabular}{|l|l|c|c|c|l|l|}
\hline
\textbf{Level}&\textbf{BAB}&\textbf{Fort}&\textbf{Ref}&\textbf{Will}&\textbf{Special}&\textbf{Spellcasting}\\
\hline
1st	&+1	&+2	&+0	&+0	&Aura of Evil, Dark Blessing	&+1 level of Paladin\\
2nd	&+2	&+3	&+0	&+0	&Unholy Gift			&+1 level of Paladin\\
3rd	&+3	&+3	&+1	&+1	&				&+1 level of Paladin\\
4th	&+4	&+4	&+1	&+1	&				&+1 level of Paladin\\
5th	&+5	&+4	&+1	&+1	&Unholy Gift			&+1 level of Paladin\\
6th	&+6	&+5	&+2	&+2	&				&+1 level of Paladin\\
7th	&+7	&+5	&+2	&+2	&				&+1 level of Paladin\\
8th	&+8	&+6	&+2	&+2	&Unholy Gift			&+1 level of Paladin\\
9th	&+9	&+6	&+3	&+3	&				&+1 level of Paladin\\
10th	&+10	&+7	&+3	&+3	&				&+1 level of Paladin\\
11th	&+11	&+7	&+3	&+3	&Unholy Gift			&+1 level of Paladin\\
12th	&+12	&+8	&+4	&+4	&				&+1 level of Paladin\\
13th	&+13	&+8	&+4	&+4	&				&+1 level of Paladin\\
14th	&+14	&+9	&+4	&+4	&Unholy Gift			&+1 level of Paladin\\
\hline
\end{tabular}
}
\end{table*}
\subsubsection{Class Features}
All the following are Class Features of the Blackguard prestige class.

\paragraph{Weapon and Armor Proficiency:} Blackguards are proficient with all simple and martial weapons, with all types of armor, and with shields.

\paragraph{Spellcasting:} When a new Blackguard level is gained, the character gains spell points per day, an increase in caster level, spells known and maximum available spell level as if he had also gained a level in the Paladin class. However, instead of using the Paladin spell list, a Blackguard uses the Blackguard spell list, and does not receive the \nameref{Spell:TouchOfVitality} spell as a bonus spell known. A character with no Paladin levels starts gaining spellcasting as a 1st level Paladin (but again, using the Blackguard spell list, and without the \nameref{Spell:TouchOfVitality} spell as a bonus spell known). See also the \nameref{sec:DarkBlessing} class feature, below.

\paragraph[Aura of Evil]{Aura of Evil: (Su)}
\label{sec:AuroOfEvil} 
At will, as a free action, a Blackguard can project an unholy aura.
While the aura is active, the Blackguard gains a +4 profane bonus on Intimidate checks, and a -4 penalty on Diplomacy checks.
All sapient creatures become instinctively aware of the Blackguard's evil alignment while the aura is active.
The Blackguard can project this aura indefinitely, or until he dismisses it (another free action). 

\paragraph[Dark Blessing]{Dark Blessing (Su):} 
\label{sec:DarkBlessing}
The alignment requirements of the Blackguard class are incompatible with those of the Paladin class, requiring a Paladin entrant to become an Ex-Paladin. However, the malignant powers that support Blackguards are fully capable of restoring a fallen Paladin's prowess. This has the following consequences:
\begin{list}{\labelitemi}{\leftmargin=1em}
 \item All spells a Paladin entrant knows as a Paladin must be switched out for Blackguard spells of the same level, except for those spells that appear on both the Paladin and Blackguard lists. The \nameref{Spell:TouchOfVitality} spell is lost, with no replacement.
 \item All Holy Gifts other than Additional Feats selected by a Paladin entrant must be switched out for Unholy Gifts (see below).
 \item A Paladin entrant's Smite and Divine Grace abilities are re-activated, using a sum of the character's Blackguard and Ex-Paladin levels to calculate the appropriate bonuses. A Blackguard's Divine Grace is usually referred to as Dark Grace.
\end{list}

\paragraph[Unholy Gift]{Unholy Gift:}
\label{sec:UnholyGift}
At 2nd level, and at every third level beyond that (5th, 8th, 11th, and 14th), the dark powers the Blackguard serves reward him with a gift of new abilities. Unless otherwise noted, a gift can only be selected once. Some gifts have specific prerequisites.

\subparagraph{Additional Feat:}
You gain a bonus feat from the list of feats noted as Fighter bonus feats.
This gift may be selected more than once.

\subparagraph{Aura of Despair (Su):}
A Blackguard with this gift radiates a malign aura that causes enemies within 10 feet of him to take a -2 penalty on all saving throws. This is a supernatural ability that functions continuously. You must be a 2nd-level Blackguard to select this gift.

\subparagraph{Brutal Smite (Su):}
Your contempt toward the weak augments your smites against them with dark power, making them especially horrifying.
When you perform a smite attack against a good creature, it must succeed on a Fortitude save (DC 10 + $1/2$ your HD + your Charisma modifier) or be \emph{nauseated} by pain for 1 round. You must have the \nameref{sec:Smite} ability and be a 2nd-level Blackguard to select this gift.

\subparagraph{Fiendish Mount}
\label{sec:FiendishMountListing}
You gain the service of a hellish animal. See the \nameref{sec:FiendishMount} creature description. You must be a 5th-level Blackguard to select this gift. If you have the \nameref{Feat:AnimalCompanion} feat, \nameref{sec:CelestialMountListing} ability, the \nameref{Feat:Familiar} feat, or the \nameref{Feat:SpellstaffUser} feat, you may not select this gift.

\subparagraph{Frightful Presence (Ex):}
The very presence of a Blackguard with this gift is unsettling to his foes. 
It takes effect automatically when you charge, make a full attack, or cast a spell. 
Opponents within 30 feet who witness the action are \emph{shaken} for 1 minute.
This ability affects only opponents with fewer Hit Dice or levels than you have. 
An affected opponent can resist the effects with a successful Will save (DC 10 + $1/2$ your HD + your Charisma modifier). 
An opponent that succeeds on the saving throw is immune to your frightful presence for 24 hours.

\subparagraph{Master of Fear (Su):}
A Blackguard with this gift is a master in every form of intimidation, capable of shaking even creatures to whom the feeling is alien. Your Fear spells and demoralize attempts can affect creatures normally immune to fear. The creatures' other defenses (such as saving throws) work normally. This is a supernatural ability activated as part of the activity in question.
You must be a 5th-level Blackguard to select this gift.

\subparagraph{No Mercy (Ex):}
You know how to strike when your opponent is at his most vulnerable. You may add your sneak attack damage when attacking a target that is \emph{prone}, even if the attack would not otherwise qualify for sneak attack. You must have the sneak attack ability to select this gift.

\subparagraph{Pale Knight (Ex):}
Though associating with the forces of darkness has taken a toll on your body, it has numbed you to some of their most debilitating attacks. A Blackguard with this gift gains immunity to energy drain and negative levels. 

\subparagraph{Poison Use (Ex):}
A Blackguard with this gift never risks accidentally poisoning himself when applying poison to a weapon or using a poisoned weapon.

\subparagraph{Sneak Attack:}
This gift functions as the Rogue class feature of the same name. 
The extra damage is +2d6. 
This qualifies as the Sneak Attack class feature for the purposes of meeting prerequisites.
If a Blackguard gets a sneak attack bonus from another source, the bonuses on damage stack.
This gift may be selected a total number of times equal to your character level divided by four, increasing the damage by +2d6 each additional time it is taken.

\subparagraph{Unholy Toughness (Ex):}
Being constantly suffused with dark energies has forced your body to develop unusual resilience. You gain damage reduction equal to one-half your Blackguard level, minimum 1 (if you have Ex-Paladin levels, use the sum of your Blackguard levels and Ex-Paladin levels instead). This damage reduction is overcome by good-aligned weapons.

\subsubsection{Ex-Blackguards:}
A Blackguard who changes to a nonevil alignment cannot advance further as a Blackguard, although he retains all Blackguard class features.

\subsubsection{Blackguard Spell List}
\label{sec:BlackguardSpellList}
\paragraph{1st-Level Blackguard Spells}
\begin{list}{\labelitemi}{\leftmargin=1em}
\item \nameref{Spell:AlignedProtection}: +2 to AC and saves, counter mind control, hedge out elementals and outsiders.
\item \nameref{Spell:Bane}: Enemies take -1 on attack rolls and saves against fear.% Instead of Bless
\item \nameref{Spell:Blackfire}*: Fires shed light for you only. %Instead of Zone of Truth
\item \nameref{Spell:CorruptWeapon}: Weapon bypasses the defenses of good foes. % Instead of Bless Weapon
\item \nameref{Spell:CureWounds}: Cures 1d8 damage +1/level.
\item \nameref{Spell:CursedBlade}*: Wounds caused by weapon refuse to heal. %Instead of Armaments of Faith
\item \nameref{Spell:DetectPoison}: Detects poison or disease in one creature or small object.
\item \nameref{Spell:DetectUndead}: Reveals the presence and strength and of undead creatures.
\item \nameref{Spell:DiscernAlignment}: Reveals the subject's alignment.
\item \nameref{Spell:DivineFavor}: You gain a luck bonus on attack and damage rolls.
\item \nameref{Spell:EndureElements}: Exist comfortably in hot or cold environments.
\item \nameref{Spell:Fear}: One creature flees for 1d4 rounds. %Instead of Remove Fear
\item \nameref{Spell:MagicWeapon}: Weapon gains +1 bonus.
\item \nameref{Spell:UncannyAccuracy}*: Your underhanded attacks are surprisingly effective. %Instead of True Strike
\item \nameref{Spell:SummonWeapon}*: Summons bonded weapon to your hand.
\item \nameref{Spell:SurgeOfStrength}*: Gain temporary strength and hit points.
\end{list}
\paragraph{2nd-Level Blackguard Spells}
\begin{list}{\labelitemi}{\leftmargin=1em}
\item \nameref{Spell:AlignWater}*: Vial of water is imbued with the power of an alignment.
\item \nameref{Spell:Blindness}: Negates one of the subject's senses. %Instead of Remove Blindness/Deafness
\item \nameref{Spell:CommandUndead}: Undead creature obeys your commands. %Instead of Aid
\item \nameref{Spell:Darkvision}: See 30 ft. in total darkness. %Instead of Light
\item \nameref{Spell:FalseLife}: Gain 1d10 temporary hp. %Instead of Virtue
\item \nameref{Spell:DeathKnell}: Kill dying creature and gain 1d8 temporary hp, and +2 to Str and your key ability score. %Instead of Divine Footstep
\item \nameref{Spell:GhoulTouch}: Paralyzes one subject, which exudes stench that makes those nearby sickened. %Instead of Restoration
\item \nameref{Spell:LionsCharge}*: You can make full attack in same round you charge.
\item \nameref{Spell:MaskAlignment}: Protects subject's alignment from being revealed via divinations.
\item \nameref{Spell:Resistance}: Grants a Resistance bonus on saving throws.
\item \nameref{Spell:ResistEnergy}: Ignores first 10 (or more) points of damage/attack from specified energy type.
\item \nameref{Spell:StrikeOfPain}*: Attack deals additional damage, and sneak attack damage if you have it. %Instead of Searing Blade
\item \nameref{Spell:WombatsBoost}: Subject gains +4 to an ability score for 1 min./level.
\end{list}
\paragraph{3rd-Level Blackguard Spells}
\begin{list}{\labelitemi}{\leftmargin=1em}
 \item \nameref{Spell:Contagion}: Infects subject with chosen disease. %Instead of Remove Disease
 \item \nameref{Spell:DispelMagic}: Cancels magical spells and effects.
 \item \nameref{Spell:FiendishFlight}*: Subject sprouts wings and flies at speed of 40 ft. %Instead of Celestial Flight
 \item \nameref{Spell:GeasQuest}: Commands subject of 7 HD or less.
 \item \nameref{Spell:GelugonsSpear}*: Creatures struck by your weapon must save or be slowed. %In place of Mace of the Astral Deva
 \item \nameref{Spell:HealMount}: As Heal on your special mount.
 \item \nameref{Spell:KeenEdge}: Doubles normal weapon's threat range.
 \item \nameref{Spell:MagicVestment}: Armor or shield gains +2 enhancement bonus.
 \item \nameref{Spell:Poison}: Creates a temporary batch of poison. %In place of Delay Poison
 \item \nameref{Spell:Question}*: Victim is compelled to answer questions truthfully. %Instead of Comprehend Languages
 \item \nameref{Spell:ShieldSelf}*: Subject takes half of your damage.
 \item \nameref{Spell:UnnaturalEdge}*: Critical hits by weapon become more dangerous. %Instead of Blade of the Sun
\end{list}
\paragraph{4th-Level Blackguard Spells}
\begin{list}{\labelitemi}{\leftmargin=1em}
 \item \nameref{Spell:BestowCurse}: -6 to an ability score; -4 on attack rolls, saves, and checks; or 50\% chance of losing each action. %Instead of Remove Curse
 \item \nameref{Spell:DeathWard}: Grants immunity to death spells and negative energy effects.
 \item \nameref{Spell:DimensionDoor}: Teleports you short distance.
 \item \nameref{Spell:FreedomOfMovement}: Subject moves normally despite impediments.
 \item \nameref{Spell:DenialOfDeathsEmbrace}*: Subject cannot die from hit point damage.
\end{list}
\paragraph{5th-Level Blackguard Spells}
\begin{list}{\labelitemi}{\leftmargin=1em}
 \item \nameref{Spell:Atonement}: Removes burden of misdeeds from subject.
 \item \nameref{Spell:BanishingWeapon}*: Melee weapon banishes outsiders.
 \item \nameref{Spell:Commune}: Deity answers three yes-or-no questions.
 \item \nameref{Spell:DispelAlignment}: Protects against creatures of the chosen alignment, discharge to drive creature away.
 \item \nameref{Spell:FormHorror}: Subject gains the form of a tentacled monstrosity. %Instead of Form of the Celestial.
 \item \nameref{Spell:WidowmakerBlade}*: Melee weapon culls out the weak. %Instead of Disruping Weapon
 \item \nameref{Spell:AlignedSword}: Weapon becomes +5, deals +2d6 damage against one alignment.
 \item \nameref{Spell:RighteousMight}: Your size increases, and you gain combat bonuses.
 \item \nameref{Spell:SlayLiving}: Touch attack kills subject. %Instead of Heal
 \item \nameref{Spell:TrueSeeing}: Lets you see all things as they really are.
\end{list}
\paragraph{6th-Level Blackguard Spells}
\begin{list}{\labelitemi}{\leftmargin=1em}
 \item \nameref{Spell:CrushingOnsetOfAge}*: Subject ages, withers, and dies before your eyes. %Instead of Assault of the Sevenfold Heaven
 \item \nameref{Spell:DweomerBlockade}*: You cut a creature off from all magic.%Instead of Bow of the Solar
 \item \nameref{Spell:FormFiend}*: You assume the form of a terrifying fiend.
 \item \nameref{Spell:PersonalMindBlank}*: You are immune to scrying and mental effects.
 \item \nameref{Spell:PlaneShift}: As many as eight subjects travel to another plane.
 \item \nameref{Spell:WordOfGod}: Kills, paralyzes, hinders, or deafens subjects not of a selected alignment.
\end{list} 
\subsection{Dragon Disciple}
\begin{quote}
\emph{Stay silent, or risk invoking the anger of the dragon.}
- Kalas, human (by birth) Dragon Disciple
\end{quote}
A Dragon Disciple is a dragon-descended spellcaster realizing the power of his heritage.
\paragraph{Hit Die:} d12
\paragraph{Requirements:}
To qualify to become a Dragon Disciple, a character must fulfill all the following criteria.
\subparagraph{Type:} Any nondragon (cannot already be a half-dragon).
\subparagraph{Skills:} Knowledge (arcana) 8 ranks.
\subparagraph{Languages:} Draconic.
\subparagraph{Spellcasting:} Ability to cast arcane spells.
\subparagraph{Special:} The player chooses a dragon variety when taking the first level in this prestige class.
\paragraph{Class Skills}
The Dragon Disciple's class skills (and the key ability for each skill) are Concentration (Con), Craft (Int), Diplomacy (Cha), 
Escape Artist (Dex), Gather Information (Cha), Intimidate (Cha), Knowledge (all skills, taken individually) (Int), 
Listen (Wis), Profession (Wis), Search (Int), Speak Language (None), Spellcraft (Int), and Spot (Wis).
\paragraph{Skill Points at each level:} 4 + Int modifier.
\begin{table*}
\centering
\caption{The Dragon Disciple}
\small
\label{tab:DragonDisciple}
\makebox[\textwidth]{
\begin{tabular}{|l|l|c|c|c|l|l|}
\hline
\multirow{2}{*}{\textbf{Level}}&\multirow{2}{*}{\textbf{BAB}}&\textbf{Fort}&\textbf{Ref}&\textbf{Will}&\multirow{2}{*}{\textbf{Special}}&\textbf{Spell points/day and}\\
&&\textbf{save}&\textbf{save}&\textbf{save}&&\textbf{max spell level}\\
\hline
1st	&+0	&+2	&+0	&+2	&Natural armor (+1)				&+1 level of existing class\\
2nd	&+1	&+3	&+0	&+3	&Ability boost (Str +2), claws and bite		&+1 level of existing class\\
3rd	&+2	&+3	&+1	&+3	&Breath weapon					&+1 level of existing class\\
4th	&+3	&+4	&+1	&+4	&Ability boost (Str +2), natural armor(+2)	&+1 level of existing class\\
5th	&+3	&+4	&+1	&+4	&Blindsense 30 ft.				&+1 level of existing class\\
6th	&+4	&+5	&+2	&+5	&Ability boost (Con +2)				&+1 level of existing class\\
7th	&+5	&+5	&+2	&+5	&Form of the Dragon, Natural armor (+3)		&+1 level of existing class\\
8th	&+6	&+6	&+2	&+6	&Ability boost (Int +2)				&+1 level of existing class\\
9th	&+6	&+6	&+3	&+6	&Wings						&+1 level of existing class\\
10th	&+7	&+7	&+3	&+7	&Blindsense 60 ft., dragon apotheosis		&+1 level of existing class\\
\hline
\end{tabular}
}
\normalsize
\end{table*}
\subsubsection{Class Features}
All of the following are class features of the Dragon Disciple prestige class.

\paragraph{Weapon and Armor Proficiency:}
Dragon Disciples gain no proficiency with any weapon or armor.

\paragraph{Spell Points/Day and Max Spell Level:} 
When a new Dragon Disciple level is gained, the character gains spell points per day, and an increase in caster level and maximum available spell level as if he had also gained a level in whatever arcane spellcasting class in which he could cast spells before he added the prestige class level. 
He does not, however, gain any other benefit a character of that class would have gained, nor does he gain spells known. 
If a character had more than one arcane spellcasting class in which he could cast arcane spells before he became a Dragon Disciple, 
he must decide to which class he adds each level of Dragon Disciple for the purpose of determining what spellcasting class gains the benefit of the spellcasting advancement.

\paragraph{Natural Armor Increase (Ex):}
At 1st, 4th, and 7th level, a Dragon Disciple gains an increase to the character's existing natural armor (a character with no natural armor bonus has a natural armor bonus of +0), as indicated on \nameref{tab:DragonDisciple} table (the numbers represent the total increase gained to that point). 
As his skin thickens, a Dragon Disciple takes on more and more of his progenitor's physical aspect.

\paragraph{Claws and Bite (Ex):}
At 2nd level, a Dragon Disciple gains two claw attacks and a bite attack if he does not already have such attacks. 
His claw attacks deal 1d4 points of damage, and his bite attack deals 1d6 points of damage (assuming a medium Dragon Disciple).
A Dragon Disciple who already has claw or bite attacks uses whichever damage values are greater.

A Dragon Disciple is considered proficient with these attacks. When making a full attack, a Dragon Disciple uses his full base attack bonus with his bite attack but takes a -5 penalty on claw attacks. 
The Multiattack feat reduces this penalty to only -2.

\paragraph{Ability Boost (Ex):}
As a Dragon Disciple gains levels in this prestige class, his ability scores increase as noted on \nameref{tab:DragonDisciple} table.

These increases stack and are gained as if through level advancement.

\paragraph{Blindsense (Ex):}
At 5th level, the Dragon Disciple gains blindsense with a range of 30 feet.
Using nonvisual senses the Dragon Disciple notices things it cannot see. 
He usually does not need to make Spot or Listen checks to notice and pinpoint the location of creatures within range of his blindsense ability, 
provided that he has line of effect to that creature.

Any opponent the Dragon Disciple cannot see still has total concealment against him, and the Dragon Disciple still has the normal miss chance when attacking foes that have concealment. 
Visibility still affects the movement of a creature with blindsense. 
A creature with blindsense is still denied its Dexterity bonus to Armor Class against attacks from creatures it cannot see. 
At 10th level, the range of this ability increases to 60 feet.
\paragraph{Breath Weapon (Su):}
At 3rd level, a Dragon Disciple gains a breath weapon.
In order to use it, he must expend his magical focus as a standard action.
The type and shape depend on the dragon variety whose heritage he enjoys (see The Dragon Disciple Breath Weapon table). 
Regardless of the ancestor, the breath weapon deals 1d6 of damage of the appropriate energy type per arcane caster level (if the Dragon Disciple has multiple arcane caster levels, use the highest).

The DC of the breath weapon is 10 + class level + Con modifier.

A line-shaped breath weapon is 5 feet high, 5 feet wide, and 60 feet long. A cone-shaped breath weapon is 30 feet long.

\begin{center}
\begin{tabular}{|p{3cm}|p{3.4cm}|}
\multicolumn{2}{c}{\textbf{The Dragon Disciple Breath Weapon}}\\
\hline
\textbf{Dragon Variety}$^1$&\textbf{Breath Weapon}\\
\hline
Black		&Line of acid\\
Blue		&Line of lightning\\
Green		&Cone of corrosive gas (acid)\\
Red		&Cone of fire\\
White		&Cone of cold\\
Brass		&Line of fire\\
Bronze		&Line of lightning\\
Copper		&Line of acid\\
Gold		&Cone of fire\\
Silver		&Cone of cold\\
\hline
\multicolumn{2}{p{6.4cm}}{\scriptsize $^1$Other varieties of Dragon Disciple are possible, using other dragon varieties as ancestors.}
\end{tabular}
\end{center}

\paragraph{Form of the Dragon:}
At 7th level, the Dragon Disciple gains \nameref{Spell:FormDragon} as a bonus spell known.
If he does not have the ability to learn or cast 6th level spells when he gains this ability,
he learns it immediately when he can do so.
If he already knows Form of the Dragon when reaching 7th level, 
he can select another spell in its stead he would have qualified for at the time he learned Form of the Dragon.
\paragraph{Wings (Ex):}
At 9th level, a Dragon Disciple grows a set of draconic wings. 
He may now fly at a speed equal to his normal land speed, with average maneuverability.
\paragraph{Dragon Apotheosis:}
At 10th level, a Dragon Disciple's transformation is complete. His type changes to dragon, and gains the augmented subtype (as normal).
This does not cause any loss of class features, despite the Dragon Disciple now no longer fulfilling the class preprequisites.
%\footnote{Note that this would cause a Dragon Disciple to no longer qualify for the prestige class. However, losing PrC prerequisites after entering the class does not cause any loss of class features.}
He gains +4 to Strength and +2 to Charisma.
His natural armor bonus increases to +4, and he acquires low-light vision, 60-foot darkvision, 
immunity to sleep and paralysis effects, and immunity to the energy type used by his breath weapon. 
\subsection{Eldritch Knight}
\begin{quote}
\emph{``But of course, I am a master of the spell \textbf{and} the blade...''}
- Kymrel, elven Eldritch Knight
\end{quote}
An Eldritch Knight is an arcane spellcaster who uses his magic to strengthen his more mundane fighting prowess.

\begin{table*}
\centering
\caption{The Eldritch Knight}
\label{tab:EldritchKnight}
\begin{tabular}{|l|l|c|c|c|l|l|}
\hline
\textbf{Level}&\textbf{BAB}&\textbf{Fort}&\textbf{Ref}&\textbf{Will}&\textbf{Special}&\textbf{Spellcasting}\\
\hline
1st	&+1	&+2	&+0	&+0	&Spellstrike		&-\\
2nd	&+2	&+3	&+0	&+0	&-			&+1 level of existing class\\
3rd	&+3	&+3	&+1	&+1	&-			&+1 level of existing class\\
4th	&+4	&+4	&+1	&+1	&Arcane Vigor		&+1 level of existing class\\
5th	&+5	&+4	&+1	&+1	&-			&+1 level of existing class\\
6th	&+6	&+5	&+2	&+2	&-			&+1 level of existing class\\
7th	&+7	&+5	&+2	&+2	&Greater Spellstrike	&+1 level of existing class\\
8th	&+8	&+6	&+2	&+2	&-			&+1 level of existing class\\
9th	&+9	&+6	&+3	&+3	&-			&+1 level of existing class\\
10th	&+10	&+7	&+3	&+3	&Effortless Spellstrike	&+1 level of existing class\\
\hline
\end{tabular}
\end{table*}

\paragraph{Hit Die:} d6
\paragraph{Requirements:}
To qualify to become an Eldritch Knight, a character must fulfill all the following criteria.
\subparagraph{Weapon Proficiency:} Must be proficient with all martial weapons.
\subparagraph{Skills:} Spellcraft 4 ranks.
\subparagraph{Feat:} \nameref{Feat:ContinuedTraining}.
\subparagraph{Spells:} Able to cast third-level arcane spells.
\paragraph{Class Skills}
The Eldritch Knight's class skills (and the key ability for each skill) are Concentration (Con), Craft (Int), Decipher Script (Int), Jump (Str), Knowledge (arcana) (Int), Knowledge (nobility and royalty) (Int), Ride (Dex), Sense Motive (Wis), Spellcraft (Int), and Swim (Str).
\paragraph{Skill Points at each level:} 4 + Int modifier.

\subsubsection{Class Features}
All of the following are features of the eldritch knight prestige class.

\paragraph{Weapon and Armor Proficiency:} Eldritch Knights gain no proficiency with any weapon or armor.

\paragraph{Spellcasting:} From 2nd level on, when a new Eldritch Knight level is gained, the character gains spell points per day, an increase in caster level, spells known and maximum available spell level as if he had also gained a level in whatever spellcasting class in which he could cast 3rd level spells before he added the prestige class level.
He does not, however, gain any other benefit a character of that class would have gained. 
If a character had more than one applicable spellcasting class before he became an Eldritch Knight, he must decide to which class he adds each level of Eldritch Knight for the purpose of determining what spellcasting class gains the benefit of the spellcasting advancement.

\paragraph{Spellstrike (Su):}
At 1st level, an Eldritch Knight gains the ability to combine a weapon attack with the casting of a targeted spell (a targeted spell is a spell with an entry of ``target'' in its spell description).
The spell may not have a casting time longer than one standard action, and may be of a level no higher than your Eldritch Knight class level (to a natural maximum of 9th level spells).

To use this ability, make a single normal weapon attack as a standard action. If the attack hits, you may expend your magical focus to cast the spell upon the target as a free action. If you have the ability to make multiple attacks as part of a single standard action, that ability works as normal, but you still only get to cast the spell once.

Aside from the action required to cast it, the spell functions as normal. You pay its spell point cost, its range limitations apply, and casting in melee provokes attacks of opportunity.

\paragraph{Arcane Vigor (Su):}
At Eldritch Knight level 4th, whenever you use your Spellstrike or Greater Spellstrike ability (see below) ability, you gain a number of temporary hit points equal to the number of spell points you spent on casting the spell.
You need not spend an action to activate this ability, it happens automatically whenever you use Spellstrike or Greater Spellstrike.
These temporary hit points last for one minute.
Temporary hit points from Arcane Vigor stack with temporary hit points from other sources.

\paragraph{Greater Spellstrike (Su):}
At Eldritch Knight level 7th, you can perform a Greater Spellstrike.
A Greater Spellstrike works as Spellstrike, above, except that instead of casting the spell after making a single normal attack as a standard action, you cast it after making a full attack as a full-round action. The spell then affects every target you hit during the full attack, even if the spell would normally only affect a single target. However, no target is affected more than once, even if you hit a target multiple times during your full attack.

This does not replace your Spellstrike ability.

\paragraph{Effortless Spellstrike (Ex):}
At Eldritch Knight level 10th, you no longer need to expend your magical focus in order to use your Spellstrike and Greater Spellstrike abilities. 
\subsection{Hierophant}
\begin{quote}
\emph{``Do not blindly place your faith in the higher powers. Understand them.''}
- Avatar, elven Hierophant
\end{quote}
A Hierophant is a divine spellcaster who has begun to understand the bonds between himself and his power source on a more fundamental level than most.

\begin{table*}
\centering
\caption{The Hierophant}
\label{tab:Hierophant}
\makebox[\textwidth]{
\begin{tabular}{llcccll}
\toprule
Level	&BAB	&Fort 	&Ref 	&Will 	&Special		&Spellcasting\\
\midrule
1st	&+0	&+0	&+0	&+2	&Special Ability	&+1 level of existing class\\
2nd	&+1	&+0	&+0	&+3	&Special Ability	&+1 level of existing class\\
3rd	&+1	&+1	&+1	&+3	&Special Ability	&+1 level of existing class\\
4th	&+2	&+1	&+1	&+4	&Special Ability	&+1 level of existing class\\
5th	&+2	&+1	&+1	&+4	&Special Ability	&+1 level of existing class\\
\bottomrule
\end{tabular}}
\end{table*}
\paragraph{Hit Die:} d8
\paragraph{Requirements:}
To qualify to become a Hierophant, a character must fulfill all the following criteria.
\subparagraph{Skills:} Knowledge (religion) 15 ranks 
\subparagraph{Feats:} Any metamagic feat.
\subparagraph{Spells:} Ability to cast 7th-level divine spells
\paragraph{Class Skills}
The Hierophant's class skills (and the key ability for each skill) are Concentration (Con), Craft (Int), Diplomacy (Cha), Heal (Wis), Knowledge (arcana) (Int), Knowledge (religion) (Int), Profession (Wis), and Spellcraft (Int).
\paragraph{Skill Points at each level:} 4 + Int modifier.

\subsubsection{Class Features}
All the following are Class Features of the Hierophant prestige class.

\paragraph{Weapon and Armor Proficiency:} Hierophants gain no proficiency with any weapon or armor.

\paragraph{Spellcasting:} When a new Hierophant level is gained, the character gains spell points per day, an increase in caster level, spells known and maximum available spell level as if he had also gained a level in whatever divine spellcasting class in which he could cast 7th-level spells before he added the prestige class level. 
He does not, however, gain any other benefit a character of that class would have gained. 
If a character had more than one divine spellcasting class in which he could cast 7th-level spells before he became a Hierophant, he must decide to which class he adds each level of Hierophant for the purpose of determining what spellcasting class gains the benefit of the spellcasting advancement.

\paragraph{Special Ability:}
A Hierophant gains the opportunity to select a special ability from among those described below by permanently eliminating a specific number of Spell Points.
These spell points are subtracted from the final number of spell points he Hierophant would otherwise have.
Effectively, these spell points are spent on ``fueling'' the Special Ability.
%When a Special Ability refers to a Hierophant's ``caster level'', it means his highest caster level.

\subparagraph{Blast Infidel (Su):}
A Hierophant can use negative energy spells to their maximum effect on creatures with an alignment opposed to the Hierophant. (See the table below for a list of which alignments are opposed to each alignment.) 
Any spell with a description that involves inflicting or channeling negative energy cast on a creature of the opposed alignment works as if under the effects of a \nameref{Feat:MaximizeSpell} feat (without using additional spell points or requiring the expenditure of the Hierophant's magical focus).

\begin{tableonecolumn}
\caption{Blast Infidel}
\label{tab:BlastInfidel}
\begin{tabular}{p{2.4cm}p{4cm}}
\toprule
Hierophant Alignment&Opposed Alignment\\
\midrule
Lawful good&		Chaotic evil\\
Neutral good&		Neutral evil\\
Chaotic good&		Lawful evil\\
Lawful neutral&		Chaotic neutral\\
Neutral&		Lawful good, chaotic good, lawful evil, chaotic evil$^1$\\
Chaotic neutral&	Lawful neutral\\
Lawful evil&		Chaotic good\\
Neutral evil&		Neutral good\\
Chaotic evil&		Lawful good\\
\bottomrule
\end{tabular}
{\scriptsize
\begin{enumerate}
 \item A neutral Hierophant chooses one of these alignments to be the one that he opposes for the purposes of this special ability.
\end{enumerate}
}
\end{tableonecolumn}

Learning this Special Ability removes 11 spell points from the Hierophant's pool.

\subparagraph{Divine Reach (Su):}
The Hierophant can use spells with a range of touch on a target up to 30 feet away. 
In the case of a spell that would ordinarily require a touch attack, the Hierophant must make a ranged touch attack instead.
Divine reach can be selected a second time as a Special Ability (paying the cost again), in which case the range increases to 60 feet. 
Learning this Special Ability removes 13 spell points from the Hierophant's pool.

\subparagraph{Faith Healing (Su):}
A Hierophant can use healing spells to their maximum effect on creatures of the same alignment as the Hierophant (including the Hierophant himself). 
Any spell of the Healing subschool cast on such creatures works as if under the effects of a \nameref{Feat:MaximizeSpell} feat (without using additional spell points or requiring the expenditure of the Hierophant's magical focus).
Learning this Special Ability removes 11 spell points from the Hierophant's pool.

\subparagraph{Gift of the Divine (Su):}
Available only to Hierophants with the \nameref{Feat:TurnUndead} feat, this ability allows a Hierophant to share the feat with another willing creature.
The creature must be touched for the transfer to take place.
The recipient can then use the feat as if he had selected the feats as one of his feats known.
The recipient does not have to meet the feat's class level requirement, but he must have a non-evil alignment and be capable of obtaining a Magical Focus.
The recipient turns undead as a creature of the Hierophant's level, but uses his own Charisma modifier.
The transfer lasts anywhere from 24 hours to one week (chosen at the time of transfer). 
Once shared, the Hierophant cannot share his ability to Turn Undead with another creature until the previous use of this ability has expired.
Learning this Special Ability removes 5 spell points from the Hierophant's pool.

\subparagraph{Metamagic Feat:}
A Hierophant can choose a metamagic feat in place of one of the Special Abilities described here if desired.
Doing so removes 17 spell points from the Hierophant's pool.

\subparagraph{Power of Nature (Su):}
Available only to Hierophants with access to at least one of the domains of Air, Animal, Earth, Fire, Plant, and Water, this ability allows a Hierophant to temporarily transfer one or more of the granted abilities of the aforementioned domains to a willing creature. 
The creature must be touched for the transfer to take place.
The transfer lasts anywhere from 24 hours to one week (chosen at the time of transfer), and while the transfer is in effect, the Hierophant cannot use the transferred granted powers.
Spellcasting and spells known are not granted powers of the domains, and can thus not be transferred.
The recipient cannot use the granted powers to fulfil prerequisites.
All transferred powers use the Hierophant's level, but otherwise use the recipient's statistics (for example, the bonus given by the Driven Arrows granted power of the Air domain would use the Hierophant's Cleric level, but the recipient's Strength and Wisdom modifiers).
Some granted powers may not be of any use to a recipient, such as an Elemental Shape ability on a recipient without spellcasting.
Learning this Special Ability removes 5 spell points from the Hierophant's pool.

\subparagraph{Spell Power:}
This ability provides the Hierophant with a +1 inherent bonus to one of his mental ability scores. 
Spell Power can be selected more than once as a Special Ability (paying the cost each time). 
Each additional time it is selected, he gains a +1 inherent bonus to another mental ability score, or one of his existing inherent bonuses to a mental ability score increases by one.
However, an inherent bonus may not exceed +5 for a single ability score.
Learning this Special Ability removes 9 spell points from the Hierophant's pool.

\subparagraph{Spell-Like Ability (Sp):}
A Hierophant who selects this Special Ability can use one of his spells known as a spell-like ability twice per day.
The caster level for this spell-like ability is equal to the Hierophant's hit dice.
The activation of the spell-like ability requires the same action as casting the spell itself.
If this Special Ability is selected more than once, it can apply to the same spell chosen the first time (increasing the number of times per day it can be used) or to a different spell.
Learning this Special Ability removes a number of spell points equal to the minimum amount required to cast the spell in question from the Hierophant's pool.

\subsection{Loremaster}
\begin{quote}
\emph{``Why, this reminds me of a story I once heard...''}
- Veyr, human Loremaster
\end{quote}
A Loremaster is a spellcaster whose magic is a means to collect ever more information about the world.
\paragraph{Hit Die:} d4
\paragraph{Requirements:}
To qualify to become a Loremaster, a character must fulfill all the following criteria.
\subparagraph{Skills:} Knowledge (any two) 10 ranks in each.
\subparagraph{Feats:} Any three metamagic or item creation feats, plus Skill Focus (Knowledge [any individual Knowledge skill]).
\subparagraph{Spells:} Able to cast three different divination spells, one of which must be 3rd level or higher.
\paragraph{Class Skills}
The Loremaster's class skills (and the key ability for each skill) are Appraise (Int), Concentration (Con), Craft (alchemy) (Int), Decipher Script (Int), Gather Information (Cha), Handle Animal (Cha), Heal (Wis), Knowledge (all skills taken individually) (Int), Perform (Cha), Profession (Wis), Speak Language (None), Spellcraft (Int), and Use Magic Device (Cha).
\paragraph{Skill Points at each level:} 4 + Int modifier.
\begin{table*}
\centering
\caption{The Loremaster}
\label{tab:Loremaster}
%\makebox[\textwidth]{
\begin{tabular}{|l|l|c|c|c|l|l|}
\hline
\textbf{Level}&\textbf{BAB}&\textbf{Fort}&\textbf{Ref}&\textbf{Will}&\textbf{Special}&\textbf{Spellcasting}\\
\hline
1st	&+0	&+0	&+0	&+2	&Secret		&+1 level of existing class\\
2nd	&+1	&+0	&+0	&+3	&Lore		&+1 level of existing class\\
3rd	&+1	&+1	&+1	&+3	&Secret		&+1 level of existing class\\
4th	&+2	&+1	&+1	&+4	&Bonus language	&+1 level of existing class\\
5th	&+2	&+1	&+1	&+4	&Secret		&+1 level of existing class\\
6th	&+3	&+2	&+2	&+5	&Greater lore	&+1 level of existing class\\
7th	&+3	&+2	&+2	&+5	&Secret		&+1 level of existing class\\
8th	&+4	&+2	&+2	&+6	&Bonus language	&+1 level of existing class\\
9th	&+4	&+3	&+3	&+6	&Secret		&+1 level of existing class\\
10th	&+5	&+3	&+3	&+7	&True lore	&+1 level of existing class\\
\hline
\end{tabular}
%}
\end{table*}
\subsubsection{Class Features}
All the following are Class Features of the Loremaster prestige class.

\paragraph{Weapon and Armor Proficiency:} Loremasters gain no proficiency with any weapon or armor.

\paragraph{Spellcasting:} When a new Loremaster level is gained, the character gains spell points per day, an increase in caster level, spells known and maximum available spell level as if he had also gained a level in whatever spellcasting class in which he could cast sufficient divination spells to enter before he added the prestige class level. 
He does not, however, gain any other benefit a character of that class would have gained. 
If a character had more than one applicable spellcasting class before he became a Loremaster, 
he must decide to which class he adds each level of Loremaster for the purpose of determining what spellcasting class gains the benefit of the spellcasting advancement.

\paragraph{Secret:}
At 1st level and every two levels higher than 1st (3rd, 5th, 7th, and 9th), the Loremaster chooses one secret from the following list.
He can't choose the same secret twice.

\subparagraph{Instant Mastery:}
You gain 4 ranks in a skill in which you previously had no ranks.\\
\subparagraph{Secret Health (Ex):}
You gain a number of hit points equal to your current number of Hit Dice. 
Each time you gain a Hit Die (such as by gaining a level), you gain 1 additional hit point. 
If you lose a Hit Die, you permanently lose 1 hit point.
\subparagraph{Secrets of Inner Strength (Ex):}
You gain a +2 bonus on Will saves.
\subparagraph{The Lore of True Stamina (Ex):}
You gain a +2 bonus on Fortitude saves.
\subparagraph{Secret Knowledge of Avoidance (Ex):}
You gain a +2 bonus on Reflex saves.
\subparagraph{Weapon Trick (Ex):}
You gain a +1 bonus on all attack rolls.
\subparagraph{Dodge Trick (Ex):}
You gain a +1 dodge bonus to AC. Unless otherwise noted, dodge bonuses stack.
\subparagraph{Applicable Knowledge:}
You gain a bonus feat for which you meet the prerequisites.
\subparagraph{Newfound Arcana:}
You gain \nameref{Feat:MagicallyGifted} as a bonus feat.

\paragraph{Lore (Ex):}
At 2nd level, a Loremaster gains the ability to know legends or information regarding various topics. This ability is identical to the \nameref{sec:Bard}'s Bardic Knowledge class feature, using your Loremaster level in place of the Bard level. This counts as Bardic Knowledge for all purposes, including those of prerequisites. If you have Bard levels, you do not use the abilities separately, but your Loremaster and Bard levels stack for purposes of determining your bonus on the check.

\paragraph{Bonus Languages:}
A Loremaster can choose any new language at 4th and 8th level.

\paragraph{Greater Lore (Sp):}
At 6th level, a Loremaster gains the ability to use \nameref{Spell:Identify} as a spell-like ability at will.
Caster level equals your number of Hit Dice.

\paragraph{True Lore (Sp):}
At 10th level, once per day a Loremaster can use her knowledge to use \nameref{Spell:LegendLore} as a spell-like ability.
\subsection{Mystic Theurge}
\begin{quote}
\emph{``I am afraid that the rest of my life would not be sufficient to share all that I have learned.''}
- Thorgeir, half-elven Mystic Theurge
\end{quote}
A Mystic Theurge is a spellcaster adept in both arcane and divine magic, sacrificing rate of advancement for unparalleled breadth of understanding.

\begin{table*}
\caption{The Mystic Theurge}
\label{tab:MysticTheurge}
\makebox[\textwidth][c]{
\begin{tabular}{|l|l|c|c|c|l|l|}
\hline
\textbf{Level}&\textbf{BAB}&\textbf{Fort}&\textbf{Ref}&\textbf{Will}&\textbf{Special}&\textbf{Spells known, maximum spell level, caster level}\\
\hline
1st	&+0	&+0	&+0	&+2	&Continued Training	&+1 level of primary and secondary spellcasting class\\
2nd	&+1	&+0	&+0	&+3	&-			&+1 level of primary and secondary spellcasting class\\
3rd	&+1	&+1	&+1	&+3	&-			&+1 level of primary and secondary spellcasting class\\
4th	&+2	&+1	&+1	&+4	&-			&+1 level of primary and secondary spellcasting class\\
5th	&+2	&+1	&+1	&+4	&Combined Knowledge	&+1 level of secondary spellcasting class\\
6th	&+3	&+2	&+2	&+5	&-			&+1 level of primary and secondary spellcasting class\\
7th	&+3	&+2	&+2	&+5	&-			&+1 level of primary and secondary spellcasting class\\
8th	&+4	&+2	&+2	&+6	&-			&+1 level of primary and secondary spellcasting class\\
9th	&+4	&+3	&+3	&+6	&-			&+1 level of primary and secondary spellcasting class\\
10th	&+5	&+3	&+3	&+7	&Bonus Feat		&+1 level of primary and secondary spellcasting class\\
11th	&+5	&+3	&+3	&+7	&-			&+1 level of primary and secondary spellcasting class\\
12th	&+6	&+4	&+4	&+8	&-			&+1 level of primary and secondary spellcasting class\\
13th	&+6	&+4	&+4	&+8	&-			&+1 level of primary and secondary spellcasting class\\
14th	&+7	&+4	&+4	&+9	&-			&+1 level of primary and secondary spellcasting class\\
15th	&+7	&+5	&+5	&+9	&Spell Union		&+1 level of secondary spellcasting class\\
16th	&+8	&+5	&+5	&+10	&-			&+1 level of primary and secondary spellcasting class\\
\hline
\end{tabular}}
\end{table*}

\paragraph{Hit Die:} d6
\paragraph{Requirements:}
To qualify to become a Mystic Theurge, a character must fulfill all the following criteria.
\subparagraph{Skills:} Knowledge (arcana) 7 ranks, Knowledge (religion) 7 ranks, Spellcraft 4 ranks.
\subparagraph{Feat:} \nameref{Feat:ContinuedTraining}.
\subparagraph{Spells:} Ability to cast second-level arcane spells and first-level divine spells OR ability to cast first-level arcane spells and second-level divine spells.
\paragraph{Class Skills}
The Mystic Theurge's class skills (and the key ability for each skill) are Concentration (Con), Craft (Int), Decipher Script (Int), Knowledge (all skills, taken individually) (Int), Profession (Wis), Sense Motive (Wis), and Spellcraft (Int).
\paragraph{Skill Points at each level:} 4 + Int modifier.

\subsubsection{Class Features}
All of the following are features of the Mystic Theurge prestige class.

\paragraph{Weapon and Armor Proficiency:} Mystic Theurges gain no proficiency with any weapon or armor.

\paragraph{Spellcasting:} In order to benefit from Mystic Theurge spellcasting advancement, the character must have levels in at least two different spellcasting classes, one arcane and one divine. One of them must be designated as the primary spellcasting class, and another as the secondary class. To be designated as a primary spellcasting class, the character must be able to cast second-level spells via that class. If the primary spellcasting class is an arcane spellcasting class, the secondary class must be a divine spellcasting class via which he can cast first level spells, and vice versa. A Mystic Theurge's primary and secondary spellcasting classes must be separate spellcasting classes. The designation must be made when taking the first level of the prestige class, and can not be changed thereafter.

At each Mystic Theurge level except at 5th and 15th, the character gains an increase in caster level, spells known, and maximum available spell level as if he had also gained a level in both his primary and secondary spellcasting classes. At Mystic Theurge levels 5th and 15th, he gains an increase in caster level, spells known, and maximum available spell level as if he had gained a level in his secondary spellcasting class only.

At every Mystic Theurge level, the character gains an increase in spell points as if he had also gained a level in his primary spellcasting class.

He does not, however, gain any other benefit a character of his primary or secondary spellcasting classes would have gained.
\paragraph{Continued Training:} A Mystic Theurge who has the \nameref{Feat:ContinuedTraining} feat for his secondary spellcasting class gains the feat for his primary spellcasting class as a bonus feat. A Mystic Theurge who has the feat for his primary spellcasting class gains the feat for his secondary spellcasting class.

\paragraph{Combined Knowledge (Ex):} At Mystic Theurge level 5th, you can use the key ability modifier of your primary spellcasting class to ddetermine the spell save DCs, ability score minimums, and bonus spell points for your secondary class, if doing so would be beneficial for you.

\paragraph{Bonus Feat:} At Mystic Theurge level 10th, you gain a bonus feat selected from the following list: \nameref{Feat:CelestialSummons}, \nameref{Feat:DomainAttunement}, \nameref{Feat:Familiar}, \nameref{Feat:FiendishSummons}, \nameref{Feat:ScribeScroll} and \nameref{Feat:TurnUndead}. You must meet the prerequisites of the feat, if any.

\paragraph{Spell Union (Su):} At Mystic Theurge level 15th, you may cast two spells as a full-round action. One of the spells must be from your primary spellcasting class, the other from your secondary class. Each spell may have a casting time no longer than one standard action. You must provide components, pay XP and spell point costs, and in all other ways follow the restrictions the spells themselves put in place, the only change is the action required to cast them. This ability can be used once per day.
\subsection{Thaumaturgist}
\begin{quote}
\emph{``This material world is but the tip of the iceberg that is the cosmos. The real battle lies far beyond.''}
- Xolon, human Thaumaturgist
\end{quote}
A Thaumaturgist is a spellcaster who is more adept than most at enlisting the aid of the outer planes.

\begin{table*}
\centering
\caption{The Thaumaturgist}
\label{tab:Thaumaturgist}
\makebox[\textwidth]{
\begin{tabular}{llcccll}
\toprule
Level	&BAB	&Fort 	&Ref 	&Will 	&Special		&Spellcasting\\
\midrule
1st	&+0	&+0	&+0	&+2	&Improved ally		&+1 level of existing class\\
2nd	&+1	&+0	&+0	&+3	&Augment Summoning	&+1 level of existing class\\
3rd	&+1	&+1	&+1	&+3	&Extended summoning	&+1 level of existing class\\
4th	&+2	&+1	&+1	&+4	&Contingent conjuration	&+1 level of existing class\\
5th	&+2	&+1	&+1	&+4	&Planar cohort		&+1 level of existing class\\
\bottomrule
\end{tabular}}
\end{table*}

\paragraph{Hit Die:} d4
\paragraph{Requirements:}
To qualify to become a Thaumaturgist, a character must fulfill all the following criteria.
\subparagraph{Skills:} Knowledge (the planes) 10 ranks 
\subparagraph{Spells:} Ability to cast \nameref{Spell:PlanarAlly}
\subparagraph{Special:} Must be a specialist Conjurer or be a Cleric with the Planes domain.
\paragraph{Class Skills}
The Thaumaturgist's class skills (and the key ability for each skill) are Concentration (Con), Craft (Int), Diplomacy (Cha), Knowledge (religion) (Int), Knowledge (the planes) (Int), Profession (Wis), Sense Motive (Wis), Speak Language (None), and Spellcraft (Int).
\paragraph{Skill Points at each level:} 4 + Int modifier.
\subsubsection{Class Features}
All the following are Class Features of the Thaumaturgist prestige class.

\paragraph{Weapon and Armor Proficiency:} Thaumaturgists gain no proficiency with any weapon or armor.

\paragraph{Spellcasting:} When a new Thaumaturgist level is gained, the character gains spell points per day, an increase in caster level, spells known and maximum available spell level as if he had also gained a level in whatever spellcasting class in which he could cast \nameref{Spell:PlanarAlly} before he added the prestige class level. 
He does not, however, gain any other benefit a character of that class would have gained. 
If a character had more than one spellcasting class in which he could cast Planar Ally before he became a Thaumaturgist, he must decide to which class he adds each level of Thaumaturgist for the purpose of determining what spellcasting class gains the benefit of the spellcasting advancement.

\paragraph{Improved Ally (Ex)}
When a Thaumaturgist casts a \nameref{Spell:PlanarAlly} spell, he, like any other caster, is entitled to making a Diplomacy check to convince the creature to aid him. If the Thaumaturgist's Diplomacy check adjusts the creature's attitude to helpful the spell's experience cost is retroactively reduced by half, as long as the task is one that is not against the creature's nature.

The Thaumaturgist's Improved ally class feature only works when the planar ally shares at least one aspect of alignment with the Thaumaturgist.

A Thaumaturgist can have only one such ally at a time, but he may negotiate tasks from other planar allies normally.

\paragraph{Augment Summoning}
At 2nd level, a Thaumaturgist gains the \nameref{Feat:AugmentSummoning} feat as a bonus feat.

\paragraph{Extended Summoning (Ex)}
At 3rd level and higher, all spells from the summoning subschool that the Thaumaturgist casts have their durations doubled, as if the \nameref{Feat:ExtendSpell} feat had been applied to them. 
The spell point cost of the summoning spells don't change and the Thaumaturgist need not expend his magical focus, however. 
This ability stacks with the effect of the Extend Spell feat (for a total duration of three times its normal duration), which does cost additional spell points.

\paragraph{Contingent Conjuration (Sp)}
Once per day, a 4th-level Thaumaturgist can use a special form of the \nameref{Spell:Contingency} spell as a spell-like ability. However, the Thaumaturgist may only cast the Contingent Conjuration spell upon himself, and he must select a Conjuration (Calling) or a Conjuration (Summoning) spell as the companion spell.

If the Thaumaturgist has the \nameref{Spell:Contingency} spell as a spell known, he can use it without interfering with his Contingent Conjuration spell-like ability.
\paragraph{Planar Cohort (Ex)}
A 5th-level Thaumaturgist can use the \nameref{Spell:PlanarAlly} spell to call a creature to act as his cohort. 

To call a planar cohort, the Thaumaturgist must cast the spell, paying the XP cost as if getting the creature to stay for one day per caster level. The improved ally class feature can't be used to reduce or eliminate this cost. 
The Thaumaturgist can then attempt to use the Diplomacy skill to persuade the creature to stay permanently. 
If it agrees, the \nameref{Spell:PlanarAlly} spell immediately ends, but the creature persists, and begins acting as the Thaumaturgist's cohort.
The planar cohort can't have more Hit Dice than the Thaumaturgist has, and must have an ECL no higher than the Thaumaturgist's character level -2.
The creature serves loyally and well as long as the Thaumaturgist continues to advance a cause important to the creature.

A Thaumaturgist can have only one planar cohort at a time, but he can continue to make agreements with other called creatures normally. 
A planar cohort replaces a Thaumaturgist's existing cohort, if he has one by virtue of the Leadership feat.

%"Now take off your shoes... yes... that'll do nicely."
%-Abracadabrus, Human Fetishist
 \newpage

\part{Skills and Feats}
Changing a d20 spellcasting system affects more than just the classes - a few skills and a multitude of feats are affected as well. The following two chapters focus on updating these aspects of the system.
\input{SkillsPsi.tex} \newpage
\input{FeatsPsi.tex}\newpage

\part{Spells}
What would a spellcasting overhaul be without overhauling the spells themselves? The following chapters present the fully reviewed spells and spell lists found in the \href{http://www.wizards.com/default.asp?x=d20/article/srd35}{d20 srd}, changed where appropriate to accommodate the changed system.

The veteran d20 reader will find spells here he has never seen before here. While it was not the author's original intention to add a lot of new material to the core document, the extensive restructuring left gaping holes in many spell lists and character concepts. Hopefully, these additions will not be out of place in your campaign.
\section{Spell Lists}
\label{sec:Spells}
Spells marked with an asterisk (*) are spells that have no immediate magical ancestor in the d20 SRD.
This may be because the spell is based on a psionic counterpart, or because it is a new spell entirely.
\subsection{Bard Spells}
\label{sec:BardSpells}
\subsubsection{1st-Level Bard Spells}
\begin{list}{\labelitemi}{\leftmargin=1em}
\item \nameref{Spell:Alarm}: Wards an area for 2 hours/level.
\item \nameref{Spell:Charm}: Makes one creature your friend.
\item \nameref{Spell:ComprehendLanguages}: You understand all spoken and written languages.
\item \nameref{Spell:ControlFall}: Objects or creatures fall slowly.
\item \nameref{Spell:CureWounds}: Cures 1d8 damage +1/level.
\item \nameref{Spell:Daze}: Target creature loses next action.
\item \nameref{Spell:DetectMagic}: Reveals the presence, strength, and school of magical auras.
\item \nameref{Spell:DetectSecretDoors}: Become aware of all secret doors within your line of sight.
\item \nameref{Spell:DisguiseSelf}: Changes your appearance.
\item \nameref{Spell:ExpeditiousRetreat}: Your speed increases by 30 ft.
\item \nameref{Spell:Fear}: One creature flees for 1d4 rounds.
\item \nameref{Spell:Grease}: Makes 10-ft. square or one object slippery.
\item \nameref{Spell:HideousLaughter}: Subject loses actions for 1 round/level.
\item \nameref{Spell:Identify}: Determines properties of magic item.
\item \nameref{Spell:Image}: Creates illusion of your design.
\item \nameref{Spell:MagicAura}: Alters object's magic aura.
\item \nameref{Spell:MentalLink}*: You forge a limited mental bond with another creature.
\item \nameref{Spell:Mount}: Summons magical riding horse for 2 hours/level.
\item \nameref{Spell:OpenClose}: Holds door shut or opens it.
\item \nameref{Spell:Repair}: Makes repairs on an object or construct.
\item \nameref{Spell:SummonInstrument}: Summons one instrument of the caster's choice.
\item \nameref{Spell:RandomAction}: Forces a creature to act randomly.
\item \nameref{Spell:RemoveFear}: Subject gains immunity to fear.
\item \nameref{Spell:Sleep}: Puts 4 HD of creatures into magical slumber.
\item \nameref{Spell:UncannyAccuracy}*: Your underhanded attacks are surprisingly effective.
\item \nameref{Spell:UnseenServant}: Invisible force obeys your commands.
\item \nameref{Spell:Ventriloquism}: Makes sounds appear out of nowhere.
\end{list}
\subsubsection{2nd-Level Bard Spells}
\begin{list}{\labelitemi}{\leftmargin=1em}
\item \nameref{Spell:AlterSelf}: Perform minor physical changes on yourself.
\item \nameref{Spell:AnimalMessenger}: Sends a Tiny animal to a specific place.
\item \nameref{Spell:Antipoison}: Stops poison from harming subject for 1 hour/level.
\item \nameref{Spell:Blindness}: Negates one of the subject's senses.
\item \nameref{Spell:Blur}: Attacks miss subject 20\% of the time.
\item \nameref{Spell:CalmEmotions}: Calms creatures, negating emotion effects.
\item \nameref{Spell:Darkness}: 20-ft. radius of supernatural shadow.
\item \nameref{Spell:Glitterdust}: Blinds creatures, outlines invisible creatures.
\item \nameref{Spell:Heroism}: Gives +2 bonus on attack rolls, saves, skill checks.
\item \nameref{Spell:HoldPerson}: Paralyzes one humanoid for 1 round/level.
\item \nameref{Spell:Invisibility}: Subject is invisible for 1 min./level or until it attacks.
\item \nameref{Spell:Locate}: Senses direction toward object (specific or type).
\item \nameref{Spell:MaskAlignment}: Protects subject's alignment from being revealed via divinations.
\item \nameref{Spell:MirrorImage}: Creates decoy duplicates of you.
\item \nameref{Spell:Pattern}: Twisting colors fascinate creatures.
%\item \nameref{Spell:Pyrotechnics}: Turns fire into blinding light or choking smoke.
\item \nameref{Spell:Rage}: Subjects are thrown into a fit of anger, with various effects.
\item \nameref{Spell:ReadThoughts}: Detect surface thoughts of creatures in range.
\item \nameref{Spell:SeeInvisibility}: Reveals invisible creatures or objects.
\item \nameref{Spell:Shatter}: Sonic vibration damages objects or crystalline creatures.
\item \nameref{Spell:Silence}: Negates sound in 20-ft. radius.
\item \nameref{Spell:SoundBurst}: Deals 1d8 sonic damage to subjects; may stun them.
\item \nameref{Spell:SummonSwarm}: Summons swarm of bats, rats, or spiders.
\item \nameref{Spell:Suggestion}: Compels subject to follow stated course of action.
\item \nameref{Spell:TwinBladeDance}: Stun opponents struck with a pair of weapons.
\item \nameref{Spell:WombatsBoost}: Subject gains +4 to an ability score for 1 min./level.
\end{list}
\subsubsection{3rd-Level Bard Spells}
\begin{list}{\labelitemi}{\leftmargin=1em}
\item \nameref{Spell:Blink}: You randomly vanish and reappear for 1 round/level.
\item \nameref{Spell:Clairvoyance}: See and hear a distant location.
\item \nameref{Spell:Confusion}: Subjects behave oddly for 1 round/level.
\item \nameref{Spell:CrushingDespair}: Subjects take -2 on attack rolls, damage rolls, saves, and checks.
\item \nameref{Spell:DispelMagic}: Cancels magical spells and effects.
\item \nameref{Spell:GaseousForm}: Subject becomes insubstantial and can fly slowly.
\item \nameref{Spell:GeasQuest}: Commands subject of 7 HD or less.
\item \nameref{Spell:Glibness}: You gain a large bonus on Bluff checks.
\item \nameref{Spell:Haste}: One creature moves faster, +1 on attack rolls, AC, and Reflex saves.
\item \nameref{Spell:Nondetection}: Masks object or creature against scrying.
\item \nameref{Spell:Piggyback}*: Latch on to a creature as it teleports, retaining your relative position.
\item \nameref{Spell:RemoveCurse}: Frees object or person from curse.
\item \nameref{Spell:Scrying}: Spies on subject from a distance.
\item \nameref{Spell:SepiaSnakeSigil}: Creates text symbol that immobilizes reader.
\item \nameref{Spell:Slow}: One creature takes only one action/round, -1 to AC, reflex saves, and attack rolls.
\item \nameref{Spell:TinyHut}: Creates shelter for ten creatures.
\end{list}
\subsubsection{4th-Level Bard Spells}
\begin{list}{\labelitemi}{\leftmargin=1em}
 \item \nameref{Spell:DetectScrying}: Alerts you of magical eavesdropping.
 \item \nameref{Spell:DimensionDoor}: Teleports you short distance.
 \item \nameref{Spell:FreedomOfMovement}: Subject moves normally despite impediments.
 \item \nameref{Spell:HallucinatoryTerrain}: Makes one type of terrain appear like another.
 \item \nameref{Spell:ModifyMemory}: Changes 5 minutes of subject's memories.
 \item \nameref{Spell:RepelVermin}: Insects, spiders, and other vermin stay 10 ft. away.
 \item \nameref{Spell:ShadowConjuration}: Mimics certain conjurations.
 \item \nameref{Spell:Shout}: Deafens all within cone and deals 7d6 sonic damage.
\end{list}
\subsubsection{5th-Level Bard Spells}
\begin{list}{\labelitemi}{\leftmargin=1em}
\item \nameref{Spell:Dream}: Contact or disturb sleeping creature.
\item \nameref{Spell:Dominate}: Controls humanoid telepathically.
\item \nameref{Spell:FalseVision}: Fools scrying with an illusion.
\item \nameref{Spell:LegendLore}: Lets you learn tales about a person, place, or thing.
\item \nameref{Spell:MindFog}: Subjects in fog suffer increasing penalties to Will saves and Wisdom checks.
\item \nameref{Spell:Mislead}: Turns you invisible and creates illusory double.
\item \nameref{Spell:ShadowEvocation}: Mimics certain Evocations.
\item \nameref{Spell:ShadowWalk}: Step into shadow to travel rapidly.
\item \nameref{Spell:SongOfDiscord}: Forces targets to attack each other.
\end{list}
\subsubsection{6th-Level Bard Spells}
\begin{list}{\labelitemi}{\leftmargin=1em}
\item \nameref{Spell:AnimateObjects}: Objects attack your foes.
\item \nameref{Spell:Eyebite}: Target becomes panicked, sickened, and/or comatose, depending on HD.
\item \nameref{Spell:HeroesFeast}: Food for four creatures cures and grants combat bonuses and immunities.
\item \nameref{Spell:IrresistibleDance}: Forces subject to dance.
\item \nameref{Spell:ProgrammedImage}: Causes Image spell to be triggered by event.
\item \nameref{Spell:SympatheticVibration}: Deals 2d10 damage/round to freestanding structure.
\end{list}
\subsection{Cleric Domains and Spells}
\label{sec:ClericDomains}
\subsubsection{Cleric Spell List}
\label{Domain:General}
Spells on this list are accessible to any Cleric, regardless of their deity and alignment. This generic spell list does not have any granted powers associated with it.
\begin{list}{\labelitemi}{\leftmargin=1em}
  \item[1] \nameref{Spell:TouchOfVitality} \textbf{(Free for Clerics)}: Cures wounds with a touch.
  \item[1] \nameref{Spell:DetectMagic}: Reveals the presence, strength, and school of magical auras.
  \item[2] \nameref{Spell:WombatsBoost}: Subject gains +4 to an ability score for 1 min./level.
  \item[2] \nameref{Spell:Augury}: Learns whether an action will be good or bad.
  \item[3] \nameref{Spell:DispelMagic}: Cancels magical spells and effects.
  \item[5] \nameref{Spell:Commune}: Deity answers three yes-or-no questions.
  \item[9] \nameref{Spell:Miracle}: Channels divine energy to produce versatile effects.
\end{list}
\subsubsection{Air Domain}
\paragraph{Granted Powers}
\subparagraph{Driven Arrows (Su):} 
You can add the lower of your Cleric level and your Wisdom modifier to ranged weapon damage rolls (but not damage caused by ranged spells, spell-like abilities, or supernatural abilities) in place of your strength modifier, if doing so is advantageous to you. 
This is a supernatural ability that functions continuously.
\subparagraph{Shield of Air (Su):} 
Starting at Cleric level 3rd, you can expend your magical focus as an immediate action to deflect a ranged weapon attack made against you so that you take no damage from it. 
Ranged attacks generated by spell effects, as well as weapons that are travelling ethereally or are composed entirely of force can't be deflected.
\subparagraph{Wind Tunnel (Su):} 
Starting at Cleric level 5th, your ranged weapons are not hampered by conditions of severe wind, including magical wind such as the one generated by the \nameref{Spell:WindWall} spell. 
\subparagraph{Elemental Shape (Su):}
Starting at Cleric level 13th, whenever you cast a \nameref{Spell:FormElemental} on yourself to assume the form of an air elemental, 
the spell's duration becomes 1 hour per caster level, rather than the spell's normal duration.
This ability is activated as part of casting the spell.
\paragraph{Air Domain Spells}
\begin{list}{\labelitemi}{\leftmargin=1em}
\item[1] \nameref{Spell:EntropicShield}: Ranged attacks against you have 20\% miss chance.
\item[1] \nameref{Spell:Fog}: Fog surrounds you.
\item[1] \nameref{Spell:SummonAerialBeast}: Summons an airborne creature.
\item[2] \nameref{Spell:GustOfWind}: Blows away and knocks down creatures.
\item[3] \nameref{Spell:CallLightning}: Calls down lightning from the sky to blast your enemies.
\item[3] \nameref{Spell:GaseousForm}: Subject becomes insubstantial and can fly slowly.
\item[3] \nameref{Spell:SleetStorm}: Sleet hampers vision and movement.
\item[3] \nameref{Spell:WindWall}: Deflects arrows, knocks down creatures, blocks gases.
\item[4] \nameref{Spell:AirWalk}: Subject treads on air as if solid (climb at 45-degree angle).
\item[4] \nameref{Spell:IceStorm}: Hail deals 5d6 damage in cylinder 40 ft. across.
\item[4] \nameref{Spell:SummonAirElemental}*: Summons an elemental to do your bidding.
\item[5] \nameref{Spell:ControlWinds}: Change wind direction and speed.
\item[6] \nameref{Spell:ChainLightning}: 1d6/level damage to multiple subjects.
\item[7] \nameref{Spell:ControlWeather}: Changes weather in local area.
\item[7] \nameref{Spell:FormElemental}: Subject becomes a living creature of the elements.
\item[8] \nameref{Spell:Whirlwind}: Cyclone deals damage and can pick up creatures.
\item[9] \nameref{Spell:StormOfVengeance}: Storm rains acid, lightning, and hail.
\end{list}
\subsubsection{Animal Domain}
\label{Domain:Animal}
\paragraph{Granted Powers}
You add Handle Animal, Knowledge (nature), and Ride to your list of Cleric class skills.
\subparagraph{Wild Empathy (Ex):}
You gain the \nameref{sec:WildEmpathy} ability, as a Ranger of a level equal to your Cleric level.
If you also have levels in Ranger, use your combined number of levels in Cleric and Ranger to determine your effective level for this ability.
\subparagraph{Wild Shape (Su):}
Starting at Cleric level 5th, whenever you cast a \nameref{Spell:FormAvian}, \nameref{Spell:FormCarnivore}, 
\nameref{Spell:FormFish} or \nameref{Spell:FormScout} on yourself, 
the spell's duration becomes 1 hour per caster level, rather than the spell's normal duration.
This ability is activated as part of casting one of the listed spells.
\paragraph{Animal Domain Spells}
\begin{list}{\labelitemi}{\leftmargin=1em}
\item[1] \nameref{Spell:AlterAnimalsSize}: Animal changes size.
\item[1] \nameref{Spell:ConverseWithNature}: Communicate with animals and other unlikely creatures and objects.
% \item[1] \nameref{Spell:SummonMonster}: Calls extraplanar creature to fight for you.
\item[1] \nameref{Spell:SummonExoticBeast}: Summons a creature from the distant lands.
\item[1] \nameref{Spell:SummonForestPredator}: Summons a predator of the forests.
\item[1] \nameref{Spell:SummonReptile}: Summons a scaled creature.
\item[2] \nameref{Spell:AnimalMessenger}: Sends a Tiny animal to a specific place.
\item[2] \nameref{Spell:AnimalsMovement}: Grants additional movement capabilities.
\item[2] \nameref{Spell:HoldAnimal}: Paralyzes one animal for 1 round/level.
\item[3] \nameref{Spell:FormAvian}: Subject gains the form of a bird.
\item[3] \nameref{Spell:FormFish}: Subject gains the form of a water-dwelling creature.
\item[4] \nameref{Spell:FormCarnivore}: Subject gains the form of a dangerous beast.
\item[5] \nameref{Spell:Awaken}: Animal or tree gains human intellect.
\item[5] \nameref{Spell:BalefulPolymorph}: Transforms subject into harmless animal.
\item[7] \nameref{Spell:AnimalShapes}: One ally/level polymorphs into a different form.
\item[9] \nameref{Spell:Shapechange}: Transforms you into any creature whose form you know, and change forms once per round.
\end{list}
\subsubsection{Chaos Domain}
\paragraph{Granted Powers}
\subparagraph{Anarchic Shield (Ex):}
You gain damage reduction equal to one-half your Cleric level (minimum 1).
Your damage reduction is overcome by lawful-aligned weapons.

\subparagraph{Gift of Fate (Su):}
Starting at Cleric level 5th, you can expend your magical focus when making any single attack roll, skill check, ability check, or saving throw.
You roll your d20 twice when making the check, and use the better result.
You must decide whether or not to use it before you make the roll in question.
Using this ability does not take an action of its own, its activation is done as part of the activity that requires the roll.
You can use this ability once per day. At Cleric level 10th, and again at 15th level, you gain an additional use per day.

\subparagraph{Anarchic Strike (Ex):}
Starting at Cleric level 8th, any weapons you use are considered chaotic-aligned for the purpose of overcoming damage reduction.
\paragraph{Chaos Domain Spells}
\begin{list}{\labelitemi}{\leftmargin=1em}
\item[1] \nameref{Spell:AlignedProtection}: +2 to AC and saves, counter mind control, hedge out elementals and outsiders.
\item[1] \nameref{Spell:DiscernAlignment}: Reveals the subject's alignment.
\item[1] \nameref{Spell:RandomAction}: Forces a creature to act randomly.
\item[2] \nameref{Spell:HideousLaughter}: Subject loses actions for 1 round/level.
\item[2] \nameref{Spell:MaskAlignment}: Protects subject's alignment from being revealed via divinations.
\item[2] \nameref{Spell:Shatter}: Sonic vibration damages objects or crystalline creatures.
\item[4] \nameref{Spell:FistOfTheDeity}: Smites creatures of opposing alignment.
\item[5] \nameref{Spell:Atonement}: Removes burden of misdeeds from subject.
\item[5] \nameref{Spell:DispelAlignment}: Protects against creatures of the chosen alignment, discharge to drive creature away.
\item[5] \nameref{Spell:Reincarnate}: Creates a new, random body for a deceased creature to inhabit.
\item[6] \nameref{Spell:AnimateObjects}: Objects attack your foes.
\item[7] \nameref{Spell:WordOfGod}: Kills, paralyzes, hinders, or deafens subjects not of a selected alignment.
\item[8] \nameref{Spell:AlignedAura}: Protects creatures, better against creatures of an opposing alignment.
\end{list}
\subsubsection{Death Domain}
\subparagraph{Class Skill:}
You add Intimidate to your list of Cleric class skills.
\paragraph{Granted Powers}
\subparagraph{Deathtouched Presence (Ex):}
You gain a +4 bonus on Intimidate checks against living creatures, and a +4 bonus on Diplomacy checks when dealing with intelligent undead creatures.
Mindless undead creatures never attack you without provocation. Unless you attack, touch, or cast a spell at a nearby mindless undead creature, it acts as if you weren't there.
This is an ability that functions continuously, requiring no activation.
\subparagraph{Leader of the Dead (Ex):}
Your Cleric levels count more when calculating the number of creatures you can control via spells with the [\nameref{sec:MinionSpells}] descriptor.
The number of HD of creatures you can control this way becomes (2 + your charisma modifier)$\times$(2$\times$ your Cleric level + your number of other HD), rather than (2 + your charisma modifier)$\times$(Your number of HD).
This is an ability that functions continuously, requiring no activation.
\subparagraph{Animation Mastery (Ex):}
Each week, you can ignore a total number of 5 $\times$ your Cleric level points of experience cost associated with casting the spells \nameref{Spell:AnimateDead} and \nameref{Spell:CreateUndead}. This is not cumulative, if you have not cast the spells sufficiently often (or not created powerful enough undead) to put you over your limit in any given week, the excess is wasted.
This ability does not require activation, it is used as part of casting one of the referenced spells.
\paragraph{Death Domain Spells}
\begin{list}{\labelitemi}{\leftmargin=1em}
\item[1] \nameref{Spell:Fear}: One creature flees for 1d4 rounds.
\item[1] \nameref{Spell:LastLaugh}*: Destroy yourself in an attempt to take your enemies with you.
\item[1] \nameref{Spell:RayOfEnfeeblement}: Ray inflicts a strength penalty of 1d6.
\item[2] \nameref{Spell:CommandUndead}: Undead creature obeys your commands.
\item[2] \nameref{Spell:DeathKnell}: Kill dying creature and gain 1d8 temporary hp, and +2 to Str and your key ability score.
\item[2] \nameref{Spell:GentleRepose}: Preserves one corpse.
\item[3] \nameref{Spell:AnimateDead}: Creates undead skeletons and zombies.
\item[3] \nameref{Spell:Contagion}: Infects subject with chosen disease.
\item[3] \nameref{Spell:SpeakWithDead}: Corpse answers three questions.
\item[4] \nameref{Spell:BalefulResurrection}*: Returns subject from the dead - mostly.
\item[4] \nameref{Spell:Blight}: Withers one plant or deals 1d6/level damage to plant creature.
\item[4] \nameref{Spell:DeathWard}: Grants immunity to death spells and negative energy effects.
\item[5] \nameref{Spell:SlayLiving}: Touch attack kills subject.
\item[6] \nameref{Spell:CreateUndead}: Creates ghouls, and more powerful creatures with augment.
\item[7] \nameref{Spell:FingerOfDeath}: Kills one subject.
\item[9] \nameref{Spell:SoulBind}: Traps dead soul to prevent resurrection.
\end{list}
\subsubsection{Destruction Domain}
\paragraph{Granted Powers}
\subparagraph{Class Skill:} 
You add Knowledge (architecture and engineering) to your list of Cleric class skills.
\subparagraph{Smite (Su):}
In order to perform a Smite, you must expend your magical focus as part of making an attack.
The attack then gains a bonus on the attack roll equal to your Charisma modifier, and a bonus on the damage roll equal to your Cleric level.
You must decide whether or not to perform a Smite before making the the attack. 
If the attack misses, you still expend your magical focus.
This is a Supernatural ability, activated as part of making an attack.
\subparagraph{Bonus Feat:}
At Cleric level 4th, you gain the Improved Sunder feat, even if you do not meet the prerequisites.
\subparagraph{Effortless Destruction (Ex):}
Your spells and attacks ignore a number of points of hardness equal to twice your Cleric level.
\paragraph{Destruction Domain Spells}
\begin{list}{\labelitemi}{\leftmargin=1em}
\item[1] \nameref{Spell:Bleed}*: Bleeding deals 1d3 points of Constitution damage.
\item[1] \nameref{Spell:InflictWounds}: Deals 1d8 damage +1/level.
\item[2] \nameref{Spell:Shatter}: Sonic vibration damages objects or crystalline creatures.
\item[2] \nameref{Spell:SoundBurst}: Deals 1d8 sonic damage to subjects; may stun them.
\item[3] \nameref{Spell:CallLightning}: Calls down lightning from the sky to blast your enemies.
\item[3] \nameref{Spell:Contagion}: Infects subject with chosen disease.
\item[4] \nameref{Spell:RustingGrasp}: Your touch corrodes iron and alloys.
\item[5] \nameref{Spell:DisruptingWeapon}: Melee weapon destroys undead.
\item[5] \nameref{Spell:SlayLiving}: Touch attack kills subject.
\item[6] \nameref{Spell:Harm}: Deals 110 points of damage to target.
\item[6] \nameref{Spell:Disintegrate}: Makes one creature or object vanish.
\item[7] \nameref{Spell:Earthquake}: Intense tremor shakes 80-ft.-radius.
\item[8] \nameref{Spell:TransmuteToAcid}*: Transforms a 10' cube of solid matter into acid.
\item[9] \nameref{Spell:Implosion}: Kills one creature/round.
\end{list}
\subsubsection{Earth Domain}
\paragraph{Granted Powers}
\subparagraph{Hardened Weapons (Su):}
As a swift action, you can expend your magical focus to have any weapons you use bypass damage reduction and hardness as if they were made of adamantine for a number of rounds equal to your Cleric level.
\subparagraph{Earth's Gentle Embrace (Su):}
Starting at Cleric level 3rd, you never take penalties when moving through terrain that is difficult due to natural rock or earth formations (such as loose stone rubble or slippery mud). Vegetation and artificial terrain features still count as difficult terrain as normal.
In addition, whenever you fall on to a natural rock or earth surface, all falling damage you take is nonlethal damage.
\subparagraph{Meld into Stone (Su):}
Starting at Cleric level 5th, whenever you cast a \nameref{Spell:MeldIntoStone} spell on yourself, the spell's duration becomes 1 day per caster level, rather than the spell's normal duration.
This ability is activated as part of casting the spell.
\subparagraph{Elemental Shape (Su):}
Starting at Cleric level 13th, whenever you cast a \nameref{Spell:FormElemental} on yourself to assume the form of an earth elemental, 
the spell's duration becomes 1 hour per caster level, rather than the spell's normal duration.
This ability is activated as part of casting the spell.
\paragraph{Earth Domain Spells}
\begin{list}{\labelitemi}{\leftmargin=1em}
\item[1] \nameref{Spell:ConverseWithNature}: Communicate with animals and other unlikely creatures and objects.
\item[1] \nameref{Spell:MagicStone}: Launches a magical stone, dealing 1d6+1 damage.
\item[2] \nameref{Spell:SoftenEarthAndStone}: Turns stone to clay or dirt to sand or mud.
\item[3] \nameref{Spell:MeldIntoStone}: Subject and its gear merge with stone.
\item[3] \nameref{Spell:MoldMaterial}: Sculpts material into any shape.
\item[3] \nameref{Spell:SummonEarthElemental}: Summons an elemental to do your bidding.
\item[4] \nameref{Spell:StepThroughEarth}*: Travel instantaneously through the ground.
\item[4] \nameref{Spell:Stoneskin}: Ignore 7 points of damage per attack.
\item[4] \nameref{Spell:SpikeStones}: Creatures in area take 4d8 damage, may be slowed.
\item[5] \nameref{Spell:CommuneWithNature}: Learn about terrain for 1 mile/level.
\item[5] \nameref{Spell:TransmuteRockAndMud}: Transforms two 10-ft. cubes per level.
\item[5] \nameref{Spell:WallOfStone}: Creates a stone wall that can be shaped.
\item[6] \nameref{Spell:MoveEarth}: Digs trenches and builds hills.
\item[7] \nameref{Spell:Earthquake}: Intense tremor shakes 80-ft.-radius.
\item[7] \nameref{Spell:FormElemental}: Subject becomes a living creature of the elements.
\item[8] \nameref{Spell:RepelMetalAndStone}: Pushes away metal and stone.
\item[8] \nameref{Spell:FormIronGolem}: Subject's body changes into a creature of living iron.
%\item[9] Elemental Swarm*: Summons multiple elementals.
\end{list}
\subsubsection{Evil Domain}
\paragraph{Granted Powers}
\subparagraph{Unholy Shield (Ex):} 
You gain damage reduction equal to one-half your Cleric level (minimum 1).
Your damage reduction is overcome by good-aligned weapons.
\subparagraph{Gift of Darkness (Su):}
Starting at Cleric level 5th, whenever you drop (usually by reducing it to -1 HP or below) 
a sentient (Int 3 or higher), living creature, score a critical hit, damage a cowering or helpless opponent, 
or cast a spell with the [Evil] descriptor, you can expend your magical focus to gain a +1 profane bonus on attack and damage rolls for one minute.
Using this ability does not take an action of its own, its activation is done as part of the activity that allowed you to use it.
At Cleric level 10th, the profane bonus increases to +2. At 15th level, it increases to +3, and at 20th level to +4.
\subparagraph{Unholy Strike (Ex):}
Starting at Cleric level 8th, any weapons you use are considered evil-aligned for the purpose of overcoming damage reduction.
\paragraph{Evil Domain Spells}
\begin{list}{\labelitemi}{\leftmargin=1em}
\item[1] \nameref{Spell:AlignedProtection}: +2 to AC and saves, counter mind control, hedge out elementals and outsiders.
\item[1] \nameref{Spell:DiscernAlignment}: Reveals the subject's alignment.
\item[1] \nameref{Spell:Fear}: One creature flees for 1d4 rounds.
\item[2] \nameref{Spell:Blindness}: Negates one of the subject's senses.
\item[2] \nameref{Spell:Darkness}: 20-ft. radius of supernatural shadow.
\item[2] \nameref{Spell:Desecrate}: Fills area with negative energy, making undead stronger.
\item[2] \nameref{Spell:MaskAlignment}: Protects subject's alignment from being revealed via divinations.
\item[3] \nameref{Spell:BestowCurse}: -6 to an ability score; -4 on attack rolls, saves, and checks; or 50\% chance of losing each action.
\item[3] \nameref{Spell:Poison}: Creates a temporary batch of poison.
\item[4] \nameref{Spell:FistOfTheDeity}: Smites creatures of opposing alignment.
\item[5] \nameref{Spell:Atonement}: Removes burden of misdeeds from subject.
\item[5] \nameref{Spell:DispelAlignment}: Protects against creatures of the chosen alignment, discharge to drive creature away.
\item[6] \nameref{Spell:CreateUndead}: Creates ghouls, and more powerful creatures with augment.
\item[7] \nameref{Spell:WordOfGod}: Kills, paralyzes, hinders, or deafens subjects not of a selected alignment.
\item[8] \nameref{Spell:AlignedAura}: Protects creatures, better against creatures of an opposing alignment.
%\item[9] \nameref{Spell:WailOfTheBanshee}: Kills multiple creatures.
\end{list}
\subsubsection{Fire Domain}
\paragraph{Granted Powers}
\subparagraph{Friendly Fire (Su):}
You gain resistance to fire equal to twice your Cleric level. This is a supernatural ability that functions continuously.
\subparagraph{Combust (Su):}
As a standard action, you can set an unattended object within 30' on fire, or deal 1d3 points of fire damage to a target within 30' by succeeding on a ranged touch attack against it.
\subparagraph{Divine Fire (Su):} 
Starting at Cleric level 5th, you can change $1/4$ of the fire damage you deal with spells that appear on the Fire domain list into divine damage, which is not subject to fire resistance or other protections from fire.
At 10th level, the ratio of damage so converted rises to $1/2$, and at 15th level to $3/4$.
This is a supernatural ability that is activated as part of casting each spell in question.
You can choose to not activate it (to deal more damage to a creature with the cold subtype, for example).
\subparagraph{Elemental Shape (Su):}
Starting at Cleric level 13th, whenever you cast a \nameref{Spell:FormElemental} on yourself to assume the form of a fire elemental, 
the spell's duration becomes 1 hour per caster level, rather than the spell's normal duration.
This ability is activated as part of casting the spell.
\paragraph{Fire Domain Spells}
\begin{list}{\labelitemi}{\leftmargin=1em}
\item[1] \nameref{Spell:FaerieFire}: Outlines subjects with light, canceling blur, concealment, and the like.
\item[1] \nameref{Spell:Light}: Causes object to shine like a torch.
\item[1] \nameref{Spell:ProduceFlame}: Create a fire in your hand that you can throw or attack with in melee.
\item[2] \nameref{Spell:FlameBlade}: Creates a weapon of pure fire.
\item[2] \nameref{Spell:FlamingSphere}: Creates rolling ball of fire, 2d6 damage, lasts 1 round/level.
\item[2] \nameref{Spell:HeatChillMetal}: Makes metal extremely hot or extremely cold, damaging those who touch it.
\item[2] \nameref{Spell:Pyrotechnics}: Turns fire into blinding light or choking smoke.
\item[3] \nameref{Spell:SummonFireElemental}: Summons an elemental to do your bidding.
\item[4] \nameref{Spell:AuraOfFire}: Enemies within range take damage, more if they attack you.
\item[4] \nameref{Spell:FlameStrike}: Smite foes with divine fire, knocking them down and dealing 7d6 damage.
\item[4] \nameref{Spell:WallOfFire}: Deals 2d4 fire damage out to 10 ft. Passing through wall deals 7d6 damage.
\item[6] \nameref{Spell:DeadlyFog}: Add elemental component to fog, causing it to deal damage.
\item[6] \nameref{Spell:FireSeeds}: Acorns and berries become grenades and bombs.
\item[7] \nameref{Spell:FireStorm}: Deals 13d6 fire damage.
\item[7] \nameref{Spell:FormElemental}: Subject becomes a living creature of the elements.
\item[9] \nameref{Spell:Soulfire}*: Links your soul to that of another creature, at great peril to both of you.
\end{list}
\subsubsection{Good Domain}
\paragraph{Granted Powers}
\subparagraph{Holy Shield (Ex):}You gain damage reduction equal to one-half your Cleric level (minimum 1).
Your damage reduction is overcome by evil-aligned weapons.

\subparagraph{Gift of Compassion (Su):}
Starting at Cleric level 5th, whenever you stabilize a dying creature, take damage due to a \nameref{Spell:ShieldOther} spell you have cast, 
use the aid another action, or cast a spell with the [Good] descriptor or one that heals another creature's hit point damage, 
you can expend your magical focus to gain a +1 sacred bonus on attack and damage rolls for one minute.
Using this ability does not take an action of its own, its activation is done as part of the activity that allowed you to use it.
At Cleric level 10th, the sacred bonus increases to +2. At 15th level, it increases to +3, and at 20th level to +4.

\subparagraph{Holy Strike (Ex):}
Starting at Cleric level 8th, any weapons you use are considered good-aligned for the purpose of overcoming damage reduction.
\paragraph{Good Domain Spells}
\begin{list}{\labelitemi}{\leftmargin=1em}
\item[1] \nameref{Spell:AlignedProtection}: +2 to AC and saves, counter mind control, hedge out elementals and outsiders.
\item[1] \nameref{Spell:Bless}: Allies gain +1 on attack rolls and +1 on saves against fear.
\item[1] \nameref{Spell:DiscernAlignment}: Reveals the subject's alignment.
\item[2] \nameref{Spell:Aid}: +1 on attack rolls, +1 on saves against fear, 1d8 temporary hp +1/level.
\item[2] \nameref{Spell:Consecrate}: Fills area with positive energy, making undead weaker.
\item[2] \nameref{Spell:MaskAlignment}: Protects subject's alignment from being revealed via divinations.
\item[2] \nameref{Spell:ShieldOther}: You take half of subject's damage.
\item[4] \nameref{Spell:FistOfTheDeity}: Smites creatures of opposing alignment.
\item[5] \nameref{Spell:Atonement}: Removes burden of misdeeds from subject.
\item[5] \nameref{Spell:DispelAlignment}: Protects against creatures of the chosen alignment, discharge to drive creature away
\item[6] \nameref{Spell:BladeBarrier}: Wall of blades deals 11d6 damage.
\item[7] \nameref{Spell:WordOfGod}: Kills, paralyzes, hinders, or deafens subjects not of a selected alignment.
\item[8] \nameref{Spell:AlignedAura}: Protects creatures, better against creatures of an opposing alignment.
\end{list}
\subsubsection{Healing Domain}
\paragraph{Granted Powers}
\subparagraph{Healer's Talents (Sp):}
You can use the following spells as spell-like abilities a combined number of times per day equal to your Wisdom modifier if you have the required number of ranks in the Heal skill, as indicated for each spell in question:
\begin{list}{\labelitemi}{\leftmargin=1em}
 \item \emph{6 ranks:} \nameref{Spell:RemoveFear}
 \item \emph{8 ranks:} \nameref{Spell:RemoveParalysis}
 \item \emph{10 ranks:} \nameref{Spell:RemoveBlindnessDeafness}
 \item \emph{10 ranks:} \nameref{Spell:RemoveDisease}
 \item \emph{10 ranks:} \nameref{Spell:RemoveCurse}
\end{list}
Your caster level for these spell-like abilities is equal to your character level.
\subparagraph{Positive Energy Infused Spellcasting (Su):}
Starting at Cleric level 6th, whenever you cast a divine spell, you can expend your magical focus to grant any one creature affected by the spell temporary hit points equal to your Cleric level. 
At Cleric level 12th, all creatures affected by your divine spells gain this benefit when you expend your magical focus.
Temporary hit points do not stack.
This is a supernatural ability whose activation is subsumed in casting the spell in question.
\subparagraph{Deathwatch (Su):}
You can determine the condition of all creatures within 30' of which you are aware. You instantly know whether each creature within the area is dead, near death (alive, with less than 1/4 of its hit points remaining), wounded (alive with less than 1/2 of its hit points remaining), lightly wounded (alive with less than 3/4 of its hit points remaining) or in good health (alive with more than 3/4 of its hit points remaining).
You also know if a creature is undead, or neither alive nor dead (such as a construct). You cannot learn the health of such creatures.
This is a supernatural ability that functions continuously, requiring no activation.
\paragraph{Healing Domain Spells}
\begin{list}{\labelitemi}{\leftmargin=1em}
\item[1] \nameref{Spell:CureWounds}: Cures 1d8 damage +1/level.
\item[1] \nameref{Spell:LifeLink}: Monitors status and position of allies.
\item[1] \nameref{Spell:RemoveFear}: Subject gains immunity to fear.
\item[2] \nameref{Spell:Antipoison}: Stops poison from harming subject for 1 hour/level.
\item[2] \nameref{Spell:RemoveParalysis}: Frees one creature from the effects of paralysis.
\item[2] \nameref{Spell:Restoration}: Dispels magical ability penalty or repairs 1d4 ability damage.
\item[3] \nameref{Spell:RemoveBlindnessDeafness}: Cures normal or magical conditions impeding senses.
\item[3] \nameref{Spell:RemoveCurse}: Frees object or person from curse.
\item[3] \nameref{Spell:RemoveDisease}: Cures all diseases affecting subject.
\item[5] \nameref{Spell:RaiseDead}: Brings a creature back from the dead.
\item[6] \nameref{Spell:Heal}: Cures great amounts of of damage, all diseases and mental conditions.
\item[7] \nameref{Spell:Regenerate}: Subject's severed limbs grow back, grants fast healing.
\end{list}
\subsubsection{Knowledge Domain}
\label{Domain:Knowledge}
\paragraph{Granted Powers}
\subparagraph{Class Skills:}
Add all Knowledge skills to your list of Cleric class skills.
\subparagraph{Lore (Ex):}
Thanks to long hours of study, you have a wide range of stray knowledge. This ability is identical to the \nameref{sec:Bard}'s Bardic Knowledge class feature, using your Cleric level in place of the Bard level. This counts as Bardic Knowledge for all purposes, including those of prerequisites. If you have Bard levels, you do not use the abilities separately, but your Cleric and Bard levels stack for purposes of determining your bonus on the check.
\subparagraph{Applied Knowledge (Ex):}
Starting at Cleric level 4th, whenever you fight a creature, you may make a knowledge check as appropriate to its type (Arcana for dragons, Local for humanoids, and so on).
This check has a DC equal to the one required to identify monsters and their special powers or vulnerabilities (10 + the monster's HD), but it is separate from that check (if you choose to also roll it).
If the check succeeds, you gain a +1 insight bonus on attack and damage rolls against that creature for the remainder of the fight.
If the check fails, you gain no bonus and may not try again.
Check separately for each creature type involved in the fight.
You must be aware of the creature you are fighting in order to roll this check.
You do not have to roll all checks at the start of the combat (although it is advantageous to do so as soon as you become aware of your combatants), you may, for example, roll again if creatures of a new type join the combat later on.

At Cleric level 8th, the insight bonus increases to +2, at 12th level it increases to +3, to +4 at 16th level, and +5 at 20th level.
\subparagraph{Gift of Knowledge (Su):}
Starting at Cleric level 5th, you can expend your magical focus when making any single Knowledge check.
You roll your d20 twice when making the check, and use the better result.
You must decide whether or not to use it before you make the roll in question.
Using this ability does not take an action of its own any more than rolling a Knowledge check usually does.
You can use this ability once per day. At Cleric level 10th, and again at 15th level, you gain an additional use per day.
\paragraph{Knowledge Domain Spells}
\begin{list}{\labelitemi}{\leftmargin=1em}
\item[1] \nameref{Spell:DetectSecretDoors}: Become aware of all secret doors within your line of sight.
\item[1] \nameref{Spell:DetectPoison}: Detects poison or disease in one creature or small object.
\item[1] \nameref{Spell:ComprehendLanguages}: You understand all spoken and written languages.
\item[2] \nameref{Spell:Clairvoyance}: See and hear a distant location.
\item[2] \nameref{Spell:Locate}: Senses direction toward object (specific or type).
\item[2] \nameref{Spell:ReadThoughts}: Detect surface thoughts of creatures in range.
\item[2] \nameref{Spell:TrapIntuition}: Notice and disable traps as a rogue does.
\item[4] \nameref{Spell:Scrying}: Spies on subject from a distance.
\item[4] \nameref{Spell:Sending}: Delivers short message anywhere, instantly.
\item[5] \nameref{Spell:TrueSeeing}: Lets you see all things as they really are.
\item[6] \nameref{Spell:LegendLore}: Lets you learn tales about a person, place, or thing.
\item[8] \nameref{Spell:AbsoluteRevelation}: Reveals exact location of creature or object.
\item[8] \nameref{Spell:MomentOfPrescience}: You gain insight bonus on single attack roll, check, or save.
\item[9] \nameref{Spell:Foresight}: ``Sixth sense`` warns of impending danger.
\end{list}
\subsubsection{Law Domain}
\paragraph{Granted Powers}
\subparagraph{Axiomatic Shield (Ex):}
You gain damage reduction equal to one-half your Cleric level (minimum 1).
Your damage reduction is overcome by chaotic-aligned weapons.

\subparagraph{Gift of Order (Su):}
Starting at Cleric level 5th, you can expend your magical focus to ''take 12`` on a single attack roll, skill check, ability check, or saving throw.
Instead of rolling a d20 for the check, calculate your result as if you had rolled a 12. 
You can use this ability even under duress.
You must decide whether or not to use it before you make the roll in question.
Using this ability does not take an action of its own, its activation is done as part of the activity that requires the roll.
You can use this ability once per day. At Cleric level 10th, and again at 15th level, you gain an additional use per day.
\subparagraph{Axiomatic Strike (Ex):}
Starting at Cleric level 8th, any weapons you use are considered lawful-aligned for the purpose of overcoming damage reduction.
\paragraph{Law Domain Spells}
\begin{list}{\labelitemi}{\leftmargin=1em}
\item[1] \nameref{Spell:Command}: One subject obeys selected command for 1 round.
\item[1] \nameref{Spell:AlignedProtection}: +2 to AC and saves, counter mind control, hedge out elementals and outsiders.
\item[1] \nameref{Spell:DiscernAlignment}: Reveals the subject's alignment.
\item[2] \nameref{Spell:CalmEmotions}: Calms creatures, negating emotion effects.
\item[2] \nameref{Spell:HoldPerson}: Paralyzes one humanoid for 1 round/level.
\item[2] \nameref{Spell:MaskAlignment}: Protects subject's alignment from being revealed via divinations.
\item[2] \nameref{Spell:ZoneOfTruth}: Subjects within field find it extremely hard to lie.
\item[4] \nameref{Spell:FistOfTheDeity}: Smites creatures of opposing alignment.
\item[4] \nameref{Spell:GeasQuest}: Commands subject of 7 HD or less.
\item[5] \nameref{Spell:Atonement}: Removes burden of misdeeds from subject. 
\item[5] \nameref{Spell:DispelAlignment}: Protects against creatures of the chosen alignment, discharge to drive creature away
\item[5] \nameref{Spell:MarkOfJustice}: Designates action that will trigger curse on subject.
\item[7] \nameref{Spell:WordOfGod}: Kills, paralyzes, hinders, or deafens subjects not of a selected alignment.
\item[8] \nameref{Spell:AlignedAura}: Protects creatures, better against creatures of an opposing alignment.
\end{list}
\subsubsection{Luck Domain}
\paragraph{Granted Powers}
\subparagraph{Extraordinary Luck (Ex):}
You can expend your magical focus to reroll any single attack roll, skill check, ability check, or saving throw.
You must decide whether or not to use this ability after you have rolled the original check, but before the GM has announced the results of that roll.
You must abide by the results of the second roll, even if it is worse than the original roll.
You cannot use this ability again to reroll a second time (even if you have two magical focuses to expend and have uses of the ability remaining).
Using this ability does not take an action of its own, its activation is done as part of the activity that requires the roll.
You can use this ability once per day. At Cleric level 4th, and again at levels 8, 12, and 16, you gain an additional use per day.
\subparagraph{Implausible Resilience (Ex):} 
When dying due to your hit points being reduced to -1 or below, you have a 50\% chance of becoming stable each round rather than a 10\% chance.
\subparagraph{Implausible Avoidance (Su)}: 
Starting at 5th level, falling objects never hit you, effectively making you immune to damage from such objects. This includes damage from objects that are purposefully dropped on top of you (but not normal ranged attacks), as well as traps that deal damage by dropping objects (or roofs, or similar) on you. Even if you never take damage from falling objects, you can still be affected by secondary effects, such as by being buried under rubble. This is a supernatural ability that functions continuously.
\paragraph{Luck Domain Spells}
\begin{list}{\labelitemi}{\leftmargin=1em}
\item[1] \nameref{Spell:Bless}: Allies gain +1 on attack rolls and +1 on saves against fear.
\item[1] \nameref{Spell:EntropicShield}: Ranged attacks against you have 20\% miss chance.
\item[1] \nameref{Spell:TrueStrike}: +20 on your next attack roll.
\item[1] \nameref{Spell:DivineFavor}: You gain a luck bonus on attack and damage rolls.
\item[2] \nameref{Spell:Aid}: +1 on attack rolls, +1 on saves against fear, 1d8 temporary hp +1/level.
\item[3] \nameref{Spell:BestowCurse}: -6 to an ability score; -4 on attack rolls, saves, and checks; or 50\% chance of losing each action.
\item[3] \nameref{Spell:RemoveCurse}: Frees object or person from curse.
\item[4] \nameref{Spell:FreedomOfMovement}: Subject moves normally despite impediments.
\item[6] \nameref{Spell:Mislead}: Turns you invisible and creates illusory double.
\item[7] \nameref{Spell:SpellTurning}: Reflect spells back at caster.
\item[8] \nameref{Spell:MomentOfPrescience}: You gain insight bonus on single attack roll, check, or save.
\item[9] \nameref{Spell:FortuneOfTheGods}*: Make all d20 rolls twice, and choose your preferred result.
\end{list}
\subsubsection{Magic Domain}
\paragraph{Granted Powers}
\subparagraph{Class Skill:}
You add Use Magic Device to your list of Cleric class skills.
\subparagraph{Cantrips (Su):}
You gain the \nameref{sec:Cantrips} class feature, as a Wizard.

\subparagraph{Immediate Counterspelling (Su):} 
At fourth level, you gain the ability to counterspell by expending your magical focus as an immediate action.
You do not need to take the ready action in your previous round, the entire counterspell attempt is performed as an immediate action.
Counterspell attempts are otherwise handled normally (see \nameref{sec:Counterspells}).
You can use this ability once per day. At Cleric levels 8, 12 and 16, you gain an additional daily use of this ability.
\paragraph{Magic Domain Spells}
\begin{list}{\labelitemi}{\leftmargin=1em}
\item[1] \nameref{Spell:MagicAura}: Alters object's magic aura.
\item[1] \nameref{Spell:MagicMissile}: Deal 1d4+1 damage, no save, no touch attack.
\item[1] \nameref{Spell:Identify}: Determines properties of magic item.
\item[2] \nameref{Spell:DweomerRip}*: Damages creatures with active magical effects on them.
\item[3] \nameref{Spell:InvisibilityPurge}: Suppresses invisibility within 5 ft./level.
\item[3] \nameref{Spell:GlyphOfWarding}: Inscription harms those who pass it.
\item[3] \nameref{Spell:SuppressMagic}*: Renders willing spellcaster incapable of casting spells.
\item[4] \nameref{Spell:ImbueWithSpellAbility}: Transfer spells to subject.
\item[5] \nameref{Spell:SpellResistance}: Subject gains SR 12 + level.
\item[6] \nameref{Spell:AntimagicField}: Negates magic within 10 ft.
\item[7] \nameref{Spell:SpellTurning}: Reflect spells back at caster.
\item[7] \nameref{Spell:LimitedWish}: Alters reality - within spell limits.
\item[9] \nameref{Spell:Disjunction}: Dispels magic, disenchants magic items.
\end{list}
\subsubsection[Moon Domain]{Moon Domain*}
\paragraph{Granted Powers}
\subparagraph{Class Skill:}
Add Hide to your list of Cleric class skills.
\subparagraph{Low-Light Vision:} You gain the low-light vision special quality. If you already have low-light vision, you gain superior low-light vision instead (quadruple the distance, instead of double).
\subparagraph{Hide in Plain Sight (Su):} Starting at Cleric level 12th, you can use the Hide skill even while being observed, as long as you are within the area of some sort of shadow.
This ability does not need any activation beyond making a hide check.
\subparagraph{Moonlight Weapons (Su):}
As a swift action, you can expend your magical focus to have any weapons you use bypass damage reduction as if they were made of silver for a number of rounds equal to your Cleric level.
\paragraph{Moon Domain Spells}
\begin{list}{\labelitemi}{\leftmargin=1em}
 \item[1] \nameref{Spell:FaerieFire}: Outlines subjects with light, canceling blur, concealment, and the like.
 \item[1] \nameref{Spell:Light}: Causes object to shine like a torch.
 \item[2] \nameref{Spell:MoonBolt}*: Blasts creature with silvery light, damaging and stunning it.
 \item[2] \nameref{Spell:Moonlust}*: You enchant a creature with visions of the moon, preventing it from acting.
 \item[2] \nameref{Spell:Darkness}: 20-ft. radius of supernatural shadow.
 \item[3] \nameref{Spell:CallOfTheMoon}*: Reveals lycanthropes as what they are.
 \item[3] \nameref{Spell:SummonNaturesAlly}: Summons a natural guardian to aid you.
 \item[3] \nameref{Spell:Wolfsbane}*: Deals debilitating damage to lycanthropes.
 \item[3] \nameref{Spell:Blink}: Subject randomly vanishes and reappears for 1 round/level.
 \item[4] \nameref{Spell:Confusion}: Subjects behave oddly for 1 round/level.
 \item[5] \nameref{Spell:Dream}: Contact or disturb sleeping creature.
 \item[6] \nameref{Spell:ShadowWalk}: Step into shadow to travel rapidly.
 \item[7] \nameref{Spell:CurseOfLycanthropy}*: You infect a creature with lycanthropy.
 \item[8] \nameref{Spell:IrresistibleDance}: Forces subject to dance.
 \item[9] \nameref{Spell:Starfall}*: Brings down a shower of Moon Bolts.
\end{list}
\subsubsection[Planes Domain]{Planes Domain*}
\paragraph{Granted Powers:}
\subparagraph{Class Skills:} You add the Speak Language skill for any language spoken by natives of planes other than the material plane to your list of Cleric class skills.
\subparagraph{Darkvision (Ex):} 
You gain the Darkvision special quality, out to 60'. 
\subparagraph{Outsider Assumption (Ex):}
Starting at Cleric level 4th, you no longer need to eat. At Cleric level 12th, you no longer need to sleep, although you must still rest to regain spell points, as before.
\subparagraph{Planar Attunement (Ex):}
Starting at Cleric level 7th, you can selectively ignore the alignment traits of any plane you visit (treating it as a mildly neutral-aligned plane instead).
\subparagraph{Planar Ward (Su):}
Starting at Cleric level 10th, you are constantly considered to be under the influence of an \nameref{Spell:AlignedProtection} spell, warding against one alignment opposite that of your own (for example, if you are a Lawful Good Cleric, you could choose to ward against Chaos or Evil), using your Cleric level as its caster level. A true neutral Cleric gains no benefit from this granted power.
You can activate or dismiss this effect as a free action. If you can select to ward against more than one alignment, you can choose to change the alignment each time you re-activate the ability.
The effect protects only yourself. You do not gain the benefit of the Aligned Protection's augment.
% \subparagraph{Planar Navigation (Ex):}
% Starting at Cleric level 10th, any \nameref{Spell:PlaneShift} spell you cast automatically gains the benefit of its augment, with no additional spell point expenditure on your behalf.

% Evil: Poison immunity, Telepathy, acid & cold res.
% Good: Petrification immunity, Tongues.
% Law: ???
% Chaos: Electricity Immunity.
% 
% Demon Traits
% Immunity to electricity and poison.
% Resistance to acid 10, cold 10, and fire 10.
% Summon (Sp): Many demons share the ability to summon others of their kind (the success chance and type of demon summoned are noted in each monster description). Demons are often reluctant to use this power until in obvious peril or extreme circumstances.
% Telepathy.
% 
% Devil Traits
% Immunity to fire and poison.
% Resistance to acid 10 and cold 10.
% See in Darkness (Su): Some devils can see perfectly in darkness of any kind, even that created by a deeper darkness spell.
% Summon (Sp): Some devils share the ability to summon others of their kind (the success chance and type of devils summoned are noted in each monster description).
% Telepathy.
% 
% Archon Traits
% Darkvision out to 60 feet and low-light vision.
% Aura of Menace (Su): A righteous aura surrounds archons that fight or get angry. Any hostile creature within a 20-foot radius of an archon must succeed on a Will save to resist its effects. The save DC varies with the type of archon, is Charisma-based, and includes a +2 racial bonus. Those who fail take a -2 penalty on attacks, AC, and saves for 24 hours or until they successfully hit the archon that generated the aura. A creature that has resisted or broken the effect cannot be affected again by the same archon's aura for 24 hours.
% Immunity to electricity and petrification.
% +4 racial bonus on saves against poison.
% Magic Circle against Evil (Su): A magic circle against evil effect always surrounds an archon (caster level equals the archon's Hit Dice). (The defensive benefits from the circle are not included in an archon's statistics block.)
% Teleport (Su): Archons can use greater teleport at will, as the spell (caster level 14th), except that the creature can transport only itself and up to 50 pounds of objects.
% Tongues (Su): All archons can speak with any creature that has a language, as though using a tongues spell (caster level 14th). This ability is always active.
% 
% Angel Traits
% Darkvision out to 60 feet and low-light vision.
% Immunity to acid, cold, and petrification.
% Resistance to electricity 10 and fire 10.
% +4 racial bonus on saves against poison.
% Protective Aura (Su): Against attacks made or effects created by evil creatures, this ability provides a +4 deflection bonus to AC and a +4 resistance bonus on saving throws to anyone within 20 feet of the angel. Otherwise, it functions as a magic circle against evil effect and a lesser globe of invulnerability, both with a radius of 20 feet (caster level equals angel's HD). This aura can be dispelled, but the angel can create it again as a free action on its next turn. (The defensive benefits from the circle are not included in an angel's statistics block.)
% Tongues (Su): All angels can speak with any creature that has a language, as though using a tongues spell (caster level equal to angel's Hit Dice). This ability is always active.
% 
% Eladrin Traits
% Immunities: Eladrins are immune to electricity and petrification.
% Energy Resistance (Ex): Eladrins have resistance to acid 10 and cold 10.
% Tongues (Su): Eladrins can speak with any creature that has a language, as though using a tongues spell cast by a 14th-level cleric.
% This ability is always active.
\subparagraph{Ascension (Ex):}
At Cleric level 20th, you are transformed into a creature of the planes.
You are forevermore treated as an outsider rather than as a humanoid (or whatever your creature type was) for the purpose of spells and magical effects.
Unlike other outsiders, you can choose to have spells affect you as if you were a member of you previous creature type (chosen when each individual spell in question is cast).
\paragraph{Planes Domain Spells}
\begin{list}{\labelitemi}{\leftmargin=1em}
\item[1] \nameref{Spell:FreeStep}*: Teleports you a very short distance.
% \item[1] \nameref{Spell:SummonMonster}: Calls extraplanar creature to fight for you.
\item[2] \nameref{Spell:SummonDevil}: Summons a devil to fight for you.
\item[3] \nameref{Spell:AstralTrap}*: Temporarily transports a creature far away.
\item[3] \nameref{Spell:Piggyback}*: Latch on to a creature as it teleports, retaining your relative position.
\item[3] \nameref{Spell:SummonDemon}: Summons a demon to fight for you.
\item[4] \nameref{Spell:DimensionalAnchor}: Bars extradimensional movement.
\item[4] \nameref{Spell:Dismissal}: Forces a creature to return to native plane.
\item[4] \nameref{Spell:PlanarAlly}: Request aid from an extraplanar creature.
\item[4] \nameref{Spell:Sending}: Delivers short message anywhere, instantly.
\item[4] \nameref{Spell:SummonCelestial}: Summons a celestial to defend you.
\item[6] \nameref{Spell:FormCelestial}*: You assume the form of a majestic celestial.
\item[6] \nameref{Spell:FormFiend}*: You assume the form of a terrifying fiend.
\item[6] \nameref{Spell:Forbiddance}: Blocks planar travel, damages creatures of selected alignments.
\item[7] \nameref{Spell:PlaneShift}: As many as eight subjects travel to another plane.
\item[8] \nameref{Spell:DimensionalLock}: Teleportation and interplanar travel blocked for one day/level within an area.
\item[9] \nameref{Spell:Gate}: Connects two planes for travel or calling.
\end{list}
\subsubsection{Plant Domain}
\paragraph{Granted Powers}
\subparagraph{Class Skill:}
Add Knowledge (nature) to your list of Cleric class skills.
\subparagraph{Woodland Stride (Ex):}
Starting at 2nd level, you may move through any sort of undergrowth (such as natural thorns, briars, overgrown areas, and similar terrain) at her normal speed and without taking damage or suffering any other impairment. However, thorns, briars, and overgrown areas that have been magically manipulated to impede motion still affect you.
\subparagraph{Trackless Step (Ex):}
Starting at 3rd level, a you leave no trail in natural surroundings and cannot be tracked. You may choose to leave a trail if so desired.
\subparagraph{Plant Venom Immunity (Ex):}
At 9th level, you gain immunity to all plant-based poison.
\subparagraph{Plant Shape (Su):}
Starting at Cleric level 9th, whenever you cast a \nameref{Spell:FormTreant} spell on yourself, the spell's duration becomes 1 hour per caster level, rather than the spell's normal duration.
This ability is activated as part of casting the spell.
\paragraph{Plant Domain Spells}
\begin{list}{\labelitemi}{\leftmargin=1em}
\item[1] \nameref{Spell:ConverseWithNature}: Communicate with animals and other unlikely creatures and objects.
\item[1] \nameref{Spell:Entangle}: Plants entangle everyone in 20-ft.-radius.
\item[1] \nameref{Spell:Woodbolt}*: Creates a spear that flies at your enemies.
\item[1] \nameref{Spell:Shillelagh}: Touched bludgeoning weapon deals damage as if it were larger than it really is.
\item[2] \nameref{Spell:Barkskin}: Grants +2 enhancement to natural armor.
\item[2] \nameref{Spell:CommandNaturesAllies}: Sway the actions of one a elemental, fey, or plant creature.
\item[2] \nameref{Spell:FormPlant}: You look exactly like a plant for 1 hour/level.
\item[3] \nameref{Spell:PlantGrowth}: You encourage or inhibit the growth of plants in a large area.
\item[3] \nameref{Spell:SpikeGrowth}: Creatures in area take 4d8 damage, may be slowed.
\item[3] \nameref{Spell:SummonNaturesAlly}: Summons a natural guardian to aid you.
\item[5] \nameref{Spell:FormTreant}: Subject gains the form of a plant creature.
\item[5] \nameref{Spell:Awaken}: Animal or tree gains human intellect.
\item[5] \nameref{Spell:WallOfThorns}: Thorns damage anyone who tries to pass.
\item[6] \nameref{Spell:AnimatePlants}: One or more trees animate and fight for you.
\item[6] \nameref{Spell:Liveoak}: Changes an oak into a treant.
\item[6] \nameref{Spell:TransportViaPlants}: Move instantly from one plant to another of the same kind.
\item[6] \nameref{Spell:RepelWood}: Pushes away wooden objects.
\item[7] \nameref{Spell:TransmuteMetalAndWood}: Transmutes a metal item to wood, and back again.
\item[7] \nameref{Spell:ControlNaturesAllies}: You link with a natural creature and make it do as you command.
\item[9] \nameref{Spell:AwakenNaturesWrath}: Animates the natural world and causes it to attack intruders.
\end{list}
\subsubsection{Protection Domain}
\paragraph{Granted Powers}
\subparagraph{Aura of Protection (Su):}
You can expend your magical focus as an immediate action to grant every ally within 30' a +2 sacred bonus to AC (when you learn this domain, an Evil or Neutral Cleric may instead choose to grant a profane bonus. Once made, this choice cannot be changed).
This bonus lasts for five rounds.
At Cleric levels 6, 12, and 18, the sacred bonus increases by +1.
\subparagraph{Retributive Ward (Su):}
Starting at 5th level, whenever a creature other than you is attacked by a creature within your threatened space, you can expend your magical focus to make an attack of opportunity against the attacker (assuming you have at least one attack of opportunity remaining to you that round).
You expend your focus as part of making the attack of opportunity, no action is required.
% You can generate a protective ward as a supernatural ability. 
% Grant someone you touch a resistance bonus equal to your Cleric level on his or her next saving throw. 
% Activating this power is a standard action. 
% The protective ward is an abjuration effect with a duration of 1 hour that is usable once per day.
\paragraph{Protection Domain Spells}
\begin{list}{\labelitemi}{\leftmargin=1em}
\item[1] \nameref{Spell:EndureElements}: Exist comfortably in hot or cold environments.
\item[1] \nameref{Spell:Sanctuary}: Opponents can't attack you, and you can't attack.
\item[1] \nameref{Spell:ShieldOfFaith}: Creates a magical field around a creature that deflects attacks.
\item[2] \nameref{Spell:Resistance}: Grants a Resistance bonus on saving throws.
\item[2] \nameref{Spell:ResistEnergy}: Ignores first 10 (or more) points of damage/attack from specified energy type.
\item[2] \nameref{Spell:ShieldOther}: You take half of subject's damage.
%\item[3] Protection from Energy: Absorb 12 points/level of damage from one kind of energy.
\item[4] \nameref{Spell:DeathWard}: Grants immunity to death spells and negative energy effects.
\item[4] \nameref{Spell:SpellImmunity}: Subject is immune to two spells of 4th level or below.
\item[4] \nameref{Spell:Stoneskin}: Ignore 7 points of damage per attack.
\item[5] \nameref{Spell:SpellResistance}: Subject gains SR 12 + level.
\item[6] \nameref{Spell:AntimagicField}: Negates magic within 10 ft.
\item[6] \nameref{Spell:AntilifeShell}: 10-ft. field hedges out living creatures.
\item[6] \nameref{Spell:WordOfRecall}: Teleports you back to designated place.
\item[8] \nameref{Spell:MindBlank}: Subject is immune to mental/emotional magic and scrying.
\item[9] \nameref{Spell:PrismaticSphere}: As prismatic wall, but surrounds on all sides.
\end{list}
\subsubsection{Strength Domain}
\paragraph{Granted Powers}
\subparagraph{Class Skills:}
You add Climb, Jump, and Swim to your list of Cleric class skills.
\subparagraph{Feat of Strength (Su):}
You can expend your magical focus as part of making a strength check, strength-based skill check, or a grapple check to gain a bonus on the check equal to your Cleric level.
Using this ability does not take an action of its own, its activation is done as part of the activity that requires the roll.
%You can use this ability once per day. At Cleric levels 4, 8, 12 and 16, you gain an additional daily use of this ability.
\subparagraph{Bonus feats:}
At Cleric levels 2nd, 5th, 9th, 13th and 17th, you gain a bonus feat, selected from the following list:

Athletic, Diehard, Endurance, Great Fortitude, Improved Grapple, Improved Overrun, Improved Unarmed Strike, Run, Skill Focus (Any strength-based skill), Toughness, Trample.

You must meet all prerequisites for the feats, if any.
\paragraph{Strength Domain Spells}
\begin{list}{\labelitemi}{\leftmargin=1em}
\item[1] \nameref{Spell:AlterSize}: Humanoid creature changes size.
\item[1] \nameref{Spell:SurgeOfStrength}*: Gain temporary strength and hit points.
\item[1] \nameref{Spell:Virtue}: Gain 5 temporary hit points.
\item[3] \nameref{Spell:MagicVestment}: Armor or shield gains +2 enhancement bonus.
\item[3] \nameref{Spell:ManeuveringHand}: Hand composed of force performs combat maneuvers.
\item[4] \nameref{Spell:HandOfForce}: Hand of force manipulates items.
\item[4] \nameref{Spell:SpellImmunity}: Subject is immune to two spells of 4th level or below.
\item[4] \nameref{Spell:Stoneskin}: Ignore 7 points of damage per attack.
\item[5] \nameref{Spell:RighteousMight}: Your size increases, and you gain combat bonuses.
\item[6] \nameref{Spell:HeroesFeast}: Food for four creatures cures and grants combat bonuses and immunities.
\item[6] \nameref{Spell:Juggernaut}*: Subject gains the ability to easily destroy objects in its path.
\item[9] \nameref{Spell:FormTitan}: The subject takes up the shape and abilities of a powerful outsider.
\end{list}
\subsubsection{Sun Domain}
\paragraph{Granted Powers}
\subparagraph{Greater Turning (Su):} 
If you have the \nameref{Feat:TurnUndead} feat, it deals 6 points of damage per Cleric level you have, and 1d6 points of damage for every non-Cleric hit die you have.
\subparagraph{Sun-blessed Sight (Ex):}
At Cleric level 3rd, you gain immunity to the conditions of being \emph{blinded} and \emph{dazzled}.
\subparagraph{Purifying Light (Su):}
Starting at Cleric level 5th, you can elect to imbue your light spells with purifying energy that harms certain creatures. At the start of your turn, any evil outsiders, undead creatures, and natives of the plane of shadow located within the bright light radius of a spell so imbued take damage equal to your Cleric level. Only spells with the light descriptor that continuously shed light can be imbued in this way. This ability is activated as part of casting the light spell in question.
\subparagraph{Power of the Sun (Su):}
Starting at Cleric level 7th, you can expend your magical focus as part of casting a spell in order to increase its save DC by 1.
At Cleric level 14th, it instead increases by 2, and by 3 at Cleric level 20th.
This ability can only be used in an area of direct, natural sunlight.
\paragraph{Sun Domain Spells}
\begin{list}{\labelitemi}{\leftmargin=1em}
\item[1] \nameref{Spell:EndureElements}: Exist comfortably in hot or cold environments.
\item[1] \nameref{Spell:Light}: Causes object to shine like a torch.
\item[1] \nameref{Spell:SearingLight}: Ray scorches foul creatures.
\item[2] \nameref{Spell:BladeOfTheSun}*: Weapon shines with brilliant energy for one round.
\item[3] \nameref{Spell:InvisibilityPurge}: Suppresses invisibility within 5 ft./level.
\item[3] \nameref{Spell:PlantGrowth}: You encourage or inhibit the growth of plants in a large area.
\item[4] \nameref{Spell:AuraOfFire}: Enemies within range take damage, more if they attack you.
\item[4] \nameref{Spell:FlameStrike}: Smite foes with divine fire, knocking them down and dealing 7d6 damage.
\item[4] \nameref{Spell:RepellingLight}*: Globe of light keeps foul creatures at bay.
\item[5] \nameref{Spell:BodyOfLight}*: Turn subject into pure energy, removing some vulnerabilities.
\item[6] \nameref{Spell:FireSeeds}: Acorns and berries become grenades and bombs.
\item[7] \nameref{Spell:ControlWeather}: Changes weather in local area.
\item[8] \nameref{Spell:Sunburst}: Blinds creatures and deals damage, more to undead and oozes.
\item[9] \nameref{Spell:PrismaticSphere}: As prismatic wall, but surrounds on all sides.
\end{list}
\subsubsection{Travel Domain}
\paragraph{Granted Powers}
\subparagraph{Class Skills:} 
You add Knowledge (Geography) and Survival to your list of Cleric class skills.
\subparagraph{Longstrider (Su):}
Whenever you cast an \nameref{Spell:ExpeditiousRetreat} spell on yourself as a standard action, the spell's duration changes to 1 hour per level rather than the spell's normal duration.
This is a supernatural ability activated as part of casting the spell in question.
\subparagraph{Efficient Travel (Su):}
You and a number of travelling companions up to your Wisdom modifier add your Cleric level to the number of miles you can move in one hour of overland travel.
This has a corresponding effect on the number of miles you can travel per day.
For example, if accompanied by a 5th level Cleric with the Travel domain, a group walking at a speed of 30' could cover 8 miles per hour of overland travel, rather than 3 miles, and 64 miles per day of overland travel, rather than 24 miles.

If travelling by mount, carriage, or similar assistance, apply this bonus speed to the means of transportation in question, rather than the group's walking speed.
\subparagraph{Safe Teleportation (Su):}
Starting at Cleric level 9th, you never suffer mishaps when casting the \nameref{Spell:Teleport} spell.
Reroll all instances of ``Mishap'' on the Teleportation table until another result is obtained.
This is a supernatural ability activated as part of casting the spell.
\paragraph{Travel Domain Spells}
\begin{list}{\labelitemi}{\leftmargin=1em}
\item[1] \nameref{Spell:ControlFall}: Objects or creatures fall slowly.
\item[1] \nameref{Spell:ExpeditiousRetreat}: Your speed increases by 30 ft.
\item[1] \nameref{Spell:FreeStep}*: Teleports you a very short distance.
\item[2] \nameref{Spell:Locate}: Senses direction toward object (specific or type).
\item[3] \nameref{Spell:Fly}: Subject flies at speed of 40 ft.
\item[3] \nameref{Spell:HelpingHand}: Ghostly hand leads subject to you.
\item[4] \nameref{Spell:DimensionDoor}: Teleports you short distance.
\item[4] \nameref{Spell:FreedomOfMovement}: Subject moves normally despite impediments.
\item[5] \nameref{Spell:Teleport}: Instantly transports you as far as 100 miles/level.
\item[6] \nameref{Spell:TransportViaPlants}: Move instantly from one plant to another of the same kind.
\item[6] \nameref{Spell:WordOfRecall}: Teleports you back to designated place.
\item[7] \nameref{Spell:FreeRun}*: Boosts overland travel speed to miraculous levels.
\item[7] \nameref{Spell:PhaseDoor}: Creates an invisible passage through wood or stone.
\item[7] \nameref{Spell:ReverseGravity}: Objects and creatures fall in a direction other than down for one round.
\item[9] \nameref{Spell:AstralProjection}: Projects you and companions onto Astral Plane.
\end{list}
\subsubsection{Trickery Domain}
\paragraph{Granted Powers}
\subparagraph{Class Skills:} Add Bluff, Disguise, and Hide to your list of Cleric class skills.
\subparagraph{Master of Diversion (Ex):}
You can create a diversion to hide or feint in combat as a move action, rather than as a standard action (see the Bluff and Hide skill descriptions, respectively). 
If you already have the ability to feint in combat as a move action, you can instead feint as a free action.
\subparagraph{Pass without Trace (Ex):}
Starting at Cleric level 3rd, you can move through any type of terrain and leave neither footprints nor scent. Tracking you is impossible. You may choose to leave a trail if so desired.
\subparagraph{Poison Use (Ex):}
Starting at Cleric level 6th, you never risk accidentally poisoning yourself when applying poison to a weapon or using a poisoned weapon.
\subparagraph{Camouflage (Ex):}
Starting at 13th level, you can use the Hide skill even without having cover or concealment.
\paragraph{Trickery Domain Spells}
\begin{list}{\labelitemi}{\leftmargin=1em}
\item[1] \nameref{Spell:DetectPoison}: Detects poison or disease in one creature or small object.
\item[1] \nameref{Spell:DisguiseSelf}: Changes your appearance.
\item[1] \nameref{Spell:MagicAura}: Alters object's magic aura.
\item[1] \nameref{Spell:ShroudTheWeakMind}: Prevents creatures with limited intelligence from perceiving warded subjects.
\item[2] \nameref{Spell:Invisibility}: Subject is invisible for 1 min./level or until it attacks.
\item[2] \nameref{Spell:Silence}: Negates sound in 20-ft. radius.
\item[3] \nameref{Spell:Nondetection}: Masks object or creature against scrying.
\item[3] \nameref{Spell:Poison}: Creates a temporary batch of poison.
\item[4] \nameref{Spell:Confusion}: Subjects behave oddly for 1 round/level.
\item[5] \nameref{Spell:FalseVision}: Fools scrying with an illusion.
\item[6] \nameref{Spell:AnimateObjects}: Objects attack your foes.
\item[6] \nameref{Spell:Mislead} : Turns you invisible and creates illusory double.
\item[7] \nameref{Spell:Screen}: Illusion hides area from vision, scrying.
\item[9] \nameref{Spell:TimeStop}: You stop time momentarily.
\end{list}
\subsubsection[Vermin Domain]{Vermin Domain*}
\paragraph{Granted Powers}
\subparagraph{Friend of Vermin (Ex):}
Vermin never attack you without provocation. Unless you attack, touch, or cast a spell at a nearby vermin, it acts as if you weren't there.
This is an ability that functions continuously, requiring no activation.
\subparagraph{Lord of Flies (Ex):}
If you have ranks in the Handle Animal skill, you can use it to handle and train monstrous vermin as if they were animals.
For this purpose, treat the vermin as if they had an Intelligence score of 1.
\subparagraph{Vermin Shape (Su):}
Starting at Cleric level 7th, whenever you cast a \nameref{Spell:FormVermin} or \nameref{Spell:FormViper} on yourself, 
the spell's duration becomes 1 hour per caster level, rather than the spell's normal duration.
This ability is activated as part of casting one of the listed spells.
\subparagraph{Vermin Venom Immunity (Ex):}
At Cleric level 9th, you gain immunity to the poison attacks of vermin, and to all poisons derived from vermin.
\paragraph{Vermin Domain Spells}
\begin{list}{\labelitemi}{\leftmargin=1em}
\item[1] \nameref{Spell:WebStrand}*: You entangle a foe in sticky goo.
\item[1] \nameref{Spell:SummonVermin}: Summons a monstrous vermin to aid you in battle.
\item[2] \nameref{Spell:Arachnophobia}*: Subject believes it is covered in spiders.
\item[2] \nameref{Spell:SummonSwarm}: Summons swarm of bats, rats, or spiders.
\item[2] \nameref{Spell:Web}: Creates sticky spiderwebs between two anchor points.
\item[3] \nameref{Spell:Poison}: Creates a temporary batch of poison.
\item[4] \nameref{Spell:FormVermin}: Subject gains the form of a vermin.
\item[4] \nameref{Spell:FormViper}: Subject gains the form of a snake.
\item[4] \nameref{Spell:GiantVermin}: Turns centipedes, scorpions, or spiders into giant vermin.
\item[4] \nameref{Spell:RepelVermin}: Insects, spiders, and other vermin stay 10 ft. away.
\item[5] \nameref{Spell:JellyfishSting}*: Slimy tentacle reaches out and paralyzes one victim.
\item[7] \nameref{Spell:CreepingDoom}: Fills subject's lungs with insects, placing it in mortal danger.
\item[9] \nameref{Spell:InsectPlague}: Swarm of locusts appears, devouring everything within an area.
\end{list}
\subsubsection{War Domain}
\paragraph{Granted Powers}
\subparagraph{Heavy Armor Proficiency:}
You are proficient in heavy armor. 
\subparagraph{Martial Devotion (Ex):}
You qualify for feats and prestige classes as if the Cleric class
had the good Base Attack Bonus progression (BAB = level), rather than the medium progression (BAB = $3/4$ level).

\subparagraph{Bonus feats:}
At Cleric level 4th, you gain the Weapon Focus feat with your deity's favored weapon as a bonus feat.
At Cleric level 8th, you gain the Weapon Specialization feat with your deity's favored weapon as a bonus feat, even if you do not meet the prerequisites.
At Cleric level 12th, you gain the Greater Weapon Focus feat with your deity's favored weapon as a bonus feat, even if you do not meet the prerequisites.
At Cleric level 16th, you gain the Greater Weapon Specialization feat with your deity's favored weapon as a bonus feat, even if you do not meet the prerequisites.
\paragraph{War Domain Spells}
\begin{list}{\labelitemi}{\leftmargin=1em}
\item[1] \nameref{Spell:DivineFavor}: You gain a luck bonus on attack and damage rolls.
\item[1] \nameref{Spell:MagicWeapon}: Weapon gains +1 bonus.
\item[2] \nameref{Spell:SpiritualWeapon}: Magic weapon attacks on its own.
\item[3] \nameref{Spell:MagicVestment}: Armor or shield gains +2 enhancement bonus.
\item[4] \nameref{Spell:DivinePower}: You gain attack bonus, +6 to Str, and 1 hp/level.
\item[4] \nameref{Spell:FlameStrike}: Smite foes with divine fire, knocking them down and dealing 7d6 damage.
\item[5] \nameref{Spell:RighteousMight}: Your size increases, and you gain combat bonuses.
\item[5] \nameref{Spell:WarriorsImpetus}*: Grants speed and strength to targeted creatures.
\item[6] \nameref{Spell:BladeBarrier}: Wall of blades deals 11d6 damage.
\item[7] \nameref{Spell:PowerWord}: Overwhelms the minds of lesser creatures with a single word of power.
\item[9] \nameref{Spell:ImmortalArmy}*: Grant virtual immortality to weaker creatures.
\end{list}
\subsubsection{Water Domain}
\paragraph{Granted Powers}
\subparagraph{Class Skill:} 
You add Swim to your list of Cleric class skills.
\subparagraph[Create Water]{Create Water (Su):}
\label{sec:CreateWater}
By expending your magical focus as a standard action, you can create up to two gallons of water per Cleric level. 
The water is wholesome and drinkable, just like clean rain water.

The water appears within 30'. Water can be created in an area as small as will actually contain the liquid, or in an area three times as large - possibly creating a downpour or filling many small receptacles.

Water weighs about 8 pounds per gallon. One cubic foot of water contains roughly 8 gallons and weighs about 60 pounds.

\subparagraph{Swim Speed:}
You gain a swim speed equal to your land speed.
You can then move through water at your speed without making Swim checks.
You gain a +8 racial bonus on any Swim check to perform a special action or avoid a hazard. 
You can always can choose to take 10 on a Swim check, even if distracted or endangered when swimming. 
You can use the run action while swimming, provided that you swim in a straight line.
\subparagraph[Water Breathing]{Water Breathing (Su):}
\label{sec:WaterBreathing}
Starting at Cleric level 5th, you gain the ability to breathe water as well as you breathe air. This allows you to speak (and provide verbal components to spells) as if you were breathing air.

If you already have the ability to breathe water, but not air, you instead gain the ability to breathe air (you become amphibious). If you are already amphibious, you gain no further benefits.

This is a supernatural ability that functions continuously.
\subparagraph{Embrace of the Deep (Su):}
Starting at Cleric level 10th, you can cause any creature within 30' to lose its ability to hold its breath while in water.
A successful Fortitude save (DC 10 + $1/2$ your Cleric level + your Wisdom modifier) negates the effect.
On a failed save the creature must, if submerged in water, immediately start making Constitution checks or start drowning\footnote{Rewriting the drowning rules is left as an exercise to the reader.}.
Creatures that do not need to breathe and creatures capable of breathing water are immune to this ability.
Using this ability is a mental free action.
You cannot force this ability upon the same creature more than once per round. 
You must have line of sight and line of effect to the creature in question. 
The creature remains unable to hold its breath until you dismiss the effect (another free action) or until it moves out of range.
Its ability to actually breathe is unaffected.
\subparagraph{Elemental Shape (Su):}
Starting at Cleric level 13th, whenever you cast a \nameref{Spell:FormElemental} on yourself to assume the form of a water elemental, 
the spell's duration becomes 1 hour per caster level, rather than the spell's normal duration.
This ability is activated as part of casting the spell.
\paragraph{Water Domain Spells}
\begin{list}{\labelitemi}{\leftmargin=1em}
\item[1] \nameref{Spell:FoodAndWater}: Purifies food and water so it is again fit for consumption.
\item[1] \nameref{Spell:Fog}: Fog surrounds you.
\item[1] \nameref{Spell:SummonAquaticCreature}: Summons a creature of the seas.
\item[2] \nameref{Spell:AlignWater}: Vial of water is imbued with the power of an alignment.
\item[2] \nameref{Spell:Squirt}*: Line of water deals nonlethal damage.
\item[3] \nameref{Spell:AnimalsMovement}: Grants additional movement capabilities.
\item[3] \nameref{Spell:SummonWaterElemental}: Summons an elemental to do your bidding.
\item[3] \nameref{Spell:WaterWalk}: Subjects tread on water as if solid.
\item[4] \nameref{Spell:IceStorm}: Hail deals 5d6 damage in cylinder 40 ft. across.
\item[4] \nameref{Spell:RustingGrasp}: Your touch corrodes iron and alloys.
\item[5] \nameref{Spell:ConeOfCold}: 9d6 points of energy damage in a cone, and all in area must save or be slowed.
\item[5] \nameref{Spell:ControlWater}: Raises or lowers bodies of water, or slows water elemental.
\item[6] \nameref{Spell:DeadlyFog}: Add elemental component to fog, causing it to deal damage.
\item[7] \nameref{Spell:ControlWeather}: Changes weather in local area.
\item[7] \nameref{Spell:FormElemental}: Subject becomes a living creature of the elements.
\item[8] \nameref{Spell:HorridWilting}: Desiccates nearby creatures.
%\item[9] Elemental Swarm*: Summons multiple elementals.
\end{list}
\subsection{Paladin Spells}
\subsubsection{1st-Level Paladin Spells}
\begin{list}{\labelitemi}{\leftmargin=1em}
\item \nameref{Spell:AlignedProtection}: +2 to AC and saves, counter mind control, hedge out elementals and outsiders.
\item \nameref{Spell:ArmamentsOfFaith}*: Arms and armor are conjured directly on to your body.
\item \nameref{Spell:Bless}: Allies gain +1 on attack rolls and +1 on saves against fear.
\item \nameref{Spell:BlessWeapon}: Weapon strikes true against evil foes.
\item \nameref{Spell:ComprehendLanguages}: You understand all spoken and written languages.
\item \nameref{Spell:CureWounds}: Cures 1d8 damage +1/level.
\item \nameref{Spell:TouchOfVitality} \textbf{(Free for Paladins)}: Cures wounds with a touch.
\item \nameref{Spell:DetectPoison}: Detects poison or disease in one creature or small object.
\item \nameref{Spell:DetectUndead}: Reveals the presence and strength of undead creatures.
\item \nameref{Spell:DiscernAlignment}: Reveals the subject's alignment.
\item \nameref{Spell:DivineFavor}: You gain a luck bonus on attack and damage rolls.
\item \nameref{Spell:EndureElements}: Exist comfortably in hot or cold environments.
\item \nameref{Spell:Light}: Causes object to shine like a torch.
\item \nameref{Spell:MagicWeapon}: Weapon gains +1 bonus.
\item \nameref{Spell:TrueStrike}: +20 on your next attack roll.
%\item \nameref{Spell:ReadMagic}: Ease reading of magical text.
\item \nameref{Spell:RemoveFear}: Subject gains immunity to fear.
\item \nameref{Spell:SummonWeapon}*: Summons bonded weapon to your hand.
\item \nameref{Spell:SurgeOfStrength}*: Gain temporary strength and hit points.
\item \nameref{Spell:Virtue}: Gain 5 temporary hit points.
\end{list}
\subsubsection{2nd-Level Paladin Spells}
\begin{list}{\labelitemi}{\leftmargin=1em}
\item \nameref{Spell:Aid}: +1 on attack rolls, +1 on saves against fear, 1d8 temporary hp +1/level.
\item \nameref{Spell:AlignWater}*: Vial of water is imbued with the power of an alignment.
\item \nameref{Spell:Antipoison}: Stops poison from harming subject for 1 hour/level.
\item \nameref{Spell:BladeOfTheSun}*: Weapon shines with brilliant energy for one round.
\item \nameref{Spell:DivineFootstep}*: Tread on air for one round.
\item \nameref{Spell:LionsCharge}*: You can make full attack in same round you charge.
\item \nameref{Spell:MaskAlignment}: Protects subject's alignment from being revealed via divinations.
\item \nameref{Spell:Resistance}: Grants a Resistance bonus on saving throws.
\item \nameref{Spell:ResistEnergy}: Ignores first 10 (or more) points of damage/attack from specified energy type.
\item \nameref{Spell:Restoration}: Dispels magical ability penalty or repairs 1d4 ability damage.
\item \nameref{Spell:SearingBlade}*: Weapon deals 3d6 extra points of damage against undead and oozes, half that against others.
\item \nameref{Spell:ShieldOther}: You take half of subject's damage.
\item \nameref{Spell:ZoneOfTruth}: Subjects within field find it extremely hard to lie.
\item \nameref{Spell:WombatsBoost}: Subject gains +4 to an ability score for 1 min./level.
\end{list}
\subsubsection{3rd-Level Paladin Spells}
\begin{list}{\labelitemi}{\leftmargin=1em}
 \item \nameref{Spell:CelestialFlight}*: Subject sprouts wings and flies at speed of 40 ft.
 \item \nameref{Spell:DispelMagic}: Cancels magical spells and effects.
 \item \nameref{Spell:GeasQuest}: Commands subject of 7 HD or less.
 \item \nameref{Spell:HealMount}: As Heal on your special mount.
 \item \nameref{Spell:KeenEdge}: Doubles normal weapon's threat range.
 \item \nameref{Spell:MagicVestment}: Armor or shield gains +2 enhancement bonus.
 \item \nameref{Spell:Piggyback}*: Latch on to a creature as it teleports, retaining your relative position.
 \item \nameref{Spell:RemoveBlindnessDeafness}: Cures normal or magical conditions impeding senses.
 \item \nameref{Spell:RemoveCurse}: Frees object or person from curse.
 \item \nameref{Spell:RemoveDisease}: Cures all diseases affecting subject.
\end{list}
\subsubsection{4th-Level Paladin Spells}
\begin{list}{\labelitemi}{\leftmargin=1em}
 \item \nameref{Spell:BodyOfLight}*: Turn subject into pure energy, removing some vulnerabilities.
 \item \nameref{Spell:DeathWard}: Grants immunity to death spells and negative energy effects.
 \item \nameref{Spell:DimensionDoor}: Teleports you short distance.
 \item \nameref{Spell:FreedomOfMovement}: Subject moves normally despite impediments.
 \item \nameref{Spell:MarkOfJustice}: Designates action that will trigger curse on subject.
\end{list}
\subsubsection{5th-Level Paladin Spells}
\begin{list}{\labelitemi}{\leftmargin=1em}
 \item \nameref{Spell:AlignedSword}: Weapon becomes +5, deals +2d6 damage against one alignment.
 \item \nameref{Spell:Atonement}: Removes burden of misdeeds from subject.
 \item \nameref{Spell:BanishingWeapon}*: Melee weapon banishes outsiders.
 \item \nameref{Spell:Commune}: Deity answers three yes-or-no questions.
 \item \nameref{Spell:DispelAlignment}: Protects against creatures of the chosen alignment, discharge to drive creature away.
 \item \nameref{Spell:DisruptingWeapon}: Melee weapon destroys undead.
 \item \nameref{Spell:MaceOfTheAstralDeva}*: Creature struck twice with weapon must save or be stunned.
 \item \nameref{Spell:RighteousMight}: Your size increases, and you gain combat bonuses.
 \item \nameref{Spell:TrueSeeing}: Lets you see all things as they really are.
\end{list}
\subsubsection{6th-Level Paladin Spells}
\begin{list}{\labelitemi}{\leftmargin=1em}
 \item \nameref{Spell:AssaultOfTheSevenFoldHeaven}*: Creature is subjected to an array of prismatic effects.
 \item \nameref{Spell:BowOfTheSolar}*: Creature must save or be destroyed.
 \item \nameref{Spell:FormCelestial}*: You assume the form of a majestic celestial.
 \item \nameref{Spell:Heal}: Cures great amounts of of damage, all diseases and mental conditions.
 \item \nameref{Spell:PersonalMindBlank}*: You are immune to scrying and mental effects.
 \item \nameref{Spell:PlaneShift}: As many as eight subjects travel to another plane.
 \item \nameref{Spell:WordOfGod}: Kills, paralyzes, hinders, or deafens subjects not of a selected alignment.
\end{list}
\subsection{Ranger Spells}
%2nd level: Arrow ignores armor (Brilliant Energy)
%2nd level: Arrow deals damage to creatures with spells
%3rd level: Arrow splits into many
%3rd level: Arrow deals ongoing damage
%4th level: Arrow paralyzes
%4th level: Arrow hits two targets
%4th level: Arrow deals strength damage
\subsubsection{1st-Level Ranger Spells}
\begin{list}{\labelitemi}{\leftmargin=1em}
\item \nameref{Spell:Alarm}: Wards an area for 2 hours/level.
\item \nameref{Spell:AlterAnimalsSize}: Animal changes size.
\item \nameref{Spell:AnimalMessenger}: Sends a Tiny animal to a specific place.
\item \nameref{Spell:AnimalsMovement}: Grants additional movement capabilities.
\item \nameref{Spell:Bloodhound}*: Grants one creature the scent special ability.
\item \nameref{Spell:ControlFall}: Objects or creatures fall slowly.
\item \nameref{Spell:CureWounds}: Cures 1d8 damage +1/level.
\item \nameref{Spell:DetectPoison}: Detects poison or disease in one creature or small object.
\item \nameref{Spell:ElementalWeapons}*: Your weapons deal an additional 1d6 points of energy damage.
\item \nameref{Spell:Elfcloak}*: You blend in with your surroundings, making it easier to hide.
\item \nameref{Spell:EndureElements}: Exist comfortably in hot or cold environments.
\item \nameref{Spell:Entangle}: Plants entangle everyone in 20-ft.-radius.
\item \nameref{Spell:ExpeditiousRetreat}: Your speed increases by 30 ft.
%Longstrider?
\item \nameref{Spell:MagicWeapon}: Weapon gains +1 bonus.
% \item Pass without Trace: One subject/level leaves no tracks.
\item \nameref{Spell:ShroudTheWeakMind}: Prevents creatures with limited intelligence from perceiving warded subjects.
\item \nameref{Spell:SummonAerialBeast}: Summons an airborne creature.
\item \nameref{Spell:SummonAquaticCreature}: Summons a creature of the seas.
\item \nameref{Spell:SummonExoticBeast}: Summons a creature from the distant lands.
\item \nameref{Spell:SummonForestPredator}: Summons a predator of the forests.
\item \nameref{Spell:SummonReptile}: Summons a scaled creature.
\item \nameref{Spell:TrueStrike}: +20 on your next attack roll.
\item \nameref{Spell:ConverseWithNature}: Communicate with animals and other unlikely creatures and objects.
%\item \nameref{Spell:SummonMonster}: Calls extraplanar creature to fight for you.
\end{list}.
\subsubsection{2nd-Level Ranger Spells}
\begin{list}{\labelitemi}{\leftmargin=1em}
\item \nameref{Spell:Antipoison}: Stops poison from harming subject for 1 hour/level.
\item \nameref{Spell:Barkskin}: Grants +2 enhancement to natural armor.
\item \nameref{Spell:CommandNaturesAllies}: Sway the actions of one a elemental, fey, or plant creature.
\item \nameref{Spell:Darkvision}: See 30 ft. in total darkness.
\item \nameref{Spell:FormPlant}: You look exactly like a plant for 1 hour/level.
\item \nameref{Spell:HoldAnimal}: Paralyzes one animal for 1 round/level.
\item \nameref{Spell:LionsCharge}*: You can make full attack in same round you charge.
\item \nameref{Spell:Resistance}: Grants a Resistance bonus on saving throws.
\item \nameref{Spell:ResistEnergy}: Ignores first 10 (or more) points of damage/attack from specified energy type.
\item \nameref{Spell:SpikeGrowth}: Creatures in area take 4d8 damage, may be slowed.
\item \nameref{Spell:StunningArrow}*: Imbue a single projectile with the power to stun the creature it strikes.
\item \nameref{Spell:TwinBladeDance}*: Stun opponents struck with a pair of weapons.
\item \nameref{Spell:WindWall}: Deflects arrows, knocks down creatures, blocks gases.
\end{list}
\subsubsection{3rd-Level Ranger Spells}
\begin{list}{\labelitemi}{\leftmargin=1em}
\item \nameref{Spell:BloodlettingArrow}*: Arrow deals constitution damage.
\item \nameref{Spell:FormAvian}: Subject gains the form of a bird.
\item \nameref{Spell:FormFish}: Subject gains the form of a water-dwelling creature.
\item \nameref{Spell:HawksFlight}*: Subject grows wings and flies at speed of 40 ft.
\item \nameref{Spell:HailOfArrows}*: Fire a ranged weapon at multiple targets within range.
\item \nameref{Spell:MagicVestment}: Armor or shield gains +2 enhancement bonus.
\item \nameref{Spell:Nondetection}: Masks object or creature against scrying.
\item \nameref{Spell:Locate}: Senses direction toward object (specific or type).
\item \nameref{Spell:PlantGrowth}: You encourage or inhibit the growth of plants in a large area.
\item \nameref{Spell:RemoveDisease}: Cures all diseases affecting subject.
\item \nameref{Spell:SummonNaturesAlly}: Summons a natural guardian to aid you.
\item \nameref{Spell:WaterWalk}: Subjects tread on water as if solid.
\end{list}
\subsubsection{4th-Level Ranger Spells}
\begin{list}{\labelitemi}{\leftmargin=1em}
\item \nameref{Spell:CommuneWithNature}: Learn about terrain for 1 mile/level.
\item \nameref{Spell:FreedomOfMovement}: Subject moves normally despite impediments.
\item \nameref{Spell:FormCarnivore}: Subject gains the form of a dangerous beast.
\item \nameref{Spell:ParalyzingArrow}*: Imbue a single projectile with the power to paralyze the creature it strikes.
\item \nameref{Spell:RepelVermin}: Insects, spiders, and other vermin stay 10 ft. away.
\item \nameref{Spell:SeekerArrow}*: Missile fired ignores line of effect.
\item \nameref{Spell:StepThroughEarth}*: Travel instantaneously through the ground.
\item \nameref{Spell:ViciousWeapons}*: Melee weapons you wield deal constitution damage.
\end{list}
\subsubsection{5th-level Ranger Spells}
\begin{list}{\labelitemi}{\leftmargin=1em}
\item \nameref{Spell:AdaptBody}*: Your body automatically adapts to hostile environments.
\item \nameref{Spell:Awaken}: Animal or tree gains human intellect.
\item \nameref{Spell:ControlWinds}: Change wind direction and speed.
\item \nameref{Spell:BalefulPolymorph}: Transforms subject into harmless animal.
\item \nameref{Spell:FormTreant}: Subject gains the form of a plant creature.
\item \nameref{Spell:PhaseArrow}*: Arrow passes through the ethereal plane on its way to its target.
\item \nameref{Spell:WallOfThorns}: Thorns damage anyone who tries to pass.
\end{list}
\subsubsection{6th-level Ranger Spells}
\begin{list}{\labelitemi}{\leftmargin=1em}
\item \nameref{Spell:AnimatePlants}: One or more trees animate and fight for you.
\item \nameref{Spell:ArrowOfDeath}*: Imbue a single projectile with the power to instantly slay the creature it strikes.
\item \nameref{Spell:FreeRun}*: Boosts overland travel speed to miraculous levels.
\item \nameref{Spell:Liveoak}: Changes an oak into a treant.
\item \nameref{Spell:RepelWood}: Pushes away wooden objects.
\item \nameref{Spell:TransportViaPlants}: Move instantly from one plant to another of the same kind.
\end{list}
\subsection{Sorcerer/Wizard Spells}
\subsubsection{1st-Level Sorcerer/Wizard Spells}
\begin{list}{\labelitemi}{\leftmargin=1em}
\item \nameref{Spell:Alarm}: Wards an area for 2 hours/level.
\item \nameref{Spell:AlignedProtection}: +2 to AC and saves, counter mind control, hedge out elementals and outsiders.
\item \nameref{Spell:ColorSpray}: Knocks unconscious, blinds, and/or stuns weak creatures.
\item \nameref{Spell:ControlFall}: Objects or creatures fall slowly.
\item \nameref{Spell:Daze}: Target creature loses next action.
\item \nameref{Spell:DetectMagic}: Reveals the presence, strength, and school of magical auras.
\item \nameref{Spell:DetectSecretDoors}: Become aware of all secret doors within your line of sight.
\item \nameref{Spell:DetectUndead}: Reveals the presence and strength of undead creatures.
\item \nameref{Spell:DisguiseSelf}: Changes your appearance.
\item \nameref{Spell:EndureElements}: Exist comfortably in hot or cold environments.
\item \nameref{Spell:ScorchingRay}: Deal 1d6 energy damage with a ranged touch attack.
\item \nameref{Spell:ShockingGrasp}: Touch delivers 1d6 energy damage.
\item \nameref{Spell:ExpeditiousRetreat}: Your speed increases by 30 ft.
\item \nameref{Spell:FloatingDisk}: Creates 3-ft.-diameter horizontal disk that holds 100 lb./level.
\item \nameref{Spell:Grease}: Makes 10-ft. square or one object slippery.
\item \nameref{Spell:Identify}: Determines properties of magic item.
\item \nameref{Spell:Light}: Causes object to shine like a torch.
\item \nameref{Spell:MagicAura}: Alters object's magic aura.
\item \nameref{Spell:MagicMissile}: Deal 1d4+1 damage, no save, no touch attack.
\item \nameref{Spell:MagicWeapon}: Weapon gains +1 bonus.
\item \nameref{Spell:MentalLink}*: You forge a limited mental bond with another creature.
\item \nameref{Spell:Mount}: Summons magical riding horse for 2 hours/level.
\item \nameref{Spell:OpenClose}: Holds door shut or opens it.
\item \nameref{Spell:RayOfEnfeeblement}: Ray inflicts a strength penalty of 1d6.
\item \nameref{Spell:Repair}: Makes repairs on an object or construct.
\item \nameref{Spell:Shield}: Invisible disc gives +4 to AC.
\item \nameref{Spell:Sleep}: Puts 4 HD of creatures into magical slumber.
\item \nameref{Spell:TouchOfFatigue}: Touch fatigues subject.
\item \nameref{Spell:TrueStrike}: +20 on your next attack roll.
\item \nameref{Spell:UnseenServant}: Invisible force obeys your commands.
\item \nameref{Spell:Ventriloquism}: Makes sounds appear out of nowhere.
\end{list}
\subsubsection{2nd-Level Sorcerer/Wizard Spells}
\begin{list}{\labelitemi}{\leftmargin=1em}
\item \nameref{Spell:AcidArrow}: Ranged touch attack for 2d4 acid damage.
\item \nameref{Spell:AlterSelf}: Perform minor physical changes on yourself.
\item \nameref{Spell:AnimalsMovement}: Grants additional movement capabilities.
\item \nameref{Spell:Blindness}: Negates one of the subject's senses.
\item \nameref{Spell:Blur}: Attacks miss subject 20\% of the time.
\item \nameref{Spell:Darkness}: 20-ft. radius of supernatural shadow.
\item \nameref{Spell:Darkvision}: See 30 ft. in total darkness.
\item \nameref{Spell:FalseLife}: Gain 1d10 temporary hp.
\item \nameref{Spell:FlamingSphere}: Creates rolling ball of fire, 2d6 damage, lasts 1 round/level.
\item \nameref{Spell:GhoulTouch}: Paralyzes one subject, which exudes stench that makes those nearby sickened.
\item \nameref{Spell:Glitterdust}: Blinds creatures, outlines invisible creatures.
\item \nameref{Spell:GustOfWind}: Blows away and knocks down creatures.
\item \nameref{Spell:HideousLaughter}: Subject loses actions for 1 round/level.
\item \nameref{Spell:HoldPerson}: Paralyzes one humanoid for 1 round/level.
\item \nameref{Spell:Levitate}: Subject moves up and down at your direction.
\item \nameref{Spell:Locate}: Senses direction toward object (specific or type).
\item \nameref{Spell:Pattern}: Twisting colors fascinate creatures.
\item \nameref{Spell:PhantomTrap}: Makes item seem trapped.
\item \nameref{Spell:ProtectionFromArrows}: Subject becomes immune to most ranged attacks.
\item \nameref{Spell:ResistEnergy}: Ignores first 10 (or more) points of damage/attack from specified energy type.
\item \nameref{Spell:SeeInvisibility}: Reveals invisible creatures or objects.
\item \nameref{Spell:Shatter}: Sonic vibration damages objects or crystalline creatures.
\item \nameref{Spell:SpectralHand}: Creates disembodied glowing hand to deliver touch attacks.
\item \nameref{Spell:SummonSwarm}: Summons swarm of bats, rats, or spiders.
\item \nameref{Spell:TouchOfIdiocy}: Subject takes 1d6 points of Int, Wis, or Cha damage.
\item \nameref{Spell:Web}: Creates sticky spiderwebs between two anchor points.
\item \nameref{Spell:WombatsBoost}\footnote{Thanks to Fax Celestis @ Giantitp.com for this joke.}: Subject gains +4 to an ability score for 1 min./level.
\end{list}

\subsubsection{3rd-Level Sorcerer/Wizard Spells}
\begin{list}{\labelitemi}{\leftmargin=1em}
\item \nameref{Spell:Blindsense}*: Subject can notice things it cannot see.
\item \nameref{Spell:Blink}: Subject randomly vanishes and reappears for 1 round/level.
\item \nameref{Spell:CrisisOfBreath}*: Disrupt subject's breathing.
\item \nameref{Spell:DispelMagic}: Cancels magical spells and effects.
\item \nameref{Spell:EnergyArrow}: Arrows deal additional energy damage.
\item \nameref{Spell:ExplosiveRunes}: Deals 6d6 damage when read.
\item \nameref{Spell:ForcedVisions}*: Useless, distracting visions flash before subject's eyes.
\item \nameref{Spell:GaseousForm}: Subject becomes insubstantial and can fly slowly.
\item \nameref{Spell:HaltUndead}: Immobilizes undead for 1 round/level.
\item \nameref{Spell:Haste}: One creature moves faster, +1 on attack rolls, AC, and Reflex saves.
\item \nameref{Spell:Heroism}: Gives +2 bonus on attack rolls, saves, skill checks.
\item \nameref{Spell:KeenEdge}: Doubles normal weapon's threat range.
\item \nameref{Spell:Piggyback}*: Latch on to a creature as it teleports, retaining your relative position.
\item \nameref{Spell:Rage}: Subjects are thrown into a fit of anger, with various effects.
\item \nameref{Spell:SepiaSnakeSigil}: Creates text symbol that immobilizes reader.
\item \nameref{Spell:ShadowWarriors}*: A group of warriors made of shadow matter appears.
\item \nameref{Spell:ShrinkItem}: Object shrinks to one-sixteenth size.
\item \nameref{Spell:SleetStorm}: Sleet hampers vision and movement.
\item \nameref{Spell:SuppressMagic}*: Renders willing spellcaster incapable of casting spells.
\item \nameref{Spell:Slow}: One creature takes only one action/round, -1 to AC, reflex saves, and attack rolls.
\item \nameref{Spell:TinyHut}: Creates shelter for ten creatures.
\item \nameref{Spell:VampiricTouch}: Touch deals 3d6 damage; caster gains damage as hp.
\item \nameref{Spell:WindWall}: Deflects arrows, knocks down creatures, blocks gases.
\end{list}
\subsubsection{4th-Level Sorcerer/Wizard Spells}
\begin{list}{\labelitemi}{\leftmargin=1em}
\item \nameref{Spell:ArcaneEye}: Invisible floating eye moves 30 ft./round.
\item \nameref{Spell:AuraOfFire}: Enemies within range take damage, more if they attack you.
\item \nameref{Spell:BestowCurse}: -6 to an ability score; -4 on attack rolls, saves, and checks; or 50\% chance of losing each action.
\item \nameref{Spell:Confusion}: Subjects behave oddly for 1 round/level.
\item \nameref{Spell:Contagion}: Infects subject with chosen disease.
\item \nameref{Spell:ControlWater}: Raises or lowers bodies of water, or slows water elemental.
\item \nameref{Spell:CrushingDespair}: Subjects take -2 on attack rolls, damage rolls, saves, and checks.
\item \nameref{Spell:DetectScrying}: Alerts you of magical eavesdropping.
\item \nameref{Spell:DimensionalAnchor}: Bars extradimensional movement.
\item \nameref{Spell:HallucinatoryTerrain}: Makes one type of terrain appear like another.
\item \nameref{Spell:HandOfForce}: Hand of force manipulates items.
\item \nameref{Spell:IceStorm}: Hail deals 5d6 damage in cylinder 40 ft. across.
\item \nameref{Spell:MoldMaterial}: Sculpts material into any shape.
\item \nameref{Spell:PhantasmalKiller}: Fearsome illusion kills subject or renders it unconscious.
\item \nameref{Spell:RemoveCurse}: Frees object or person from curse.
\item \nameref{Spell:Shout}: Deafens all within cone and deals 7d6 sonic damage.
\item \nameref{Spell:Stoneskin}: Ignore 7 points of damage per attack.
\item \nameref{Spell:Telekinesis}: Telekinetically throw things around.
\item \nameref{Spell:WallOfFire}: Deals 2d4 fire damage out to 10 ft. Passing through wall deals 7d6 damage.
\item \nameref{Spell:WallOfIce}: Ice forms a translucent, shapeable wall.
\end{list}
\subsubsection{5th-level Sorcerer/Wizard Spells}
\begin{list}{\labelitemi}{\leftmargin=1em} 
\item \nameref{Spell:BalefulPolymorph}: Transforms subject into harmless animal.
\item \nameref{Spell:Blight}: Withers one plant or deals 1d6/level damage to plant creature.
\item \nameref{Spell:Dream}: Contact or disturb sleeping creature.
\item \nameref{Spell:FalseVision}: Fools scrying with an illusion.
\item \nameref{Spell:Feeblemind}: Subject's Int and Cha drop to 1.
\item \nameref{Spell:InstantSummons}: Prepared object appears in your hand.
\item \nameref{Spell:InterposingHand}: Hand provides cover against one opponent.
\item \nameref{Spell:MindFog}: Subjects in fog suffer increasing penalties to Will saves and Wisdom checks.
\item \nameref{Spell:PrivateSanctum}: Prevents anyone from viewing or scrying an area for 24 hours.
\item \nameref{Spell:PryingEyes}: Floating eyes scout for you.
\item \nameref{Spell:Sending}: Delivers short message anywhere, instantly.
\item \nameref{Spell:Teleport}: Instantly transports you as far as 100 miles/level.
\item \nameref{Spell:TransmuteRockAndMud}: Transforms two 10-ft. cubes per level.
\item \nameref{Spell:XRayVision}*: Grants you the ability to see through solid matter.
\item \nameref{Spell:WallOfForce}: Create Wall which is immune to damage.
\item \nameref{Spell:WallOfStone}: Creates a stone wall that can be shaped.
\item \nameref{Spell:WavesOfFatigue}: Several targets become fatigued.
\end{list}
\subsubsection{6th-level Sorcerer/Wizard Spells}
\begin{list}{\labelitemi}{\leftmargin=1em}
\item \nameref{Spell:AntilifeShell}: 10-ft. field hedges out living creatures.
\item \nameref{Spell:Contingency}: Allows you to store one spell to be swiftly cast when the need is dire.
\item \nameref{Spell:DeadlyFog}: Add elemental component to fog, causing it to deal damage.
\item \nameref{Spell:Disintegrate}: Makes one creature or object vanish.
\item \nameref{Spell:Eyebite}: Target becomes panicked, sickened, and/or comatose, depending on HD.
\item \nameref{Spell:FreezingSphere}: Freezes in place and deals cold damage.
\item \nameref{Spell:Hardening}: Increases an object's hardness.
\item \nameref{Spell:ShadowWalk}: Step into shadow to travel rapidly.
\item \nameref{Spell:Shun}*: Subject is forced away.
\item \nameref{Spell:Transformation}: You transform into a a hulk skilled in combat.
\item \nameref{Spell:TransmuteFleshAndStone}: Turns subject creature into statue, or the other way around.
\item \nameref{Spell:TrueSeeing}: Lets you see all things as they really are.
\end{list}
\subsubsection{7th-Level Sorcerer/Wizard Spells}
\begin{list}{\labelitemi}{\leftmargin=1em}
\item \nameref{Spell:ControlWeather}: Changes weather in local area.
\item \nameref{Spell:EtherealJaunt}: Touched creature becomes ethereal for 1 round/level.
\item \nameref{Spell:FingerOfDeath}: Kills one subject.
\item \nameref{Spell:LimitedWish}: Alters reality - within spell limits.
\item \nameref{Spell:MagesSword}: Floating magic blade strikes opponents.
\item \nameref{Spell:PhaseDoor}: Creates an invisible passage through wood or stone.
\item \nameref{Spell:PrismaticSpray}: Rays hit subjects with variety of effects.
\item \nameref{Spell:Refuge}: Alters item to transport its possessor to you.
\item \nameref{Spell:ReverseGravity}: Objects and creatures fall in a direction other than down for one round
\item \nameref{Spell:Sequester}: Subject is invisible to sight and scrying; renders creature comatose.
\end{list}
\subsubsection{8th-level Sorcerer/Wizard Spells}
\begin{list}{\labelitemi}{\leftmargin=1em}
\item \nameref{Spell:Binding}: Utilizes an array of techniques to imprison a creature.
\item \nameref{Spell:MindBlank}: Subject is immune to mental/emotional magic and scrying.
\item \nameref{Spell:MomentOfPrescience}: You gain insight bonus on single attack roll, check, or save.
\item \nameref{Spell:PrismaticWall}: Wall's colors have array of effects.
\item \nameref{Spell:Sunburst}: Blinds creatures and deals damage, more to undead and oozes.
\item \nameref{Spell:TelepathicBeacon}: Object or location attracts or repels certain creatures.
\item \nameref{Spell:TemporalStasis}: Puts subject into suspended animation.
\end{list}
\subsubsection{9th-level Sorcerer/Wizard Spells}
\begin{list}{\labelitemi}{\leftmargin=1em}
\item \nameref{Spell:AstralProjection}: Projects you and companions onto Astral Plane.
\item \nameref{Spell:Freedom}: Releases creature from imprisonment.
\item \nameref{Spell:Genesis}: You instigate a new demiplane on the Astral Plane.
\item \nameref{Spell:Imprisonment}: Entombs subject beneath the earth.
\item \nameref{Spell:PrismaticSphere}: As prismatic wall, but surrounds on all sides.
\item \nameref{Spell:TeleportationCircle}: Circle teleports any creature inside to designated spot.
\item \nameref{Spell:TimeStop}: You stop time momentarily.
\item \nameref{Spell:WailOfTheBanshee}: Kills multiple creatures.
\item \nameref{Spell:Wish}: As limited wish, but with fewer limits.
\end{list}
\subsubsection{Abjurer Spells}
\begin{list}{\labelitemi}{\leftmargin=1em}
\item[1] \nameref{Spell:MageArmor}: Gives subject +4 armor bonus.
\item[2] \nameref{Spell:Resistance} : Grants a Resistance bonus on saving throws.
\item[3] \nameref{Spell:Nondetection}: Masks object or creature against scrying.
\item[4] \nameref{Spell:GlobeOfInvulnerability}: Stops low-powered spell effects.
\item[5] \nameref{Spell:Dismissal}: Forces a creature to return to native plane.
\item[6] \nameref{Spell:AntimagicField}: Negates magic within 10 ft.
\item[7] \nameref{Spell:SpellTurning}: Reflect spells back at caster.
\item[8] \nameref{Spell:DimensionalLock}: Teleportation and interplanar travel blocked for one day/level within an area.
\item[9] \nameref{Spell:Disjunction}: Dispels magic, disenchants magic items.
\end{list}
\subsubsection{Conjurer Spells}
\begin{list}{\labelitemi}{\leftmargin=1em}
\item[1] \nameref{Spell:Fog}: Fog surrounds you.
% \item[1] \nameref{Spell:SummonMonster}: Calls extraplanar creature to fight for you.
\item[2] \nameref{Spell:MatterCreation}: Creates one cloth or wood object.
\item[2] \nameref{Spell:SummonDemon}: Summons a demon to fight for you.
\item[2] \nameref{Spell:SummonDevil}: Summons a devil to fight for you.
\item[3] \nameref{Spell:NoxiousVapors}: Nauseating vapors, 1 round/level.
\item[4] \nameref{Spell:BlackTentacles}: Tentacles grapple all within 20 ft. spread.
\item[4] \nameref{Spell:DimensionDoor}: Teleports you short distance.
\item[4] \nameref{Spell:SummonCelestial}: Summons a celestial to defend you.
\item[5] \nameref{Spell:PlanarBinding}: Traps extraplanar creature of 6 HD or less.
\item[6] \nameref{Spell:WallOfIron}: Flat iron wall appears, may topple on to foes.
\item[7] \nameref{Spell:PlaneShift}: As many as eight subjects travel to another plane.
\item[8] \nameref{Spell:Maze}: Traps subject in extradimensional maze.
\item[9] \nameref{Spell:Gate}: Connects two planes for travel or calling.
\end{list}
\subsubsection{Diviner Spells}
\begin{list}{\labelitemi}{\leftmargin=1em}
\item[1] \nameref{Spell:ComprehendLanguages}: You understand all spoken and written languages.
\item[2] \nameref{Spell:Clairvoyance}: See and hear a distant location.
\item[3] \nameref{Spell:MnemonicEnhancer}: You magically enhance your own ability to recall information.
\item[4] \nameref{Spell:Scrying}: Spies on subject from a distance.
\item[5] \nameref{Spell:ContactOtherPlane}: Lets you ask question of extraplanar entity.
\item[6] \nameref{Spell:LegendLore}: Lets you learn tales about a person, place, or thing.
\item[7] \nameref{Spell:Precognition}: You see the future of one creature.
\item[8] \nameref{Spell:AbsoluteRevelation}: Reveals exact location of creature or object.
\item[9] \nameref{Spell:Foresight}: ``Sixth sense`` warns of impending danger.
\end{list}
\subsubsection{Enchanter Spells}
\begin{list}{\labelitemi}{\leftmargin=1em}
\item[1] \nameref{Spell:Charm}: Makes one creature your friend.
\item[2] \nameref{Spell:ReadThoughts}: Detect surface thoughts of creatures in range.
\item[3] \nameref{Spell:Suggestion}: Compels subject to follow stated course of action.
\item[4] \nameref{Spell:GeasQuest}: Commands subject of 7 HD or less.
\item[5] \nameref{Spell:Dominate}: Controls humanoid telepathically.
\item[6] \nameref{Spell:DeadlyFright}*: Humanoid dies of fright.
\item[7] \nameref{Spell:PowerWord}: Overwhelms the minds of lesser creatures with a single word of power.
\item[8] \nameref{Spell:IrresistibleDance}: Forces subject to dance.
\item[9] \nameref{Spell:CrushingTheEssence}*: Suppresses all immunities to enchantments for one round.
\end{list}
\subsubsection{Evoker Spells}
\begin{list}{\labelitemi}{\leftmargin=1em}
\item[1] \nameref{Spell:Burn}: Creature catches on fire if it takes fire damage.
\item[2] \nameref{Spell:Fireball}: Deal 3d6 energy damage in a burst.
\item[2] \nameref{Spell:LightningBolt}: Deal 3d6 energy damage in a line.
\item[2] \nameref{Spell:Pyrotechnics}: Turns fire into blinding light or choking smoke.
\item[3] \nameref{Spell:ManeuveringHand}: Hand composed of force performs combat maneuvers.
\item[4] \nameref{Spell:ResilientSphere}: Force globe protects but traps one subject.
\item[5] \nameref{Spell:ConeOfCold}: 9d6 points of energy damage in a cone, and all in area must save or be slowed.
\item[6] \nameref{Spell:ChainLightning}: 1d6/level damage to multiple subjects.
\item[8] \nameref{Spell:PolarRay}:: Ranged touch attack deals cold damage and immobilizes.
\item[9] \nameref{Spell:Meteor}: Sphere descends, destroys everything.
\end{list}
\subsubsection{Illusionist Spells}
\begin{list}{\labelitemi}{\leftmargin=1em}
\item[1] \nameref{Spell:Image}: Creates illusion of your design.
\item[2] \nameref{Spell:Invisibility}: Subject is invisible for 1 min./level or until it attacks.
\item[2] \nameref{Spell:MirrorImage}: Creates decoy duplicates of you.
\item[3] \nameref{Spell:HallOfMirrors}*: Subject's movement is randomized.
\item[4] \nameref{Spell:ShadowConjuration}: Mimics certain conjurations.
\item[5] \nameref{Spell:ShadowEvocation}: Mimics certain Evocations.
\item[6] \nameref{Spell:Mislead}: Turns you invisible and creates illusory double.
\item[6] \nameref{Spell:ProgrammedImage}: Causes Image spell to be triggered by event.
\item[7] \nameref{Spell:Simulacrum}: Creates partially real double of a creature.
\item[8] \nameref{Spell:Screen}: Illusion hides area from vision, scrying.
\item[9] \nameref{Spell:RealityVeil}*: Creatures live forevermore in world of your imagination.
\end{list}
\subsubsection{Necromancer Spells}
\begin{list}{\labelitemi}{\leftmargin=1em}
\item[1] \nameref{Spell:Fear}: One creature flees for 1d4 rounds.
\item[2] \nameref{Spell:CommandUndead}: Undead creature obeys your commands.
\item[3] \nameref{Spell:GentleRepose}: Preserves one corpse.
\item[4] \nameref{Spell:AnimateDead}: Creates undead skeletons and zombies.
\item[4] \nameref{Spell:Enervation}: Subject gains 1d4 negative levels.
\item[5] \nameref{Spell:BalefulResurrection}: Returns subject from the dead - mostly.
\item[5] \nameref{Spell:Possession}: You assume a spirit form and take control of another creature's body.
\item[6] \nameref{Spell:CreateUndead}: Creates ghouls, and more powerful creatures with augment.
\item[6] \nameref{Spell:LifeAndDeath}: Kills living creatures or destroys undead creatures.
\item[7] \nameref{Spell:ControlUndead}: You force an undead creature to bow to your will.
\item[8] \nameref{Spell:HorridWilting}: Dessicates nearby creatures.
\item[9] \nameref{Spell:SoulBind}: Traps dead soul to prevent resurrection.
\end{list}
\subsubsection{Transmuter Spells}
\begin{list}{\labelitemi}{\leftmargin=1em}
\item[1] \nameref{Spell:AlterSize}: Humanoid creature changes size.
\item[2] \nameref{Spell:FormScout}: Subject gains the form of a nimble creature.
\item[3] \nameref{Spell:Fly}: Subject flies at speed of 40 ft.
\item[3] \nameref{Spell:FormAvian}: Subject gains the form of a bird.
\item[3] \nameref{Spell:FormFish}: Subject gains the form of a water-dwelling creature.
\item[4] \nameref{Spell:FormCarnivore}: Subject gains the form of a dangerous beast.
\item[4] \nameref{Spell:FormVermin}: Subject gains the form of a vermin.
\item[4] \nameref{Spell:FormViper}: Subject gains the form of a snake.
\item[5] \nameref{Spell:FormHorror}: Subject gains the form of a tentacled monstrosity.
\item[5] \nameref{Spell:FormTreant}: Subject gains the form of a plant creature.
\item[6] \nameref{Spell:FormDragon}: Subject gains the form of a mighty dragon.
\item[7] \nameref{Spell:FormElemental}: Subject becomes a living creature of the elements.
\item[8] \nameref{Spell:FormIronGolem}: Subject's body changes into a creature of living iron.
\item[9] \nameref{Spell:Shapechange}: Transforms you into any creature whose form you know, and change forms once per round.
\end{list}








\subsection{'A' Spells}
\subsubsection{Adapt Body}
\label{Spell:AdaptBody}
Transmutation
\\ \textbf{Level:} Ranger 5
\\ \textbf{Display:} V, S
\\ \textbf{Casting Time:} 1 standard action
\\ \textbf{Range:} Personal
\\ \textbf{Target:} You
\\ \textbf{Duration:} 1 hour/level (D)
\\ \textbf{Spell Points:} 9

\emph{Your body automatically adapts to hostile environments.} 

You gain the ability to adapt to underwater, extremely hot, extremely cold, or airless environments, allowing you to survive as if you were a creature native to that environment. 
You can breathe and move (though penalties to movement and attacks, if any for a particular environment, remain), and you take no damage simply from being in that environment.
You need not specify what environment you are adapting to when you manifest this power; simply activate it, and your body will instantly adapt to any hostile environment as needed throughout the duration.

You can somewhat adapt to extreme environmental features such as acid, lava, fire, and electricity. 
Any environmental feature that normally directly deals 1 or more dice of damage per round deals you only half the usual amount of damage.

\subsubsection{Arcane Flux}
\label{Spell:ArcaneFlux}
Abjuration
\\ \textbf{Level:} Spellbreaker 1
\\ \textbf{Components:} V
\\ \textbf{Casting Time:} 1 immediate action
\\ \textbf{Range:} Close (25 ft. + 5 ft./2 levels)
\\ \textbf{Target:} One creature
\\ \textbf{Duration:} Instantaneous
\\ \textbf{Saving Throw:} None; see text
\\ \textbf{Spell Resistance:} Yes
\\ \textbf{Spell Points:} 1

\emph{``You cause a momentary disruption in the normal flows of magic surrounding a creature, causing spellcasters trouble.''}

A creature affected by this spell while casting a spell is distracted. It must succeed on a \nameref{sec:Concentration} skill check against the spell's save DC or lose the spell.

\subsubsection{Armaments of Faith}
\label{Spell:ArmamentsOfFaith}
Conjuration (Creation)
\\ \textbf{Level:} Paladin 1
\\ \textbf{Components:} V, S
\\ \textbf{Casting Time:} 1 standard action
\\ \textbf{Range:} 0ft.
\\ \textbf{Effect:} A suit of armor AND/OR a shield AND/OR a weapon; see text
\\ \textbf{Duration:} 1 min./level
\\ \textbf{Saving Throw:} None
\\ \textbf{Spell Resistance:} No
\\ \textbf{Spell Points:} 1

\emph{You extend your arms and spread your legs as a suit of golden, slightly translucent armor assembles itself around your body, a shield appears in your left hand, and a warhammer in your right.}

You conjure the arms and armor needed to see you through a fight.
You can create one suit of armor, one shield, and one weapon with a single casting of this spell. 
You do not need to create all the items you are entitled to create.
These items can be of any kind with which you are proficient, are of masterwork quality, and sized and shaped for you. 
All the items appear on your body. 
If you conjure armor, you do not take penalties for donning your armor hastily.
You must have space on your body for all the items you conjure, for example, you could not conjure armor if you are already wearing armor, nor could you simultaneously conjure a shield and a greatsword (unless you have some ability to simultaneously use a shield and a greatsword).
An item removed from your body disappears instantly.
Regardless of appearance, treat the items as being made of steel for all purposes.

\paragraph{Augment:} You can augment this spell in one or more of the following ways:
\begin{enumerate}
 \item If you spend 4 additional spell points, this spell's duration increases to 24 hours.
 \item If you spend 4 additional spell points, the items created by this spell are treated as being made of cold iron rather than steel, allowing the weapon to bypass the damage reduction of some creatures.
 \item If you spend 4 additional spell points, the items created by this spell are treated as being made of alchemical silver rather than steel, allowing the weapon to bypass the damage reduction of some creatures.
 \item If you spend 8 additional spell points, the items created by this spell are treated as being made of adamantine rather than steel, increasing their resiliance and allowing the weapon to bypass the damage reduction of some creatures.
\end{enumerate}

\subsubsection{Arrow of Death}
\label{Spell:ArrowOfDeath}
Necromancy [Death]
\\ \textbf{Level:} Ranger 6
\\ \textbf{Components:} V
\\ \textbf{Casting Time:} 1 swift action
\\ \textbf{Range:} Touch
\\ \textbf{Target:} One arrow or bolt touched
\\ \textbf{Duration:} Until end of round; see text
\\ \textbf{Saving Throw:} None, Fort negates; see text
\\ \textbf{Spell Resistance:} No, Yes; see text
\\ \textbf{Spell Points:} 11

\emph{The arrow turns black as the midnight sky as you finish the incantation.}

A creature struck by the projectile must succeed on a fortitude save or be instantly slain.
Spell resistance applies against the death effect, as does immunity or resistance to death or necromancy effects.
The projectile remains so enhanced only until the end of the current turn. 
If the projectile does not hit a creature before the end of the turn, the spell drains away harmlessly.

\subsubsection{Assault of the Sevenfold Heaven}
\label{Spell:AssaultOfTheSevenFoldHeaven}
Evocation
\\ \textbf{Level:} Paladin 6
\\ \textbf{Components:} V, S
\\ \textbf{Casting Time:} 1 standard action
\\ \textbf{Range:} Touch
\\ \textbf{Target:} Weapon touched
\\ \textbf{Duration:} 1 round; See text
\\ \textbf{Saving Throw:} None; See text
\\ \textbf{Spell Resistance:} No
\\ \textbf{Spell Points:} 11

\emph{Seven bursts of light flash from your weapon upon impact, searing your target with the righteous power of the upper planes.}

The target of your next successful attack with the weapon 
(if it is made before the end of the spell's duration)
is subjected to each of the following effects, in order:
\begin{enumerate}
 \item 10 points of fire damage (Reflex half)
 \item 20 points of acid damage (Reflex half)
 \item 40 points of electricity damage (Reflex half)
 \item 1d8 points of constitution damage (Fortitude half)
 \item Turned to stone, as if by a \nameref{Spell:TransmuteFleshAndStone} spell, except the duration is only 1 round (Fortitude negates)
 \item \emph{Confused}, as if by a \nameref{Spell:Confusion} spell (Will negates)
 \item Sent to one of the upper planes for 1 round (Will negates). Unless killed or retained by the powers of those planes in the interim, the subject returns at the end of this duration.
\end{enumerate}

\subsubsection{Astral Trap}
\label{Spell:AstralTrap}
Conjuration (Teleportation)
\\ \textbf{Level:} Planes 3
\\ \textbf{Components:} V, S
\\ \textbf{Casting Time:} 1 standard action
\\ \textbf{Range:} Medium (100 ft. + 10 ft./level)
\\ \textbf{Target:} One creature
\\ \textbf{Duration:} 1 round/level (D); see text
\\ \textbf{Saving Throw:} Will negates; see text
\\ \textbf{Spell Resistance:} Yes
\\ \textbf{Spell Points:} 5

\emph{A rip to the Astral Plane opens next to the creature, pulling it in.}

The subject is (safely) teleported to a random location on the Astral Plane unless it succeeds on a Will save.
It reappears when the duration of this spell expires, in the same orientation and condition as before.

In each round of the spell's duration, it can attempt a Will save as a full-round action. 
Success allows the subject to return. 
The subject can act normally on its next turn after this spell ends.

If the subject is capable of planar travel, he can use those abilities normally. Traveling out of the trap in this way ends the 

If the space from which the subject departed is occupied upon his return, he appears in the closest unoccupied space, still in his original orientation. Determine the closest space randomly if necessary.

\paragraph{Augment:} For every 2 additional spell points you spend, the spell can target an additional creature within range.

\subsubsection{Aura of Retribution}
\label{Spell:AuraOfRetribution}
Abjuration 
\\ \textbf{Level:} Dread Knight 1
\\ \textbf{Components:} V, S
\\ \textbf{Casting Time:} 1 standard action
\\ \textbf{Range:} 20 ft.
\\ \textbf{Area:} 20-ft.-radius emanation, centered on you
\\ \textbf{Duration:} 1 round/level
\\ \textbf{Saving Throw:} Will partial
\\ \textbf{Spell Resistance:} Yes

\emph{An aura of your own particular justice surrounds you, doing to others as they would do to you.}

Any creature within range that directly deals damage to you while the spell is in effect is fatigued for 1 round. A successful Will save negates the fatigue, but an attacker must save again should he damage you again.

\paragraph{Augment:} This spell can be augmented in one of the following ways:
\begin{enumerate}
 \item If you spend two additional spell points, the target is exhausted rather than fatigued.
 \item If you spend six additional spell points, the target is stunned rather than fatigued.
\end{enumerate}

\subsection{'B' Spells}

\subsubsection{Baleful Resurrection}
\label{Spell:BalefulResurrection}
Necromancy (Healing) [Evil]
\\ \textbf{Level:} Death 4, Necromancer 5
\\ \textbf{Casting Time:} 1 hour
\\ \textbf{Spell Points:} Death 7, Necromancer 9; XP; see text

\emph{''I don't know. He just came back... wrong, somehow.``}

This spell functions as the \nameref{Spell:RaiseDead} spell, but with a few significant differences, as outlined here.

In addition to losing a level or point of constitution, the creature comes back to life with one of the curses on the \nameref{tab:BalefulResurrection} table, determined randomly. If it already suffers from that curse (or has the corresponding penalties for another reason), roll again. If no curses remain, the spell fails. Finally, the subject's alignment is shifted one step towards Evil. A creature who is already Evil has its aligment shifted one step towards Chaotic. A Chaotic Evil creature does not suffer a change in alignment.

\begin{table*}
\label{tab:BalefulResurrection}
\caption{Baleful Resurrection curses}
\centering
\begin{tabular}{p{0.065\textwidth}p{0.15\textwidth}p{0.7\textwidth}}
\toprule
d\%&  Curse name&  Curse description\\
\midrule
1-13&Bloody cough& The creature occasionally coughs up blood, often with a horrid hacking sound. It takes a -3 penalty on Fortitude saves.\\
14-25&Dead eyes& Most of the color drains out of the creature's eyes, making them resemble that of a corpse. It takes a -6 penalty on Spot checks, and gains the Light Blindness trait, as a Drow.\\
26-38&Dead skin& The creature's skin turns pale and cold, and its sense of touch is greatly reduced. It takes a -3 penalty on Reflex saves.\\
39-50&Hallucinations& The creature suffers from intermittent hallucinations, causing it difficulties in differentiating figment from reality. The creature takes a -3 penalty on Will saves.\\
51-62&Headaches& The creature suffers from constant, distracting headaches. It takes a -2 penalty on Initiative checks and all intelligence-based skill checks.\\
63-75&Social outcast& Intelligent creatures subconsciously dislike the subject. The initial attitude of all non-Evil non-player characters towards the subject is shifted one step towards hostile, and it takes a -6 penalty on Gather Information checks.\\
76-87&Undeath's taint& The creature's brush with death has left it vulnerable to sacred energies. It is treated as an undead creature when caught in the area of a \nameref{Feat:TurnUndead} effect, although it only takes half damage (and may save to halve that damage again). In addition, Necromancy (Healing) spells only heal the creature of half their normal number of hit points.\\
88-100&Wrong smell& Nonintelligent creatures dislike the subject. The initial response of all animals when encountering the subject is either try to flee or to attack it. It takes a -6 penalty on Handle Animal and Ride checks.\\
\bottomrule
\end{tabular}
\end{table*}

A \nameref{Spell:Wish} or \nameref{Spell:Miracle} spell can remove the creature's curse. A normal \nameref{Spell:RemoveCurse} spell is effective only when cast by a caster of 11th level or higher. An \nameref{Spell:Atonement} spell can restore its original alignment (following the usual rules for the Atonement spell). The level/HD loss or Constitution loss can still not be repaired by any means.

A creature successfully raised with this spell becomes alive and stable at -9 hit points rather than with hit points equal to its HD. Any wounds and disease symptoms repaired as part of the resurrection process are only repaired to the absolute minimum functionality required to return the creature to life. Even if the creature's hit point total is fully restored, such wounds still leave scars, unless removed with the \nameref{Spell:Regenerate} spell. Creatures with visible scars from this process gain a +4 bonus in Intimidate checks, but suffer a -4 penalty on Diplomacy checks.

\subsubsection{Banishing Weapon}
\label{Spell:BanishingWeapon}
Transmutation [see text]
\\ \textbf{Level:} Blackguard 5, Paladin 5
\\ \textbf{Components:} V, S
\\ \textbf{Casting Time:} 1 standard action
\\ \textbf{Range:} Touch
\\ \textbf{Targets:} One melee weapon
\\ \textbf{Duration:} 1 round/level
\\ \textbf{Saving Throw:} Will negates (harmless, object); see text
\\ \textbf{Spell Resistance:} Yes (harmless, object)
\\ \textbf{Spell Points:} 9

\emph{Your weapon becomes a tool to banish outsiders.}

When casting this spell, choose an alignment with which to imbue your weapon. The spell gains that alignment as a descriptor, and banishes outsiders of the opposed alignment.

Any outsider of the appropriate alignment with HD equal to or less than 9 must succeed on a Will save or be banished back to its home plane if struck in combat with a weapon affected by this spell. 
Spell resistance does not apply against the banishment effect.

\paragraph{Augment:} For every additional spell point you spend, the weapon can affect an outsider with 2 more HD.
\subsubsection{Blade of the Sun}
\label{Spell:BladeOfTheSun}
Evocation [Good, Light]
\\ \textbf{Level:} Paladin 2, Sun 2
\\ \textbf{Components:} V
\\ \textbf{Casting Time:} 1 swift action
\\ \textbf{Range:} Touch
\\ \textbf{Target:} Melee weapon touched
\\ \textbf{Duration:} 1 round
\\ \textbf{Saving Throw:} None
\\ \textbf{Spell Resistance:} No
\\ \textbf{Spell Points:} 3

\emph{As you raise your blade high, it begins to shine with the white-hot power of the sun.}

For the duration of the spell, the touched weapon gains the benefits of the \emph{Brilliant Energy} weapon enhancement in addition to any other enchancements it may already have.
The weapon also shines bright light out to a 40' radius, and dim light another 40' beyond that. 
Even after the spell ends, the weapon sheds light as a torch for 1 round.

Casting this spell on a weapon that already has the Brilliant Energy enhancement has no additional effects, 
except with respect to the lighting and interaction with \nameref{Spell:DispelMagic} and similar effects.

\subsubsection{Blade Storm}
\label{Spell:BladeStorm}
Evocation [Force]
\\ \textbf{Level:} War 8
\\ \textbf{Components:} V, S
\\ \textbf{Casting Time:} 1 standard action
\\ \textbf{Range:} 20 ft.
\\ \textbf{Area:} 20-ft.-radius burst, centered on you
\\ \textbf{Duration:} Instantaneous
\\ \textbf{Saving Throw:} Reflex half
\\ \textbf{Spell Resistance:} Yes
\\ \textbf{Spell Points:} 15

\emph{Whirling blades of force explode out from your outstretched arms, eviscerating your enemies.} 

Creatures within the affected area take 15d6 points of force damage and are stunned for 1 round.
A successful Reflex saving throw halves the damage and negates the stun.

You can specifically select any number of creatures within the spell area to exclude from the spell's effect.

\paragraph{Augment:} For every additional spell point you spend, this spell's damage increases by one die (d6).

\subsubsection{Bleed}
\label{Spell:Bleed}
Necromancy
\\ \textbf{Level:} Destruction 1
\\ \textbf{Components:} V, S
\\ \textbf{Casting Time:} 1 standard action
\\ \textbf{Range:} Touch
\\ \textbf{Target:} One living creature
\\ \textbf{Duration:} Instantaneous
\\ \textbf{Saving Throw:} Fortitude half
\\ \textbf{Spell Resistance:} Yes
\\ \textbf{Spell Points:} 1

\emph{The subject suddenly begins bleeding from every orifice.}

The target takes 1d3 points of Constitution damage. A successful Fortitude save reduces the Constitution damage by half (half of one point of damage is no damage).

\paragraph{Augment:} For every 3 additional spell points you spend, this spell's Constitution damage increases by one die (d3).

\subsubsection{Blindsense}
\label{Spell:Blindsense}
Divination
\\ \textbf{Level:} Assassin 3, Sor/Wiz 3
\\ \textbf{Components:} V, S
\\ \textbf{Casting Time:} 1 standard action
\\ \textbf{Range:} Touch
\\ \textbf{Target:} Creature touched
\\ \textbf{Duration:} 1 min/level
\\ \textbf{Saving Throw:} Will negates (harmless)
\\ \textbf{Spell Resistance:} Yes (harmless)
\\ \textbf{Spell Points:} 3

\emph{Your senses widen, revealing things that were once hidden.}

The subject of this spell gains Blindsense out to 30'. 

\paragraph{Augment:} You can Augment the spell in one or both of the following ways:
\begin{enumerate}
 \item For every additional spell point you spend, the range of the blindsense increases by 10'.
 \item If you spend two additional spell points, the spell's dueation increases to 10 minutes per level.
\end{enumerate}

\subsubsection{Bloodhound}
\label{Spell:Bloodhound}
Transmutation
\\ \textbf{Level:} Ranger 1
\\ \textbf{Components:} V, S
\\ \textbf{Casting Time:} 1 standard action
\\ \textbf{Range:} Touch
\\ \textbf{Target:} Creature touched
\\ \textbf{Duration:} 10 min./level
\\ \textbf{Saving Throw:} None
\\ \textbf{Spell Resistance:} Yes (harmless)

\emph{``Your nose twitches, and the world gains a whole new dimension.''}

The subject gains the scent special ability, and a +2 enhancement bonus on Survival checks when tracking.

\paragraph{Augment:} For every additional spell point you spend, this spell's enhancement bonus on Survival checks increases by 1.

\subsubsection{Bloodletting Arrow}
\label{Spell:BloodlettingArrow}
Transmutation
\\ \textbf{Level:} Ranger 3
\\ \textbf{Components:} V
\\ \textbf{Casting Time:} 1 swift action
\\ \textbf{Range:} Touch
\\ \textbf{Target:} One arrow or bolt touched
\\ \textbf{Duration:} Until end of round; see text
\\ \textbf{Saving Throw:} None, Fort half; see text
\\ \textbf{Spell Resistance:} No
\\ \textbf{Spell Points:} 5

\emph{Blood pours out of the wound inflicted by your arrow like a stream.}

A creature struck by the projectile takes 1d6 points of Constitution damage.
A successful Fortitude save reduces the Constitution damage by half.
The projectile remains so enhanced only until the end of the current turn. 
If the projectile does not hit a creature before the end of the turn, the spell drains away harmlessly.

\paragraph{Augment:} For every 6 additional spell points you spend, this spell deals an additional 1d6 points of Constitution damage.

\subsubsection{Bow of the Solar}
\label{Spell:BowOfTheSolar}
Necromancy [Death, Good]
\\ \textbf{Level:} Paladin 6
\\ \textbf{Components:} V, S
\\ \textbf{Casting Time:} 1 standard action
\\ \textbf{Range:} Long (400 ft. + 40 ft./level)
\\ \textbf{Effect:} Ray
\\ \textbf{Duration:} Instantaneous
\\ \textbf{Saving Throw:} Fortitude negates
\\ \textbf{Spell Resistance:} Yes
\\ \textbf{Spell Points:} 11

\emph{A great, partially translucent bow appears in your hands, lasting just long enough for you to fire a single arrow.}

You must succeed on a ranged touch attack to hit your opponent.
On a hit, the opponent must succeed on a Fortitude saving throw or be instantly slain.
An evil creature takes a -4 penalty on the save.
If the save is successful, the creature instead takes damage as if struck by a +2 arrow shot from a composite longbow sized for you, including your full strength modifier and all other situational modifiers that may apply.

\subsection{'C' Spells}
\subsubsection{Call of the Moon}
\label{Spell:CallOfTheMoon}
Enchantment (Compulsion) [Mind-Affecting]
\\ \textbf{Level:} Moon 3
\\ \textbf{Components:} V
\\ \textbf{Casting Time:} 1 standard action
\\ \textbf{Range:} Close (25 ft. + 5 ft./2 levels)
\\ \textbf{Target:} Up to five living creatures, no two of which are more than 30 feet apart
\\ \textbf{Duration:} Instantaneous
\\ \textbf{Saving Throw:} Will partial; see text
\\ \textbf{Spell Resistance:} Yes
\\ \textbf{Spell Points:} 5

\emph{You bring out the image of a full moon in the minds of your subjects.}

A lycanthrope affected by this spell must succeed on a will save or be forced to assume its animal form at the earliest opportunity - usually on its next action. A lycanthrope already in its animal form is not affected.
Even if the lycanthrope succeeds on the will save, you know that the spell had an effect on the subject - that is, you know with absolute certainty that the creature is in fact a lycanthrope.

Non-lycanthropes suffer no ill effect from this spell.
\paragraph{Augment:}
For every additional spell point you spend, this spell can affect an additional target.%, and its saving throw DC increases by 1. 
No target of the spell can be more than 30 feet from another target of the spell.

\subsubsection{Celestial Flight}
\label{Spell:CelestialFlight}
Transmutation [Good]
\\ \textbf{Level:} Paladin 3

\emph{The subject grows great, white, angelic wings.}

This spell functions as the \nameref{Spell:Fly} spell (including its augmentation option), except as noted here, and that the flight is due to physical wings the subject grows (which matters for effects like those cause by tanglefoot bags).

\subsubsection{Corpse Explosion}
\label{Spell:CorpseExplosion}
Necromancy
\\ \textbf{Level:} Death 7
\\ \textbf{Components:} V, S
\\ \textbf{Casting Time:} 1 standard action OR 1 immediate action; see text
\\ \textbf{Range:} Close (25 ft. + 5 ft./2 levels)
\\ \textbf{Target:} One fresh corpse OR recently destroyed undead; see text
\\ \textbf{Duration:} Instantaneous
\\ \textbf{Saving Throw:} None, Reflex half; see text
\\ \textbf{Spell Resistance:} None
\\ \textbf{Spell Points:} 13

\emph{You extend your arm, and the corpse begins to bulge.}

The corpse explodes in a \hyperref[sec:AreaShapes]{burst} of bone and flesh.
Any creature within 20' of the corpse takes 1d6 points of damage per HD the corpse had at the time of death. A successful Reflex save halves this damage. 

The targeted corpse must be fresh, dead for no more than one hour per caster level. It must be mostly in one piece prior to the casting of this spell. A corpse that never was alive to begin with is not a corpse for the purposes of this spell. 

Alternatively, you can target a recently destroyed undead creature, destroyed for no more than one hour per caster level. When targeting a destroyed undead creature, use the number of HD it had at the time of destruction for purposes of calculating the damage dealt by this spell.

Ordinarily, casting this spell is a standard action. However, if you catch a creature at the moment of its death (or destruction, in the case of an undead creature), you may cast this spell as an immediate action, causing it to explode on the spot.

\subsubsection{Crisis of Breath}
\label{Spell:CrisisOfBreath}
Necromancy [Mind-Affecting]
\\ \textbf{Level:} Sor/Wiz 3
\\ \textbf{Components:} V, S
\\ \textbf{Casting Time:} 1 standard action
\\ \textbf{Range:} Medium (100 ft. + 10 ft./ level)
\\ \textbf{Target:} One breathing humanoid
\\ \textbf{Duration:} 1 round/level
\\ \textbf{Saving Throw:} Will negates, Fortitude partial; see text
\\ \textbf{Spell Resistance:} Yes
\\ \textbf{Spell Points:} 5

\emph{You force the subject to purge its entire store of air in one explosive exhalation, and prevent it from drawing breath again.}

The subject's lungs do not automatically function again while the spell's duration lasts.
If the target succeeds on a Will save when crisis of breath is cast, it is unaffected by this spell. 
If it fails its Will save, it can still continue to breathe by taking a standard action in each round to gasp for breath.
An affected creature can attempt to take actions normally (instead of consciously controlling its breathing), 
but each round it does so, beginning in the round when it failed its Will save, the subject risks blacking out from lack of oxygen. 
It must succeed on a Fortitude save at the end of any of its turns in which it did not consciously take a breath. 
The DC of this save increases by 1 in every consecutive round after the first one that goes by without a breath; 
the DC drops back to its original value if the subject spends an action to take a breath.
If a subject fails a Fortitude save, it is disabled (0 hp). 
In the following round, it drops to -1 hit points and is dying. 
Curing spells can revive a dying subject normally, so long as this spell's duration has expired; 
if the spell is still in effect, a revived creature is still subject to Fortitude saves in each round when it does not consciously breathe.

\paragraph{Augment:} You can augment this spell in one or more of the following ways.
\begin{enumerate}
 \item If you spend 2 additional spell points, this spell can also affect an animal, fey, giant, magical beast, or monstrous humanoid.
 \item If you spend 4 additional spell points, this spell can also affect an aberration, dragon, elemental, or outsider in addition to the creature types mentioned above.
 \item If you spend 6 additional spell points, this spell can affect up to four creatures all within a 20-ft.-radius burst.
\end{enumerate}

\subsubsection{Curse of Lycanthropy}
\label{Spell:CurseOfLycanthropy}
Transmutation
\\ \textbf{Level:} Moon 7
\\ \textbf{Components:} V, S
\\ \textbf{Casting Time:} 1 hour
\\ \textbf{Range:} Touch
\\ \textbf{Target:} Humanoid or giant touched
\\ \textbf{Duration:} Instantaneous
\\ \textbf{Saving Throw:} Fortitude negates
\\ \textbf{Spell Resistance:} Yes
\\ \textbf{Spell Points:} 13

\emph{''Welcome to being a monster.''}

The creature becomes an afflicted lycanthrope, of a type of your choosing. You cannot afflict a creature who is already a lycanthrope with a second strain of lycanthropy.
\subsection{'D' Spells}

\subsubsection{Dancing Blade}
\label{Spell:DancingBlade}
Transmutation 
\\ \textbf{Level:} Dread Knight 2, Spellbreaker 2
\\ \textbf{Components:} V, S, AF, 
\\ \textbf{Casting Time:} 1 standard action
\\ \textbf{Range:} 60 ft.
\\ \textbf{Effect:} 60-ft. line
\\ \textbf{Duration:} Instantaneous
\\ \textbf{Saving Throw:} None
\\ \textbf{Spell Resistance:} No
\\ \texttt{Spell Points:} 3

\emph{Your weapon gains momentary life of its own, slashing and stabbing your enemies to the best of its ability before resuming its ordinary function.}

Your weapon flies out, attacks enemies in a line, and unerringly returns to your hand. 

Resolve one normal, melee attack against each enemy within the spell's area, with the exception that you may use your key ability modifier in place of your strength modifier on the weapon's attack and damage rolls. The weapon's attacks are resolved just as if you had performed normal melee attacks, allowing you to apply all feats and other effects normally associated with your melee attacks.

You must hold a melee weapon to set loose in order to cast this spell.

\subsubsection{Dimensional Barrier}
\label{Spell:DimensionalBarrier}
Abjuration
\\ \textbf{Level:} Spellbreaker 5
\\ \textbf{Range:} 20 ft.
\\ \textbf{Area:} 20-ft.-radius emanation, centered on you
\\ \textbf{Duration:} One round/level (D)
\\ \textbf{Spell Points:} 9

\emph{You emanate a pulsating emerald field of energy, blocking extradimensional travel }

This spell functions as \nameref{Spell:DimensionalLock} (excluding augmentation options), except as noted here.

The spherical field you bring into being is centered on you and moves with you.
\subsubsection{Discern Spellcaster}
\label{Spell:DiscernSpellcaster}
Divination
\\ \textbf{Level:} Spellbreaker 1
\\ \textbf{Components:} V
\\ \textbf{Casting Time:} 1 standard action
\\ \textbf{Range:} Close (25 ft. + 5 ft./2 levels)
\\ \textbf{Target:} One creature
\\ \textbf{Duration:} Instantaneous
\\ \textbf{Saving Throw:} Will negates
\\ \textbf{Spell Resistance:} Yes
\\ \textbf{Spell Points:} 1

\emph{``A WITCH! BURN HER!''}

This spell reveals to you whether the target of the spell has any spellcasting ability. Spell-like and Supernatural abilities do not register.

If the subject resists the spell with a successful Will save or you fail to overcome its spell resistance, the spell reveals no information. Otherwise, the spell reveals an absolute answer (either ``no spellcasting ability'' or ``spellcasting ability'').

\paragraph{Augment:} For every additional spell point you spend, this spell can affect an additional target within range.

\subsubsection{Dismantle Magic}
Abjuration
\\ \textbf{Level:} Abjurer 1, Magic 1, Spellbreaker 1
\\ \textbf{Casting Time:} 1 minute

\emph{One by one, you pull the threads of magic apart.}

This spell functions like \nameref{Spell:DispelMagic} (excluding the augmentation option), except as noted here.

\paragraph{Augment:} For every additional spell point you spend, you gain a +2 bonus on your dispel check. You may spend no more than six additional spell points in this way.
\subsubsection{Divine Footstep}
\label{Spell:DivineFootstep}
Transmutation [Air, Good]
\\ \textbf{Level:} Paladin 2
\\ \textbf{Components:} V
\\ \textbf{Casting Time:} 1 swift action
\\ \textbf{Range:} Personal
\\ \textbf{Target:} You
\\ \textbf{Duration:} 1 round
\\ \textbf{Spell Points:} 3

\emph{You take a step into the air, allowing your faith to support your foot as you leap forward.}

This spell functions as the \nameref{Spell:AirWalk} spell (the Air Walk augment not included), except as noted here.

\paragraph{Augment:} For every 3 additional spell points you spend, this spell's duration increases by 1 round.

\subsubsection{Dweomer Rip}
\label{Spell:DweomerRip}
Abjuration
\\ \textbf{Level:} Magic 2
\\ \textbf{Components:} V, S
\\ \textbf{Casting Time:} 1 standard action
\\ \textbf{Range:} Close (25 ft. + 5 ft./2 levels)
\\ \textbf{Target:} One creature
\\ \textbf{Duration:} Instantaneous
\\ \textbf{Saving throw:} Fortitude half
\\ \textbf{Spell Resistance:} No
\\ \textbf{Spell Points:} 3

\emph{You pull at the strings of magic surrounding a creature, each cutting it like a thread.}

If a creature targeted by this spell has active spell effects on it at the time of casting,
it takes damage equal to twice the combined number of spell points that were spent on the spells, 
excluding those spell points spent to apply metamagic feats to the spells.
For example, a Wizard under the protection of a \nameref{Spell:MageArmor} spell augmented to cost 3 spell points and an unaugmented, quickened \nameref{Spell:Shield} spell
would take 8 points of damage.

A successful fortitude save halves the damage. The spells are not dispelled.
\paragraph{Augment:} For every 3 additional spell points you spend, this spell can affect an additional target.

\subsection{'E' Spells}

\subsubsection{Eldritch Orb}
\label{Spell:EldritchOrb}
Evocation
\\ \textbf{Level:} Dread Knight 1, Sor/Wiz 1
\\ \textbf{Components:} V, S
\\ \textbf{Casting Time:} 1 swift action
\\ \textbf{Range:} 0 ft.
\\ \textbf{Effect:} Orb in your palm
\\ \textbf{Duration:} 1 min./level (D)
\\ \textbf{Saving Throw:} None
\\ \textbf{Spell Resistance:} Yes
\\ \textbf{Spell Points:} 1

\emph{``The warrior drew his sword, and an orb of angry, purple energy appeared in the mage's hand. The time for negotion was over.''}

The evoked orb can be hurled or used to touch enemies in melee. It provides illumination as a candle.

Used as a melee weapon, it deals 1d6 points of damage on a successful touch attack.
As a thrown weapon, it can be hurled up to 120 feet (with no range penalty), and deal the same damage on a successful ranged touch attack. Immediately after throwing the orb, another appears in your hand.

Each attack you make reduces the remaining duration by 1 minute. 
If an attack reduces the remaining duration to 0 minutes or less, the spell ends after the attack resolves.

An eldritch orb leaves scorch marks on creatures and surfaces it strikes, but doesn't ignite anything but the most flammable materials (such as oil, paper, or dry straw).

\paragraph{Augment:} For every 3 additional spell points you spend, this spell's damage increases by one die (d6).

\subsubsection{Elemental Weapons}
\label{Spell:ElementalWeapons}
Evocation [see text]
\\ \textbf{Level:} Ranger 1
\\ \textbf{Components:} V
\\ \textbf{Casting Time:} 1 swift action
\\ \textbf{Range:} Personal
\\ \textbf{Target:} You
\\ \textbf{Duration:} 1 round
\\ \textbf{Spell Points:} 1

\emph{Your weapons hiss with the primal forces of nature.}

For the duration of this spell, all of your weapon attacks (including ranged attacks, but not your spells, spell-like abilities, or supernatural abilities) deal an additional 1d6 points of energy damage. Choose one of the following energy types upon casting: \emph{Cold}, \emph{Electricity}, or \emph{Fire}.
This spell's descriptor is the same as the type of energy you selected. 
The damage dealt by this spell stacks with any other elemental damage your weapon may deal.

\paragraph{Augment:} For two additional spell points you spend, this spell's damage increases by one die (d6).

\emph{Special:} Casting this spell does not provoke Attacks of Opportunity.

\subsubsection{Elfcloak}
\label{Spell:Elfcloak}
Divination
\\ \textbf{Level:} Ranger 1
\\ \textbf{Components:} V, S
\\ \textbf{Casting Time:} 1 standard action
\\ \textbf{Range:} Personal
\\ \textbf{Target:} You
\\ \textbf{Duration:} 10 min./level (D)
\\ \textbf{Spell Points:} 1

\emph{Your clothes and equipment subtly takes on the color and texture of nearby objects, including floors and walls.}

You gain a +5 competence bonus on Hide checks.

\paragraph{Augment:} For every additional spell point you spend, the Hide check bonus granted by this spell increases by 1.

\subsection{'F' Spells}

\subsubsection{Feedback Orb}
\label{Spell:FeedbackOrb}
Abjuration
\\ \textbf{Level:} Sor/Wiz 4, Spellbreaker 3
\\ \textbf{Components:} V, S
\\ \textbf{Casting Time:} 1 standard action
\\ \textbf{Range:} Close (25 ft. + 5 ft./2 levels)
\\ \textbf{Target:} One creature
\\ \textbf{Duration:} Instantaneous
\\ \textbf{Saving Throw:} None
\\ \textbf{Spell Resistance:} Yes
\\ \textbf{Spell Points:} Sor/Wiz 7, Spellbreaker 5

\emph{The brilliant white orb speeds towards the mage, erupting into sparks upon impact.}

You fire the orb as a ranged touch attack at any creature within range.

On a successful hit, the target takes 1d4 points of damage. If the target has spell points, he loses five of their daily spell points (to a minimum of zero). He then takes an additional 1d4 points of damage for each spell point lost.

\paragraph{Augment:} For every additional spell point you spend, the target loses an additional spell point, with a corresponding increase in potential damage.

\subsubsection[Swift Fly]{Fly, Swift}
\label{Spell:SwiftFly}
Transmutation
\\ \textbf{Level:} Spellbreaker 2
\\ \textbf{Components:} V, S
\\ \textbf{Casting Time:} 1 swift action
\\ \textbf{Range:} Personal
\\ \textbf{Target:} You
\\ \textbf{Duration:} 2 rounds
\\ \textbf{Spell Points:} 3

\emph{You rise from the ground in a swirl of rapidly burning arcane energy.}

This spell functions as the \nameref{Spell:Fly} spell (excluding augmentation options), except as noted here.

If the spell expires while the target is still aloft, it only floats for one round, rather than 1d6 rounds.

\paragraph{Augment:} For every 2 additional spell points you spend, this spell's duration is increased by one round.

\subsubsection{Forced Visions}
\label{Spell:ForcedVisions}
Divination [Mind-Affecting]
\\ \textbf{Level:} Sor/Wiz 3
\\ \textbf{Components:} V, S
\\ \textbf{Casting Time:} 1 standard action
\\ \textbf{Range:} Medium (100 ft. + 10 ft./ level)
\\ \textbf{Target:} One creature with an intelligence score of 3 or more
\\ \textbf{Duration:} 1 round/level
\\ \textbf{Saving Throw:} Will negates; see text
\\ \textbf{Spell Resistance:} Yes
\\ \textbf{Spell Points:} 5

\emph{Your victim staggers around, unable to tell the images in its head from what is happening around it.}

Unlike most divination spells, this spell is less about retrieving information than it is to force it upon someone.
For the duration of this spell, the subject's mind is haunted by fairly useless (but factually accurate) 
visions of his past, present, and sometimes even future.
Unless he succeeds on a Will save, the subject is \emph{stunned} for the first round of the spell's duration, the images
momentarily overwhelming its conscious mind. Every round thereafter, at the start of his turn, he must succeed on a will save or be \emph{confused} for that round. 
When such a save succeeds, the subject can suppress the images to the point where it no
longer interferes with his actions, effectively ending the spell.

If desired, the GM can use the following table to determine the kind of vision the subject suffers:
\begin{tableonecolumn}
\caption{Random Visions}
\label{tab:RandomVisions}
\begin{tabular}{p{1cm}p{5.4cm}}
\toprule
d\% &Vision\\
\midrule
1-30 & Childhood memories\\
31-60 & Memories regarding the subject's love life\\
61-70 & Memory resulting in the character shouting out an embarrassing fact about himself\\
71-80 & Memories regarding the character's line of work or training\\
81-90 & A vision of the character as an old man (Does not have to imply that he \emph{will} be old, only that it's a possibility.)\\
91-95 & A vision of the character's surroundings, distorted and confusing\\
96-99 & A vision of the character, as seen from the caster's eyes\\
100 & A truly useful vision of the future\\
\bottomrule
\end{tabular}
\end{tableonecolumn}

\paragraph{Augment:} For every 2 additional spell points you spend, the subject is stunned for one additional round of the
spell's duration, rather than just the first. A successful will save in a subsequent round still ends the spell.

\subsubsection{Form of the Celestial}
\label{Spell:FormCelestial}
Transmutation (Polymorph) [Good]
\\ \textbf{Level:} Paladin 6, Planes 6
\\ \textbf{Components:} V,S
\\ \textbf{Casting Time:} 1 standard action
\\ \textbf{Range:} Personal
\\ \textbf{Target:} You
\\ \textbf{Duration:} 10 min./level (D)
\\ \textbf{Spell Points:} 11

\emph{You assume the form of a celestial creature, sprouting angelic wings and your skin taking on a different hue.}

You undergo the following changes:
\\ You grow physical wings and can fly at a speed of 100', with good maneuverability. 
\\ You gain darkvision out to 60', and low-light vision.
\\ You gain immunity to acid, cold, and petrification.
\\ You gain resistance to electricity 10 and fire 10.
\\ You gain immunity to poison.
\\ You gain the ability to understand, speak, and write all languages, as if by a \nameref{Spell:ComprehendLanguages} spell with the first and second augments.
\\ You gain immunity to some spells, as if you were under the protection of a \nameref{Spell:GlobeOfInvulnerability} spell augmented to cost as many spell points as you spent on this spell. You do not emanate a globe, you only gain the protection yourself.

Your form is humanoid, with functioning hands and legs.

\subsubsection{Free Run}
\label{Spell:FreeRun}
Transmutation
\\ \textbf{Level:} Ranger 6, Travel 7
\\ \textbf{Components:} V, S
\\ \textbf{Casting Time:} 1 standard action
\\ \textbf{Range:} Personal and touch
\\ \textbf{Target:} You and up to 5 touched willing creatures
\\ \textbf{Duration:} 1 hour/level
\\ \textbf{Saving Throw:} None
\\ \textbf{Spell Resistance:} No
\\ \textbf{Spell Points:} Ranger 11, Travel 13

\emph{''A distant drumbeat sounds in your ears, and you feel an irresistible urge to run.``}

For the duration of this spell, the overland distance traveled by the subjects is multiplied by 10 for any given time period.
This multiplication applies only if traveling by foot, not when mounted or using another form of assisted transportation (such as a ship).

Further, the subjects' overland movement is not hindered by negative terrain conditions (as long as the terrain is actually passable), traveling as easily over difficult terrain as a well-made highway.
For example, the subjects could travel across a trackless jungle at full (tenfold) speed, rather than $1/4$ speed.

This does not affect the subjects' tactical movement speed.

\paragraph{Augment:} For every 2 additional spell points you spend, this spell can affect an additional target.

\subsection{'G' Spells}
\subsubsection{Goldskin}
\label{Spell:Goldskin}
Conjuration (Teleportation)
\\ \textbf{Level:} Commerce 4
\\ \textbf{Components:} V, S
\\ \textbf{Casting Time:} 1 standard action
\\ \textbf{Range:} Personal
\\ \textbf{Target:} You
\\ \textbf{Duration:} 10 min./level
\\ \textbf{Spell Points:} 7

\emph{The shimmering golden glow protects you from being molested by proletarians.}

Each time you are struck with a manufactured weapon, the attacker loses 7 gold pieces from those stored on his person.
The gold pieces are transferred on to your person. Extradimensional storage spaces accessed via magic items (such as Bags of Holding) count as being on a character's person for purposes of this spell.

Should the attacker not have 7 gold pieces to spare, the attack deals no damage.

\paragraph{Augment:} You can augment this spell in one or both of the following ways:
\begin{enumerate}
 \item For every additional spell point spent, the attacker loses an additional gold piece per hit.
\end{enumerate}

\subsection{'H' Spells}
\subsubsection{Hail of Arrows}
\label{Spell:HailOfArrows}
Transmutation
\\ \textbf{Level:} Ranger 3
\\ \textbf{Components:} V
\\ \textbf{Casting Time:} 1 swift action
\\ \textbf{Range:} Personal
\\ \textbf{Target:} You
\\ \textbf{Duration:} 1 round
\\ \textbf{Spell Points:} 5

\emph{You let magic take control of your drawing arm, and watch the arrows fly.}

While under the influence of this spell, you can take a full-round action to fire a ranged weapon at up to five targets within range.
Each attack uses your primary attack bonus, and each enemy may only be targeted by a single attack.
With those exceptions, this full-round action is similar to making a full attack with a ranged weapon.

\subsubsection{Hand of Midas}
\label{Spell:HandOfMidas}
Transmutation
\\ \textbf{Level:} Commerce 7
\\ \textbf{Components:} V, S
\\ \textbf{Casting Time:} 1 standard action
\\ \textbf{Range:} Touch
\\ \textbf{Target:} One creature OR one object; see text
\\ \textbf{Duration:} Instantaneous
\\ \textbf{Saving Throw:} Fortitude negates OR fortitude negates (object); see text
\\ \textbf{Spell Resistance:} Yes (object)
\\ \textbf{Spell Points:} 13, XP; see text

\emph{``I did not say you would be rich. I said you would be worth a lot!''}

A targeted creature is turned into a mindless, inert statue of gold.
It is unaware of its surroundings, as well as the passage of time. It does not age. Its gear is not affected by the transformation.

The creature and each of its items retains its own hit point total throughout the transformation, and its hardness changes to 5.
Destroying the statue kills the creature.

The transmuted creature is considered a Construct for the purposes of targeting it with spells and other abilities.

An augmented \nameref{Spell:RemoveCurse} spell can restore the transmuted creature.

Alternatively, you can affect a nonmagical object, or part of a larger nonmagical object distinctive enough to be considered an object in its own right (such as a doorknob or a wagon wheel).

\emph{Experience Cost:} 10 XP per pound of the creature or object to be transmuted.

Unlike most spells with an experience cost, you must pay this cost when the spell takes effect, not when the spell is cast.
If expending this amount of experience would cause you to go down a level, you spend no experience and the spell fails (but the action and spell points are spent).

When placing this spell within \nameref{Item:Matrices} or \nameref{Item:Wands}, you must pay the spell's experience cost (per charge, in the case of wands) at the time of crafting. In such a case, the spell fails if not imbued with sufficient experience to transmute a creature or object of the target's weight.
\subsubsection[Swift Haste]{Haste, Swift}
\label{Spell:SwiftHaste}
Transmutation
\\ \textbf{Level:} Spellbreaker 2
\\ \textbf{Components:} V
\\ \textbf{Casting Time:} 1 swift action
\\ \textbf{Range:} Personal
\\ \textbf{Targets:} You
\\ \textbf{Duration:} 2 rounds
\\ \textbf{Spell Points:} 3

This spell functions as the \nameref{Spell:Haste} spell (excluding augmentation options), except as noted here.

\paragraph{Augment:} For every 2 additional spell points you spend, this spell's duration is increased by one round.

\subsubsection{Hawk's Flight}
\label{Spell:HawksFlight}
Transmutation
\\ \textbf{Level:} Ranger 3

\emph{The subject grows enormous, gray wings, resembling those of a bird of prey.}

This spell functions as the \nameref{Spell:Fly} spell (including its augmentation option), except as noted here, and that the flight is due to physical wings the subject grows (which matters for effects like those caused by tanglefoot bags).

\subsubsection{Hexenstrike}
\label{Spell:Hexenstrike}
Divination
\\ \textbf{Level:} Dread Knight 1
\\ \textbf{Components:} V
\\ \textbf{Casting Time:} 1 swift action
\\ \textbf{Range:} Personal
\\ \textbf{Target:} You
\\ \textbf{Duration:} 1 round
\\ \textbf{Spell Points:} 1

\emph{Your enemies' cursed bones turn brittle beneath your blows.}

For the duration of the spell, whenever you make a weapon attack (including natural attacks and ranged attacks, but not spells, spell-like abilities, or supernatural abilities) against a Cursed creature, you gain a bonus on your damage roll equal to your Charisma modifier.

\paragraph{Augment:} For every 2 additional spell points you spend, this spell's duration is increased by one round.

\subsection{'I' Spells}

\subsubsection{Inexorable Onset of Death}
\label{Spell:InexorableOnsetOfDeath}
Necromancy
\\ \textbf{Level:} Dread Knight 6
\\ \textbf{Components:} V, S
\\ \textbf{Casting Time:} 1 standard action
\\ \textbf{Range:} Close (25 ft. + 5 ft./2 levels)
\\ \textbf{Target:} One creature
\\ \textbf{Duration:} Permanent (D)
\\ \textbf{Saving Throw:} None
\\ \textbf{Spell Resistance:} Yes
\\ \textbf{Spell Points:} 11

\emph{``You may be able to kill me, but you need me if you want to see the next minute!''}

The creature gains one negative level. One round later, and each round thereafter, it gains an additional negative level.

The curse bestowed by this spell cannot be dispelled, but it can be removed with a \nameref{Spell:LimitedWish}, \nameref{Spell:Miracle}, \nameref{Spell:RemoveCurse}, or \nameref{Spell:Wish} spell.

\subsubsection{Insightful Strikes}
\label{Spell:InsightfulStrikes}
Divination
\\ \textbf{Level:} Arcane Archer 1\footnote{Add the spell to the list of spells available to the class through its Advanced Learning class feature.}, Spellbreaker 1
\\ \textbf{Components:} V
\\ \textbf{Casting Time:} 1 swift action
\\ \textbf{Range:} Personal
\\ \textbf{Target:} You
\\ \textbf{Duration:} 1 round
\\ \textbf{Spell Points:} 1

\emph{Your magical foresight allows you to strike far more effectively than mundane warriors can.}

For the duration of this spell, you gain an insight bonus on weapon damage rolls (including your attacks with natural weapons and ranged weapons, but not your spells, spell-like abilities, or supernatural abilities) equal to your Intelligence modifier.

\paragraph{Augment:} You can augment this spell in one of the following ways:
\begin{enumerate}
 \item If you spend four additional spell points, the insight bonus becomes equal to twice your Intelligence modifier.
 \item If you spend eight additional spell points, the insight bonus becomes equal to three times your Intelligence modifier.
 \item If you spend twelve additional spell points, the insight bonus becomes equal to four times your Intelligence modifier.
 \item If you spend sixteen additional spell points, the insight bonus becomes equal to five times your Intelligence modifier.
\end{enumerate}

\emph{Special:} Casting this spell does not provoke Attacks of Opportunity.

\subsection{'L' Spells}
\subsubsection{Last Laugh}
\label{Spell:LastLaugh}
Necromancy
\\ \textbf{Level:} Death 1
\\ \textbf{Components:} V, S
\\ \textbf{Casting Time:} 1 immediate action
\\ \textbf{Range:} 30 ft.
\\ \textbf{Area:} 30-ft.-radius burst, centered on you
\\ \textbf{Duration:} Instantaneous
\\ \textbf{Saving Throw:} None
\\ \textbf{Spell Resistance:} Yes
\\ \textbf{Spell Points:} 1

\emph{With your deity's name on your lips, you expend the last of your energy in an attempt to take your enemies with you.}

This spell deals 1d8 points of negative energy damage to all creatures within its area, but you die as part of the casting process (you dying is not considered a negative energy effect, or a death effect). If you are a nonliving creature, you are instead destroyed. If you can neither die nor be destroyed by the spell (such as due to you being immune to necromancy spells), the spell fails.

The \nameref{Spell:SpeakWithDead} spell cannot be used to interrogate your corpse. You may be raised from the dead normally.

\paragraph{Augment:} For every additional spell point you spend, this spell's damage increases by one die (d8).

\subsubsection{Life Tap}
\label{Spell:LifeTap}
Necromancy
\\ \textbf{Level:} Dread Knight 3
\\ \textbf{Components:} V, S
\\ \textbf{Casting Time:} 1 standard action
\\ \textbf{Range:} Close (25 ft. + 5 ft./2 levels)
\\ \textbf{Targets:} Three living creatures, no two of which can be more than 30 ft. apart
\\ \textbf{Duration:} 1 round/level
\\ \textbf{Saving Throw:} Fortitude negates
\\ \textbf{Spell Resistance:} Yes
\\ \textbf{Spell Points:} 5

\emph{As life flows out of your enemies, you redirect it to flow back into you.}

Whenever a creature makes a weapon attack (including natural attacks and ranged attacks, but not spells, spell-like abilities, or supernatural abilities) that successfully deals damage to a creature under the effect of this Curse, the attacker is cured of a number of points of damage equal to one-half the amount of damage dealt.

Hit points can only be gained while the Cursed creature remains at 1 hit point or higher. Any damage that would reduce the creature to 0 or fewer hit points confers no benefit.

No damage is healed by attacks that deal nonlethal damage, such as when attacking a Cursed creature with regeneration.

\paragraph{Augment:} For every additional spell point you spend, this spell can affect an additional creature.

\subsection{'M' Spells}
\subsubsection{Mace of the Astral Deva}
\label{Spell:MaceOfTheAstralDeva}
Transmutation [Good]
\\ \textbf{Level:} Paladin 5
\\ \textbf{Components:} V, S
\\ \textbf{Casting Time:} 1 standard action
\\ \textbf{Range:} Touch
\\ \textbf{Target:} Melee weapon touched
\\ \textbf{Duration:} 1 round/level
\\ \textbf{Saving Throw:} None; see text
\\ \textbf{Spell Resistance:} No; see text
\\ \textbf{Spell Points:} 9

\emph{You invoke the powers of the mighty Astral Devas to bring down your enemies.}

A creature struck twice in the same round with a weapon imbued with this spell must make a Fortitude save or be stunned for 1d3 rounds.
This effect is subject to spell resistance.

\subsubsection{Mental Link}
\label{Spell:MentalLink}
Enchantment
\\ \textbf{Level:} Bard 1, Sor/Wiz 1
\\ \textbf{Components:} V, S
\\ \textbf{Casting Time:} 1 standard action
\\ \textbf{Range:} Close (25 ft. + 5 ft./2 levels); see text
\\ \textbf{Targets:} You and one other willing creature within range that has an Intelligence score of 3 or higher
\\ \textbf{Duration:} 10 min./level
\\ \textbf{Saving Throw:} None; see text
\\ \textbf{Spell Resistance:} Yes (harmless)
\\ \textbf{Spell Points:} 1

\emph{You feel your allies like a nugget embedded in your skull.}

The spell creates a limited mental link between you and the target creature, joining your senses.
The effect is that anything heard by one of you is heard by the other.

Once the bond is formed, it works over any distance (although not from one plane to another).

Assuming you can hear the words you say yourself, this spell allows conversation at a distance.

\paragraph{Augment:} You can augment this spell in one or both of the following ways: 
\begin{enumerate}
 \item For every additional spell point you spend, this spell can affect an additional target. 
 Any additional target cannot be more than 15 feet from another target of the spell at the time of casting.
 \item If you spend 4 additional spell points, you can attempt to create a bond with a creature that is not willing (Will save negates).
 \item If you spend 4 additional spell points, the link does work from one plane to another.
\end{enumerate}

\subsubsection{Merciful Weapon}
\label{Spell:MercifulWeapon}
Transmutation
\\ \textbf{Level:} Paladin 2
\\ \textbf{Components:} V, S
\\ \textbf{Casting Time:} 1 standard action or 1 swift action; see text
\\ \textbf{Range:} Close (25 ft. + 5 ft./2 levels)
\\ \textbf{Target:} One weapon
\\ \textbf{Duration:} 1 min./level (D) or 1 round; see text
\\ \textbf{Saving Throw:} Will negates (object)
\\ \textbf{Spell Resistance:} Yes (object)
\\ \textbf{Spell Points:} 3

\emph{``No blood shall be shed this day.''}

All damage dealt by the targeted weapon for the duration of the spell is considered nonlethal damage.

At the time of casting, you make a choice. If you cast the spell as a standard action, the duration is 1 minute per level. 
If you cast it as a swift action, the duration is one round.

\paragraph{Augment:} For every 3 additional spell points you spend, this spell can affect an additional weapon within range.

\subsubsection{Moon Bolt}
\label{Spell:MoonBolt}
Evocation [Light]
\\ \textbf{Level:} Moon 2
\\ \textbf{Components:} V, S
\\ \textbf{Casting Time:} 1 standard action
\\ \textbf{Range:} Medium (100 ft. + 10 ft./level)
\\ \textbf{Target:} One creature
\\ \textbf{Duration:} Instantaneous
\\ \textbf{Saving Throw:} Fortitude partial
\\ \textbf{Spell Resistance:} Yes
\\ \textbf{Spell Points:} 3

\emph{You blast your enemy with a ball of pure, silvery light.}

The creature takes takes 1d8 points of damage and must succeed on a Fortitude save to avoid being stunned for 1 round. Lycanthropes take a -4 penalty on this saving throw.

\paragraph{Augment:} For every 2 additional spell points you spend, this spell's damage increases by one die (d8).
\subsubsection{Moonlust}
\label{Spell:Moonlust}
Enchantment (Compulsion) [Mind-Affecting]
\\ \textbf{Level:} Moon 2
\\ \textbf{Components:} V, S
\\ \textbf{Casting Time:} 1 standard action
\\ \textbf{Range:} Touch
\\ \textbf{Target:} Creature touched
\\ \textbf{Duration:} 3 rounds
\\ \textbf{Saving Throw:} Will negates
\\ \textbf{Spell Resistance:} Yes
\\ \textbf{Spell Points:} 3

\emph{The creature sits enthralled, staring into the air, as if looking at something no one else can see.}

The caster may \emph{daze} one living creature by making a successful touch attack. 
If the target creature does not make a successful Will save, its mind is clouded and it takes no action for 3 rounds. 
The dazed subject is not stunned (so attackers get no special advantage against it), but it can't move, cast spells, use mental abilities, and so on.

\paragraph{Augment:} For every additional spell point you spend, the subject is dazed for an additional round.

\subsection{'O' Spells}
\subsubsection{Orb of Light}
\label{Spell:OrbOfLight}
Evocation [Light]
\\ \textbf{Level:} Sun 4
\\ \textbf{Components:} V, S
\\ \textbf{Casting Time:} 1 standard action
\\ \textbf{Range:} Close (25 ft. + 5 ft./2 levels)
\\ \textbf{Effect:} One orb of light
\\ \textbf{Duration:} Instantaneous
\\ \textbf{Saving Throw:} Fortitude partial; see text
\\ \textbf{Spell Resistance:} Yes
\\ \textbf{Spell Points:} 7

\emph{Your hand glows with every color of the rainbow before the light coalesces into a ball of the purest light}

An orb of energy about 3 inches across shoots from your palm at its target, dealing 7d6 points of damage. You must succeed on a ranged touch attack to hit your target.
Any creature struck by the orb must succeed on a Fortitude save or be blinded for one round.

The orb radiates bright light out to 30', and provides shadowy illumination for another 30'. While the orb is too short-lived to serve as an effective light source, it is sufficient to illuminate your surroundings while you select your target.

\paragraph{Augment:} For every additional spell point you spend, this spell's damage increases by one die (d6).

\subsection{'P' Spells}
\subsubsection{Paralyzing Arrow}
\label{Spell:ParalyzingArrow}
Transmutation
\\ \textbf{Level:} Ranger 4
\\ \textbf{Components:} V
\\ \textbf{Casting Time:} 1 swift action
\\ \textbf{Range:} Touch
\\ \textbf{Target:} One arrow or bolt touched
\\ \textbf{Duration:} Until end of round; see text
\\ \textbf{Saving Throw:} None, Fort negates; see text
\\ \textbf{Spell Resistance:} No, Yes; see text
\\ \textbf{Spell Points:} 7

\emph{You infuse the arrow with the power to turn your opponents' muscles to water.}

A creature struck by the projectile must succeed on a fortitude save or be \emph{paralyzed} for one round.
Spell resistance applies against the paralysis effect.
The projectile remains so enhanced only until the end of the current turn. 
If the projectile does not hit a creature before the end of the turn, the spell drains away harmlessly.

\paragraph{Augment:} For every 2 additional spell points you spend, the paralysis lasts for an additional round.

\subsubsection{Phase Arrow}
\label{Spell:PhaseArrow}
Divination
\\ \textbf{Level:} Ranger 5
\\ \textbf{Components:} V
\\ \textbf{Casting Time:} 1 standard action
\\ \textbf{Range:} Personal
\\ \textbf{Target:} You
\\ \textbf{Duration:} 1 round; See text
\\ \textbf{Spell Points:} 9

\emph{The arrow turns invisible when you release the string, and only reappears when it sprouts from your opponent's chest.}

Your next ranged weapon attack (if it is made before the end of the spell's duration) passes through the ethereal plane on its way to its target.
This causes it to ignore all forms of obstacles in its path, except for those that extend into the ethereal plane (such as force effects and abjurations).
It therefore ignores all kinds of cover (including total cover, although you must still be somehow aware of the target's location in such a case), conditions of severe wind, and even shield or armor bonuses to the target's AC (unless those bonuses result from effects that block ethereal assaults).
Concealment and miss chances and non-physical sources of AC still apply.
You can hit an ethereal or incorporeal creature with a Phase Arrow without incurring any kind of miss chance.

\subsubsection{Piggyback}
\label{Spell:Piggyback}
Conjuration (Teleportation)
\\ \textbf{Level:} Bard 3, Paladin 3, Planes 3, Sor/Wiz 3
\\ \textbf{Components:} V
\\ \textbf{Casting Time:} 1 immediate action
\\ \textbf{Range:} Close (25 ft. + 5 ft./2 levels)
\\ \textbf{Target:} You and one creature; see text
\\ \textbf{Duration:} Instantaneous
\\ \textbf{Saving Throw:} Will negates; see text
\\ \textbf{Spell Resistance:} None
\\ \textbf{Spell Points:} 5

\emph{``You use the residual energies of the teleportation spell to fuel your own.''}

This spell must be cast in response to another creature finishing casting a Conjuration (Teleportation) spell that transports that creature to another location. You are then instantaneously teleported along with the creature, so that your relative positioning is maintained. If the creature does not wish for you to travel with it, it may attempt a Will save to negate your spell.

You can bring along objects as long as their weight doesn't exceed your maximum load. You cannot bring along creatures other than yourself\footnote{Note that the Piggyback spell is a valid target for another instance of the spell. Rumor has it that Teleportation Konga is a sport commonly practiced in Wizards' celebrations.}.

If the travel would cause you to arrive in a place that is already occupied by a solid body, the Piggyback spell fails.

\paragraph{Augment:} If you spend 6 additional spell points, this spell does not allow a Will save.
\subsubsection{Pox}
\label{Spell:Pox}
Necromancy
\\ \textbf{Level:} Dread Knight 1
\\ \textbf{Components:} V, S
\\ \textbf{Casting Time:} 1 standard action
\\ \textbf{Range:} Close (25 ft. + 5 ft./2 levels)
\\ \textbf{Target:} One humanoid
\\ \textbf{Duration:} Instantaneous and Permanent (D); see text
\\ \textbf{Saving Throw:} Fortitude negates
\\ \textbf{Spell Resistance:} Yes
\\ \textbf{Spell Points:} 1

\emph{``Not so pretty now...''}

The subject instantaneously takes 2 points of Constitution damage.

This damage is caused be huge, painful and hideous pustules that erupt all over the subject's skin. The pox is so disfiguring that creatures need to succeed on a DC 15 Spot check to even recognize the subject as itself.
This part of the spell is permanent unless dismissed by the caster or removed with \nameref{Spell:RemoveCurse}, persisting even if the Constitution damage is healed.

Protections against magical diseases are effective against this spell.

\paragraph{Augment:} For every additional spell point you spend, you can affect an additional subject within range.

\emph{Special:} This spell immediately ends if the caster dies. 

\subsection{'R' Spells}
\subsubsection{Repelling Light}
\label{Spell:RepellingLight}
Abjuration [Light]
\\ \textbf{Level:} Sun 4
\\ \textbf{Components:} V, S
\\ \textbf{Casting Time:} 1 standard action
\\ \textbf{Range:} 10 ft.
\\ \textbf{Area:} 10-ft.-radius emanation centered on you
\\ \textbf{Duration:} 1 round/level (D)
\\ \textbf{Saving Throw:} Will negates; see text
\\ \textbf{Spell Resistance:} Yes
\\ \textbf{Spell Points:} 7

\emph{I am a servant of the Secret Fire, wielder of the flame of Anor. The dark fire will not avail you, flame of Udûn. Go back to the Shadow! You cannot pass.} 

You bring into being a mobile, globe of light that prevents the entrance of many types of foul creatures.
The globe moves around with you.

The effect hedges out evil outsiders, undead creatures, natives of the plane of shadow, and creatures who take damage or penalties from bright light or sunlight. A creature of the listed types can penetrate the barrier if it succeeds on a Will save. 
Even so, crossing the barrier deals the creature 2d6 points of damage.

This spell may be used only defensively, not aggressively. Forcing the globe against a creature that the spell keeps at bay lets the creature in (without dealing damage), and prevents the spell from affecting it again until it exits the emanation and attempts to re-enter.

Creatures with reach sufficient to stand outside the globe and attack the caster can do so, even with natural weapons (the globe does not do its work instantaneously).

The spell's area is illuminated to the point of bright illumination. In addition, shadowy illumination extends out to twice the spell's listed radius.
If an area of magical light and an area of magical darkness overlap, the spell on which more spell points were spent prevails.
If an equal number of spell points were spent on both spells, ambient light conditions remain.

\paragraph{Augment:} You can augment this spell in one or both of the following ways.
\begin{enumerate}
 \item For every 4 additional spell points you spend, the globe's range and radius increases by 5'.
 \item If you spend 4 additional spell points, creatures that attempt to move into the spell's area are \emph{blown away} from the caster if they fail the will save.
\end{enumerate}

\subsubsection{Roguespace}
\label{Spell:Roguespace}
Conjuration (Teleportation)
\\ \textbf{Level:} Spellbreaker 3
\\ \textbf{Components:} V
\\ \textbf{Casting Time:} 1 immediate action
\\ \textbf{Range:} Personal
\\ \textbf{Target:} You
\\ \textbf{Duration:} Instantaneous
\\ \textbf{Spell Points:} 5

\emph{You instantaneously slide into a pocket dimension, allowing you to avoid harmful effects.}

When you cast this spell in conjunction with making a successful Reflex save against an attack that normally deals half damage on a successful save, you instead take no damage.

You can cast this spell quickly enough to save yourself if you unexpectedly come within range of a dangerous effect.

\paragraph{Augment:} If you spend 4 additional spell points, you take only half damage even on a failed Reflex save.

\subsection{'S' Spells}
\subsubsection{Searing Blade}
\label{Spell:SearingBlade}
Evocation [Light]
\\ \textbf{Level:} Paladin 2
\\ \textbf{Components:} V, S
\\ \textbf{Casting Time:} 1 swift action
\\ \textbf{Range:} Touch
\\ \textbf{Target:} Weapon touched
\\ \textbf{Duration:} 1 round; See text
\\ \textbf{Saving Throw:} None
\\ \textbf{Spell Resistance:} Yes
\\ \textbf{Spell Points:} 3

\emph{You imbue a weapon with the power to blast the dark and unclean.}

Your next successful attack with the weapon (if it is made before the end of the spell's duration) deals an additional 3d6 points of damage if the target of the attack is an undead creature or an ooze, half that amount otherwise.
A creature that is harmed by or takes penalties from sunlight or bright light takes double damage
(meaning that a non-undead, non-ooze creature that takes penalties from light takes normal damage).

\paragraph{Augment:} For every additional spell point you spend, the damage against undead creatures and oozes increases by 1d6 (with a corresponding increase in damage against other creatures).

\subsubsection{Seeker Arrow}
\label{Spell:SeekerArrow}
Divination
\\ \textbf{Level:} Ranger 4
\\ \textbf{Components:} V
\\ \textbf{Casting Time:} 1 standard action
\\ \textbf{Range:} Personal
\\ \textbf{Target:} You
\\ \textbf{Duration:} 1 round; See text
\\ \textbf{Spell Points:} 7

\emph{``There is no such thing as an impossible shot. Only somewhat trickier ones.''}

Your next ranged weapon attack (if it is made before the end of the spell's duration) can work even if you do not have line of effect or line of sight to your intended target.
As long as you visualize the correct square when firing and the projectile has some possible path along which it could travel to the target, you may make the attack normally.
Additionally, this attack is not affected by the miss chance that applies to attackers trying to strike a concealed target.
Other detrimental factors, such as range, conditions of severe wind, and the target's AC apply normally.

For example, you could try to fire an arrow around a boulder and at a target you know to be hiding behind it, if you know (or can guess) the square in which the target is.
You could, however, not try to fire an arrow at a creature trapped in a \nameref{Spell:ResilientSphere}, even if you can see it, since no path exists through a Resilient Sphere.

\subsubsection{Sepia Snake Ward}
\label{Spell:SepiaSnakeWard}
Conjuration (Creation) [Force]
\\ \textbf{Level:} Commerce 3, Sor/Wiz 3
\\ \textbf{Target:} One touched pouch or lock

\emph{''I almost feel sorry for the thieves.``}

This spell functions as \nameref{Spell:SepiaSnakeSigil} (including the augmentation option), except as noted here.

When cast on a pouch, the snake strikes those who would attempt to use Sleight of Hand to sneak into the pouch or steal the pouch as a whole using the Sleight of Hand skill.

When cast on a lock, the snake strikes those who would attempt to force it open with the Open Lock skill.

Simply attempting to access the container normally does not trigger the spell.

The hidden ward cannot be detected by normal observation, and \nameref{Spell:DetectMagic} reveals only that the pouch or lock is magical.

\emph{Note:} Magic traps such as Sepia Snake Wards are hard to detect and disable. 
A rogue (only) can use the Search skill to find the runes and Disable Device to thwart them. 
The DC in each case is 25 + spell level, or 28 for a Sepia Snake Ward.

\subsubsection[Swift See Invisibility]{See Invisibility, Swift}
\label{Spell:SwiftSeeInvisibility}
Divination
\\ \textbf{Level:} Spellbreaker 2
\\ \textbf{Components:} V
\\ \textbf{Casting Time:} 1 swift action
\\ \textbf{Range:} Personal
\\ \textbf{Target:} You
\\ \textbf{Duration:} 2 rounds
\\ \textbf{Spell Points:} 3

\emph{Hidden things reveal themselves to you in a flash, but the vision fades rapidly.}

This spell functions as the \nameref{Spell:SeeInvisibility} spell (excluding augmentation options), except as noted here.

\paragraph{Augment:} For every 2 additional spell points you spend, this spell's duration is increased by one round.

\subsubsection{Shadow Warriors}
\label{Spell:ShadowWarriors}
Illusion (Shadow)
\\ \textbf{Level:} Sor/Wiz 3
\\ \textbf{Components:} V, S
\\ \textbf{Casting Time:} 1 round
\\ \textbf{Range:} Close (25 ft. + 5 ft./2 levels)
\\ \textbf{Effect:} 5 phantom warriors
\\ \textbf{Duration:} 1 round/level (D)
\\ \textbf{Saving Throw:} None
\\ \textbf{Spell Resistance:} No
\\ \textbf{Spell Points:} 5

\emph{Several shadows appear on the ground, which then immediately stand up, defying all logic.}

You draw forth matter from the plane of shadows to form several phantom warriors. 
You place the warriors independently within the spell's range.
They can share another's creature's space.
These soldiers appear fully armed, and are clad in glistening black full plate armors.

Once created, the warriors stay in their square, standing in an imposing manner, weapons drawn.
Each warrior threatens the spaces around it, and may take one attack of opportunity per round
(but no other attacks). They can flank with each other, as well as with other allied creatures.

The warriors do not block line of sight or line of effect, nor do they hinder the movement of
any creature. They are not creatures, and are not subject to targeted spells.

While the spell is active, you gain a circumstance bonus on intimidate checks equal to the number
of active warriors.

The relevant statistics of a shadow warrior are given on the \nameref{tab:ShadowWarriors} table.
\begin{table*}
\label{tab:ShadowWarriors}
\caption{Shadow Warrior}
\makebox[\textwidth]{
\begin{tabular}{ll}
\toprule
\textbf{Size:}&Medium\\
\textbf{Hit Points}&CL$^1$ $\times$ 2\\
\textbf{Armor Class:}& 18 + KAM$^1$, touch 10+ KAM, flatfooted 18\\
\textbf{Attack:}& -\\
\textbf{Space/Reach:}& 5 ft./5 ft.\\
\textbf{Special Attacks:}&Attack of opportunity: Longsword +CL melee (1d8+KAM) 19-20/x2\\
\textbf{Saves:}&Fort +CL, Ref +CL, Will +CL\\
\bottomrule
\end{tabular}}
\begin{enumerate}
 \item KAM is the caster's key ability modifier, and CL is his caster level.
\end{enumerate}
\end{table*}

\paragraph{Augment:} For every additional spell point you spend, you gain an additional warrior when casting this spell.

\subsubsection{Shun}
\label{Spell:Shun}
Enchantment (Compulsion) [Mind-Affecting]
\\ \textbf{Level:} Sor/Wiz 6
\\ \textbf{Components:} V, S
\\ \textbf{Casting Time:} 1 standard action
\\ \textbf{Range:} Close (25 ft. + 5 ft./2 levels)
\\ \textbf{Target:} One creature
\\ \textbf{Duration:} 1 hour/level
\\ \textbf{Saving Throw:} Will partial; see text
\\ \textbf{Spell Resistance:} Yes
\\ \textbf{Spell Points:} 11

\emph{''SHUN! Shun the nonbeliever!``}

The targeted creature is compelled to avoid the presence of others.
It must spend its actions physically distancing itself from every creature within line of sight.
It need not expend any resources to do so (including spell points or magic item charges) unless doing so is the only way to retreat.
A creature engaged in melee combat typically takes the withdraw action. 
A creature not aware of any other creatures within line of sight typically does its best to hide.

One round after the spell is cast, the subject receives a Will saving throw. 
If the saving throw is successful, the subject may act normally from there on,
otherwise it continues to eschew contact with others while the spell lasts.

\paragraph{Augment:} For every 3 additional spell points you spend, this spell can affect an additional target.
No target of the spell can be more than 15 feet from another target of the spell.

\subsubsection{Spell Karma}
\label{Spell:SpellKarma}
Abjuration
\\ \textbf{Level:} Dread Knight 6
\\ \textbf{Components:} V, S
\\ \textbf{Casting Time:} 1 standard action
\\ \textbf{Range:} Close (25 ft. + 5 ft./2 levels)
\\ \textbf{Target:} One creature
\\ \textbf{Duration:} 1 round/level (D)
\\ \textbf{Saving Throw:} None
\\ \textbf{Spell Resistance:} Yes
\\ \textbf{Spell Points:} 11

\emph{``You think you scare me? Give me your best shot!''}

The spells of a spellcaster under the influence of this spell rebound back upon him, as if all creatures he targets were under a \nameref{Spell:SpellTurning} spell.

\subsubsection{Squirt}
\label{Spell:Squirt}
Conjuration (Creation) [Water]
\\ \textbf{Level:} Water 2
\\ \textbf{Components:} V, S
\\ \textbf{Casting Time:} 1 standard action
\\ \textbf{Range:} 60 ft.
\\ \textbf{Area:} 60-ft. line
\\ \textbf{Duration:} Instantaneous
\\ \textbf{Saving Throw:} Fortitude half
\\ \textbf{Spell Resistance:} No
\\ \textbf{Spell Points:} 3

\emph{A geyser of cold water springs out from your outstretched hand.}

All creatures caught in the area of the spell take 3d6 points of nonlethal damage.
A successful fortitude save halves the damage.
Objects (and some creatures) are immune to nonlethal damage.

Most of the water you create dissipates as soon as the line reaches its end point, leaving only a thin layer of wetness on the surface beneath.

\paragraph{Augment:} For every additional spell point you spend, this spell's damage increases by one die (d6).

\subsubsection{Starfall}
\label{Spell:Starfall}
Evocation [Light]
\\ \textbf{Level:} Moon 9
\\ \textbf{Components:} V, S
\\ \textbf{Casting Time:} 1 standard action
\\ \textbf{Range:} Long (400 ft. + 40 ft./level)
\\ \textbf{Effect:} A shower of Moon Bolts
\\ \textbf{Duration:} 6 rounds
\\ \textbf{Saving Throw:} See text
\\ \textbf{Spell Resistance:} Yes
\\ \textbf{Spell Points:} 17

\emph{You bring down half the sky on top of your enemies, with devastating results.}

Immediately upon the completion of the spell, and at the start of your turn each round of the spell's duration thereafter, all creatures within the spell's range except for those you specifically exclude are subjected to a \nameref{Spell:MoonBolt}, cast as if augmented to cast a number of spell points equal to the number of spell points you spent on the Starfall spell. You do not need line of sight or line of effect to the creatures struck by the bolts, the Starfall seeks them out on its own.

\subsubsection{Step through Earth}
\label{Spell:StepThroughEarth}
Transmutation
\\ \textbf{Level:} Earth 4, Ranger 4
\\ \textbf{Components:} V
\\ \textbf{Casting Time:} 1 standard action
\\ \textbf{Range:} Long (400 ft. + 40 ft./level)
\\ \textbf{Target:} You
\\ \textbf{Duration:} Instantaneous
\\ \textbf{Saving Throw:} None
\\ \textbf{Spell Resistance:} No
\\ \textbf{Spell Points:} 7

\emph{You fall forwards on to the ground, but instead of hitting it, you fall through it and appear somewhere else.}

You instantly transfer yourself from your current location to another spot within range.
You always arrive at exactly the spot desired-whether by simply visualizing the area or by stating direction.
However, both the destination square and the square from which you transfer yourself must contain earth or unworked stone, and a straight ''passage`` of such terrain must exist between the two squares. You could not, for example, transport yourself into or out of a box containing dirt, or a castle courtyard whose foundations and underlying dungeons block have replaced the natural soil.

After using this spell, you can't take any other actions until your next turn. 
You can bring along objects as long as their weight doesn't exceed your maximum load. 

If you arrive in a place that is already occupied by a solid body, the spell fails.

\paragraph{Augment:} You can augment this spell in one or both of the following ways:
\begin{enumerate}
 \item If you spend an additional 6 spell points, you can cast this spell as a move action.
 \item If you spend an additional 6 spell points, you can act after casting this spell (provided you have actions to do so).
\end{enumerate}

\subsubsection{Stunning Arrow}
\label{Spell:StunningArrow}
Evocation
\\ \textbf{Level:} Ranger 2
\\ \textbf{Components:} V
\\ \textbf{Casting Time:} 1 swift action
\\ \textbf{Range:} Touch
\\ \textbf{Target:} One arrow or bolt touched
\\ \textbf{Duration:} Until end of round; see text
\\ \textbf{Saving Throw:} None, Fort negates; see text
\\ \textbf{Spell Resistance:} No, Yes; see text
\\ \textbf{Spell Points:} 3

\emph{You imbue a single arrow with a magical charge powerful enough to overwhelm the senses of its victims.}

A creature struck by the projectile must succeed on a fortitude save or be \emph{stunned} for one round.
Spell resistance applies against the stunning effect.
The projectile remains so enhanced only until the end of the current round. 
If the projectile does not hit a creature before the end of the round, the spell drains away harmlessly.

\subsubsection{Summon Weapon}
\label{Spell:SummonWeapon}
Conjuration (Teleportation)
\\ \textbf{Level:} Blackguard 1, Paladin 1
\\ \textbf{Components:} V, S
\\ \textbf{Casting Time:} 1 standard action
\\ \textbf{Range:} Touch
\\ \textbf{Target:} One masterwork weapon with which you are proficient
\\ \textbf{Duration:} Permanent until discharged
\\ \textbf{Saving Throw:} None
\\ \textbf{Spell Resistance:} No
\\ \textbf{Spell Points:} 1

\emph{You call your weapon directly to your hand.}

First, you must cast this spell on a weapon. This weapon can be a magic weapon.
The weapon cannot be attended by another creature at the time the spell is cast.

Thereafter, you can summon the weapon by speaking a special word (set by you when the spell is cast) as an immediate action. 
The weapon appears instantly in your hand, and the spell ends.

If the weapon is being attended by another creature at the time you speak the special word, the spell does not work, 
but you know who the possessor is and where that creature is located when the summoning is attempted.
This does not discharge the spell.

The weapon cannot be summoned across planar boundaries.
Attempting to do so does not discharge the spell.

\subsubsection{Suppress Magic}
\label{Spell:SuppressMagic}
Abjuration
\\ \textbf{Level:} Magic 3, Sor/Wiz 3
\\ \textbf{Components:} V, S
\\ \textbf{Casting Time:} 1 round
\\ \textbf{Range:} Touch
\\ \textbf{Target:} Willing spellcaster touched
\\ \textbf{Duration:} 10 min./level
\\ \textbf{Saving Throw:} None, Will negates; see text
\\ \textbf{Spell Resistance:} Yes
\\ \textbf{Spell Points:} 5

\emph{''Magic is strictly forbidden in the presence of the king.``}

The targeted spellcaster becomes incapable of casting spells. Spell-like and supernatural abilities are not affected.

Should the spellcaster later become unwilling, he may attempt a Will saving throw with a -4 penalty to shake off the effects of the spell, once.

The spellcaster is surrounded by a visible aura while the spell is in effect, making it obvious (to those with sufficient \nameref{sec:Spellcraft}) when it has been thrown off, or expired.

\paragraph{Augment:} If you spend 4 additional spell points, this spell's duration changes to 24 hours.

\subsubsection{Surge of Strength}
\label{Spell:SurgeOfStrength}
Transmutation
\\ \textbf{Level:} Blackguard 1, Paladin 1, Strength 1
\\ \textbf{Components:} V, S
\\ \textbf{Casting Time:} 1 swift action
\\ \textbf{Range:} Personal
\\ \textbf{Target:} You
\\ \textbf{Duration:} 1 round
\\ \textbf{Spell Points:} 1

\emph{Your muscles momentarily ripple with divine strength.}

You gain a +2 sacred bonus to strength, and two temporary hit points.

\paragraph{Augment:} For every 3 additional spell points you spend, the sacred bonus and number of temporary hit points increase by 1.
\subsection{'T' Spells}
\subsubsection{Twin Blade Dance}
\label{Spell:TwinBladeDance}
Illusion (Pattern) [Mind-Affecting]
\\ \textbf{Level:} Bard 2, Ranger 2
\\ \textbf{Components:} V
\\ \textbf{Casting Time:} 1 swift action
\\ \textbf{Range:} Touch
\\ \textbf{Targets:} Two melee weapons
\\ \textbf{Duration:} 1 round/level (D)
\\ \textbf{Saving Throw:} None, Will negates; see text
\\ \textbf{Spell Resistance:} No, Yes; see text
\\ \textbf{Spell Points:} 3

\emph{Your weapons appear to swirl around and leave trailing lines, creating a dizzying display of skill.}

An opponent struck by both of the weapons affected by this spell in the same round by the same creature must succeed on a Will save or be \emph{stunned} for one round.
Spell resistance applies against the stunning effect, as does immunity or resistance to mind-affecting spells, illusions, or patterns.

\subsection{'U' Spells}
\subsubsection{Uncanny Accuracy}
\label{Spell:UncannyAccuracy}
Divination
\\ \textbf{Level:} Bard 1, Blackguard 1
\\ \textbf{Components:} V
\\ \textbf{Casting Time:} 1 swift action
\\ \textbf{Range:} Personal
\\ \textbf{Target:} You
\\ \textbf{Duration:} 1 round
\\ \textbf{Spell Points:} 1

\emph{Your opponent's weak points suddenly seem more accessible.}

Whenever you make a weapon attack against a flanked opponent or against an opponent denied his dexterity bonus, you deal an extra 1d6 points of damage and gain a +2  bonus on the attack roll.

\paragraph{Augment:} For every 3 additional spell points you spend, this spell's damage increases by one die (d6).

\subsection{'V' Spells}
\subsubsection{Vicious Weapons}
\label{Spell:ViciousWeapons}
Necromancy
\\ \textbf{Level:} Ranger 4
\\ \textbf{Components:} V
\\ \textbf{Casting Time:} 1 standard action
\\ \textbf{Range:} Personal
\\ \textbf{Target:} You
\\ \textbf{Duration:} 1 min./level
\\ \textbf{Spell Points:} 7

\emph{The actual appearance of the weapons you wield does not change, but somehow they seem more cruel and dangerous.}

For the duration of this spell, all melee weapons you use gain the benefit of the Wounding weapon enhancement.

\subsection{'W' Spells}
\subsubsection{Warrior's Impetus}
\label{Spell:WarriorsImpetus}
Transmutation
\\ \textbf{Level:} War 5
\\ \textbf{Components:} V, S
\\ \textbf{Casting Time:} 1 standard action
\\ \textbf{Range:} Close (25 ft. + 5 ft./2 levels)
\\ \textbf{Target:} One creature
\\ \textbf{Duration:} 1 round/level
\\ \textbf{Saving Throw:} None
\\ \textbf{Spell Resistance:} Yes
\\ \textbf{Spell Points:} 9

\emph{You imbue your allies with a portion of the speed and power of all warriors who have ever followed your cause.}

When making a full attack action, a targeted creature may make one extra attack with any weapon he is holding. 
The attack is made using the creature's full base attack bonus, plus any modifiers appropriate to the situation. 
(This effect is not cumulative with similar effects, such as that provided by a \nameref{Spell:Haste}, nor does it actually grant an extra action, so you can't use it to cast a second spell or otherwise take an extra action in the round.)

The targets also gain a +3 morale bonus on attack rolls and damage rolls. (This bonus on attack rolls stacks with the untyped bonus provided by \nameref{Spell:Haste}.)
\paragraph{Augment:} For every additional spell point you spend, this spell can affect an additional creature.

\subsubsection{Winged Weapon}
\label{Spell:WingedWeapon}
Transmutation
\\ \textbf{Level:} Spellbreaker 1
\\ \textbf{Components:} V
\\ \textbf{Casting Time:} 1 swift action
\\ \textbf{Range:} Touch
\\ \textbf{Target:} One melee weapon
\\ \textbf{Duration:} 1 round
\\ \textbf{Saving Throw:} Will negates (harmless, object)
\\ \textbf{Spell Resistance:} Yes (harmless, object)
\\ \textbf{Spell Points:} 1

\emph{The weapon grows golden, faintly translucent wings that flutter rapidly.}

The touched weapon gains the benefits of the Throwing and Returning enhancements for the duration of the spell.

\paragraph{Augment:} This spell can be augmented in one or both of the following ways:

\begin{enumerate}
 \item If you spend 6 additional spell points, the weapon gains the benefit of the Dancing enhancement rather than the Throwing and Returning enhancements, and the base duration of the spell is increased to 4 rounds.
 \item For every 2 additional spell points you spend, this spell's duration is increased by one round.
\end{enumerate}

\subsubsection{Wolfsbane}
\label{Spell:Wolfsbane}
Necromancy
\\ \textbf{Level:} Moon 3
\\ \textbf{Components:} V, S
\\ \textbf{Casting Time:} 1 standard action
\\ \textbf{Range:} 120 ft.
\\ \textbf{Area:} 120-ft. line
\\ \textbf{Duration:} Instantaneous
\\ \textbf{Saving Throw:} Will partial; see text
\\ \textbf{Spell Resistance:} Yes
\\ \textbf{Spell Points:} 5

\emph{The horde of lycanthropes emits a howl of pain.}

Every lycanthrope caught in the area takes 50 points of damage and is \emph{paralyzed} for one round. A successful will save halves the damage and negates the paralysis.

\subsubsection{Woodbolt}
\label{Spell:Woodbolt}
Conjuration (Creation)
\\ \textbf{Level:} Plant 1
\\ \textbf{Components:} V, S
\\ \textbf{Casting Time:} 1 standard action
\\ \textbf{Range:} Close (25 ft. + 5 ft./2 levels)
\\ \textbf{Effect:} Ray
\\ \textbf{Duration:} Instantaneous
\\ \textbf{Saving Throw:} None
\\ \textbf{Spell Resistance:} None
\\ \textbf{Spell Points:} 1

\emph{You create a spearlike branch that hurls itself at your enemies.}

You create a wooden bolt shoots forth from your outstretched arm and strikes a target within range, dealing 1d6 points of damage.
You must succeed on a ranged attack (not a ranged touch attack) with the bolt for it to deal damage.

\paragraph{Augment:} For every additional spell point you spend, this spell's damage increases by one die (d6).

\subsection{'X' Spells}
\subsubsection{X-ray Vision}
\label{Spell:XRayVision}
Divination
\\ \textbf{Level:} Sor/Wiz 5
\\ \textbf{Components:} V, S
\\ \textbf{Casting Time:}  1 standard action
\\ \textbf{Range:} Personal
\\ \textbf{Target:} You
\\ \textbf{Duration:} 1 min./level
\\ \textbf{Spell Points:} 9

\emph{''The toughest part is to remember to open your door.``}

You gain the ability to see into and through solid matter.

The range of the X-ray vision is 20'. Within this range, you see as if in normal light even if there is no illumination (magical darkness is ignored as well). X-ray vision can penetrate 1 foot of stone, 1 inch of metal, or up to 3 feet of wood or dirt. A thin sheet of lead blocks the vision.
X-ray vision is black and white only but otherwise functions like normal sight.

Note that although X-ray vision may pierce certain forms of concealment, it does not grant line of effect.

In addition, you gain a +4 bonus on Heal checks made to re-set broken bones.

\paragraph{Augment:} You can augment this spell in one or both of the following ways:
\begin{enumerate}
 \item For every 2 additional spell points you spend, the radius of your X-ray vision increases by 5 feet.
 \item If you spend 4 additional spell points, stone, metal, wood and dirt do not block the X-ray vision. 
\end{enumerate}\newpage

\part{Creatures}
Player characters are not the only creatures around - they interact with monsters and minions.

This chapter includes examples of monsters converted from the \href{http://www.wizards.com/default.asp?x=d20/article/srd35}{d20 srd}, and a full conversion of all the kinds of magical minions PCs can collect. Converting all monsters from the d20 SRD is beyond the scope of this project.
\input{MagicalCreatures.tex}\newpage
\section{Monsters}
\subsection{Angel}
\label{sec:Angel}
Angels are a race of celestials, beings who live on the good-aligned Outer Planes.

Angels can be of any good alignment. Regardless of their alignment, angels never lie, cheat, or steal. They are impeccably honorable in all their dealings and often prove the most trustworthy and diplomatic of all the celestials.

All angels are blessed with comely looks, though their actual appearances vary widely.

Angels speak Celestial, Infernal, and Draconic, though they can speak with almost any creature because of their tongues ability. 
\subsubsection{Astral Deva}
An astral deva is about 7 $1/2$ feet tall and weighs about 250 pounds. 

An astral deva is not afraid to enter melee combat. It takes a fierce joy in bashing evil foes with its powerful mace, which it has invariably made more powerful by using its \nameref{Spell:MagicWeapon} spell. 

An astral deva's natural weapons, as well as any weapons it wields, are treated as good-aligned for the purpose of overcoming damage reduction. 
\begin{table*}
\label{tab:AstralDeva}
\caption{Astral Deva}
\makebox[\textwidth]{%\resizebox{\textwidth}{!}{
\begin{tabular}{p{0.3\textwidth}p{0.7\textwidth}}
\toprule
\textbf{Size/Type:}		&Medium Outsider (Angel, Extraplanar, Good)\\
\textbf{Hit Dice}		&12d8+48 (102 hp)\\
\textbf{Initiative}		&+8\\
\textbf{Speed}			&50 ft. (10 squares), fly 100 ft. (good)\\
\textbf{Armor Class:}		&29 (+4 Dex, +15 natural), touch 14, flat-footed 25\\
\textbf{BAB/Grapple:}		&+12/+18\\
\textbf{Attack:}		&Heavy mace +22 melee (1d8+12 plus stun, includes \nameref{Spell:MagicWeapon} augmented to provide a +4 bonus) or slam +18 melee (1d8+9)\\
\textbf{Full Attack:}		&Heavy mace +22/+17/+12 melee (1d8+12 plus stun, includes \nameref{Spell:MagicWeapon} augmented to provide a +4 bonus) or slam +18 melee (1d8+9)\\
\textbf{Space/Reach:}		&5 ft./5 ft.\\
\textbf{Special Attacks:}	&Spells, stun\\
\textbf{Special Qualities:}	&Change shape, damage reduction 10/evil, darkvision 60 ft., low-light vision, immunity to acid, cold, and petrification, protective aura, resistance to electricity 10 and fire 10, spell resistance 30, tongues, uncanny dodge\\
\textbf{Saves:}			&Fort +12 (+16 against poison), Ref +12, Will +12\\
\textbf{Abilities:}		&Str 22, Dex 18, Con 18, Int 18, Wis 18, Cha 20\\
\textbf{Skills:}		&Concentration +19, Craft or Knowledge (any three) +19, Diplomacy +22, Escape Artist +19, Hide +19, Intimidate +20, Listen +23, Move Silently +19, Sense Motive +19, Spot +23, Use Rope +4 (+6 with bindings)\\
\textbf{Feats:}			&\nameref{Feat:ExpandedKnowledge}: \nameref{Spell:FistOfTheDeity}, \nameref{Feat:ExpandedKnowledge}: \nameref{Spell:Invisibility}, Improved Initiative, \nameref{Feat:QuickenSpell}, Power Attack\\
\textbf{Environment:}		&Any good-aligned plane\\
\textbf{Organization:}		&Solitary, pair, or squad (3-5)\\
\textbf{Challenge Rating}	&14\\
\textbf{Treasure:}		&No coins; double goods; standard items\\
\textbf{Alignment:}		&Always good (any)\\
\textbf{Advancement:}		&By character class\\
\textbf{Level Adjustment:}	&+8\\
\bottomrule 
\end{tabular}}%}
\end{table*}
\paragraph{Spells:} Astral Devas cast spells as 17th-level \nameref{sec:Paladin}s. A typical astral deva might know the following spells:

% \nameref{Spell:Aid}, \nameref{Spell:DeathWard}, \nameref{Spell:DiscernAlignment}, \nameref{Spell:DispelAlignment}, \nameref{Spell:DispelMagic}, \nameref{Spell:DisruptingWeapon}, \nameref{Spell:FistOfTheDeity}, \nameref{Spell:Heal}, \nameref{Spell:Light}, \nameref{Spell:MagicWeapon}, \nameref{Spell:MarkOfJustice}, \nameref{Spell:PlaneShift}, \nameref{Spell:RemoveCurse}, \nameref{Spell:RemoveDisease}, \nameref{Spell:RemoveFear}, \nameref{Spell:TouchOfVitality}, \nameref{Spell:TrueSeeing}, and \nameref{Spell:ZoneOfTruth}.
1st - \nameref{Spell:DiscernAlignment}, \nameref{Spell:Light}, \nameref{Spell:MagicWeapon}, \nameref{Spell:RemoveFear}, \nameref{Spell:TouchOfVitality};
2nd - \nameref{Spell:Aid}, \nameref{Spell:ZoneOfTruth};
3rd - \nameref{Spell:DispelMagic} \nameref{Spell:RemoveDisease}, \nameref{Spell:RemoveCurse};
4th - \nameref{Spell:DeathWard}, \nameref{Spell:FistOfTheDeity}, \nameref{Spell:MarkOfJustice};
5th - \nameref{Spell:DispelAlignment}, \nameref{Spell:DisruptingWeapon}, \nameref{Spell:TrueSeeing};
6th - \nameref{Spell:Heal}, \nameref{Spell:PlaneShift};

They have 117 spell points when encountered (91 for casting as a 17th-level Paladin, 42 for their high Charisma score, 16 spent on casting \nameref{Spell:MagicWeapon} at the start of the day).
\paragraph{Change Shape (Su):}
An astral deva can assume the form of any Small or Medium humanoid.

\paragraph{Stun (Su):}
If an astral deva strikes an opponent twice in one round with its mace, that creature must succeed on a DC 22 Fortitude save or be stunned for 1d6 rounds. The save DC is Strength-based. The \nameref{Spell:MaceOfTheAstralDeva} spell is a mortal's imitation of this ability - Astral Devas gain no further benefit even if wielding a weapon under the influence of this spell.
\paragraph{Uncanny Dodge (Ex)}
An astral deva retains its Dexterity bonus to AC when flat-footed, and it cannot be flanked except by a rogue of at least 16th level. It can flank characters with the uncanny dodge ability as if it were a 12th-level rogue. 

\subsubsection{Planetar}
A planetar is nearly 9 feet tall and weighs about 500 pounds. Despite their vast array of magical powers, planetars are likely to wade into melee with their magically enhanced greatswords. They particularly enjoy fighting fiends.

A planetar's natural weapons, as well as any weapons it wields, are treated as good-aligned for the purpose of overcoming damage reduction.
\begin{table*}
\label{tab:Planetar}
\caption{Planetar}
\makebox[\textwidth]{%\resizebox{\textwidth}{!}{
\begin{tabular}{p{0.3\textwidth}p{0.7\textwidth}}
\toprule
\textbf{Size/Type:}		&Large Outsider (Angel, Extraplanar, Good)\\
\textbf{Hit Dice}		&14d8+70 (133 hp)\\
\textbf{Initiative}		&+8\\
\textbf{Speed}			&30 ft. (6 squares), fly 90 ft. (good)\\
\textbf{Armor Class:}		&36 (-1 size, +4 Dex, +19 natural, \nameref{Spell:MagicVestment} augmented to provide a +4 bonus), touch 13, flat-footed 28\\
\textbf{BAB/Grapple:}		&+14/+25\\
\textbf{Attack:}		&Greatsword +24 melee (3d6+14/19-20, includes \nameref{Spell:MagicWeapon} augmented to provide a +4 bonus) or slam +20 melee (2d8+10)\\
\textbf{Full Attack:}		&Greatsword +24/+19/+14 melee (3d6+14/19-20, includes \nameref{Spell:MagicWeapon} augmented to provide a +4 bonus) or slam +20 melee (2d8+10)\\
\textbf{Space/Reach:}		&10 ft./10 ft.\\
\textbf{Special Attacks:}	&Spells\\
\textbf{Special Qualities:}	&Change shape, damage reduction 10/evil, darkvision 60 ft., low-light vision, immunity to acid, cold, and petrification, protective aura, regeneration 10, resistance to electricity 10 and fire 10, spell resistance 30, tongues\\
\textbf{Saves:}			&Fort +14 (+18 against poison), Ref +13, Will +15\\
\textbf{Abilities:}		&Str 25, Dex 19, Con 20, Int 22, Wis 23, Cha 22\\
\textbf{Skills:}		&Concentration +22, Craft or Knowledge (any four) +23, Diplomacy +25, Escape Artist +21, Hide +17, Intimidate +23, Listen +23, Move Silently +21, Sense Motive +23, Search +23, Spot +23, Use Rope +4 (+6 with bindings)\\
\textbf{Feats:}			&\nameref{Feat:ExpandedKnowledge}:\nameref{Spell:CureWounds},\nameref{Feat:ExpandedKnowledge}:\nameref{Spell:RaiseDead}, Improved Initiative, \nameref{Feat:QuickenSpell}, Power Attack\\
\textbf{Environment:}		&Any good-aligned plane\\
\textbf{Organization:}		&Solitary or pair\\
\textbf{Challenge Rating}	&16\\
\textbf{Treasure:}		&No coins; double goods; standard items\\
\textbf{Alignment:}		&Always good (any)\\
\textbf{Advancement:}		&15-21 HD (Large); 22-42 HD (Huge)\\
\textbf{Level Adjustment:}	&-\\
\bottomrule
\end{tabular}}
\end{table*}
\paragraph{Spell-like Abilities:}
At will: \nameref{Spell:DiscernAlignment}, \nameref{Spell:DispelMagic}, \nameref{Spell:Light}.

In addition, the following abilities are always active on the planetar's person, as the spells (caster level 17th): \nameref{Spell:SeeInvisibility}, and \nameref{Spell:TrueSeeing}. They can be dispelled, but the planetar can reactivate them as a free action.
\paragraph{Spells:} Planetars can cast divine spells as 17th-level clerics. A planetar selects his spells known from the domains of Air, Destruction, Good, Law, or War, but does not gain their granted powers. A typical Planetar might know the following spells:

1st - \nameref{Spell:AlignedProtection}, \nameref{Spell:Bless}, \nameref{Spell:CureWounds}, \nameref{Spell:DivineFavor}, \nameref{Spell:InflictWounds},\nameref{Spell:MagicWeapon}, \nameref{Spell:TouchOfVitality};
2nd - \nameref{Spell:Aid}, \nameref{Spell:Consecrate}, \nameref{Spell:MaskAlignment};
3rd - \nameref{Spell:Contagion}, \nameref{Spell:MagicVestment}, \nameref{Spell:WindWall};
4th - \nameref{Spell:FistOfTheDeity}, \nameref{Spell:FlameStrike};
5th - \nameref{Spell:DispelAlignment}, \nameref{Spell:MarkOfJustice}, \nameref{Spell:RaiseDead}, \nameref{Spell:RighteousMight}, \nameref{Spell:WarriorsImpetus};
6th - \nameref{Spell:Disintegrate}, \nameref{Spell:Harm};
7th - \nameref{Spell:PowerWord}, \nameref{Spell:WordOfGod};
8th - \nameref{Spell:AlignedAura}, \nameref{Spell:Whirlwind};
9th - \nameref{Spell:Implosion};

They have 250 spell points when encountered (231 for casting as a 17th-level Cleric, 51 for their high Wisdom score, 16 and 16 spent on casting \nameref{Spell:MagicWeapon} and \nameref{Spell:MagicVestment} at the start of the day).
\paragraph{Change Shape (Su):}
A planetar can assume the form of any Small or Medium humanoid.
\paragraph{Regeneration (Ex):} A planetar takes damage from evil-aligned weapons and from spells and effects with the evil descriptor.
\subsection{Nymph}
A Nymph is about the height and weight of a female elf.

Nymphs speak Sylvan and Common.

\begin{table*}
\label{tab:Nymph}
\caption{Nymph}
\makebox[\textwidth]{%\resizebox{\textwidth}{!}{
\begin{tabular}{p{0.3\textwidth}p{0.7\textwidth}}
\toprule
\textbf{Size/Type:}		&Medium Fey\\
\textbf{Hit Dice}		&6d6+6 (27 hp)\\
\textbf{Initiative}		&+3\\
\textbf{Speed}			&30 ft. (6 squares), swim 20 ft.\\
\textbf{Armor Class:}		&17 (+3 Dex, +4 deflection), touch 17, flat-footed 14\\
\textbf{BAB/Grapple:}		&+3/+3\\
\textbf{Attack:}		&Dagger +3 melee (1d4/19-20)\\
\textbf{Full Attack:}		&Dagger +3 melee (1d4/19-20)\\
\textbf{Space/Reach:}		&5 ft./5 ft.\\
\textbf{Special Attacks:}	&Blinding beauty, spells, spell-like abilities, stunning glance\\
\textbf{Special Qualities:}	&Damage reduction 10/cold iron, low-light vision, unearthly grace, wild empathy\\
\textbf{Saves:}			&Fort +7, Ref +12, Will +12\\
\textbf{Abilities:}		&Str 10, Dex 17, Con 12, Int 16, Wis 17, Cha 19\\
\textbf{Skills:}		&Concentration +10, Diplomacy +6, Escape Artist +12, Handle Animal +13, Heal +12, Hide +12, Listen +12, Move Silently +12, Ride +5, Sense Motive +12, Spot +12, Swim +8, Use Rope +3 (+5 with bindings)\\
\textbf{Feats:}			&\nameref{Feat:ExpandedKnowledge}:\nameref{Spell:CureWounds}, \nameref{Feat:ExtendSpell}, \nameref{Feat:SplitRay}\\
\textbf{Environment:}		&Temperate forests\\
\textbf{Organization:}		&Solitary\\
\textbf{Challenge Rating}	&7\\
\textbf{Treasure:}		&Standard\\
\textbf{Alignment:}		&Usually Chaotic Good\\
\textbf{Advancement:}		&7-12 HD (Medium)\\
\textbf{Level Adjustment:}	&+7\\
\bottomrule 
\end{tabular}}
\end{table*}

\subsubsection{Combat}
\paragraph{Blinding Beauty (Su):}
This ability affects all humanoids within 30 feet of a Nymph. 
Those who look directly at a Nymph must succeed on a DC 17 Fortitude save or be blinded permanently as though by the blindness spell. 
A Nymph can suppress or resume this ability as a free action.

The save DC is Charisma-based.

\paragraph{Spell-Like Abilities:}
1/day - \nameref{Spell:StepThroughEarth}. Caster level 7th.

\paragraph{Spells:}
Nymphs can cast divine spells as 7th-level Clerics.
A Nymph selects her spells known from the domains of Air, Plant, and Water, but does not gain their granted powers. A typical Nymph might know the following spells:

1st - \nameref{Spell:ConverseWithNature}, \nameref{Spell:TouchOfVitality}, \nameref{Spell:Entangle}, \nameref{Spell:Fog}, \nameref{Spell:Woodbolt};
2nd - \nameref{Spell:Barkskin}, \nameref{Spell:CommandNaturesAllies}, \nameref{Spell:FormPlant};
3rd - \nameref{Spell:CallLightning}, \nameref{Spell:SpikeGrowth}, \nameref{Spell:SummonWaterElemental};
4th - \nameref{Spell:RustingGrasp};

They have 53 spell points when encountered (43 for casting as a 7th-level Cleric, and 10 for their high Wisdom score).

\paragraph{Stunning Glance (Su):}
As a standard action, a wrathful Nymph can stun a creature within 30 feet with a look. The target creature must succeed on a DC 17 Fortitude save or be stunned for 2d4 rounds. The save DC is Charisma-based.

\paragraph{Unearthly Grace (Su):}
A Nymph adds her Charisma modifier as a bonus on all her saving throws, and as a deflection bonus to her Armor Class. (The statistics block already reflects these bonuses).

\paragraph{Wild Empathy (Ex):}
This power works like the \nameref{sec:WildEmpathy} ability, using the Nymph's number of HD as her Ranger level. A Nymph has a +6 racial bonus on the check.

\paragraph{Skills:}
A Nymph has a +8 racial bonus on any Swim check to perform some special action or avoid a hazard. She can always choose to take 10 on a Swim check, even if distracted or endangered. She can use the run action while swimming, provided she swims in a straight line.
\subsection{Ogre Mage}
\label{sec:OgreMage}
The ogre mage is a more intelligent and dangerous variety of its mundane cousin.

An ogre mage stands about 10 feet tall and weighs up to 700 pounds. Its skin varies in color from light green to light blue, and its hair is black or very dark brown. Ogre mages favor loose, comfortable clothing and lightweight armor.

Ogre mages speak Giant and Common.
\subsubsection{Combat}
Ogre mages rely on their spells to see them through, resorting to physical combat only when necessary.
When faced with obviously superior forces, they prefer to retreat using gaseous form rather than fight a losing battle.
\begin{table*}
\label{tab:OgreMage}
\caption{Ogre Mage}
\makebox[\textwidth]{%\resizebox{\textwidth}{!}{
\begin{tabular}{|p{0.3\textwidth}|p{0.7\textwidth}|}
\hline
\textbf{Size/Type:}		&Large Giant\\
\textbf{Hit Dice}		&8d8+24 (60 HP)\\
\textbf{Initiative}		&+4\\
\textbf{Speed}			&40 ft. (8 squares), can cast \nameref{Spell:Fly}\\
\textbf{Armor Class:}		&18 (-1 size, +5 natural, +4 chain shirt), touch 9, flat-footed 18, can cast \nameref{Spell:Shield}\\
\textbf{BAB/Grapple:}		&+6/+15\\
\textbf{Attack:}		&Greatsword +10 melee (3d6+7/19-20) or longbow +5 ranged (2d6/$\times$3)\\
\textbf{Full Attack:}		&Greatsword +10/+5 melee (3d6+7/19-20) or longbow +5/+0 ranged (2d6/$\times$3)\\
\textbf{Space/Reach:}		&10 ft./10 ft.\\
\textbf{Special Attacks:}	&Spells\\
\textbf{Special Qualities:}	&Darkvision 60', low-light vision, regeneration 5, spell resistance 20\\
\textbf{Saves:}			&Fort +9, Ref +2, Will +4\\
\textbf{Abilities:}		&Str 21, Dex 10, Con 17, Int 14, Wis 14, Cha 17\\
\textbf{Skills:}		&Concentration +14, Listen +13, \nameref{sec:Spellcraft} +13, Spot +13\\
\textbf{Feats:}			&\nameref{Feat:ExpandedKnowledge}: \nameref{Spell:Invisibility}, Improved Initiative, \nameref{Feat:SilentSpell}\\
\textbf{Environment:}		&Cold hills\\
\textbf{Organization:}		&Solitary, pair, or troupe (1-2 plus 2-4 ogres)\\
\textbf{Challenge Rating}	&8\\
\textbf{Alignment:}		&Usually lawful evil\\
\textbf{Treasure:}		&Double standard\\
\textbf{Advancement:}		&By character class\\
\textbf{Level Adjustment:}	&+4\\
\hline \end{tabular}}%}
\end{table*}
\paragraph{Change Shape (Su):}
An ogre mage can assume the form of any Small, Medium, or Large humanoid or giant.
\paragraph{Regeneration (Ex):}
Fire and acid deal normal damage to an ogre mage.

An ogre mage that loses a limb or body part can reattach it by holding the severed member to the stump. Reattachment takes 1 minute. If the head or some other vital organ is severed, it must be reattached within 10 minutes or the creature dies. An ogre mage cannot regrow lost body parts.
\paragraph{Spells:} Ogre Mages cast spells as 9th-level \nameref{sec:Sorcerer}s.
They typically know the spells \nameref{Spell:ArcaneEye}, \nameref{Spell:Charm}, \nameref{Spell:ConeOfCold}, \nameref{Spell:Darkness}, \nameref{Spell:FalseLife}, \nameref{Spell:Fly}, \nameref{Spell:GaseousForm}, \nameref{Spell:HallucinatoryTerrain}, \nameref{Spell:Invisibility}, \nameref{Spell:Shield}, and \nameref{Spell:Sleep}.

They have 104 spell points (91 for casting as a 9th-level Sorcerer, 13 for their high Charisma score).
\paragraph{Ogre Mages As Characters:}
Ogre mage characters possess the following racial traits.

\begin{itemize}
 \item +10 Strength, +6 Constitution, +4 Intelligence, +4 Wisdom, +6 Charisma.
 \item Large size. -1 penalty to Armor Class, -1 penalty on attack rolls, -4 penalty on Hide checks, +4 bonus on grapple checks, lifting and carrying limits double those of Medium characters.
 \item Space/Reach: 10 feet/10 feet.
 \item An ogre mage's base land speed is 40 feet.
 \item Darkvision: Ogre mages can see in the dark up to 60 feet.
 \item Racial Hit Dice: An ogre mage begins with eight levels of giant, which provide 8d8 Hit Dice, a base attack bonus of +6, and base saving throw bonuses of Fort +6, Ref +2, and Will +2.
 \item Racial Skills: An ogre mage's giant levels give it skill points equal to 11 $\times$ (2 + Int modifier [minimum 1]). Its class skills are Concentration, Listen, Spellcraft, and Spot.
 \item Racial Feats: An ogre mage's giant levels give it three feats.
 \item +5 natural armor bonus.
 \item Spellcasting: Ogre mages cast spells as 9th-level Sorcerers.
 \item Special Qualities (see above): Regeneration 5, spell resistance equal to 12 + number of hit dice the Ogre Mage has (including both racial HD and class levels).
 \item Automatic Languages: Common, Giant. Bonus Languages: Dwarven, Goblin, Infernal, Orc.
 \item Favored Class: Sorcerer.
 \item Level adjustment +4.
\end{itemize}\newpage

\part{Items}
Due to changes in the underlying magic system, many magic items must be updated.
Those items that directly affect the casting of spells (such as scrolls and wands) are wholly altered.
Some magic items that replicate the effects of spells have been updated, others stand unchanged.
Note that even if the rules text of a magic item that references a spell mechanic is unchanged, the rules text of the spell the item references may not be.

Unless noted otherwise in this chapter, use the rules text presented in the \href{http://www.wizards.com/default.asp?x=d20/article/srd35}{d20 srd}. See \nameref{sec:MissingItems} for a summary on the changes from the d20 SRD.
\section{Magic Items}
\subsection{Overview}
Due to changes in the underlying magic system, many magic items must be updated.
Those items that directly affect the casting of spells (such as scrolls and wands) are wholly altered.
Some magic items that replicate the effects of spells have been updated, others stand unchanged.
Note that even if the rules text of a magic item that references a spell mechanic is unchanged, the rules text of the spell the item references may not be.

Unless noted otherwise in this chapter, use the rules text presented in the \href{http://www.wizards.com/default.asp?x=d20/article/srd35}{d20 srd}. See \nameref{sec:MissingItems} for a summary on the changes from the d20 SRD.
\subsubsection{Augments}
\subsubsection{Augments}
Unless otherwise noted, assume that any item that refers to a spell refers to the unaugmented form of that spell.
\subsubsection{Saving throws}
Some items duplicate spells that allow saving throws (or allow the casting of such spells directly). Unless otherwise noted, the DC for such a saving throw is calculated as if it were cast using the number of spell points spent on the spell (the minimum, unless otherwise noted) by a spellcaster with a key ability score equal to 10+the total number of spell points spent on the spell.

For example, an item duplicating an unaugmented \nameref{Spell:FlameStrike} would have a default save DC of 14 (the base for a spell costing 7 spell points) + 3 (the key ability modifier of a spellcaster with a $10 + 7 = 17$ in his key ability score) = 17.

Were the Flame Strike augmented to cost 10 spell points, the default save DC for the item would be 15 (the base for a spell costing 10 spell points) + 5 (the key ability modifier of a spellcaster with a $10 + 10 = 20$ in his key ability score) = 20.
\subsection{Special Magical Materials}
\subsubsection{Spellsteel}
Iron (particularly cold iron) is usually disruptive to the flow of magical energies, to the point where Fey and some other supernatural creatures are harmed by it. 
Spellsteel is the exception. 
When iron is mined close to a powerful nexus of ley lines, spellcasters have found that sometimes the ambient magic has imprinted the metal, fundamentally altering the properties it has when forged to form steel. While a weapon made of spellsteel is no different from a mundane steel weapon for a nonmagical character, a spellcaster who wields a spellsteel weapon can focus magical power through it, increasing the damage that weapon deals.
 
As a free action that does not provoke attacks of opportunity, the wielder can channel magical power into a melee weapon or ranged weapon made of spellsteel. For 2 spell points, the spellsteel weapon deals an extra 2d6 points of damage. The weapon will stay charged for 1 minute or until it scores its next hit. Bows, crossbows, and slings bestow this power on their ammunition. All missile weapons lose this effect if they miss. However, they may be recovered and charged again.

Any weapon made of spellsteel costs 1,000 gp more than its mundane counterpart. Any item that can be made out of ordinary steel could potentially be made out of spellsteel. 

Spellsteel has 30 hit points per inch of thickness and a hardness of 10. 
\subsubsection{Poisons}
Most poisons affect all mortal creatures equally, causing everyone pain and illness. However, alchemists and hedge witches have long known of the powers of the forkroot. The forkroot, when ingested in sufficient amounts, gives anyone a strong feeling of drowsiness, but in addition, spellcasters feel their grip on magic slip away as the root takes hold.

If a caster fails a saving throw against a forkroot poison's initial damage, his caster level is immediately reduced by the amount indicated on the \nameref{tab:Poisons} table for 1d4 hours, to a minimum of 0. 
If multiple batches of a forkroot poison are consumed, the reduction stacks.
Hiding more than one batch of forkroot poison in a single meal or drink is not feasible.
The root has a bitter taste that is easily recognized by anyone who has experienced the poison before - but not before it has begun to take effect.
\begin{table*}
\label{tab:Poisons}
\caption{Poisons}
\centering
\begin{tabular}{|l|l|l|l|l|}
\hline
\textbf{Poison}		&\textbf{Type}	&\textbf{Initial Damage}&\textbf{Secondary Damage}	&\textbf{Price}\\
\hline
Forkroot mash		&Ingested DC 14	&Caster level -1d2	&Unconsciousness		&100GP\\
Forkroot powder		&Ingested DC 18	&Caster level -1d4	&Unconsciousness		&300GP\\
Purified forkroot powder&Ingested DC 22	&Caster level -2d4	&Unconsciousness		&1000GP\\
\hline
\end{tabular}

\end{table*}
\subsection{Matrices, Potions, and Oils}
\label{Item:Matrices}
Matrices, potions, and oils are items that store a single charge of spell energy.
The terms are interchangeable - any spell that could be stored in one type of the items could conceivably be stored in another.
Of these, potions are best known and most common, but ``Matrix'' is the most general term. 
For this reason, the whole family of items will be referred to as ``matrices'' throughout this chapter.

Using a matrix is basically like casting a spell.
It can be used only once, and then it is depleted.

\paragraph{Physical Description:}
Matrices can come in almost any shape and form, but they are always small (2 cubic inches at most), relatively fragile, and have negligible weight.
They have AC 13, 1 hit point, hardness 1, and a break DC of 12.
\begin{list}{\labelitemi}{\leftmargin=1em}
 \item \emph{Potions and oils:} A typical potion or oil consists of 1 ounce of liquid held in a ceramic or glass vial fitted with a tight stopper. How to use the contents of the vial depends on the spell stored in it, but most are either drunk or smeared on to the target of the spell.
 \item \emph{Other matrices:} Matrices other than potions and oils often take the form of lesser gems, animal parts (such as skulls or femurs), or something specific to the spell in question.
\end{list}
See \nameref{tab:MatrixAppearances} for examples on what matrices can look like.
Regardless of their physical appearance, the activation mechanics of matrices remain the same.

\begin{table*}
\caption{Matrix appearances}
\label{tab:MatrixAppearances}
\centering
\begin{tabular}{|l|l|l|}
\hline
\textbf{Spell name}&\textbf{Appearance}&\textbf{Usage example}\\
\hline
\nameref{Spell:BlackTentacles}	&Severed tentacle&Thrown, grows on impact\\
\nameref{Spell:CureWounds}	&Vial of oil 	&Smeared on wounds\\
\nameref{Spell:CureWounds}	&Small ruby 	&Touched to forehead of recipient\\
\nameref{Spell:DivinePower}	&Skull		&Crushed in hand\\
\nameref{Spell:Fireball}	&Glass bead	&Flies to target area and explodes\\
\nameref{Spell:RemoveDisease}	&Potion		&Imbibed\\
\hline
\end{tabular}
\end{table*}

\paragraph{Identifying the Matrix:} 
A matrix must be identified before a character can safely use it or know exactly what spell it contains. 
Doing this requires a successful \nameref{sec:Spellcraft} check (DC 25) or a spell such as \nameref{Spell:Identify}. 
Once a particular matrix has been identified, it is not necessary to do so again. 
Identifying a matrix in advance lets a character proceed directly to the next step when the time comes to use it.
Once the matrix is identified, the character becomes aware of what spell it stores within, and how to activate it. 
Identifying a matrix via Spellcraft requires one minute of study.
A failed check to identify a matrix may not be retried until the character gains another rank in spellcraft.

An experienced character learns to identify matrices by memory - for example, the last time he tasted a liquid that reminded him of almonds, it turned out to be a potion of \nameref{Spell:CureWounds}.

\paragraph{Activation:} 
To safely activate a matrix, a caster must have identified it, as described above.
To activate it blindly, the caster must make a DC 10 Wisdom check or suffer a mishap (see below).
Once the matrix been identified (or successfully activated through guesswork), it is thereafter treated as a spell completion item.

Using the stored spell of a matrix requires holding it and performing its specific rite of activation (which requires the same kind of action as casting the spell ordinarily).
Activating matrix is subject to disruption just as casting a spell normally would be.
Additionally, the user must meet the following requirements.
\begin{list}{\labelitemi}{\leftmargin=1em}
 \item The spell must be of the correct type (arcane or divine). Arcane spellcasters (Bards, Sorcerers, and Wizards) can only use matrices containing arcane spells, and divine spellcasters (Clerics, Paladins, and Rangers) can only use matrices containing divine spells. (The type of matrix a character creates is also determined by his or her class.)
 \item The user must have the spell on his class list.
 \item The user must have the requisite key ability score.
\end{list}
If the user meets these requirements and has a caster level at least equal to the spell's caster level, he can automatically cast the stored spell without a check. 
If he meets all requirements but his own level is lower than the caster level of the matrix, she has to make a caster level check (1d20 + user's caster level), against a DC equal to the caster level of the matrix +1 to cast the spell successfully. 
On a failure, the user must succeed on a DC 5 Wisdom check to avoid a mishap (see below).
A natural roll of 1 on a check to avoid a mishap is always a failure.
\paragraph{Determine Effect:} A spell successfully cast from a matrix works exactly as if cast normally. 
Assume the caster level of the matrix is always the minimum level required to cast the spell for the character who infused the matrix, unless the creator specifically desires
otherwise.
\paragraph{Matrix Mishaps:} When a mishap occurs, the spell in the matrix has a reversed or harmful effect. Possible mishaps are given below.
\begin{list}{\labelitemi}{\leftmargin=1em}
 \item A surge of uncontrolled magical energy deals 1d6 points of damage per spell level to the matrix user.
 \item Spell strikes the matrix user or an ally instead of the intended target, or a random target nearby if the matrix user was the intended recipient.
 \item Spell takes effect at some random location within spell range.
 \item Spell's effect on the target is contrary to the spell's normal effect.
 \item The matrix user suffers some minor but bizarre effect related to the spell in some way. Most such effects should last only as long as the original spell's duration, or 2d10 minutes for instantaneous spells.
 \item Some innocuous item or items appear in the spell's area.
 \item Spell has delayed effect. Sometime within the next 1d12 hours, the spell activates. If the matrix user was the intended recipient, the spell takes effect normally. If the user was not the intended recipient, the spell goes off in the general direction of the original recipient or target, up to the spell's maximum range, if the target has moved away.
\end{list}

\begin{table*}
\caption{Matrices}
\label{tab:Matrices}
\centering
\begin{tabular}{|c|r|r|}
\hline
\textbf{Spell level}&\textbf{Market Price}$^1$&\textbf{Cost to Create}$^2$\\
\hline
1&	25gp	&12gp 5sp, 1XP\\
2&	150gp	&75gp, 6XP\\
3&	375gp	&187gp 5sp, 15XP\\
4&	700gp	&350gp, 28XP\\
5&	1125gp	&562gp 5sp, 45XP\\
6&	1650gp	&825gp, 66XP\\
7&	2275gp	&1137gp 5sp, 91XP\\
8&	3000gp	&1500gp, 120XP\\
9&	3825gp	&1912gp 5sp, 153XP\\
\hline
\end{tabular}
\scriptsize
\begin{enumerate}
 \item Any matrix that contains a spell with an experience point cost costs an additional 5 GP for each point of XP required.
 \item Any matrix that contains a spell with an experience point cost also has that experience cost in addition to that noted here.
\end{enumerate}
\end{table*}
\paragraph{Foolproof Matrices}
\label{Item:FoolproofMatrices}
Some matrices are ``foolproof''. These cost twice the amount indicated on the \nameref{tab:Matrices} table, but are usable by anyone, regardless of class, ability scores, and level.
Using such a matrix requires no special skill, they are use-activated rather than spell completion items.
It is always obvious how to use a Foolproof Matrix.

However, not any spell can be squeezed into a Foolproof Matrix. A Foolproof Matrix must fulfill all of the following requirements:
\begin{list}{\labelitemi}{\leftmargin=1em}
 \item The spell must be of 3rd level or below.
 \item The spell must target one or more creatures.
 \item The spell must have a range of Personal or Touch.
\end{list}
Most Foolproof Matrices are potions or oils.
\subsection{Scrolls}
\label{Item:Scrolls}
More than a mundane inscription, a scroll is a magical item that, when read by a character with the appropriate knowledge, bestows complete information on how to cast a certain spell (or collection of spells) upon the reader.
% A stored spell can be used only once. Using a scroll is
% basically like casting a spell.\footnote{Instead of activating a scroll directly, 
% it is possible to use the knowledge contained on it to cast the spell using the caster's own spell point reserve.}

\paragraph{Physical Description:} A scroll is a heavy sheet of fine vellum or high-quality paper. 
An area about 8 1/2 inches wide and 11 inches long is sufficient to hold one spell. 
The sheet is reinforced at the top and bottom with strips of leather slightly longer than the sheet is wide. 
A scroll holding more than one spell has the same width (about 8 1/2 inches) but is an extra foot or so long for each extra spell. 
Scrolls that hold three or more spells are usually fitted with reinforcing rods at each end rather than simple strips of leather. 
A scroll has AC 9, 1 hit point, hardness 0, and a break DC of 8.

To protect it from wrinkling or tearing, a scroll is rolled up from both ends to form a double cylinder. 
(This also helps the user unroll the scroll quickly.)
The scroll is placed in a tube of ivory, jade, leather, metal, or wood.
Most scroll cases are inscribed with magic symbols which often identify the owner or the spells stored on the scrolls inside. 
The symbols often hide magic traps.

\paragraph{Deciphering the Scroll:} 
A scroll must be deciphered before a character can use it or know exactly what spell it contains. 
Doing this requires a successful \nameref{sec:Spellcraft} check (DC 15 + spell level) or the use of \nameref{sec:Cantrips}. 
Once a particular scroll has been deciphered, it is not necessary to do so again. 
Deciphering a scroll in advance lets a character proceed directly to the next step when the time comes to use it.
Once the scroll is deciphered, the character becomes aware of all the spells stored on the scroll. 
Deciphering a scroll requires one minute of study (or one standard action, if \nameref{sec:Cantrips} are used).
A failed check to decipher a scroll may be retried once per day.

% \paragraph{Activation:} To activate a scroll, a caster must have deciphered
% it, as described above. 
% Once the scroll has been deciphered, it is treated as a
% spell completion item, except as noted below.
% 
% Using a scroll's stored spell after deciphering it requires holding
% it and reading its inscription (a which requires the same kind of action as casting the spell ordinarily). 
% Activating a scroll is subject to disruption just as
% casting a spell normally would be.
% Additionally, the user must meet the following requirements.
% \begin{list}{\labelitemi}{\leftmargin=1em}
%  \item The user must have the spell on his class list.
%  \item The user must have the requisite key ability score.
% \end{list}
% If the user meets these requirements and has a caster level at least equal
% to the spell's caster level, he can automatically cast the stored spell
% without a check. 
% If he meets both requirements but his own level is lower than
% the scroll's caster level, she has to make a caster level check
% (1d20 + user's level), against a DC equal to the scroll's caster level
% +1 to cast the spell successfully. On a failure, the user must succeed on a
% DC 5 Wisdom check to avoid a mishap (see below). A natural roll of 1 on this
% check is always a failure.

\paragraph{Activation:} 
After deciphering the scroll, the spellcasting character must choose one of the spells available on the scroll and read it.
As part of reading the spell, make a \nameref{sec:Spellcraft} check  (DC 15 + the spell's level) to see if the spell will be correctly cast. 
If the spell is not on the caster's class list, he automatically fails this check.
This check requires one full round, which provokes attacks of opportunity.

Upon successfully making the check, the character can attempt to cast that spell normally on his next turn, even if he doesn't know it (assuming he has spell points left for the day). 
He retains the ability to cast the selected spell for only 1 round. 
If he doesn't cast the spell, fails the Spellcraft check, or casts a different spell, 
he loses his chance to cast that spell unless the source is read again.

\paragraph{Determine Effect:} 
A spell successfully cast from a scroll works exactly as if cast normally. 
Assume the scroll's caster level is always the minimum level required to cast the spell for the character who wrote the scroll, unless the creator specifically desires otherwise.

\begin{table*}
\caption{Scrolls}
\label{tab:Scrolls}
\centering
\begin{tabular}{|c|r|r|}
\hline
\textbf{Spell point cost}&\textbf{Market Price}$^1$&\textbf{Cost to Create}$^2$\\
\hline
1&	1000gp	&500gp, 40XP\\
3&	4000gp	&2000gp, 160XP\\
5&	9000gp	&4500gp, 360XP\\
7&	16000gp	&8000gp, 640XP\\
9&	25000gp	&12500gp, 1000XP\\
11&	36000gp	&18000gp, 1440XP\\
13&	49000gp	&24500gp, 1960XP\\
15&	64000gp	&32000gp, 2560XP\\
17&	81000gp	&40500gp, 3240XP\\
\hline
\end{tabular}
\scriptsize
\begin{enumerate}
 \item Any scroll that has a spell with an experience point cost written on it costs an additional 5 GP for each point of XP spent.
 \item Any scroll that has a spell with an experience point cost written on it also has an XP cost in addition to that noted here.
\end{enumerate}
\end{table*}
\subsection{Rings}
\subsubsection{Animal Friendship}
\label{Item:AnimalFriendship}
   \textbf{Price:} 10800gp
\\ \textbf{Body Slot:} Finger
\\ \textbf{Caster Level:} 3rd
\\ \textbf{Activation:} -
\\ \textbf{Weight:} -

While wearing this ring, you gain the \nameref{sec:WildEmpathy} class feature, as a 3rd-level Ranger. If you already have the class feature, you instead gain a +3 bonus on Wild Empathy checks.

%\paragraph{Physical Description:} STUFF

\paragraph{Prerequisites:} \nameref{Feat:ForgeRing}, \nameref{sec:WildEmpathy} class feature

\paragraph{Cost to Create:} 5400gp, 432 XP
% \subsubsection{Djinni Calling}
% \label{Item:DjinniCalling}
%    \textbf{Price:} 50000gp
% \\ \textbf{Body Slot:} Finger
% \\ \textbf{Caster Level:} 17th
% \\ \textbf{Activation:} Standard (Command)
% \\ \textbf{Weight:} -
% 
% One of the many rings of fable, this ``genie'' ring is most useful indeed. It serves as a special gate by means of which a specific djinni can be called from the Elemental Plane of Air. When the ring is rubbed (a standard action), the call goes out, and the djinni appears on the next round. The djinni faithfully obeys and serves the wearer of the ring, but never for more than 1 hour per day. If the djinni of the ring is ever killed, the ring becomes nonmagical and worthless.
% 
% \paragraph{Prerequisites:} \nameref{Feat:ForgeRing}, \nameref{Spell:Gate}
% 
% \paragraph{Cost to Create:} 2500gp, 2000 XP

\subsubsection{Elemental Command}
\label{Item:ElementalCommand}
   \textbf{Price:} 200000gp
\\ \textbf{Body Slot:} Finger
\\ \textbf{Caster Level:} 15th
\\ \textbf{Activation:} see text
\\ \textbf{Weight:} -

All four kinds of elemental command rings are very powerful. 
Each appears to be nothing more than a lesser magic ring until fully activated (by meeting a special condition, such as single-handedly slaying an elemental of the appropriate type or exposure to a sacred material of the appropriate element), but each has certain other powers as well as the following common properties.

Elementals of the plane to which the ring is attuned can't attack the wearer, or even approach within 5 feet of him. If the wearer desires, he may forego this protection and instead attempt to charm the elemental (as \nameref{Spell:Charm}, Will DC 17 negates). If the charm attempt fails, however, absolute protection is lost and no further attempt at charming can be made.

Creatures from the plane to which the ring is attuned who attack the wearer take a -1 penalty on their attack rolls. 
The ring wearer makes applicable saving throws from the extraplanar creature's attacks with a +2 resistance bonus. 
He gains a +4 morale bonus on all attack rolls against such creatures. 
Any weapon he uses bypasses the damage reduction of such creatures, regardless of any qualities the weapon may or may not have.

The wearer of the ring is able to converse with creatures from the plane to which his ring is attuned. 
These creatures recognize that he wears the ring. 
They show a healthy respect for the wearer if alignments are similar. 
If alignments are opposed, creatures fear the wearer if he is strong. If he is weak, they hate and desire to slay him.

In addition to the powers described above, each specific ring gives its wearer the following abilities according to its kind.

\paragraph{Ring of Elemental Command (Air)}
\begin{list}{\labelitemi}{\leftmargin=1em}
\item \nameref{Spell:ControlFall} (unlimited use, wearer only)
\item \nameref{Spell:ResistEnergy} (electricity only) (unlimited use, wearer only)
\item \nameref{Spell:GustOfWind} (twice per day)
\item \nameref{Spell:WindWall} (unlimited use)
\item \nameref{Spell:AirWalk} (once per day, wearer only)
\item \nameref{Spell:ChainLightning} (once per day, electricity only)
\end{list}
The ring appears to be a ring of \nameref{Item:FeatherFall} until a certain condition is met to activate its full potential. It must be reactivated each time a new wearer acquires it.

\paragraph{Ring of Elemental Command (Earth)}
\begin{list}{\labelitemi}{\leftmargin=1em}
\item \nameref{Spell:MeldIntoStone} (unlimited use, wearer only)
\item \nameref{Spell:SoftenEarthAndStone} (unlimited use)
\item \nameref{Spell:MoldMaterial} (twice per day)
\item \nameref{Spell:Stoneskin} (once per week, wearer only)
\item \nameref{Spell:WallOfStone} (once per day)
\end{list}
The ring appears to be a ring of meld into stone until the established condition is met.

\paragraph{Ring of Elemental Command (Fire)}
\begin{list}{\labelitemi}{\leftmargin=1em}
\item \nameref{Spell:ResistEnergy} (fire only) (as a major ring of energy resistance [fire])
\item \nameref{Spell:ShockingGrasp} (fire only)
\item \nameref{Spell:FlamingSphere} (twice per day)
\item \nameref{Spell:Pyrotechnics} (twice per day)
\item \nameref{Spell:WallOfFire}  (once per day)
\item \nameref{Spell:FlameStrike} (twice per week)
\end{list}
The ring appears to be a major ring of energy resistance (fire) until the established condition is met.

\paragraph{Ring of Elemental Command (Water)}
\begin{list}{\labelitemi}{\leftmargin=1em}
\item \nameref{Spell:WaterWalk} (unlimited use)
\item \nameref{sec:CreateWater}, as the Water Domain ability (unlimited use). Does not work if you do not possess a magical focus.
\item \nameref{sec:WaterBreathing}, as the Water Domain ability (unlimited use)
\item \nameref{Spell:WallOfIce} (once per day)
\item \nameref{Spell:IceStorm} (twice per week)
\item \nameref{Spell:ControlWater} (twice per week)
\end{list}
The ring appears to be a ring of water walking until the established condition is met.

\paragraph{Prerequisites:} \nameref{Feat:ForgeRing}, appropriate elemental summoning spell, all appropriate spells.

\paragraph{Cost to Create:} 100000gp, 8000 XP
\subsubsection{Feather Fall}
\label{Item:FeatherFall}
   \textbf{Price:} 2200gp
\\ \textbf{Body Slot:} Finger
\\ \textbf{Caster Level:} 1st
\\ \textbf{Activation:} -
\\ \textbf{Weight:} -

Whenever the wearer of this ring falls more than 5', this ring protects him as if by a \nameref{Spell:ControlFall} spell. However, the wearer does not gain a bonus on jump checks.

\paragraph{Physical Description:} The ring is crafted with a feather pattern all around its edge.

\paragraph{Prerequisites:} \nameref{Feat:ForgeRing}, \nameref{Spell:ControlFall}

\paragraph{Cost to Create:} 1100gp, 88 XP
\subsubsection{Mind Shielding}
\label{Item:MindShielding}
   \textbf{Price:} 8000gp
\\ \textbf{Body Slot:} Finger
\\ \textbf{Caster Level:} 5th
\\ \textbf{Activation:} -
\\ \textbf{Weight:} -

The wearer of this ring is continually immune to \nameref{Spell:ReadThoughts} and any attempt to magically discern his alignment, and takes only half the normal penalty on Bluff checks from \nameref{Spell:ZoneOfTruth}.

\paragraph{Physical Description:} This ring is usually of fine workmanship and wrought from heavy gold.

\paragraph{Prerequisites:} \nameref{Feat:ForgeRing}, \nameref{Spell:Nondetection}

\paragraph{Cost to Create:} 4000gp, 320 XP
\subsubsection{Ram}
\label{Item:Ram}
   \textbf{Price:} 8600gp
\\ \textbf{Body Slot:} Finger
\\ \textbf{Caster Level:} 5th
\\ \textbf{Activation:} Standard (Command)
\\ \textbf{Weight:} -

The ring of the ram has three charges, which are renewed each day at dawn. It allows its wielder to use \nameref{Spell:ManeuveringHand}, as the spell, except that the force takes the shape of the head of a ram or a goat, and does not allow grapple or disarm attempts. Unlike simpler command-activated items, the power of the ram-like force depends on the number of charges expended at the time of activation:
\begin{list}{\labelitemi}{\leftmargin=1em}
\item \emph{1 charge:} The spell acts as a medium-sized creature with strength 20, for a total strength check modifier of +5.
\item \emph{2 charges:} The spell acts as a large-sized creature with strength 20, for a total strength check modifier of +9.
\item \emph{3 charges:} The spell acts as a huge-sized creature with strength 20, for a total strength check modifier of +13.
\end{list}
The spell lasts for 5 rounds, as normal.

\paragraph{Physical Description:} The ring of the ram is an ornate ring forged of hard metal, usually iron or an iron alloy. It has the head of a ram as its device.

\paragraph{Prerequisites:} \nameref{Feat:ForgeRing}, \nameref{Spell:ManeuveringHand}

\paragraph{Cost to Create:} 4300gp, 344 XP
\subsubsection{Regeneration, Minor}
\label{Item:RegenerationMinor}
   \textbf{Price:} 8000gp
\\ \textbf{Body Slot:} Finger
\\ \textbf{Caster Level:} 1st
\\ \textbf{Activation:} -
\\ \textbf{Weight:} -

While worn, this ring grants its wearer Fast Healing 1.
The ring must be worn for a 24 hours before it begins to work. If it is removed, the owner must wear it for another 24 hours to reattune it to himself.

\paragraph{Physical Description:} This ring is forged of white gold, and feels warm to the touch.

\paragraph{Prerequisites:} \nameref{Feat:ForgeRing}, \nameref{Spell:Regenerate}

\paragraph{Cost to Create:} 4000gp, 320 XP

\subsubsection{Regeneration, Major}
\label{Item:RegenerationMajor}
   \textbf{Price:} 40760gp
\\ \textbf{Body Slot:} Finger
\\ \textbf{Caster Level:} 13th
\\ \textbf{Activation:} - and Standard (Command)
\\ \textbf{Weight:} -

As the Ring of \nameref{Item:RegenerationMinor}. In addition, an attuned wearer can use the \nameref{Spell:Regenerate} spell as a spell-like ability once per day.

\paragraph{Physical Description:} This ring is forged of white gold, and feels warm and tingly to the touch.

\paragraph{Prerequisites:} \nameref{Feat:ForgeRing}, \nameref{Spell:Regenerate}

\paragraph{Cost to Create:} 20380gp, 1630 XP
\subsubsection{Shooting Stars}
\label{Item:ShootingStars}
   \textbf{Price:} 43200gp
\\ \textbf{Body Slot:} Finger
\\ \textbf{Caster Level:} 12th
\\ \textbf{Activation:} Standard (Command), free (mental); see text
\\ \textbf{Weight:} -

At will as a standard action, the wearer of this ring can use \nameref{Spell:MoonBolt}, as the spell. In addition, as a free action, he may cause his the ring to shed light as a torch. He can extinguish the light as another free action.

\paragraph{Physical Description:} The ring is forged of pure silver, inset with a dark stone of unfamiliar material. When the ring's glow function is activated, it is the stone that shines. The light is as pale as starlight rather than reddish yellow like that of an open flame.

\paragraph{Prerequisites:} \nameref{Feat:ForgeRing}, \nameref{Spell:MoonBolt}

\paragraph{Cost to Create:} 21600gp, 1728 XP
\subsubsection{Spell Storing, Minor}
\label{Item:SpellStoringMinor}
   \textbf{Price:} 18000gp
\\ \textbf{Body Slot:} Finger
\\ \textbf{Caster Level:} 5th
\\ \textbf{Activation:} - and Standard (Command)
\\ \textbf{Weight:} -

A minor ring of spell storing contains one or more spells that the wearer can cast.  
The user need not provide any components, or pay an XP cost to cast the spells. 
The activation time for the ring is same as the casting time for the relevant spell, with a minimum of 1 standard action.

A spellcaster can cast any spells into the ring, so long as the total spell point cost does not add up to more than five. 

For a randomly generated ring, treat it as a matrix to determine what spells are stored in it. Each spell has spell points spent on it equal to the minimum needed to cast that spell (no augments).
If you roll a spell that would put the ring over the five-spell point limit, ignore that roll; the ring has no more spells in it. (Not every newly discovered ring need be fully charged.)

A spellcaster can use a matrix to put a spell into the minor ring of spell storing.

The ring magically imparts to the wearer the names of all spells currently stored within it.

%\paragraph{Physical Description:} STUFF

\paragraph{Prerequisites:} \nameref{Feat:ForgeRing}, \nameref{Spell:ImbueWithSpellAbility}

\paragraph{Cost to Create:} 9000gp, 720 XP
\subsubsection{Spell Storing}
\label{Item:SpellStoring}
   \textbf{Price:} 50000gp
\\ \textbf{Caster Level:} 9th

As a \nameref{Item:SpellStoringMinor}, except it holds up to nine spell points.

\paragraph{Cost to Create:} 25000gp, 2000 XP
\subsubsection{Spell Storing, Major}
\label{Item:SpellStoringMajor}
   \textbf{Price:} 200000gp
\\ \textbf{Caster Level:} 17th

As a \nameref{Item:SpellStoringMinor}, except it holds up to nineteen spell points.

\paragraph{Cost to Create:} 100000gp, 8000 XP
\subsubsection{Spell Turning}
\label{Item:SpellTurning}
   \textbf{Price:} 98280gp
\\ \textbf{Body Slot:} Finger
\\ \textbf{Caster Level:} 13th
\\ \textbf{Activation:} Standard (Command)
\\ \textbf{Weight:} -

Three times per day, the bearer of this ring can use \nameref{Spell:SpellTurning}, as the spell.

%\paragraph{Physical Description:} STUFF

\paragraph{Prerequisites:} \nameref{Feat:ForgeRing}, \nameref{Spell:SpellTurning}

\paragraph{Cost to Create:} 49140gp, 3931 XP
\subsubsection{Telekinesis}
\label{Item:Telekinesis}
   \textbf{Price:} 50400gp
\\ \textbf{Body Slot:} Finger
\\ \textbf{Caster Level:} 7th
\\ \textbf{Activation:} Standard (Command)
\\ \textbf{Weight:} -

At will, the bearer of this ring can use \nameref{Spell:Telekinesis}, as the spell.

%\paragraph{Physical Description:} STUFF

\paragraph{Prerequisites:} \nameref{Feat:ForgeRing}, \nameref{Spell:Telekinesis}

\paragraph{Cost to Create:} 25200gp, 2016 XP
\subsubsection{X-ray Vision}
\label{Item:XRayVision}
   \textbf{Price:} 16200gp
\\ \textbf{Body Slot:} Finger
\\ \textbf{Caster Level:} 9th
\\ \textbf{Activation:} Standard (Command)
\\ \textbf{Weight:} -

Once per day, you can use \nameref{Spell:XRayVision}, as the spell.

\paragraph{Physical Description:} The ring is unusually heavy for its size, and inset with a dull, metallic-looking stone. When in dark surroundings, it emits an almost imperceptible green glow.

\paragraph{Prerequisites:} \nameref{Feat:ForgeRing}, \nameref{Spell:XRayVision}

\paragraph{Cost to Create:} 8100gp, 648 XP
\subsection{Rods}
\subsubsection{Absorbtion}
\textbf{Price:} 50000gp\\
\textbf{Body Slot:} -\\
\textbf{Caster Level:} 15th\\
\textbf{Activation:} -\\
\textbf{Weight:} 5lbs.

This rod acts as a magnet, drawing spells or spell-like abilities into itself. 
The magic absorbed must be a single-target spell or a ray directed at either the character possessing the rod or her gear. 
The rod then nullifies the spell's effect and stores its potential until the wielder releases this energy in the form of spells of her own. 
She can instantly detect the number of spell points spent on the spell as the rod absorbs that spell's energy. 
Absorption requires no action on the part of the user if the rod is in hand at the time.

A running total of absorbed (and used) spell points should be kept. 
The wielder of the rod can use captured spell energy to cast any spell he knows.
See \nameref{sec:UsingStoredSpellPoints} for more information.

A rod of absorption absorbs a maximum of 200 spell points and can thereafter only discharge any remaining potential it might have. The rod cannot be recharged. The wielder knows the rod's remaining absorbing potential and current amount of stored energy.

To determine the absorption potential remaining in a newly found rod, roll 2d\%. Then roll d\% again: On a result of 71-100, half the spell points already absorbed by the rod are still stored within.

\paragraph{Prerequisites:} \nameref{Feat:CraftRod}, \nameref{Spell:SpellTurning}

\paragraph{Cost to Create:} 25000gp, 2000 XP

\subsubsection{Alertness}
\textbf{Price:} 60000gp\\
\textbf{Body Slot:} -\\
\textbf{Caster Level:} 11th\\
\textbf{Activation:} -\\
\textbf{Weight:} 5lbs.

This rod has eight flanges on its macelike head. 
The rod bestows a +1 insight bonus on initiative checks. 
If grasped firmly, the rod enables the holder to use \nameref{Spell:DetectMagic}, \nameref{Spell:DiscernAlignment}, \nameref{Spell:ZoneOfTruth}, \nameref{Spell:Light}, or \nameref{Spell:SeeInvisibility}, as the spells, a combined number of five times per day. 

By concentrating on the rod (a standard action), you gain Blindsense out to 120' until the start of your next turn.

\paragraph{Prerequisites:} \nameref{Feat:CraftRod}, \nameref{Spell:SpellTurning}

\paragraph{Cost to Create:} 30000gp, 2400 XP
\subsubsection{Guards and Wards}
\textbf{Price:} 50000gp\\
\textbf{Body Slot:} -\\
\textbf{Caster Level:} 12th\\
\textbf{Activation:} -; see text\\
\textbf{Weight:} 5lbs.

This powerful magic item is primarily used to defend your stronghold.
It takes the form of a magic rod, which can be attuned to a location of a size of up to 2000 square feet (as much as 20 feet high), and shaped as you desire. 
You can ward several stories of a stronghold by dividing the area among them.  
Attuning the item to a location is a process that takes 30 minutes (and requires you be somewhere within the area to be warded).
After it has been attuned, the ward creates the following magical effects within the warded area:
\paragraph{Fog}
Magical fog fills all corridors, obscuring all sight, including darkvision, beyond 5 feet. 
A creature within 5 feet has concealment (attacks have a 20\% miss chance).
Creatures farther away have total concealment (50\% miss chance, and the attacker cannot use sight to locate the target). 
\emph{Saving Throw: None. Spell Resistance: No.}
\paragraph{Arcane Locks}
All doors in the warded area are arcane locked (as the augment of the \nameref{Spell:OpenClose} spell). \emph{Saving Throw: None. Spell Resistance: No.}
\paragraph{Webs}
Webs fill all stairs from top to bottom. These strands are identical with those created by the \nameref{Spell:Web} spell (DC 17), 
except that they regrow in 10 minutes if they are burned or torn away while the guards and wards spell lasts. 
\emph{Saving Throw: see text for web. Spell Resistance: No.}
\paragraph{Confusion}
Where there are choices in direction—such as a corridor intersection or side passage - 
a minor confusion-type effect functions so as to make it 50\% probable that intruders believe they are going in the opposite direction from the one they actually chose. 
This is an Illusion (phantasm), mind-affecting effect. \emph{Saving Throw: None. Spell Resistance: Yes.}
\paragraph{Lost Doors}
12 doors are covered by an \nameref{Spell:Image} spell (DC 16) to appear as if they were a plain wall. \emph{Saving Throw: Will disbelief (if interacted with). Spell Resistance: No.}

In addition, you can place your choice of one of the following five magical effects:
\begin{enumerate}
\item Change in ambient light levels in four corridors. 
You can designate a simple program that causes the light level changes to repeat as long as the guards and wards effect is in place. 
\emph{Saving Throw: None. Spell Resistance: No.}
\item A magic mouth in two places, as per the augment of the \nameref{Spell:Ventriloquism} spell. 
\emph{Saving Throw: None. Spell Resistance: No.}
\item A \nameref{Spell:NoxiousVapors} in two places (no augments in effect, save DC 17). 
The vapors appear in the places you designate; they return within 10 minutes if dispersed by wind while the guards and wards effect is in place. 
\emph{Saving Throw: see text for Noxious Vapors. Spell Resistance: No.}
\item A \nameref{Spell:GustOfWind} in one corridor or room. Save DC 17, fires once per round.
\emph{Saving Throw: see gust of wind. Spell Resistance: Yes.}
\item A \nameref{Spell:Suggestion} in one place. 
You select an area of up to 5 feet square, and any creature who enters or passes through the area receives the suggestion mentally. 
\emph{Saving Throw: Will negates. Spell Resistance: Yes.}
\end{enumerate}
You can redesignate which of the magical effects is active (as well its location) by re-attuning the item to the area.

A \nameref{Spell:DispelMagic} cast on a specific effect, if successful, removes only that effect. A successful \nameref{Spell:Disjunction} destroys the entire guards and wards effect. 

\paragraph{Prerequisites:} \nameref{Feat:CraftRod}, \nameref{Spell:Light}, \nameref{Spell:Fog}, \nameref{Spell:OpenClose}, \nameref{Spell:Image}, \nameref{Spell:Web}, \nameref{Spell:HallOfMirrors},\nameref{Spell:Ventriloquism}, \nameref{Spell:NoxiousVapors}, \nameref{Spell:GustOfWind}, \nameref{Spell:Suggestion}, \nameref{sec:Cantrips} class feature.

\paragraph{Cost to Create:} 25000gp, 2000 XP
\subsubsection{Negation}
\textbf{Price:} 37000gp\\
\textbf{Body Slot:} -\\
\textbf{Caster Level:} 15th\\
\textbf{Activation:} Standard (Command)\\
\textbf{Weight:} 5lbs.

This device negates the spell or spell-like function or functions of magic items. 
The wielder points the rod at the magic item, and a pale gray beam shoots forth to touch the target device, attacking as a ray (a ranged touch attack). 
The ray functions as a \nameref{Spell:DispelMagic} spell, except it only affects magic items. 
To negate instantaneous effects from an item, the rod wielder needs to have used a ready action. 
The dispel check uses the rod's caster level (15th) and is augmented to cost 14 spell points, for a total bonus of +21. 
The target item gets no saving throw, although the rod can't negate artifacts (even minor artifacts). 
The rod can function three times per day.

\paragraph{Prerequisites:} \nameref{Feat:CraftRod}, and \nameref{Spell:LimitedWish} or \nameref{Spell:Miracle}

\paragraph{Cost to Create:} 18500gp, 1480 XP
\subsubsection{Wonder}
\textbf{Price:} 12000gp\\
\textbf{Body Slot:} -\\
\textbf{Caster Level:} 10th\\
\textbf{Activation:} Standard (Command)\\
\textbf{Weight:} 5lbs.

A rod of wonder is a strange and unpredictable device that randomly generates any number of weird effects each time it is used.  Typical powers of the rod are shown on the \nameref{tab:RodOfWonder} table.

If the rod's effect can't take place for some reason (such as due to the rod being pointed at an object when the effect requires targeting a creature, or the creature the rod is pointed at is out of range), the activation fails.

\begin{table*}
\caption{Rod of Wonder}
\label{tab:RodOfWonder}
\centering
\begin{tabular}{|c|p{0.92\textwidth}|}
\hline
\textbf{d\%}&\textbf{Wondrous Effect}\\
\hline
01-05&\nameref{Spell:Slow} creature pointed at for 10 rounds (Will DC 16 negates).\\
06-10&\nameref{Spell:FaerieFire}, centered on the target.\\
11-15&Deludes wielder for 1 round into believing the rod functions as indicated by a second die roll (no save).\\
16-20&\nameref{Spell:GustOfWind} in a line emanating out from the rod (strength DC 16).\\
21-25&Wielder learns target's surface thoughts (as with \nameref{Spell:ReadThoughts}) for 1d4 rounds (no save).\\
26-30&\nameref{Spell:NoxiousVapors} centered on target (Fortitude DC 16).\\
31-33&Heavy rain falls for 1 round in 60-ft. radius centered on rod wielder.\\
34-36&Summon an animal - a rhino (01-25 on d\%), elephant (26-50), or mouse (51-100). The animal disappears after one minute.\\
37-46&Lightning bolt (70 ft. long, 5 ft. wide), 10d6 damage (Reflex DC 17 for half damage).\\
47-49&Stream of 600 large butterflies pours forth and flutters around for 2 rounds, blinding everyone (including wielder) within 25 ft. (Reflex DC 14 negates blindness).\\
50-53&Creature pointed at grows in size, as if by the \nameref{Spell:AlterSize} spell with the fourth augment (Fortitude DC 15 negates).\\
54-58&\nameref{Spell:Darkness} centered on target.\\
59-62&Grass grows in 160-sq.-ft. area before the rod, or grass existing there grows to ten times normal size.\\
63-65&Turn ethereal any nonliving object of up to 1,000 lb. mass and up to 30 cu. ft. in size for one hour.\\
66-69&Wielder is reduced in size, as if by the \nameref{Spell:AlterSize} spell with the fourth augment (Fortitude DC 15 negates).\\
70-79&\nameref{Spell:Fireball} at target, fire version (Reflex DC 17 half).\\
80-84&\nameref{Spell:Invisibility} covers rod wielder.\\
85-87&Leaves grow from target if within 60 ft. of rod. These last 24 hours.\\
88-90&10-40 gems, value 1 gp each, shoot forth in a 30-ft.-long line. Each gem deals 1 point of damage to any creature in its path: Roll 5d4 for the number of hits and divide them among the available targets.\\
91-95&Shimmering colors dance and play over a 40-ft.-by-30-ft. area in front of rod. Creatures therein are blinded for 1d6 rounds (Fortitude DC 15 negates).\\
96-97&Wielder (50\% chance) or target (50\% chance) turns permanently blue, green, or purple (no save). A \nameref{Spell:RemoveCurse} spell can restore the creature's original color.\\
98-100&\nameref{Spell:TransmuteFleshAndStone} on target, \emph{Flesh to Stone} if the target is a creature made of flesh, \emph{Stone to Flesh} if the target is an object made of stone or a creature that is already petrified.\\
\hline
\end{tabular}
\end{table*}
The powers of the rod are Spell-like abilities if they duplicate the effects of spells, Supernatural abilities otherwise.

\paragraph{Prerequisites:} \nameref{Feat:CraftRod}, \nameref{Spell:Confusion}, creator must be chaotic; 

\paragraph{Cost to Create:} 6000gp, 480 XP
\subsection{Wands}
A wand is a slender piece of wood that contains a single spell. Each wand has 50
charges when created, and each charge expended allows one use of that spell. A
wand that runs out of charges is just a simple stick.

\paragraph{Physical Description:} A typical wand is a piece of wood,
between 8 inches and 10 inches long and about 1/2 inch thick, which
can weigh up to 1/4 pound. Occasionally, a wand is decorated with carvings or
inscribed runes. A typical wand has AC 7, 7 hit
points, a hardness of 8, and a break DC of 18.

\paragraph{Activation:} Wands use the spell trigger activation method, so casting a
spell from a wand is usually a standard action that does not provoke attacks of
opportunity. (If the spell being cast has a casting time longer than 1
standard action, however, it takes that long to cast the spell from a
wand.) The user must have the spell on his class list, even if he knows the
command word. Additionally, to activate a wand, a character must hold it in
hand and point it in the general direction of the target or area to be affected.
Wands are normally created at the minimum caster level required to cast
the spell, and spells that can be augmented are not augmented when stored in a
wand. A wand's wielder cannot augment the spell contained within the wand.
However, wands can be created at a higher caster level than required to
cast the spell. In this case, the wand that holds an augmentable spell is
augmented, to the limit of the caster level and the spell's augmentation
maximums, if any. The caster level of a wand cannot be more than five
higher than the minimum caster level to use the spell it contains.

\begin{table*}
\caption{Wands}
\label{tab:Wands}
\centering
\begin{tabular}{|c|r|}
\hline
\textbf{Spell Level}$^1$&\textbf{Market Price}$^2$\\
\hline
1st&750gp\\
2nd&4500gp\\
3rd&11250gp\\
4th&21000gp\\
5th&33750gp\\
6th&49500gp\\
7th&68250gp\\
8th&90000gp\\
9th&114750gp\\
\hline
\end{tabular}
\scriptsize
\begin{enumerate}
 \item Some wands have higher caster levels than the minimum spell level, which gives them commensurately higher costs.
 \item Any wand that stores a spell with an experience point cost also has an XP cost in addition to that noted here.
\end{enumerate}
\end{table*}
\subsection{Wondrous Items}
\subsubsection{Amulet of the Planes}
   \textbf{Price:} 35000gp
\\ \textbf{Body Slot:} Throat
\\ \textbf{Caster Level:} 17th
\\ \textbf{Activation:} Standard (Command)
\\ \textbf{Weight:} -

Once per day, the amulet allows its wearer to utilize the augmented version of \nameref{Spell:PlaneShift}.
By speaking a command word, the amulet takes its wearer (and his companions, as described in the \nameref{Spell:PlaneShift} spell description) to the specific location on any plane that he wants. 
However, this is a difficult item to master. The user must make a DC 15 Intelligence check in order to get 
If she fails, the amulet transports her and all those traveling with her to the correct plane, but as if with an unaugmented \nameref{Spell:PlaneShift} or to a random plane (61-100).

\paragraph{Physical Description:} This device usually appears to be a black circular amulet, although any character looking closely at it sees a dark, moving swirl of color.

\paragraph{Activation:} Standard (Command).

\paragraph{Prerequisites:} \nameref{Feat:CraftWondrousItem}, \nameref{Spell:PlaneShift}

\paragraph{Cost to Create:} 17500gp, 1400 XP
\subsubsection{Amulet of Proof against Detection and Location}
   \textbf{Price:} 15000gp (lesser); 30000gp (standard); 60000gp (greater)
\\ \textbf{Body Slot:} Throat
\\ \textbf{Caster Level:} 5th (lesser); 10th (standard); 15th (greater)
\\ \textbf{Activation:} -
\\ \textbf{Weight:} -

This amulet protects the wearer from scrying and magical location just as a \nameref{Spell:Nondetection} spell does. If a divination (Scrying) spell is attempted against the wearer, the caster of the divination must succeed on a caster level check (1d20 + caster level), as described in the \nameref{Spell:Nondetection} spell entry. The DC of the check varies according to the version of the amulet selected, DC 16 for the lesser version, DC 26 for the normal version, and DC 36 for the greater version (the increased DC is due to the standard and lesser versions duplicating augmented forms of the \nameref{Spell:Nondetection} spell).

\paragraph{Physical Description:} A silver amulet.

\paragraph{Activation:} None. The amulet provides its benefits continuously while worn, no activation required.

\paragraph{Prerequisites:} \nameref{Feat:CraftWondrousItem}, \nameref{Spell:Nondetection}

\paragraph{Cost to Create:} 7500gp, 600XP (lesser); 15000gp, 1200XP (standard); 30000gp, 2400XP (greater)

\subsubsection{Bead of Force}
   \textbf{Price:} 1400gp
\\ \textbf{Body Slot:} -
\\ \textbf{Caster Level:} 7th
\\ \textbf{Activation:} Use-activated
\\ \textbf{Weight:} -

You use this bead as you would use a thrown weapon with which you are proficient, with a range increment of 20'. If you succeed on a ranged touch attack against a medium-sized or smaller creature, it must succeed on a reflex save (DC 17) or be trapped inside a \nameref{Spell:ResilientSphere}, as the spell.
If you miss, or the creature is too large or otherwise incapable of being trapped inside a Resilient Sphere, the bead explodes, dealing 5d6 points of force damage to all creatures within a 10' radius burst (center of effect chosen by you, although the targeted creature must be within the burst). You may miss intentionally.

The bead is completely destroyed upon impact, making this a one-use item.

Moderate evocation; CL 10th; Craft Wondrous Item, resilient sphere; Price 3,000 gp.

\paragraph{Physical Description:} This small black sphere appears to be a lusterless pearl.

\paragraph{Activation:} Use-activated. Throw as you would a thrown weapon.

\paragraph{Prerequisites:} \nameref{Feat:CraftWondrousItem}, \nameref{Spell:ResilientSphere}

\paragraph{Cost to Create:} 700gp, 56XP
\subsubsection{Magical Restraints}
\textbf{Price:} See the \nameref{tab:MagicalRestraints} table\\
\textbf{Body Slot:} -\\
\textbf{Caster Level:} 16th\\
\textbf{Activation:} -\\
\textbf{Weight:} 1 lb.

 These restraints limit the total number of spell points a magical creature wearing it can use in 1 round (regardless of the creature's total spell point reserve), or completely damps the ability to use magic. 
All types of magical restraints prevent the free casting of spells, such as by Spell-Like abilities. Supernatural abilities are not affected.

\paragraph{Physical Description:} Each of the various magical restraints is an iron cuff that cunningly locks around the wrist. Opening it without the key requires a DC 27 Open Lock check. The break and escape artist DCs, as well as the restraints' hit points and hardness are the same as those of masterwork manacles. The restraints are ineffective on creatures that do not have arms or armlike appendages.

\paragraph{Activation:} None. The restraints function continuously while worn.

\begin{table*}
\caption{Magical Restraints}
\label{tab:MagicalRestraints}
\centering
\begin{tabular}{|c|c|r|r|}
\hline
\textbf{Restraint Type}&\textbf{Allowed}&\textbf{Market Price}&\textbf{Cost to Create}\\
&\textbf{ SP/round}&&\\
\hline
Lesser&5&1000gp&500gp, 40XP\\
Average&3&6000gp&3000gp, 240XP\\
Greater&1&12000gp&6000gp, 480XP\\
Damping&0&24000gp&12000gp, 960XP\\
\hline
\end{tabular}
\end{table*}

\paragraph{Prerequisites:} \nameref{Feat:CraftWondrousItem}, \nameref{Spell:LimitedWish}, \nameref{Spell:DispelMagic}

\paragraph{Cost to Create:} See the \nameref{tab:MagicalRestraints} table.


\subsubsection{Pearl of Power}
\label{Item:PearlOfPower}
\textbf{Price:} See the \nameref{tab:PearlsOfPower} table.\\
\textbf{Body Slot:} -\\
\textbf{Caster Level:} Equal to maximum spell point storage\\
\textbf{Activation:} -; see text\\
\textbf{Weight:} -

Pearls of Power store spell points that spellcasting characters can use to pay
for casting their spells.

\paragraph{Physical Description:} This is a pearl of average size.
It looks normal, except for a faint glow (which is insufficient to provide real
illumination). It has negligible weight, has AC 7, 10 hit points, a hardness of
8, and a break DC of 16.

\paragraph{Activation:} The user must merely hold or have a pearl on her person for a
period of at least 10 minutes (which is long enough to attune oneself to the
pearl). Thereafter, the owner can use spell points stored in the pearl to
cast spells she knows.
The maximum number of points a pearl of power can store is always an odd
number and is never more than 17. It can store only as many spell points as its
original maximum, set at the time of its creation. When a pearl of power's
spell points are used up, the glow of the pearl dims. 
However, the user can
recharge it by paying spell points on a 1-for-1 basis. While doing this takes
from the user's own spell point reserve for the day, those spell points remain
available in the pearl of power until used.

A user cannot directly replenish her personal spell points from those stored in
a pearl of power, nor can he draw spell points from more than one source to
cast a spell. See \nameref{sec:UsingStoredSpellPoints} for more information.

\begin{table*}
\caption{Pearls of Power}
\label{tab:PearlsOfPower}
\centering
\begin{tabular}{|c|r|r|}
\hline
\textbf{Maximum SP Storage}&\textbf{Market Price}&\textbf{Cost to Create}\\
\hline
1&	1000gp	&500gp, 40XP\\
3&	4000gp	&2000gp, 160XP\\
5&	9000gp	&4500gp, 360XP\\
7&	16000gp	&8000gp, 640XP\\
9&	25000gp	&12500gp, 1000XP\\
11&	36000gp	&18000gp, 1440XP\\
13&	49000gp	&24500gp, 1960XP\\
15&	64000gp	&32000gp, 2560XP\\
17&	81000gp	&40500gp, 3240XP\\
\hline
\end{tabular}
\end{table*}

\paragraph{Prerequisites:} \nameref{Feat:CraftWondrousItem}

\paragraph{Cost to Create:} See the \nameref{tab:PearlsOfPower} table.
\subsubsection{Spell Focus}
\textbf{Price:} 8000gp\\
\textbf{Body Slot:} Throat\\
\textbf{Caster Level:} 8th\\
\textbf{Activation:} -\\
\textbf{Weight:} -

Every school of magic (Abjuration, Conjuration, Divination, Enchantment, Evocation, Illusion, Necromancy and Transmutation) has a type of spell focus associated with it. This focus is an item worn around the neck, and wearing one adds a +1 enhancement bonus to the save DCs of spells of the corresponding school.

\paragraph{Physical Description:} Typical spell focuses are unobtrusive ornaments.

\paragraph{Activation:} None. A spell focus provides its benefit continuously, no activation required.

\paragraph{Prerequisites:} \nameref{Feat:CraftWondrousItem}, creator must be a Specialist Wizard in the relevant school of magic.

\paragraph{Cost to Create:} 4000gp, 320 XP
\subsubsection{Symbol}
\textbf{Price:} See the \nameref{tab:Symbols} table\\
\textbf{Body Slot:} -\\
\textbf{Caster Level:} Varies; see the \nameref{tab:Symbols} table\\
\textbf{Activation:} When triggered; see text\\
\textbf{Weight:} -

A symbol is a potent rune of power scribed upon a surface. 
When triggered, a symbol has a particular, harmful effect on one or more creatures within 60 feet of the symbol (treat as a burst).
The symbol affects the closest creatures first.%, skipping creatures with too many hit points to affect. 
Once triggered, a symbol becomes active and glows, 
lasting for 10 minutes per caster level, %or until it has affected 150 hit points' worth of creatures, whichever comes first.
after which it is burned out and useless.
Any creature that enters the area while the symbol is active is subject to its effect, 
whether or not that creature was in the area when it was triggered. 
A creature need save against the symbol only once as long as it remains within the area, 
though if it leaves the area and returns while the symbol is still active, it must save again.

Until it is triggered, the symbol of is inactive (though visible and legible at a distance of 60 feet). 
To be effective, a symbol must always be placed in plain sight and in a prominent location. 
Covering or hiding the rune renders the symbol ineffective, unless a creature removes the covering, in which case the symbol works normally.

As a default, a symbol is triggered whenever a creature does one or more of the following, as you select: 
looks at the rune; reads the rune; touches the rune; passes over the rune; or passes through a portal bearing the rune. 
Regardless of the trigger method or methods chosen, a creature more than 60 feet from a symbol can't trigger it 
(even if it meets one or more of the triggering conditions, such as reading the rune). 
Once a symbol is created, its triggering conditions cannot be changed.

In this case, ``reading`` the rune means any attempt to study it, identify it, or fathom its meaning. 
Throwing a cover over a symbol to render it inoperative triggers it if the symbol reacts to touch. 
% You can't use a symbol of death offensively; 
% for instance, a touch-triggered symbol remains untriggered if an item bearing the symbol of death is used to touch a creature. 
% Likewise, a symbol cannot be placed on a weapon and set to activate when the weapon strikes a foe.

You can also set special triggering limitations of your own. 
These can be as simple or elaborate as you desire. 
Special conditions for triggering a symbol can be based on a creature's name, identity, or alignment, 
but otherwise must be based on observable actions or qualities. 
Intangibles such as level, class, Hit Dice, and hit points don't qualify.

When scribing a symbol, you can specify a password or phrase that prevents a creature using it from triggering the effect. 
Anyone using the password remains immune to that particular rune's effects so long as the creature remains within 60 feet of the rune. 
If the creature leaves the radius and returns later, it must use the password again.

You also can attune any number of creatures to the symbol. %, but doing this can extend the casting time. 
These creatures must either be present at the time you scribe the symbol, or you must have some way of unambiguously identifying them
for the purpose of this magic item.
% Attuning one or two creatures takes negligible time, and attuning a small group (as many as ten creatures) extends the casting time to 1 hour. 
% Attuning a large group (as many as twenty-five creatures) takes 24 hours. Attuning larger groups takes proportionately longer. 
Any creature attuned to a symbol cannot trigger it and is immune to its effects, even if within its radius when triggered. 
You are automatically considered attuned to your own symbols, and thus always ignore the effects and cannot inadvertently trigger them.

% Read magic allows you to identify a symbol of death with a DC 19 Spellcraft check. 
% Of course, if the symbol of death is set to be triggered by reading it, this will trigger the symbol.

A symbol can be destroyed by a successful dispel magic targeted solely on the rune. 
%An erase spell has no effect on a symbol of death.
Destruction of the surface where a symbol is inscribed destroys the symbol but also triggers it. 

The effects of each symbol is detailed below. See also \nameref{tab:Symbols}.

\begin{list}{\labelitemi}{\leftmargin=1em}
 \item \textbf{Symbol of Fire:} The viewer bursts into flames, taking 7d6+7 points of fire damage.
 A successful reflex save halves the damage.
 \item \textbf{Symbol of Sleep:} All viewers of 10 HD or less fall into catatonic slumber for
 3d6$\times$10 minutes. Unlike with the sleep spell, sleeping creatures cannot be awakened by nonmagical means before this time expires.
 A successful will save negates the unconsciousness.
 \item \textbf{Symbol of Pain:} The viewer suffers wracking pains that impose a -4 penalty on attack rolls, 
 skill checks, and ability checks. These effects last for 1 hour after the creature moves farther than 60 feet from the symbol.
 A successful fortitude save reduces the penalty to -2.
 \item \textbf{Symbol of Persuasion:} The viewer becomes charmed by the caster, as if subjected to a version of the \nameref{Spell:Charm}
 spell that can affect any type of creature.
 A successful Will Save negates the charm.
 \item \textbf{Symbol of Fear:} The viewer becomes panicked for one minute. A successful will save negates the fear.
 \item \textbf{Symbol of Stunning:} The viewer is stunned for 1d6 rounds. A successful will save negates the stun.
 \item \textbf{Symbol of Weakness:} The viewer takes 3d6 points of strength damage. A successful fortitude save negates the ability damage.
 \item \textbf{Symbol of Insanity:} The viewer is rendered permanently confused, as if by an augmented Confusion spell. A successful will save negates the insanity.
 \item \textbf{Symbol of Death:} The viewer dies. This is a [death] effect. A successful fortitude save negates the death effect.
\end{list}

\begin{table*}
\caption{Symbols}
\label{tab:Symbols}
\centering
\makebox[\textwidth]{
\begin{tabular}{|l|l|l|c|c|l|}
\hline
\textbf{Symbol}&\textbf{Market}&\textbf{Cost to}&\textbf{Save DC}&\textbf{Caster}&\textbf{Associated}\\
&\textbf{Price}&\textbf{Create}&&\textbf{Level}&\textbf{Spell}\\
\hline
Fire &		100gp	&50gp, 4XP	&19&7&Fireball\\
Sleep&		1000gp	&500gp, 40XP	&22&9&Sleep\\
Pain&		1000gp	&500gp, 40XP	&22&9&Crushing Despair\\
Persuasion&	5000gp	&2500gp, 200XP	&24&11&Charm\\
Fear&		1000gp	&500gp, 40XP	&24&11&Fear\\
Stunning&	5000gp	&2500gp, 200XP	&25&13&Daze\\
Weakness&	5000gp	&2500gp, 200XP	&25&13&Ray of Enfeeblement\\
Insanity&	5000gp	&2500gp, 200XP	&27&15&Confusion\\
Death&		5000gp	&2500gp, 200XP	&27&15&Finger of Death\\
\hline
\end{tabular}}
\end{table*}
\emph{Note:} Symbols are a form of magic traps. 
A character with the Trapfinding class feature can use the Search skill to find a symbol and Disable Device to thwart it. 
The DC in each case is 25 + 1/2 the symbol's caster level. 

\paragraph{Prerequisites:} \nameref{Feat:CraftWondrousItem}, associated spell.

\paragraph{Cost to Create:} 4000gp, 320 XP
% \subsubsection{Torc of Spell Preservation}
% \label{Item:TorcOfSpellPreservation}
%    \textbf{Price:} 4000gp
% \\ \textbf{Body Slot:} Throat
% \\ \textbf{Caster Level:} 8th
% \\ \textbf{Activation:} -
% \\ \textbf{Weight:} -
% 
% Five times per day, you can cast a spell by paying spell points equal to the standard cost minus 1 (minimum 1).
% 
% Your caster level must still be high enough to pay the unmodified spell point cost.
% 
% \paragraph{Physical Description:} This item is a band inlaid with precious metal, worn around the neck or upper arm. This
% choice does not affect the body slot the torc occupies.
% 
% \paragraph{Activation:} Activated as part of casting a spell, no action required.
% 
% \paragraph{Prerequisites:} \nameref{Feat:CraftWondrousItem}, \nameref{Spell:LimitedWish}
% 
% \paragraph{Cost to Create:} 2000gp, 160 XP
\subsubsection{Tome of Inherent Improvement}
\textbf{Price:} 27,500 gp (+1), 55,000 gp (+2), 82,500 gp (+3), 110,000 gp (+4), 137,500 gp (+5)\\
\textbf{Body Slot:} -\\
\textbf{Caster Level:} 17th\\
\textbf{Activation:} See text\\
\textbf{Weight:} 5lbs

This book contains instruction on a specific form of self-improvement, but entwined within the words is a powerful magical effect. Each individual Tome is associated with one of the six ability scores. Anyone reading the book gains an inherent bonus of from +1 to +5 (depending on the type of tome) to the Tome's associated ability score. Alternatively, it can increase an existing inherent bonus to the associated ability score by the same amount, but never to more than a total of +5.

Once the book is read, the magic disappears from the pages and it becomes a normal book.

Tomes of Inherent Improvement are usually referred to by the names of their associated ability scores: A Tome associated with Strength is referred to as a Manual of Gainful Exercise, one associated with Dexterity is referred to as a Manual of Quickness of Action, with Constitution as Manual of Bodily Health, with Intelligence as Tome of clear thought, with Wisdom as Tome of understanding, and those associated with Charisma as Tome of leadership and influence.

\paragraph{Physical Description:} Tomes of Inherent Improvement are simply large, heavy tomes. The inexpert eye is likely to miss one if hidden in library shelves.

\paragraph{Activation:} Reading a Tome of Inherent Improvement takes a total of 48 hours over a minimum of six days.

\paragraph{Prerequisites:} \nameref{Feat:CraftWondrousItem}, \nameref{Spell:Miracle} or \nameref{Spell:Wish}.

\paragraph{Cost to Create:} 1,250 gp + 5,100 XP (+1), 2,500 gp + 10,200 XP (+2), 3,750 gp + 15,300 XP (+3), 5,000 gp + 20,400 XP (+4), 6,250 gp + 25,500 XP (+5);
\subsection{Where's my favorite item?}
\label{sec:MissingItems}
To deal with the changes in the spell system (and for general ease of use), the item categories have been shuffled around.
\begin{list}{\labelitemi}{\leftmargin=1em}
 \item \textbf{Magic Arms and Armor} work mostly as before.
 \item \textbf{Potions and Oils} have been folded into a new category of items called \emph{Matrices}, which also includes all single-use spell completion items.
 \item \textbf{Rings} have been individually updated as appropriate. The Rings of Wizardry have been removed due to incompatability reasons.
 \item \textbf{Rods} are mostly unchanged. However, metamagic rods will not be implemented, as they are a form of free metamagic. The Rod of Lordly Might and Rod of Thunder and Lightning have not been updated at this point, due to being supremely useless anyway.
 \item \textbf{Scrolls} now serve as a way for spellcasters to increase their repertoire of spells known, rather than being one-use spell completion items.
 \item \textbf{Staffs} no longer exist in their prior form. However, see Spellstaffs.
 \item \textbf{Wondrous Items} have received individual conversions. Those that directly emulate radically changed spells have been rewritten (or will be). In addition, the following items have been removed:
 \begin{list}{\labelitemii}{\leftmargin=1em}
  \item Blessed Book. Does not exist due to Wizards no longer using spellbooks.
  \item Candle of Invocation. Does not exist. Caster level increases such as those granted by the Candle of Invocation should not exist under this system.
  %\item Does not exist (pending rewrite).
  \item Ioun Stones. As normal, but Orange Prism Ioun Stone do not exist. Caster level increases such as those granted by the Orange Prism should not exist under this system.
  \item Prayer Beads. As normal, but Bead of Karma does not exist. Caster level increases such as those granted by the Bead of Karma should not exist under this system.
  \item 
 \end{list}
\end{list}\newpage

\part{New Content}
The previous chapters have dealt with converting what the \href{http://www.wizards.com/default.asp?x=d20/article/srd35}{d20 srd} has to offer - the next ones are devoted to content that is entirely the author's own. While some new content has crept in already, the point of those additions were to replace or augment existing options. The following content is what would, in a conventional distribution scheme, be delegated to a new splatbook. 
\section{New Spellcasting Classes}
%\subsection[Dread Knight]{The Dread Knight}
\begin{quote}
\emph{``You do not know pain! Not yet.''}
- [NAME], Minotaur Dread Knight
\end{quote}

Dread Knights are spellcasters born with the magical spark in them - but in a dark and twisted form. Some embrace it, becoming merciless killers.
Others try to turn their powers towards good, becoming dark avengers of sorts.
The thing all Dread Knights have in common is that when one shows up, things are about to go horribly wrong for someone.
\paragraph{Alignment:} Any nongood
\paragraph{Hit Die:} d10
\paragraph{Class skills:}
The Dread Knight's class skills (and the key ability for each skill) are Bluff (Cha), Concentration (Con), Craft (Int), Disguise (Cha), Forgery (Cha), Gather Information (Cha), Intimidate (Cha), Hide (Dex) Knowledge (Arcana) (Int), Knowledge (local) (Int), Knowledge (religion) (Int), Knowledge (the planes) (Int), Move Silently (Dex), Profession (Wis), Ride (Dex), Sense Motive (Wis), and Spellcraft (Int).

\paragraph{Skill Points at 1st Level:} (4 + Int modifier) $\times$ 4.
\paragraph{Skill Points at each additional Level:} 4 + Int modifier.
\begin{table*}
\caption{The Dread Knight}
\label{tab:DreadKnight}
\makebox[\textwidth]{
\begin{tabular}{llccclccc}
\toprule
	&	&	&	&	&				&\multicolumn{3}{c}{Spellcasting}\\ \cmidrule(r){7-9}
Level	&BAB	&Fort 	&Ref 	&Will 	&Special			&SP/day	&Known&Max level\\
\midrule
1st &+1			&+2 &+0 &+2	&Dread Knight's Curse,		&0 &1 &1st\\
    &			&&&		&Dark Grace			&&&\\
2nd &+2 		&+3 &+0 &+3 	&Dark Power 			&1 &2 &1st\\
3rd &+3 		&+3 &+1 &+3 	&-    				&3 &3 &1st\\
4th &+4 		&+4 &+1 &+4 	&-    				&5 &4 &2nd\\
5th &+5 		&+4 &+1 &+4 	&Dark Power   			&7 &5 &2nd\\
6th &+6/+1 		&+5 &+2 &+5 	&-    				&11 &6 &2nd\\
7th &+7/+2 		&+5 &+2 &+5 	&-    				&15 &7 &3rd\\
8th &+8/+3 		&+6 &+2 &+6 	&Dark Power   			&19 &8 &3rd\\
9th &+9/+4 		&+6 &+3 &+6 	&-    				&23 &9 &3rd\\
10th &+10/+5		&+7 &+3 &+7 	&-    				&27 &10 &4th\\
11th &+11/+6/+1		&+7 &+3 &+7 	&Dark Power   			&35 &11 &4th\\
12th &+12/+7/+2 	&+8 &+4 &+8 	&-    				&43 &12 &4th\\
13th &+13/+8/+3 	&+8 &+4 &+8 	&-    				&51 &13 &5th\\
14th &+14/+9/+4 	&+9 &+4 &+9 	&Dark Power   			&59 &14 &5th\\
15th &+15/+10/+5	&+9 &+5 &+9 	&-    				&67 &15 &5th\\
16th &+16/+11/+6/+1 	&+10 &+5 &+10 	&-    				&79 &16 &6th\\
17th &+17/+12/+7/+2 	&+10 &+5 &+10 	&Dark Power   			&91 &17 &6th\\
18th &+18/+13/+8/+3 	&+11 &+6 &+11 	&-    				&103 &18 &6th\\
19th &+19/+14/+9/+4 	&+11 &+6 &+11 	&-    				&115 &19 &6th\\
20th &+20/+15/+10/+5	&+12 &+6 &+12 	&Dark Power   			&127 &20 &6th\\
\bottomrule
\end{tabular}}
\end{table*}
\subsubsection{Class Features}
All the following are class features of the Dread Knight.

\paragraph{Weapon and Armor Proficiency:} 
Dread Knights are proficient with all simple and martial weapons, with light and medium armor, and with shields (except tower shields and exotic shields).

\paragraph{Spell Points/Day:} A Dread Knight's ability to cast spells is limited by the spell points he has available. 
His base daily allotment of spell points is given on \nameref{tab:DreadKnight} table. 
In addition, he receives bonus spell points per day if he has a high Charisma score.
His race may also provide bonus spell points per day, as may certain feats and items.

\paragraph{Spells Known:} A Dread Knight begins play knowing one Dread Knight spell of your choice. 
Each time he achieves a new level, he unlocks the knowledge of new spells.
Choose the spells known from the Dread Knight spell list (Exception: The feats Expanded Knowledge and Epic Expanded Knowledge do allow a Dread Knight to learn spells of other classes, even specialist Wizard spells.).
A Dread Knight can cast any spell he knows that has a spell point cost equal to or lower than his caster level.
The number of times a Dread Knight can cast spells in a day is limited only by his daily spell points. 
A Dread Knight simply knows his spells; they are ingrained in his mind, though he must get a good night's sleep each day to regain all his spent spell points.
The Difficulty Class for saving throws against Dread Knight spells is 10 + one-half the number of spell points spent on the spell (round up) + the Dread Knight's Charisma modifier. 

Spells learned via the Dread Knight class are arcane spells.

\paragraph{Maximum Spell Level Known:} A Dread Knight begins play with the ability to learn 1st-level spells. 
As he attains higher levels, a Dread Knight may gain the ability to master more complex spells, as shown on the \nameref{tab:DreadKnight} table.
To learn or cast a spell, a Dread Knight must have a Charisma score of at least 10 + the spell's level.

\paragraph[Dread Knight's Curse]{Dread Knight's Curse: (Su)}
\label{sec:DreadKnightsCurse}
By expending your magical focus as a swift action, you can call upon dark powers to target one or more creatures within 30' with a curse.
The subjects must succeed on a Will save (DC 10 + $1/2$ your character level + your Charisma modifier) or take a -2 penalty on all attack rolls, saving throws, ability checks and skill checks for one hour.
\footnote{Several of the Dread Knight's class features and available spells refer to ``Cursed'' targets. A Cursed target is any creature affected by the Dread Knight's Curse ability, or any creature under the effect of a spell with the [\nameref{sec:Curses}] descriptor.}
% any creature affected by one of the following spells: \nameref{Spell:Bane}, \nameref{Spell:BestowCurse},\nameref{Spell:Blindness}, \nameref{Spell:CrushingOnsetOfAge}, \nameref{Spell:CursedBlade}, \nameref{Spell:DenialOfDeathsEmbrace}, \nameref{Spell:Fear}, \nameref{Spell:InexorableOnsetOfDeath} and \nameref{Spell:Pox}.}

A \nameref{Spell:RemoveCurse} spell can prematurely end this curse.

\paragraph{Dark Grace: (Su)} A Dread Knight gains a bonus on all saving throws equal to his Charisma modifier or his Dread Knight level, whichever is lower.
This is a Supernatural ability that functions continuously, requiring no activation.

\paragraph{Dark Power:}
At 2nd level, the dark heritage of a Dread Knight manifests itself in a gift of new powers. He gains an additional power at Dread Knight levels 5th, 8th, 11th, 17th, and 20th. Unless otherwise noted, a power can only be selected once. Some powers have specific prerequisites.

\subparagraph{Additional Feat:}
You gain a bonus feat from the list of feats noted as Fighter bonus feats. You may also select the \nameref{Feat:Familiar} feat. You must meet the feat's prerequisites, if any.
This power may be selected more than once.

\subparagraph{Aura of Unluck (Su):} 
By expending his magical focus as a free action, a Dread Knight can create a tangible aura of bad luck to foil his enemies. 

Any melee or ranged attack made against the Dread Knight while this aura of unluck is active has a 20\% miss chance (as if the Dread Knight were concealed). The aura lasts for a number of rounds equal to 3 + the Dread Knight's Charisma bonus (if any).

\subparagraph{Bastard's Strike (Su):}
You can automatically confirm all critical threats you roll against Cursed creatures.

This is a supernatural ability that requires no special activation.

\subparagraph{Bestow Curse (Sp):}
You can use \nameref{Spell:BestowCurse} at will as a standard action. Caster level equals your number of Hit Dice.

You must be a 5th-level Dread Knight to select this power.
\subparagraph{Bleeding Edge (Su):}
You can inflict a wound cursed to bleed incessantly. 

Whenever you make a successful weapon attack (including natural attacks and ranged attacks, but not spells, spell-like abilities, or supernatural abilities) against a Cursed creature, it must make a Fortitude save (DC 10 + $1/2$ your HD + your Charisma modifier). On a failed save, the target takes one point of Constitution damage from bleeding.

If it starts bleeding, the save must be repeated once per round thereafter (with the appropriate penalties on failure) until the target receives magical healing that restores hit point damage or ceases to be Cursed.

Striking a creature that is already bleeding immediately enforces a new save.

\subparagraph{Debilitating Curse (Su):}
The penalties imposed by your Dread Knight's Curse increase to -8.

You must be a 17th-level Dread Knight and have the \nameref{sec:PowerfulCurse} power in order to select this power.

\subparagraph{Fiendish Tongue (Ex):} You gain a +4 profane bonus on Bluff, Diplomacy and Intimidate checks. Against evil outsiders, these bonuses are doubled.

You also gain knowledge of the languages Abyssal and Infernal.

\subparagraph{Lasting Doom (Su):}
As a full-round action, you can render a Dread Knight's Curse already in place on a target permanent, preventing it from running out when its usual duration is up.

At your option, the duration of the curse can be changed to something other than ``Permanent'' by using this ability, such as a year and a day, or a hundred years.

Also, you may set a certain condition that must be met to make the curse expire. This condition may be almost anything you can imagine, but is typically something extraordinarily difficult to bring about, such as slaying a particular great wyrm, scaling the world's tallest mountain, or lifting the king's castle above the clouds.

\subparagraph{Unnatural Resilience (Ex):}  
Your unearthly connections ward you from attacks against your body and soul. 

If you make a successful Will or Fortitude save against an effect that normally has a reduced effect on a successful save, you suffer no ill effects. This includes all spells that have saving throw entries of Will half, Will partial, Fortitude half or Fortitude partial.

This extraordinary ability functions continuously, requiring no special activation.

\subparagraph[Powerful Curse]{Powerful Curse (Su):}
\label{sec:PowerfulCurse}
The penalties imposed by your Dread Knight's Curse increase to -4 (from -2).

You must be an 8th-level Dread Knight to select this power.
% \subparagraph{No Survivors (Ex):}
% Whenever you reduce a foe to negative hit points, you can choose to reduce him to -10 hit points, killing him instantly.

\subsubsection{Ex-Dread Knights:}
A Dread Knight who changes to a good alignment loses his ability to cast Dread Knight spells, the Dark Grace class feature, and all supernatural Dark Powers.
The spellcasting and other abilities remain dormant until he atones (see the \nameref{Spell:Atonement} spell description).


% \subsubsection{Cerebral Disturbance}
% Touch attack, no-save slow.
% \subsubsection{Darkbolt}
% Single-target damage + stun.
% \subsubsection{Rotting Curse}
% Con damage per hour.
\subsection[Spellbreaker]{The Spellbreaker}
\label{sec:Spellbreaker}
\begin{quote}
\emph{``I thirst... for magic!''.}
-Kael, elven Spellbreaker
\end{quote}

\paragraph{Hit Die:} d10
\paragraph{Class skills:}
The Spellbreaker's class skills (and the key ability for each skill) are Balance (Dex), Bluff (Cha), Climb (Str), Concentration (Con), Craft (Int), Diplomacy (Cha), Disguise (Cha), Gather Information (Cha), Intimidate (Cha), Jump (Str), Knowledge (all skills, taken individually) (Int), Listen (Wis), Profession (Wis), Sense Motive (Wis), Spellcraft (Int), Spot (Wis), Swim (Str), Tumble (Dex).

\paragraph{Skill Points at 1st Level:} (4 + Int modifier) $\times$ 4.
\paragraph{Skill Points at each additional Level:} 4 + Int modifier.

%[01:11:51 GMT] Ërnir: Langaði til að sjá ýmislegt frá Spellbreakerunum í WC3. Magic immunity/resistance thingy, feedback/mana burn (lemur caster, hann missir galdramojo og tekur auka skaða), stela buffum/tennisa debuffs til baka, stela summons.
%[01:13:04 GMT] Ërnir: Það sem mig langaði til að bæta við er stuff til að leech-a frá göldrum sem er castað nálægt honum (ó, castaðirðu transmutation galdri? Þá læri ég að fljúga í smá stund), og það að stela concentration göldrum eins og Incantatrix.

\begin{table*}
\caption{The Spellbreaker}
\label{tab:Spellbreaker}
\makebox[\textwidth]{
\begin{tabular}{llccclccc}
\toprule
	&	&	&	&	&					&\multicolumn{3}{c}{Spellcasting}\\ \cmidrule(r){7-9}
Level	&BAB	&Fort 	&Ref 	&Will 	&Special				&SP/day	&Known&Max level\\
\midrule
1st  &+1 		&+2 &+0 &+2	&Siphon Energy, Spell Resistance	&0   &2 &1st \\
2nd  &+2 		&+3 &+0 &+3 	&Bonus Feat				&1   &3 &1st \\
3rd  &+3 		&+3 &+1 &+3 	&Leech Magic +1				&3   &3 &1st \\
4th  &+4 		&+4 &+1 &+4 	&Selective Resistance			&5   &4 &1st \\
5th  &+5 		&+4 &+1 &+4 	&Dispel Magic (standard)		&7   &5 &2nd \\
6th  &+6/+1 		&+5 &+2 &+5 	&Bonus Feat				&11  &6 &2nd \\
7th  &+7/+2 		&+5 &+2 &+5 	&Manipulate Magic			&15  &6 &2nd \\
8th  &+8/+3 		&+6 &+2 &+6 	&Leech Magic +2				&19  &7 &2nd \\
9th  &+9/+4 		&+6 &+3 &+6 	&Steal Control				&23  &8 &3rd \\
10th &+10/+5		&+7 &+3 &+7 	&Bonus Feat				&27  &8 &3rd \\
11th &+11/+6/+1		&+7 &+3 &+7 	&Antimagic Affinity			&35  &9 &3rd \\
12th &+12/+7/+2		&+8 &+4 &+8 	&-					&43  &10&3rd \\
13th &+13/+8/+3		&+8 &+4 &+8 	&Dispel Magic (swift)			&51  &11&4th \\
14th &+14/+9/+4		&+9 &+4 &+9 	&Bonus Feat				&59  &11&4th \\
15th &+15/+10/+5	&+9 &+5 &+9 	&Arcane Vigor				&67  &12&4th \\
16th &+16/+11/+6/+1	&+10&+5 &+10 	&Leech Magic +3				&79  &13&4th \\
17th &+17/+12/+7/+2	&+10&+5 &+10 	&Disjunction				&91  &14&5th \\
18th &+18/+13/+8/+3	&+11&+6 &+11 	&Bonus Feat				&103 &14&5th \\
19th &+19/+14/+9/+4	&+11&+6 &+11 	&Dispel Magic (immediate)		&115 &16&5th \\
20th &+20/+15/+10/+5	&+12&+6 &+12 	&Arcanist's Doom			&127 &16&5th \\
\bottomrule
\end{tabular}}
\end{table*}

\subsubsection{Class Features}
All the following are class features of the Spellbreaker.

\paragraph{Weapon and Armor Proficiency:} 
Spellbreakers are proficient with all simple and martial weapons, with light and medium armor, and with shields (including tower shields, but not exotic shields).

\paragraph{Spell Resistance (Ex):} A Spellbreaker has spell resistance equal to 12 + his class level. If he already has spell resistance from another source (such as from his race), use the highest value with a +2 bonus.

\paragraph{Spells Known:} A Spellbreaker begins play knowing two Spellbreaker spells of your choice.
At the levels indicated on \nameref{tab:Spellbreaker} table, he unlocks the knowledge of a new spell.
Choose the spells known from the full Spellbreaker spell list.
(Exception: The feats Expanded Knowledge and Epic Expanded Knowledge do allow a Spellbreaker to learn spells of other classes, including spells restricted to specialist Wizards.) 

A Spellbreaker can cast any spell he knows that has a spell point cost equal to or lower than his caster level.
The number of times a Spellbreaker can cast spells in a day is limited only by his daily spell points. 
A Spellbreaker simply knows his spells; they are ingrained in his mind, though he must obtain spell points from other spellcasters in order to take advantage of them (see Spell Points, below).
The Difficulty Class for saving throws against Spellbreaker spells is 10 + one-half the number of spell points spent on the spell (round up) + the Spellbreaker's Intelligence modifier.

Spells learned via the Spellbreaker class are arcane spells.
\paragraph{Spell Points:} A Spellbreaker's ability to cast spells is limited by the spell points he has available. 
Unlike most spellcasters, a Spellbreaker can \emph{not} regain spell points simply by concentrating at the start of the day - the only way a Spellbreaker can regain spent spell points is using his class features to leech them from other spellcasters.
His base spell point capacity is given on the \nameref{tab:Spellbreaker} table. In addition, he receives bonus spell points per day if he has a high Intelligence score.
Provided he can obtain the required spell points, a Spellbreaker has no daily limit on the number of spells he can cast.

A multiclass Spellbreaker with levels in other spellcasting classes faces this same restriction - he can never (again) regain spell points by concentration.
\paragraph{Siphon Energy (Su):} Whenever a Spellbreaker successfully deals damage to a spellcasting opponent with a weapon (not a spell, even if the spell is a ray or otherwise requires a touch attack), the Spellbreaker steals some of his spell energy for himself. The opponent loses a number of spell points equal to the Spellbreaker's class level, the Spellbreaker gains a number of spell points equal to the number the opponent lost, and the opponent takes a number of points of damage equal to twice the number of spell points lost. The struck opponent loses spell points and takes damage even if the Spellbreaker is already at his maximum number of spell points.

Alternatively, a Spellbreaker can siphon any number of spell points from a willing spellcaster by touching him as a standard action (the spellcaster knows the number of spell points being siphoned). This use does not deal damage to the ''donating`` spellcaster.

A spellcaster's spell point reserve can not be reduced below zero by the use of this ability.

Using this ability does not require an action, it is activated as part of making the attack (or standard action) in question.
\paragraph{Bonus Feat:} 
At Spellbreaker levels 2nd, 6th, 10th, 14th and 18th, the Spellbreaker may select any Magical feat as a bonus feat. He must meet the feat's prerequisites, if any.
\paragraph{Leech Magic (Su):} Starting at Spellbreaker level 3rd, you emit an aura that takes its toll on all spellcasting in your vicinity. Each spell cast within 30' has its spell point cost increased by 1. This additional spell point does not count towards the spell's augmentation or other potential benefits of increased spell point expenditure (such as the calculation of the spell's save DC), but it does count towards the limit of it being impossible to spend more spell points on a spell than the caster's caster level. This may make it impossible for a spellcaster to cast his highest level spells. You add the additional spell point to your Spell Point reserve, 

Any spellcaster is aware of this dampening effect as soon as he enters the aura. Should he choose to cast a spell anyway, the additional spell point spent is added to your spell point pool.

As a free action, you can prevent the aura from affecting specified spellcasters within range for one round. You must be aware of the spellcaster and his location to do so.

At Spellbreaker level 8th, the spell point cost is instead increased by 2, and at Spellbreaker level 16th, by 3.
\paragraph{Selective Resistance (Ex):} Starting at Spellbreaker level 4th, you can lower your spell resistance as an immediate action. You retain the ability to lower your spell resistance as a standard action, should you desire to do so.

\paragraph{Dispel Magic (Sp):} Starting at Spellbreaker level 5th, a Spellbreaker can use \nameref{Spell:DispelMagic} at will as a standard action. Caster level equals your number of Hit Dice.

At Spellbreaker level 14th, you can use this ability as a swift action three times per day.

At Spellbreaker level 19th, you can use this ability as an immediate action. You can use the ability as a swift action or as an immediate action a combined number of three times per day.
\paragraph{Manipulate Magic (Su):} Starting at Spellbreaker level 7th, whenever you succeed on a dispel check to dispel an ongoing spell effect affecting a creature, you can instead move the spell to a different creature. The spell continues to affect its new recipient, as if it had been cast on that creature originally.

The spell's recipient must be within 30' of both you and the creature it affected originally (the target of the dispelling effect).
The recipient must be a legal target for the spell in question. You can not move spells with a range of Personal. If the spell being moved allows saving throws and/or spell resistance, the recipient is allowed to apply those defenses as normal. In the case of spells that establish a special connection between the spell's target and its caster (for example: \nameref{Spell:Charm} and \nameref{Spell:ShieldOther}), you are thereafter considered to be the spell's caster.

The spell's duration, caster level, augmentations, metamagic effects, and other variable factors are not reset or changed when the spell is moved.
\paragraph{Steal Control (Su):} Starting at Spellbreaker level 9th, whenever you succeed on a dispel check to dispel an ongoing Conjuration (Summoning) spell that summoned a creature, you can instead assume control of the spell. The spell is then not dispelled, continuing to function as normal, with the exception that it now serves you and obeys your commands as it obeyed the spell's original caster.

\paragraph{Antimagic Affinity (Ex):} Starting at Spellbreaker level 11th, you can use the Supernatural abilities granted by the Spellbreaker class inside the area of an \nameref{Spell:AntimagicField} (or in similar locations, such as planes with dead magic). 

You also gain the ability to shield one of your magic items from the effects of an Antimagic Field as a swift action. The shield lasts for one round. While shielded, the item functions normally. If the item leaves your possession for the duration of the effect, the shield instantly ends.

You continue to lose access to your spellcasting, spell-like abilities, and other supernatural abilities while within an Antimagic Field, as normal.
\paragraph{Arcane Vigor (Su):} Starting at Spellbreaker level 15th, whenever a caster fails to affect you due to your spell resistance class feature, you gain temporary hit points equal to twice the number of spell points spent on the spell. These temporary hit points last for a number of rounds equal to your Spellbreaker class level.

\paragraph{Disjunction (Sp):} Starting at Spellbreaker level 17th, a Spellbreaker can use \nameref{Spell:Disjunction} three times per day. Caster level equals your number of Hit Dice.

\paragraph{Arcanist's Doom (Su):} At Spellbreaker level 20th, you learn to forever destroy a spellcaster's ability to do magic.
Using this ability requires touching the spellcaster throughout one minute of concentration. If carried out to completion, the spellcaster instantly loses all of his spell points, and permanently loses his ability to regain them. Only a \nameref{Spell:Wish} or \nameref{Spell:Miracle} spell can restore the spellcaster's ability to regain spell points. \newpage
%\section{New Prestige Classes}
%\subsection{Metamagician}
\textbf{Hit Die:} d4\\
\textbf{Requirements:}
To qualify to become a metamagician, a character must fulfill all the following criteria.
\begin{itemize}
 \item \textbf{Skills:} Concentration 8 ranks, Knowledge (arcana) 8 ranks, Spellcraft 8 ranks.
 \item \textbf{Feats:} Iron Will, at least one metamagic feat
 \item \textbf{Spells:} Ability to cast 3rd-level arcane spells, including at least two abjuration spells
\end{itemize}
\textbf{Class Skills}

The metamagician's class skills (and the key ability for each skill) are Concentration (Con), Craft (Int), 
Knowledge (all skills taken individually) (Int), Profession (Wis), and Spellcraft (Int). 

\textbf{Skill Points at each level:} 2 + Int modifier.
\begin{table}
\centering
\caption{The Metamagician}
\small
\label{tab:Metamagician}
\resizebox{\textwidth}{!}{
\begin{tabular}{|l|l|c|c|c|l|p{4.7cm}|}
\hline
\textbf{Level}&\textbf{BAB}&\textbf{Fort}&\textbf{Ref}&\textbf{Will}&\textbf{Special}&\textbf{Spellcasting}\\
\hline
1st	&+0	&+0	&+0	&+2	&Focused Specialization	&+1 level of existing arcane spellcasting class\\
2nd	&+1	&+0	&+0	&+3	&			&+1 level of existing arcane spellcasting class\\
3rd	&+1	&+1	&+1	&+3	&			&+1 level of existing arcane spellcasting class\\
4th	&+2	&+1	&+1	&+4	&			&+1 level of existing arcane spellcasting class\\
5th	&+2	&+1	&+1	&+4	&Easy Metamagic 1/day	&+1 level of existing arcane spellcasting class\\
6th	&+3	&+2	&+2	&+5	&			&+1 level of existing arcane spellcasting class\\
7th	&+3	&+2	&+2	&+5	&			&+1 level of existing arcane spellcasting class\\
8th	&+4	&+2	&+2	&+6	&			&+1 level of existing arcane spellcasting class\\
9th	&+4	&+3	&+3	&+6	&Easy Metamagic 2/day	&+1 level of existing arcane spellcasting class\\
10th	&+5	&+3	&+3	&+7	&Metamagic Mastery	&+1 level of existing arcane spellcasting class\\
\hline
\end{tabular}}
\normalsize
\end{table}
\subsubsection{Class Features}
All the following are Class Features of the metamagician prestige class.

\textbf{Weapon and Armor Proficiency:} Metamagicians gain no proficiency with any weapon or armor.

\textbf{Spellcasting:} When a new metamagician level is gained, 
the character gains spell points per day, an increase in caster level, and spells known
as if he had also gained a level in whatever arcane spellcasting class in which he could cast 3rd-level spells before he added the prestige class level. 
He does not, however, gain any other benefit a character of that class would have gained. 
If a character had more than one arcane spellcasting class in which he could cast 3rd-level spells before he became a metamagician, 
he must decide to which class he adds each level of metamagician for the purpose of determining what spellcasting class gains the benefit of the spellcasting advancement.

\textbf{Focused Specialization (Ex):}
A metamagician gives up some of his versatility in magic to focus on that which remains to him.
Upon first entering the metamagician class, choose a school of magic to give up.
From this point onward, you may never learn spells from that school of magic.
You can not give up your school of specialization, or the abjuration school.

\textbf{Easy Metamagic (Su):}
At 5th level, the metamagician gains a limited ability to apply metamagic without interrupting his concentration.
Once per day, he may apply a metamagic feat to a spell without expending his magical focus.
At 9th level, he can do so twice per day.

\textbf{Metamagic Mastery (Su):}
At 10th level, the metamagician's ability to manipulate magic has come to its peak.
Whenever the metamagician applies a metamagic feat to a spell, he pays one less spell point for the metamagic adjustment than normal.
Metamagic feats with a cost of one or no spell points are not affected by this ability. \newpage
\section{New Races}
\input{NewRaces.tex} \newpage
\section{New Feats}
\input{NewFeats.tex}\newpage
\section{New Spell Lists}
\subsection{Spellbreaker Spell List}
\subsubsection{1st-Level Spellbreaker Spells}
\begin{list}{\labelitemi}{\leftmargin=1em}
  \item \nameref{Spell:ArcaneFlux}*: Causes spellcaster to lose concentration.
  \item \nameref{Spell:Bloodhound}*: Grants one creature the scent special ability.
  \item \nameref{Spell:DiscernSpellcaster}*: Reveals whether target creature is a spellcaster.
  \item \nameref{Spell:ExpeditiousRetreat}: Your speed increases by 30 ft.
  \item \nameref{Spell:InsightfulStrikes}*: Your magical insight grants you vision enough into the future to strike more effectively.
  \item \nameref{Spell:MagicAura}: Alters object's magic aura.
  \item \nameref{Spell:MagicWeapon}: Weapon gains +1 bonus.
  \item \nameref{Spell:TrueStrike}: +20 on your next attack roll.
  \item \nameref{Spell:WingedWeapon}*: Allows weapon to fly for one round.
\end{list}
\subsubsection{2nd-Level Spellbreaker Spells}
\begin{list}{\labelitemi}{\leftmargin=1em}
  \item \nameref{Spell:Resistance} : Grants a Resistance bonus on saving throws.
  \item \nameref{Spell:ResistEnergy}: Ignores first 10 (or more) points of damage/attack from specified energy type.
  \item \nameref{Spell:Silence}: Negates sound in 20-ft. radius.
  \item \nameref{Spell:SwiftFly}*: Fly for a few moments.
  \item \nameref{Spell:SwiftHaste}*: Act with increased speed for a few moments.
  \item \nameref{Spell:SwiftSeeInvisibility}*: See the invisible for a few moments.
  \item \nameref{Spell:WombatsBoost}: Subject gains +4 to an ability score for 1 min./level.
\end{list}
\subsubsection{3rd-Level Spellbreaker Spells}
\begin{list}{\labelitemi}{\leftmargin=1em}
  \item \nameref{Spell:DispelMagic}: Cancels magical spells and effects.
  \item \nameref{Spell:KeenEdge}: Doubles normal weapon's threat range.
  \item \nameref{Spell:MagicVestment}: Armor or shield gains +2 enhancement bonus.
  \item \nameref{Spell:Nondetection}: Masks object or creature against scrying.
  \item \nameref{Spell:Piggyback}*: Latch on to a creature as it teleports, retaining your relative position.
  \item \nameref{Spell:RemoveCurse}: Frees object or person from curse.
  \item \nameref{Spell:Roguespace}*: Allows you to completely dodge certain effects that allow a reflex save.
  \item \nameref{Spell:SuppressMagic}*: Renders willing spellcaster incapable of casting spells.
\end{list}
\subsubsection{4th-Level Spellbreaker Spells}
\begin{list}{\labelitemi}{\leftmargin=1em}
  \item \nameref{Spell:DetectScrying}: Alerts you of magical eavesdropping.
  \item \nameref{Spell:DimensionDoor}: Teleports you short distance.
  \item \nameref{Spell:DimensionalAnchor}: Bars extradimensional movement.
  \item \nameref{Spell:GlobeOfInvulnerability}: Stops low-powered spell effects.
  \item \nameref{Spell:Stoneskin}: Ignore 7 points of damage per attack.
\end{list}
\subsubsection{5th-Level Spellbreaker Spells}
\begin{list}{\labelitemi}{\leftmargin=1em}
  \item \nameref{Spell:AntimagicField}: Negates magic within 10 ft.
  \item \nameref{Spell:Dismissal}: Forces a creature to return to native plane.
  \item \nameref{Spell:DweomerBlockade}*: You cut a creature off from all magic.
  \item \nameref{Spell:SpellResistance}: Subject gains SR 12 + level.
\end{list}\newpage
\section{New Spells}
\subsection{'A' Spells}
\subsubsection{Arcane Flux}
\label{Spell:ArcaneFlux}
Abjuration
\\ \textbf{Level:} Spellbreaker 1
\\ \textbf{Components:} V
\\ \textbf{Casting Time:} 1 immediate action
\\ \textbf{Range:} Close (25 ft. + 5 ft./2 levels)
\\ \textbf{Target:} One creature
\\ \textbf{Duration:} Instantaneous
\\ \textbf{Saving Throw:} None; see text
\\ \textbf{Spell Resistance:} Yes
\\ \textbf{Spell Points:} 1

\emph{``You cause a momentary disruption in the normal flows of magic surrounding a creature, causing spellcasters trouble.''}

A creature affected by this spell while casting a spell is distracted. It must succeed on a \nameref{sec:Concentration} skill check against the spell's save DC or lose the spell.
\subsection{'C' Spells}
\subsubsection{Call of the Moon}
\label{Spell:CallOfTheMoon}
Enchantment (Compulsion) [Mind-Affecting]
\\ \textbf{Level:} Moon 3
\\ \textbf{Components:} V
\\ \textbf{Casting Time:} 1 standard action
\\ \textbf{Range:} Close (25 ft. + 5 ft./2 levels)
\\ \textbf{Target:} Up to five living creatures, no two of which are more than 30 feet apart
\\ \textbf{Duration:} Instantaneous
\\ \textbf{Saving Throw:} Will partial; see text
\\ \textbf{Spell Resistance:} Yes
\\ \textbf{Spell Points:} 5

\emph{You bring out the image of a full moon in the minds of your subjects.}

A lycanthrope affected by this spell must succeed on a will save or be forced to assume its animal form at the earliest opportunity - usually on its next action. A lycanthrope already in its animal form is not affected.
Even if the lycanthrope succeeds on the will save, you know that the spell had an effect on the subject - that is, you know with absolute certainty that the creature is in fact a lycanthrope.

Non-lycanthropes suffer no ill effect from this spell.
\paragraph{Augment:}
For every additional spell point you spend, this spell can affect an additional target.%, and its saving throw DC increases by 1. 
No target of the spell can be more than 30 feet from another target of the spell.

\subsubsection{Curse of Lycanthropy}
\label{Spell:CurseOfLycanthropy}
Transmutation
\\ \textbf{Level:} Moon 7
\\ \textbf{Components:} V, S
\\ \textbf{Casting Time:} 1 hour
\\ \textbf{Range:} Touch
\\ \textbf{Target:} Humanoid or giant touched
\\ \textbf{Duration:} Instantaneous
\\ \textbf{Saving Throw:} Fortitude negates
\\ \textbf{Spell Resistance:} Yes
\\ \textbf{Spell Points:} 13

\emph{''Welcome to being a monster.''}

The creature becomes an afflicted lycanthrope, of a type of your choosing. You cannot afflict a creature who is already a lycanthrope with a second strain of lycanthropy.
\subsection{'D' Spells}

\subsubsection{Dimensional Barrier}
\label{Spell:DimensionalBarrier}
Abjuration
\\ \textbf{Level:} Spellbreaker 5
\\ \textbf{Range:} 20 ft.
\\ \textbf{Area:} 20-ft.-radius emanation, centered on you
\\ \textbf{Duration:} One round/level (D)
\\ \textbf{Spell Points:} 9

\emph{You emanate a pulsating emerald field of energy, blocking extradimensional travel }

This spell functions as \nameref{Spell:DimensionalLock} (excluding augmentation options), except as noted here.

The spherical field you bring into being is centered on you and moves with you.
\subsubsection{Discern Spellcaster}
\label{Spell:DiscernSpellcaster}
Divination
\\ \textbf{Level:} Spellbreaker 1
\\ \textbf{Components:} V
\\ \textbf{Casting Time:} 1 standard action
\\ \textbf{Range:} Close (25 ft. + 5 ft./2 levels)
\\ \textbf{Target:} One creature
\\ \textbf{Duration:} Instantaneous
\\ \textbf{Saving Throw:} Will negates
\\ \textbf{Spell Resistance:} Yes
\\ \textbf{Spell Points:} 1

\emph{``A WITCH! BURN HER!''}

This spell reveals to you whether the target of the spell has any spellcasting ability. Spell-like and Supernatural abilities do not register.

If the subject resists the spell with a successful Will save or you fail to overcome its spell resistance, the spell reveals no information. Otherwise, the spell reveals an absolute answer (either ``no spellcasting ability'' or ``spellcasting ability'').

\paragraph{Augment:} For every additional spell point you spend, this spell can affect an additional target within range.
\subsection{'F' Spells}
\subsubsection[Swift Fly]{Fly, Swift}
\label{Spell:SwiftFly}
Transmutation
\\ \textbf{Level:} Spellbreaker 2
\\ \textbf{Components:} V, S
\\ \textbf{Casting Time:} 1 swift action
\\ \textbf{Range:} Personal
\\ \textbf{Target:} You
\\ \textbf{Duration:} 2 rounds
\\ \textbf{Spell Points:} 3

\emph{You rise from the ground in a swirl of rapidly burning arcane energy.}

This spell functions as the \nameref{Spell:Fly} spell (excluding augmentation options), except as noted here.

If the spell expires while the target is still aloft, it only floats for one round, rather than 1d6 rounds.

\paragraph{Augment:} For every 2 additional spell points you spend, this spell's duration is increased by one round.
\subsection{'G' Spells}
\subsubsection{Goldskin}
\label{Spell:Goldskin}
Conjuration (Teleportation)
\\ \textbf{Level:} Commerce 4
\\ \textbf{Components:} V, S
\\ \textbf{Casting Time:} 1 standard action
\\ \textbf{Range:} Personal
\\ \textbf{Target:} You
\\ \textbf{Duration:} 10 min./level
\\ \textbf{Spell Points:} 7

\emph{The shimmering golden glow protects you from being molested by proletarians.}

Each time you are struck with a manufactured weapon, the attacker loses 7 gold pieces from those stored on his person.
The gold pieces are transferred on to your person. Extradimensional storage spaces accessed via magic items (such as Bags of Holding) count as being on a character's person for purposes of this spell.

Should the attacker not have 7 gold pieces to spare, the attack deals no damage.

\paragraph{Augment:} You can augment this spell in one or both of the following ways:
\begin{enumerate}
 \item For every additional spell point spent, the attacker loses an additional gold piece per hit.
\end{enumerate}

\subsection{'H' Spells}
\subsubsection{Hand of Midas}
\label{Spell:HandOfMidas}
Transmutation
\\ \textbf{Level:} Commerce 7
\\ \textbf{Components:} V, S
\\ \textbf{Casting Time:} 1 standard action
\\ \textbf{Range:} Touch
\\ \textbf{Target:} One creature OR one object; see text
\\ \textbf{Duration:} Instantaneous
\\ \textbf{Saving Throw:} Fortitude negates OR fortitude negates (object); see text
\\ \textbf{Spell Resistance:} Yes (object)
\\ \textbf{Spell Points:} 13, XP; see text

\emph{``I did not say you would be rich. I said you would be worth a lot!''}

A targeted creature is turned into a mindless, inert statue of gold.
It is unaware of its surroundings, as well as the passage of time. It does not age. Its gear is not affected by the transformation.

The creature and each of its items retains its own hit point total throughout the transformation, and its hardness changes to 5.
Destroying the statue kills the creature.

The transmuted creature is considered a Construct for the purposes of targeting it with spells and other abilities.

An augmented \nameref{Spell:RemoveCurse} spell can restore the transmuted creature.

Alternatively, you can affect a nonmagical object, or part of a larger nonmagical object distinctive enough to be considered an object in its own right (such as a doorknob or a wagon wheel).

\emph{Experience Cost:} 10 XP per pound of the creature or object to be transmuted.

Unlike most spells with an experience cost, you must pay this cost when the spell takes effect, not when the spell is cast.
If expending this amount of experience would cause you to go down a level, you spend no experience and the spell fails (but the action and spell points are spent).

When placing this spell within \nameref{Item:Matrices} or \nameref{Item:Wands}, you must pay the spell's experience cost (per charge, in the case of wands) at the time of crafting. In such a case, the spell fails if not imbued with sufficient experience to transmute a creature or object of the target's weight.
\subsubsection[Swift Haste]{Haste, Swift}
\label{Spell:SwiftHaste}
Transmutation
\\ \textbf{Level:} Spellbreaker 2
\\ \textbf{Components:} V
\\ \textbf{Casting Time:} 1 swift action
\\ \textbf{Range:} Personal
\\ \textbf{Targets:} You
\\ \textbf{Duration:} 2 rounds
\\ \textbf{Spell Points:} 3

This spell functions as the \nameref{Spell:Haste} spell (excluding augmentation options), except as noted here.

\paragraph{Augment:} For every 2 additional spell points you spend, this spell's duration is increased by one round.
\subsection{'I' Spells}
\subsubsection{Insightful Strikes}
\label{Spell:InsightfulStrikes}
Divination
\\ \textbf{Level:} Arcane Archer 1\footnote{Add the spell to the list of spells available to the class through its Advanced Learning class feature.}, Spellbreaker 1
\\ \textbf{Components:} V
\\ \textbf{Casting Time:} 1 swift action
\\ \textbf{Range:} Personal
\\ \textbf{Target:} You
\\ \textbf{Duration:} 1 round
\\ \textbf{Spell Points:} 1

\emph{Your magical foresight allows you to strike far more effectively than mundane warriors can.}

For the duration of this spell, you gain an insight bonus on weapon attacks (including ranged attacks, but not your spells, spell-like abilities, or supernatural abilities) equal to your Intelligence modifier.

\paragraph{Augment:} You can augment this spell in one of the following ways:
\begin{enumerate}
 \item If you spend four additional spell points, the insight bonus becomes equal to twice your Intelligence modifier.
 \item If you spend eight additional spell points, the insight bonus becomes equal to three times your Intelligence modifier.
 \item If you spend twelve additional spell points, the insight bonus becomes equal to four times your Intelligence modifier.
 \item If you spend sixteen additional spell points, the insight bonus becomes equal to five times your Intelligence modifier.
\end{enumerate}

\emph{Special:} Casting this spell does not provoke Attacks of Opportunity.
\subsection{'M' Spells}
\subsubsection{Moon Bolt}
\label{Spell:MoonBolt}
Evocation [Light]
\\ \textbf{Level:} Moon 2
\\ \textbf{Components:} V, S
\\ \textbf{Casting Time:} 1 standard action
\\ \textbf{Range:} Medium (100 ft. + 10 ft./level)
\\ \textbf{Target:} One creature
\\ \textbf{Duration:} Instantaneous
\\ \textbf{Saving Throw:} Fortitude partial
\\ \textbf{Spell Resistance:} Yes
\\ \textbf{Spell Points:} 3

\emph{You blast your enemy with a ball of pure, silvery light.}

The creature takes takes 1d8 points of damage and must succeed on a Fortitude save to avoid being stunned for 1 round. Lycanthropes take a -4 penalty on this saving throw.

\paragraph{Augment:} For every 2 additional spell points you spend, this spell's damage increases by one die (d8).
\subsubsection{Moonlust}
\label{Spell:Moonlust}
Enchantment (Compulsion) [Mind-Affecting]
\\ \textbf{Level:} Moon 2
\\ \textbf{Components:} V, S
\\ \textbf{Casting Time:} 1 standard action
\\ \textbf{Range:} Touch
\\ \textbf{Target:} Creature touched
\\ \textbf{Duration:} 3 rounds
\\ \textbf{Saving Throw:} Will negates
\\ \textbf{Spell Resistance:} Yes
\\ \textbf{Spell Points:} 3

\emph{The creature sits enthralled, staring into the air, as if looking at something no one else can see.}

The caster may \emph{daze} one living creature by making a successful touch attack. 
If the target creature does not make a successful Will save, its mind is clouded and it takes no action for 3 rounds. 
The dazed subject is not stunned (so attackers get no special advantage against it), but it can't move, cast spells, use mental abilities, and so on.

\paragraph{Augment:} For every additional spell point you spend, the subject is dazed for an additional round.
\subsection{'R' Spells}
\subsubsection{Roguespace}
\label{Spell:Roguespace}
Conjuration (Teleportation)
\\ \textbf{Level:} Spellbreaker 3
\\ \textbf{Components:} V
\\ \textbf{Casting Time:} 1 immediate action
\\ \textbf{Range:} Personal
\\ \textbf{Target:} You
\\ \textbf{Duration:} Instantaneous
\\ \textbf{Spell Points:} 5

\emph{You instantaneously slide into a pocket dimension, allowing you to avoid harmful effects.}

When you cast this spell in conjunction with making a successful Reflex save against an attack that normally deals half damage on a successful save, you instead take no damage.

You can cast this spell quickly enough to save yourself if you unexpectedly come within range of a dangerous effect.

\paragraph{Augment:} If you spend 4 additional spell points, you take only half damage even on a failed Reflex save.

\subsection{'S' Spells}

\subsubsection{Sepia Snake Ward}
\label{Spell:SepiaSnakeWard}
Conjuration (Creation) [Force]
\\ \textbf{Level:} Commerce 3, Sor/Wiz 3
\\ \textbf{Target:} One touched pouch or lock

\emph{''I almost feel sorry for the thieves.``}

This spell functions as \nameref{Spell:SepiaSnakeSigil} (including the augmentation option), except as noted here.

When cast on a pouch, the snake strikes those who would attempt to use Sleight of Hand to sneak into the pouch or steal the pouch as a whole using the Sleight of Hand skill.

When cast on a lock, the snake strikes those who would attempt to force it open with the Open Lock skill.

Simply attempting to access the container normally does not trigger the spell.

The hidden ward cannot be detected by normal observation, and \nameref{Spell:DetectMagic} reveals only that the pouch or lock is magical.

\emph{Note:} Magic traps such as Sepia Snake Wards are hard to detect and disable. 
A rogue (only) can use the Search skill to find the runes and Disable Device to thwart them. 
The DC in each case is 25 + spell level, or 28 for a Sepia Snake Ward.

\subsubsection[Swift See Invisibility]{See Invisibility, Swift}
\label{Spell:SwiftSeeInvisibility}
Divination
\\ \textbf{Level:} Spellbreaker 2
\\ \textbf{Components:} V
\\ \textbf{Casting Time:} 1 swift action
\\ \textbf{Range:} Personal
\\ \textbf{Target:} You
\\ \textbf{Duration:} 2 rounds
\\ \textbf{Spell Points:} 3

\emph{Hidden things reveal themselves to you in a flash, but the vision fades rapidly.}

This spell functions as the \nameref{Spell:SeeInvisibility} spell (excluding augmentation options), except as noted here.

\paragraph{Augment:} For every 2 additional spell points you spend, this spell's duration is increased by one round.

\subsubsection{Starfall}
\label{Spell:Starfall}
Evocation [Light]
\\ \textbf{Level:} Moon 9
\\ \textbf{Components:} V, S
\\ \textbf{Casting Time:} 1 standard action
\\ \textbf{Range:} Long (400 ft. + 40 ft./level)
\\ \textbf{Effect:} A shower of Moon Bolts
\\ \textbf{Duration:} 6 rounds
\\ \textbf{Saving Throw:} See text
\\ \textbf{Spell Resistance:} Yes
\\ \textbf{Spell Points:} 17

\emph{You bring down half the sky on top of your enemies, with devastating results.}

Immediately upon the completion of the spell, and at the start of your turn each round of the spell's duration thereafter, all creatures within the spell's range except for those you specifically exclude are subjected to a \nameref{Spell:MoonBolt}, cast as if augmented to cast a number of spell points equal to the number of spell points you spent on the Starfall spell. You do not need line of sight or line of effect to the creatures struck by the bolts, the Starfall seeks them out on its own.
\subsection{'W' Spells}
\subsubsection{Winged Weapon}
\label{Spell:WingedWeapon}
Transmutation
\\ \textbf{Level:} Spellbreaker 1
\\ \textbf{Components:} V
\\ \textbf{Casting Time:} 1 swift action
\\ \textbf{Range:} Touch
\\ \textbf{Target:} One melee weapon
\\ \textbf{Duration:} 1 round
\\ \textbf{Saving Throw:} Will negates (harmless, object)
\\ \textbf{Spell Resistance:} Yes (harmless, object)
\\ \textbf{Spell Points:} 1

\emph{The weapon grows golden, faintly translucent wings that flutter rapidly.}

The touched weapon gains the benefits of the Throwing and Returning enhancements for the duration of the spell.

\paragraph{Augment:} For every 2 additional spell points you spend, this spell's duration is increased by one round.

\subsubsection{Wolfsbane}
\label{Spell:Wolfsbane}
Necromancy
\\ \textbf{Level:} Moon 3
\\ \textbf{Components:} V, S
\\ \textbf{Casting Time:} 1 standard action
\\ \textbf{Range:} 120 ft.
\\ \textbf{Area:} 120-ft. line
\\ \textbf{Duration:} Instantaneous
\\ \textbf{Saving Throw:} Will partial; see text
\\ \textbf{Spell Resistance:} Yes
\\ \textbf{Spell Points:} 5

\emph{The horde of lycanthropes emits a howl of pain.}

Every lycanthrope caught in the area takes 50 points of damage and is \emph{paralyzed} for one round. A successful will save halves the damage and negates the paralysis.\newpage
\section{New Magic}
\input{NewMagic.tex}\newpage\newpage
\part{End notes}
\section{End notes}
\subsection{Thanks}
Thanks go to 
\begin{list}{\labelitemi}{\leftmargin=1em}
 \item Xzoltar @ Competitor Games, who gave the system its first playtesting and me extraordinary amounts of feedback
 \item Swooper @ GitP, who has found more wording inconsistencies and \LaTeX bugs than I'd like to admit
 \item Garryl @ GitP, who gave me the first full readthrough, and as a result about 30 inconsistencies in the text
 \item Seharvepernfan and eggs @ GitP, who have repeatedly showered me with feedback and praise
 \item The players in my current (\today, 2013) campaign, who have actually bothered with reading some of this crap
 \item and everyone else who has commented on the system and listened to me prattle on about it as it was coming up.
\end{list}
\subsection{Version history}
\begin{list}{\labelitemi}{\leftmargin=1em}
 \item January 13th, 2013. Beta 1.10c. Bug fixes and updates, mostly in prestige classes (Dragon Disciple, Eldritch Knight, Loremaster and Mystic Theurge). Length 308 pages.
 \item December 29th, 2012: Beta 1.10b. High elf race added, and racial descriptions expanded. Mystic Theurge class added. The project's source code is now available on \href{https://github.com/Ernir/VancianToPsionics}{Github}, licenced under GPL version 3. Length: 308 pages.
 \item December 25th, 2012: Beta 1.10. Circle magic and magic duels added, as well as a multitude of bug fixes, mostly thanks to Seharvepernfan. Polymorph subschool changed to allow spellcasting while morphed. Font changed from Computer Modern to Times, which resulted in further document shortening. Still working on my Spellbreaker (Anti-mage) class, which is not included. Length: 308 pages.
 \item May 17, 2012: Beta 1.09. The Adept class added. Magic rings have been added. Summon Monster has undergone a paradigm change, and is now split into many, many smaller spells. Further condensation w.r.t. vertical paragraph, item list and section spacing, length now 316 pages.
 \item January 26, 2012: Beta 1.08. The Ranger class and its spell list added. The Arcane Archer, Blackguard (including its spell list), Eldritch Knight, Heirophant and Loremaster prestige classes added. Stick figure drawings (by yours truly) added at random places in the document. The Sun domain added, completing the main list of SRD domains. Many magic items updated. The document has been changed to a two-column layout, and as a result, is now down to 315 pages.
 \item September 23, 2011: Beta 1.07. The Air, Animal, Destruction, Earth, Fire, Luck, Moon, Planes, Plant, Travel, Trickery, and Vermin domains added. The Cleric class is now complete. Length: 440 pages.
 \item August 21, 2011: Beta 1.06. The Bard class added! Knowledge and Protection domains added. A line of fluff has been added to nearly all spells. An example monster (the Ogre Mage) added. The gnome race updated to take into account the changes to its SLAs. Length: 389 pages.
 \item June 12, 2011: Beta 1.05a: More domains (Death, Healing, Strength and War) added.
 Also, I've been adding fluff to spells! Length: 356 pages.
 \item May 21, 2011: Beta 1.05: The Dragon Disciple class added. Cantrips are back in. The first few domains, the alignment domains and the Magic domain, have been added.
 More spells creeping in, several minor errors fixed. Length: 338 pages.
 \item May 4, 2011: Beta 1.04. Cleric class added, although the domains are still missing.
 Assassin class added. The Sorcerer given the Wild Magic class feature. Work on Cleric spells and domains continues.
 Length: 310 pages.
 \item April 24, 2011: Beta 1.03. Paladin class added! Since the project now includes a divine class, project name changed to ``From Vancian to Psionic''
 (from ``From Arcane to Psionic''). A few more wording errors fixed. Length: 288 pages.
 \item April 17, 2011: Beta 1.02. Bug fix release, thanks to Garryl.
 \item April 13, 2011: Beta 1.01. Sorcerer class added! Wizard class slightly changed w.r.t. SP/day in response to the Sorcerer. Heroism spell inconsistency fixed.
 Preperations for the inclusion of the Cleric class have begun.
 Length: 253 pages.
 \item April 5, 2011: Beta 1. Wizard spells of levels 6, 7, 8 and 9 included. The Archmage prestige class included.
 Base system more or less finalized. Polymorph subschool changed. The ``Minion'' type of spells created.
 Length: 249 pages.
 \item January 16th, 2011: Alpha 2. Includes the reinstatement of Familiars, Wizard spells of levels 3, 4, and 5, numerous typo fixes, and more magic items.
 Lots of cleanup with regards to formatting and typos. Length: 177 pages.
 \item December 8th, 2010: Alpha 1. Includes the base system, the Wizard class, Wizard spells of levels 1 and 2, SRD feats,
 and the most basic magic items. Length: 99 pages.
 \item December 3rd, 2010: The project begins.
\end{list}
\subsection{Contact Info}
Problems, questions and concerns can be sent to
\href{mailto:e.ernir@gmail.com}{e.ernir@gmail.com}, or posted publicly at \href{http://forum.faxcelestis.net/viewtopic.php?f=1&t=186&sid=8991c93e2a863857ec84c4674df8481b}{Competitor Games}
or \href{http://www.giantitp.com/forums/showthread.php?t=194002}{Giant in the Playground}.
\subsection{OGL}
This document is made with material from the \href{http://www.wizards.com/default.asp?x=d20/article/srd35}{d20 srd},
in accordance with the Open Game License that follows.
\subsubsection{OPEN GAME LICENSE Version 1.0a}
\label{sec:OGL}
\small
The following text is the property of Wizards of the Coast, Inc. and is Copyright 2000 Wizards of the Coast, Inc ("Wizards"). All Rights Reserved.

\begin{enumerate}\item Definitions: 
\begin{enumerate}
 \item "Contributors" means the copyright and/or trademark owners who have contributed Open Game Content;
 \item "Derivative Material" means copyrighted material including derivative works and translations (including into other computer languages), potation, modification, correction, addition, extension, upgrade, improvement, compilation, abridgment or other form in which an existing work may be recast, transformed or adapted;
 \item "Distribute" means to reproduce, license, rent, lease, sell, broadcast, publicly display, transmit or otherwise distribute; 
 \item "Open Game Content" means the game mechanic and includes the methods, procedures, processes and routines to the extent such content does not embody the Product Identity and is an enhancement over the prior art and any additional content clearly identified as Open Game Content by the Contributor, and means any work covered by this License, including translations and derivative works under copyright law, but specifically excludes Product Identity.
 \item "Product Identity" means product and product line names, logos and identifying marks including trade dress; artifacts; creatures characters; stories, storylines, plots, thematic elements, dialogue, incidents, language, artwork, symbols, designs, depictions, likenesses, formats, poses, concepts, themes and graphic, photographic and other visual or audio representations; names and descriptions of characters, spells, enchantments, personalities, teams, personas, likenesses and special abilities; places, locations, environments, creatures, equipment, magical or supernatural abilities or effects, logos, symbols, or graphic designs; and any other trademark or registered trademark clearly identified as Product identity by the owner of the Product Identity, and which specifically excludes the Open Game Content;
 \item "Trademark" means the logos, names, mark, sign, motto, designs that are used by a Contributor to identify itself or its products or the associated products contributed to the Open Game License by the Contributor
 \item "Use", "Used" or "Using" means to use, Distribute, copy, edit, format, modify, translate and otherwise create Derivative Material of Open Game Content. 
 \item "You" or "Your" means the licensee in terms of this agreement.
\end{enumerate}
 \item \textbf{The License:} This License applies to any Open Game Content that contains a notice indicating that the Open Game Content may only be Used under and in terms of this License. You must affix such a notice to any Open Game Content that you Use. No terms may be added to or subtracted from this License except as described by the License itself. No other terms or conditions may be applied to any Open Game Content distributed using this License.
 \item \textbf{Offer and Acceptance:} By Using the Open Game Content You indicate Your acceptance of the terms of this License.
 \item \textbf{Grant and Consideration:} In consideration for agreeing to use this License, the Contributors grant You a perpetual, worldwide, royalty-free, non-exclusive license with the exact terms of this License to Use, the Open Game Content.
 \item \textbf{Representation of Authority to Contribute:} If You are contributing original material as Open Game Content, You represent that Your Contributions are Your original creation and/or You have sufficient rights to grant the rights conveyed by this License.
 \item \textbf{Notice of License Copyright:} You must update the COPYRIGHT NOTICE portion of this License to include the exact text of the COPYRIGHT NOTICE of any Open Game Content You are copying, modifying or distributing, and You must add the title, the copyright date, and the copyright holder's name to the COPYRIGHT NOTICE of any original Open Game Content you Distribute.
 \item \textbf{Use of Product Identity:} You agree not to Use any Product Identity, including as an indication as to compatibility, except as expressly licensed in another, independent Agreement with the owner of each element of that Product Identity. You agree not to indicate compatibility or co-adaptability with any Trademark or Registered Trademark in conjunction with a work containing Open Game Content except as expressly licensed in another, independent Agreement with the owner of such Trademark or Registered Trademark. The use of any Product Identity in Open Game Content does not constitute a challenge to the ownership of that Product Identity. The owner of any Product Identity used in Open Game Content shall retain all rights, title and interest in and to that Product Identity.
 \item \textbf{Identification:} If you distribute Open Game Content You must clearly indicate which portions of the work that you are distributing are Open Game Content.
 \item \textbf{Updating the License:} Wizards or its designated Agents may publish updated versions of this License. You may use any authorized version of this License to copy, modify and distribute any Open Game Content originally distributed under any version of this License.
 \item \textbf{Copy of this License:} You MUST include a copy of this License with every copy of the Open Game Content You Distribute.
 \item \textbf{Use of Contributor Credits:} You may not market or advertise the Open Game Content using the name of any Contributor unless You have written permission from the Contributor to do so.
 \item \textbf{Inability to Comply:} If it is impossible for You to comply with any of the terms of this License with respect to some or all of the Open Game Content due to statute, judicial order, or governmental regulation then You may not Use any Open Game Material so affected.
 \item \textbf{Termination:} This License will terminate automatically if You fail to comply with all terms herein and fail to cure such breach within 30 days of becoming aware of the breach. All sublicenses shall survive the termination of this License.
 \item \textbf{Reformation:} If any provision of this License is held to be unenforceable, such provision shall be reformed only to the extent necessary to make it enforceable.
 \item \textbf{COPYRIGHT NOTICE} Open Game License v 1.0 Copyright 2000, Wizards of the Coast, Inc.
\end{enumerate}
\subsection{Source code}
This is an \emph{open source} project, released under the \textbf{GNU General Public License, version 3} (GPL-3.0). The source (\LaTeX code and images) is accessible at \href{https://github.com/Ernir/VancianToPsionics}{Github}.
\subsubsection{GPL-3.0}
\input{LICENCE.tex}

\end{multicols}
\end{document}
