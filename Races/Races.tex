\section{Races}
Many creatures have innate magical powers on the racial level, every member of the species possessing some measure of magical talent. None of the common races have powers as pronounced as those of strong magical beings like dragons and rakshasas, but some, like gnomes and some elves, still have sufficient abilities to affect their daily lives and preferred fighting styles.

Unless noted otherwise in this chapter, use the rules text presented in the \href{http://www.wizards.com/default.asp?x=d20/article/srd35}{d20 srd}.\footnote{To summarize the changes to races: Elves gain 2 bonus spell points. Gnomes have their spell-like abilities updated to reflect the changes in the spells they refer to.}
\subsection{Elves}
\subsubsection{Description}
Elves average 5 feet tall and typically weigh just over 100 pounds. They live on fruits and grains, though they occasionally hunt for fresh meat. Elves prefer colorful clothes, usually with a green-and-gray cloak that blends well with the colors of the forest.

Elves speak Elven, and most also know Common and Sylvan.
\subsubsection{Magic}
Elves have a natural talent for magic, the race having spawned many of the world's most famous (and oldest) archmages. The race is especially well known for its numerous Wizards.
\subsubsection{Combat}
Elves are cautious warriors and take time to analyze their opponents and the location of the fight if at all possible, maximizing their advantage by using ambushes, snipers, and camouflage. They prefer to fire from cover and retreat before they are found, repeating this maneuver until all of their enemies are dead.

They prefer longbows, shortbows, rapiers, and longswords. In melee, elves are graceful and deadly, using complex maneuvers that are beautiful to observe. Their wizards often use \nameref{Spell:Sleep} spells during combat because these won't affect other elves.
\subsubsection{Racial Traits}
Elves possess the following racial traits.

\begin{itemize}
 \item +2 Dexterity, -2 Constitution.
 \item Medium size.
 \item An elf's base land speed is 30 feet.
 \item Immunity to sleep spells and effects, and a +2 racial saving throw bonus against enchantment spells or effects.
 \item Low-light vision.
 \item Weapon Proficiency: Elves are automatically proficient with the longsword, rapier, longbow, composite longbow, shortbow, and composite shortbow.
 \item +2 racial bonus on Listen, Search, and Spot checks. An elf who merely passes within 5 feet of a secret or concealed door is entitled to a Search check to notice it as if she were actively looking for it.
 \item Naturally magical: Elves gain 2 bonus spell points at 1st level. This does not grant them the ability to cast spells unless they gain that ability through another source, such as levels in a spellcasting class.
 \item Automatic Languages: Common, Elven. Bonus Languages: Draconic, Gnoll, Gnome, Goblin, Orc, Sylvan.
 \item Favored Class: \nameref{sec:Wizard}.
\end{itemize}
\subsection{Gnomes}
\subsubsection{Description}
Gnomes stand 3 to 3½ feet tall and weigh 40 to 45 pounds. Their skin color ranges from dark tan to woody brown, their hair is fair, and their eyes can be any shade of blue. Gnome males prefer short, carefully trimmed beards. Gnomes generally wear leather or earth tones, though they decorate their clothes with intricate stitching or fine jewelry. Gnomes reach adulthood at about age 40, and they live about 350 years, though some can live almost 500 years.

Gnomes speak their own language, Gnome. Most gnomes who travel outside gnome lands (as traders, tinkers, or adventurers) know Common, while warriors in gnome settlements usually learn Goblin.
\subsubsection{Magic}
Gnomes have specific affinity with illusion magic, and the ability to communicate with burrowing animals. Some are able to use Cantrips.

Most gnome spellcasters are Bards. Those with a more scholarly bent usually become Illusionists.
\subsubsection{Combat}
Gnomes prefer misdirection and deception over direct confrontation.

They would rather befuddle or embarrass foes (other than goblinoids or kobolds) than kill them.

Gnomes make heavy use of illusion magic and carefully prepared ambushes and traps whenever they can.
\subsubsection{Racial Traits}
Gnomes have the following racial traits: 

\begin{itemize}
 \item +2 Constitution, -2 Strength.
 \item Small: As a Small creature, a gnome gains a +1 size bonus to Armor Class, a +1 size bonus on attack rolls, and a +4 size bonus on Hide checks, but he uses smaller weapons than humans use, and his lifting and carrying limits are three-quarters of those of a Medium character.
 \item Gnome base land speed is 20 feet.
 \item Low-Light Vision: A gnome can see twice as far as a human in starlight, moonlight, torchlight, and similar conditions of poor illumination. He retains the ability to distinguish color and detail under these conditions.
 \item Weapon Familiarity: Gnomes may treat gnome hooked hammers as martial weapons rather than exotic weapons.
 \item +2 racial bonus on saving throws against illusions.
 \item Add +1 to the Difficulty Class for all saving throws against illusion spells cast by gnomes. This adjustment stacks with those from similar effects.
 \item +1 racial bonus on attack rolls against kobolds and goblinoids.
 \item +4 dodge bonus to Armor Class against monsters of the giant type. Any time a creature loses its Dexterity bonus (if any) to Armor Class, such as when it's caught flat-footed, it loses its dodge bonus, too.
 \item +2 racial bonus on Listen checks.
 \item +2 racial bonus on Craft (alchemy) checks.
 \item Automatic Languages: Common and Gnome. Bonus Languages: Draconic, Dwarven, Elven, Giant, Goblin, and Orc. In addition, a gnome can speak with a burrowing mammal (see below).
 \item Spell-Like Abilities: 1/day-\nameref{Spell:ConverseWithNature}. Caster level 1st (regardless of character level). This spell-like ability only allows the gnome to communicate with burrowing mammals, not any kind of animal. This spell-like ability qualifies gnomes as magical creatures, but still does not allow them to obtain a magical focus unless they obtain an actual spell point reserve.
 \item Cantrips: A gnome who obtains a spell point reserve (gnomes frequently have the \nameref{Feat:MagicalSpark} feat for \nameref{Spell:Ventriloquism}) can use \nameref{sec:Cantrips} as a Wizard can, albeit in a limited fashion. A gnome cannot use his racial Cantrip to deal damage, to decipher inscriptions, or to increase his reading speed. A gnome who gains Cantrips as a class feature can access all of its functions, as normal.
 \item Favored Class: \nameref{sec:Bard}.
\end{itemize}
