\subsection{Spiritwalker Orc}
\subsubsection{Description}
Spiritwalker orcs are physically similar to their more common brethren, although they tend to be slightly less thick of chest. Most favor unobtrusive, earth-colored clothing when not dressed for war. % Those who brave interacting with them will find that they possess, if not keen minds, at least surprisingly honed senses and trust in intuition.

Like other orcs, spiritwalker orcs speak Orc. If given the opportunity, they are likely to learn Common, Sylvan, or Terran.
\subsubsection{Magic}
Spiritwalker orcs have a great affinity for many forms of divine magic thanks to their high Wisdom. In addition, each spiritwalker orc has a natural ability to communicate with a certain type of animal. According to the spiritwalker orcs themselves, this is due to them not finding it very hard to talk to their ancestors, even if said ancestors have been reborn in beastial form.
\subsubsection{Combat}
Spiritwalker orcs are a proud and stubborn lot - if insulted or mistreated, they usually attempt to regain face by humiliating the offender in a duel or contest of strength. 

If forced to full-out battle, they prefer overwhelming assaults, seeking to end the fight quickly. They rarely retreat or flee once committed to a fight.
\subsubsection{Racial Traits}
Spiritwalker orcs possess all orc racial traits, with additions and exceptions as noted below.
\begin{itemize}
 \item +2 Strength, -2 Intelligence, +2 Wisdom, -2 Charisma: Spiritwalker orcs are less brutish than some others of their kind, and more in tune with their surroundings.
 \item No Light Sensitivity: Spiritwalker Orcs have adapted to life outside underground dungeons.
 \item Spell-Like Ability: 1/day - \nameref{Spell:ConverseWithNature}. Caster level 1st (regardless of character level). This spell-like ability only allows the spiritwalker orc to communicate with one certain kind of animal (selected at character creation), usually an animal that serves as the totem animal of the orc's tribe or family.
\end{itemize}