Additions to the spellcasting system beyond simple psionic conversion can be found here.
\subsection{Modifications to the System}

\subsubsection{A New Descriptor}
\paragraph[Curse]{Curse} 
\subsection{Circle Magic}
\label{sec:CircleMagic}
Circle Magic is the term used to describe spellcasting performed through a special bond formed between two or more spellcasters.

A magic Circle is formed when a spellcaster with the ability to serve as a \nameref{Feat:CircleMagicLeader} wills it into existence as a standard action. Any spellcasting character can then join this proto-Circle as a standard action. The abilities of the Circle Leader limit the potential number of Circle members, as well as the distance between them. A Circle Leader can choose not to accept spellcasters to his Circle, causing the attempt to join the Circle to fail.

After forming or joining the Circle, each member (including the Circle Leader) must spend a free action at the start of his turn to maintain the connection for another round, or lose the bond to the Circle. While in the Circle, a spellcaster must make a \nameref{sec:Concentration} check or lose the bond should he be damaged or otherwise distracted.
Should the Circle Leader lose concentration or stop maintaining the Circle, the Circle immediately ends unless another member has the ability to be a Circle Leader. If such a member exists, that member becomes a Circle Leader in its stead. If the Circle contains two or more members so qualified, they must immediately and unanimously choose a successor amongst themselves, or the Circle ends.

When a spellcaster forms or joins a Circle, he commits one or more spell points to a shared pool. While the spellcaster remains a member of the Circle, he can choose to use either his own spell points or those from the shared pool to fuel his spellcasting.
A Circle member who leaves the Circle regains a number of spell points equal to the number of spell points in the shared pool divided by the number of members (prior to the member leaving), up to a maximum equal to the number of points that member originally committed. The shared pool loses the same number of spell points as the leaving member regains.

Likewise, when a spellcaster forms or joins a Circle, he makes one or more of his spells known available for use by the Circle. Any member of the Circle can then use this spell as if it were their own known spell (note that not all spells may be usable by all Circle members, such as due to caster level and spell point spending restrictions, or alignment restrictions). The spells so shared remain available to the Circle for as long as the member who contributed them remains a member of the Circle.
\subsection{Mage Duels}
\label{sec:MageDuels}
Mage duels are a way for spellcasters to test their mettle against one another.

\paragraph{Starting a Mage Duel:} To start a mage duel, a spellcaster issues a challenge to another spellcaster as a standard action. The challenged must then accept the challenge as a standard action for the duel to begin.

As soon as the duel begins, make a Test of Power (see below).

The challenger and the challenged must be able to perceive one another, but the duel can be performed at any distance.

You can never start a mage duel with a nonspellcaster or an unwilling spellcaster (Exception: see \nameref{Feat:DeadlyDuelist}).
\paragraph{Duel Checks:} Repeatedly in a mage duel, the contestants must make opposed \emph{duel checks}. A duel check is an ability check, using that spellcaster's key ability modifier. In addition, a spellcaster may spend spell points on a one-to-one basis to increase his duel check. The choice of whether to spend spell points must be made before the duel check is rolled. As when casting spells, a spellcaster may never spend more spell points on a given duel check than his caster level.

A natural 20 or natural 1 is not an automatic success or failure on a duel check. On a tie, roll again (using the same spell point investiture) until the tie is resolved. You can forfeit a duel check that you do not wish to win.
\paragraph{Dueling Consequences:} While you're in a mage duel, your ability to attack others and defend yourself is limited.
\begin{itemize}
 \item \textbf{No Threatened Squares:} You don't threaten any squares.
 \item \textbf{No Dexterity Bonus:} You lose your Dexterity bonus to AC (if you have one).
 \item \textbf{No Movement:} You can't move.
\end{itemize}
\paragraph{If You're Dueling:} When you are dueling (regardless of who started the duel), you can perform the following actions.
\begin{itemize}
 \item \textbf{Cast a Spell:} You can attempt to cast a spell while dueling, provided its casting time is no more than 1 standard action and it either has no verbal or somatic components or its components have been suppressed.
 \item \textbf{Quit the Duel:} You can break off the duel by winning a duel check against your opponent as a standard action.
 \item \textbf{Test of Power:} Make an opposed duel check as a standard action. If you win, you deal 1d6 points of nonlethal damage plus your key ability modifier to your opponent. If you lose, you take 1d6 points of nonlethal damage plus your opponent's key ability modifier.
\end{itemize}
If the duelists are rendered incapable of perceiving one another at any point, the duel ends immediately.
\paragraph{Multiple Duelists:} A spellcaster can duel any number of opponents simultaneously. Resolve the duels separately.
\paragraph{Appearance:} A mage duel is evidenced by visual clues that can be identified with a DC 10 Knowledge (arcana) or Spellcraft check. These clues may be subtle or obvious, depending on the personalities, power and specializations of the spellcasters involved in the duel.

\subsection{New Companion Types}
\subsection{Undead Lackey}
\label{sec:UndeadLackey}
An Undead Lackey is a loyal skeleton or zombie granted a measure of intelligence by a spellcaster with the \nameref{Feat:UndeadLackey} feat.

The Undead Lackey is based on one of the creatures on the following list: bat, cat, rat, raven, wolf, or human. If a human is selected, it is based on a 1st-level Warrior with three ability scores of 10 and three of 11.
Only a normal, unmodified creature may become an Undead Lackey.

See below for details on how familiars work.
\subsubsection{Undead Lackey Statistics}
To create an Undead Lackey, apply the skeleton or zombie template to the base creature, with the following exceptions:
\begin{itemize}
 \item \textbf{Hit Dice:} An Undead Lackey becomes a creature with a number of hit dice equal to the master's number of levels in spellcasting classes (levels of different spellcasting classes stack), 
 regardless of how many hit dice the original creature had. These hit dice are undead hit dice, with its base attack bonus and base saving throws being modified accordingly.
 When the master gains additional levels in a spellcasting class, the Undead Lackey gains additional magical beast hit dice.
 \subitem \textbf{Hit Points:} Undead Lackeys gain hit points as characters do, gaining the maximum possible number of hit points at first HD, 
 and rolling thereafter. As an undead creature, an Undead Lackey does not add any Constitution bonus to its hit points.
 \item \textbf{Ability Scores:} An Undead Lackey's mental ability scores are all changed to 10, rather than as normal for the skeleton or zombie template.
 It gains ability score increases as its number of HD increases as any other creature does.
 \item \textbf{Skills:} The Undead Lackey gains 4 skill points per HD, as normal for creatures of the undead type with an intelligence score. The Undead Lackey's master may rearrange the base creature's skill ranks when the Undead Lackey is raised. 
 The Undead Lackey's ``class'' skills are the same as that of its master. The Undead Lackey retains any racial skill bonuses it may have.
 \item \textbf{Feats:} The Undead Lackey's master may rearrange the base creature's feats (other than racial bonus feats) when the creature is summoned. The Undead Lackey gains a bonus feat from the chosen template (Improved Initiative for Skeletons and Toughness for Zombies) as normal.
 The Undead Lackey gains additional feats as it gains extra hit dice, as other creatures do. The Undead Lackey may choose any feat for which it qualifies, including special \nameref{sec:CompanionFeats}.
\end{itemize}

An Undead Lackey can speak one language of its master's choice (so long as it is a language the owner knows). 
An Undead Lackey can understand all other languages known by its master, but cannot speak them. This is a supernatural ability. 
\subsubsection{Raising an Undead Lackey}
Raising an Undead Lackey requires the expenditure of magical components costing 100 GP, and performing a ritual on the corpse of the base creature that takes 1 hour to complete.
\subsubsection{Undead Lackeys and Destruction}
Resurrecting or replacing a destroyed Undead Lackey requires this same ritual as raising one (including the material cost). Resurrecting an Undead Lackey requires access to the remains of the destroyed creature.

This ``resurrection'' or replacement fully restores it, it does not have experience to lose as characters do, its abilities are a function of its master.

If a spellcaster who is the master of an Undead Lackey dies, the companion is destroyed, becoming an inert corpse.
\subsubsection{Dismissing an Undead Lackey}
A master may dismiss his Undead Lackey at any time. The Undead Lackey then immediately de-animates, becoming a normal corpse.

The Undead Lackey itself, despite its sentience, is forever fully loyal to its master, and is incapable of choosing to sever ties with its master.
\subsubsection{Vermin Companion}
\label{sec:VerminCompanion}
A vermin companion is the trusty, non-intelligent sidekick of a character with the \nameref{Feat:VerminCompanion} feat.

This vermin is a monstrous centipede, scorpion, or spider (or another monstrous vermin of similar power, at the GM's discretion).

A vermin companion functions precisely as a \nameref{sec:AnimalCompanion} does, with the following exceptions:
\begin{itemize}
 \item Body and size: A vermin companion's base size depends on its number of hit dice. Use the base statistics of the largest monstrous vermin of the appropriate type whose unmodified number of hit dice is at most equal to the vermin companion's number of hit dice. For example, a spider companion with 1 HD is small (and uses the Monstrous Spider, Small statistics block), a spider companion with 2 hit dice is medium, a spider companion with 3 to 4 HD is large, and so on.
 \item \textbf{Type:} Replace all references to the animal type with corresponding references to the vermin type.
 \item \textbf{Ability Scores:} A vermin companion retains its own ability scores.
 It does \emph{not} gain ability score increases as it gains additional hit dice. Use the base ability scores of a monstrous vermin of the creature's current size.
 \item \textbf{Skills:} A vermin companion does not have any skill ranks. It retains any racial skill bonuses it may have. 
 \item \textbf{Feats:} A vermin companion does not have any feats.
\end{itemize}
Vermin companions are non-intelligent. They can follow very simple instructions issued by their master (such as ``go over there'' and ``don't eat the horse''), but otherwise revert to their natural behaviour.

Bonding a vermin companion involves finding the creature in question, and weaning it to the master's presence through 1 hour of interaction.