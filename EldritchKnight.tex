\subsection{Eldritch Knight}
\begin{quote}
\emph{``But of course, I am a master of the spell \textbf{and} the blade...''}
- Kymrel, elven Eldritch Knight
\end{quote}
An Eldritch Knight is an arcane spellcaster who uses his magic to strengthen his more mundane fighting prowess.

\begin{table*}
\centering
\caption{The Eldritch Knight}
\label{tab:EldritchKnight}
\begin{tabular}{|l|l|c|c|c|l|l|}
\hline
\textbf{Level}&\textbf{BAB}&\textbf{Fort}&\textbf{Ref}&\textbf{Will}&\textbf{Special}&\textbf{Spellcasting}\\
\hline
1st	&+1	&+2	&+0	&+0	&Spellstrike		&-\\
2nd	&+2	&+3	&+0	&+0	&-			&+1 level of existing class\\
3rd	&+3	&+3	&+1	&+1	&-			&+1 level of existing class\\
4th	&+4	&+4	&+1	&+1	&Arcane Vigor		&+1 level of existing class\\
5th	&+5	&+4	&+1	&+1	&-			&+1 level of existing class\\
6th	&+6	&+5	&+2	&+2	&-			&+1 level of existing class\\
7th	&+7	&+5	&+2	&+2	&Greater Spellstrike	&+1 level of existing class\\
8th	&+8	&+6	&+2	&+2	&-			&+1 level of existing class\\
9th	&+9	&+6	&+3	&+3	&-			&+1 level of existing class\\
10th	&+10	&+7	&+3	&+3	&Effortless Spellstrike	&+1 level of existing class\\
\hline
\end{tabular}
\end{table*}

\paragraph{Hit Die:} d6
\paragraph{Requirements:}
To qualify to become an Eldritch Knight, a character must fulfill all the following criteria.
\subparagraph{Weapon Proficiency:} Must be proficient with all martial weapons.
\subparagraph{Skills:} Spellcraft 4 ranks.
\subparagraph{Feat:} \nameref{Feat:ContinuedTraining}.
\subparagraph{Spells:} Able to cast third-level arcane spells.
\paragraph{Class Skills}
The Eldritch Knight's class skills (and the key ability for each skill) are Concentration (Con), Craft (Int), Decipher Script (Int), Jump (Str), Knowledge (arcana) (Int), Knowledge (nobility and royalty) (Int), Ride (Dex), Sense Motive (Wis), Spellcraft (Int), and Swim (Str).
\paragraph{Skill Points at each level:} 4 + Int modifier.

\subsubsection{Class Features}
All of the following are features of the eldritch knight prestige class.

\paragraph{Weapon and Armor Proficiency:} Eldritch Knights gain no proficiency with any weapon or armor.

\paragraph{Spellcasting:} From 2nd level on, when a new Eldritch Knight level is gained, the character gains spell points per day, an increase in caster level, spells known and maximum available spell level as if he had also gained a level in whatever spellcasting class in which he could cast 3rd level spells before he added the prestige class level.
He does not, however, gain any other benefit a character of that class would have gained. 
If a character had more than one applicable spellcasting class before he became an Eldritch Knight, he must decide to which class he adds each level of Eldritch Knight for the purpose of determining what spellcasting class gains the benefit of the spellcasting advancement.

\paragraph{Spellstrike (Su):}
At 1st level, an Eldritch Knight gains the ability to combine a weapon attack with the casting of a targeted spell (a targeted spell is a spell with an entry of ``target'' in its spell description).
The spell may not have a casting time longer than one standard action, and may be of a level no higher than your Eldritch Knight class level (to a natural maximum of 9th level spells).

To use this ability, make a single normal weapon attack as a standard action. If the attack hits, you may expend your magical focus to cast the spell upon the target as a free action. If you have the ability to make multiple attacks as part of a single standard action, that ability works as normal, but you still only get to cast the spell once.

Aside from the action required to cast it, the spell functions as normal. You pay its spell point cost, its range limitations apply, and casting in melee provokes attacks of opportunity.

\paragraph{Arcane Vigor (Su):}
At Eldritch Knight level 4th, whenever you use your Spellstrike or Greater Spellstrike ability (see below) ability, you gain a number of temporary hit points equal to the number of spell points you spent on casting the spell.
You need not spend an action to activate this ability, it happens automatically whenever you use Spellstrike or Greater Spellstrike.
These temporary hit points last for one minute.
Temporary hit points from Arcane Vigor stack with temporary hit points from other sources.

\paragraph{Greater Spellstrike (Su):}
At Eldritch Knight level 7th, you can perform a Greater Spellstrike.
A Greater Spellstrike works as Spellstrike, above, except that instead of casting the spell after making a single normal attack as a standard action, you cast it after making a full attack as a full-round action. The spell then affects every target you hit during the full attack, even if the spell would normally only affect a single target. However, no target is affected more than once, even if you hit a target multiple times during your full attack.

This does not replace your Spellstrike ability.

\paragraph{Effortless Spellstrike (Ex):}
At Eldritch Knight level 10th, you no longer need to expend your magical focus in order to use your Spellstrike and Greater Spellstrike abilities.