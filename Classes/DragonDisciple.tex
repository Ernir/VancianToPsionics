\subsection{Dragon Disciple}
\begin{quote}
\emph{Stay silent, or risk invoking the anger of the dragon.}
- Kalas, human (by birth) Dragon Disciple
\end{quote}
A Dragon Disciple is a dragon-descended spellcaster realizing the power of his heritage.
\begin{table*}
\centering
\caption{The Dragon Disciple}
\small
\label{tab:DragonDisciple}
\makebox[\textwidth]{
\begin{tabular}{llcccll}
\toprule
Level	&BAB	&Fort 	&Ref 	&Will 	&Special					&Spellcasting\\
\midrule
1st	&+0	&+2	&+0	&+2	&Natural armor (+1)				&+1 level of existing class\\
2nd	&+1	&+3	&+0	&+3	&Ability boost (Str +2), claws and bite		&+1 level of existing class\\
3rd	&+2	&+3	&+1	&+3	&Breath weapon					&+1 level of existing class\\
4th	&+3	&+4	&+1	&+4	&Ability boost (Str +2), natural armor(+2)	&+1 level of existing class\\
5th	&+3	&+4	&+1	&+4	&Blindsense 30 ft.				&+1 level of existing class\\
6th	&+4	&+5	&+2	&+5	&Ability boost (Con +2)				&+1 level of existing class\\
7th	&+5	&+5	&+2	&+5	&Form of the Dragon, Natural armor (+3)		&+1 level of existing class\\
8th	&+6	&+6	&+2	&+6	&Ability boost (Int +2)				&+1 level of existing class\\
9th	&+6	&+6	&+3	&+6	&Wings						&+1 level of existing class\\
10th	&+7	&+7	&+3	&+7	&Blindsense 60 ft., dragon apotheosis		&+1 level of existing class\\
\bottomrule
\end{tabular}}
\normalsize
\end{table*}
\paragraph{Hit Die:} d12
\paragraph{Requirements:}
To qualify to become a Dragon Disciple, a character must fulfill all the following criteria.
\subparagraph{Type:} Any nondragon (cannot already be a half-dragon).
\subparagraph{Skills:} Knowledge (arcana) 8 ranks.
\subparagraph{Languages:} Draconic.
\subparagraph{Spellcasting:} Ability to cast arcane spells.
\subparagraph{Special:} The player chooses a dragon variety when taking the first level in this prestige class.
\paragraph{Class Skills}
The Dragon Disciple's class skills (and the key ability for each skill) are Concentration (Con), Craft (Int), Diplomacy (Cha), 
Escape Artist (Dex), Gather Information (Cha), Intimidate (Cha), Knowledge (all skills, taken individually) (Int), 
Listen (Wis), Profession (Wis), Search (Int), Speak Language (None), Spellcraft (Int), and Spot (Wis).
\paragraph{Skill Points at each level:} 4 + Int modifier.
\subsubsection{Class Features}
All of the following are class features of the Dragon Disciple prestige class.

\paragraph{Weapon and Armor Proficiency:}
Dragon Disciples gain no proficiency with any weapon or armor.

\paragraph{Spell Points/Day and Max Spell Level:} 
When a new Dragon Disciple level is gained, the character gains spell points per day, and an increase in caster level and maximum available spell level as if he had also gained a level in whatever arcane spellcasting class in which he could cast spells before he added the prestige class level. 
He does not, however, gain any other benefit a character of that class would have gained, nor does he gain spells known. 
If a character had more than one arcane spellcasting class in which he could cast arcane spells before he became a Dragon Disciple, 
he must decide to which class he adds each level of Dragon Disciple for the purpose of determining what spellcasting class gains the benefit of the spellcasting advancement.

\paragraph{Natural Armor Increase (Ex):}
At 1st, 4th, and 7th level, a Dragon Disciple gains an increase to the character's existing natural armor (a character with no natural armor bonus has a natural armor bonus of +0), as indicated on \nameref{tab:DragonDisciple} table (the numbers represent the total increase gained to that point). 
As his skin thickens, a Dragon Disciple takes on more and more of his progenitor's physical aspect.

\paragraph{Claws and Bite (Ex):}
At 2nd level, a Dragon Disciple gains two claw attacks and a bite attack if he does not already have such attacks. 
His claw attacks deal 1d4 points of damage, and his bite attack deals 1d6 points of damage (assuming a medium Dragon Disciple).
A Dragon Disciple who already has claw or bite attacks uses whichever damage values are greater.

A Dragon Disciple is considered proficient with these attacks. When making a full attack, a Dragon Disciple uses his full base attack bonus with his bite attack but takes a -5 penalty on claw attacks. 
The Multiattack feat reduces this penalty to only -2.

\paragraph{Ability Boost (Ex):}
As a Dragon Disciple gains levels in this prestige class, his ability scores increase as noted on \nameref{tab:DragonDisciple} table.

These increases stack and are gained as if through level advancement.

\paragraph{Blindsense (Ex):}
At 5th level, the Dragon Disciple gains blindsense with a range of 30 feet.
Using nonvisual senses the Dragon Disciple notices things it cannot see. 
He usually does not need to make Spot or Listen checks to notice and pinpoint the location of creatures within range of his blindsense ability, 
provided that he has line of effect to that creature.

Any opponent the Dragon Disciple cannot see still has total concealment against him, and the Dragon Disciple still has the normal miss chance when attacking foes that have concealment. 
Visibility still affects the movement of a creature with blindsense. 
A creature with blindsense is still denied its Dexterity bonus to Armor Class against attacks from creatures it cannot see. 
At 10th level, the range of this ability increases to 60 feet.
\paragraph{Breath Weapon (Su):}
At 3rd level, a Dragon Disciple gains a breath weapon.
In order to use it, he must expend his magical focus as a standard action.
The type and shape depend on the dragon variety whose heritage he enjoys (see The Dragon Disciple Breath Weapon table). 
Regardless of the ancestor, the breath weapon deals 1d6 of damage of the appropriate energy type per arcane caster level (if the Dragon Disciple has multiple arcane caster levels, use the highest).

The DC of the breath weapon is 10 + class level + Con modifier.

A line-shaped breath weapon is 5 feet high, 5 feet wide, and 60 feet long. A cone-shaped breath weapon is 30 feet long.

\begin{tableonecolumn}
%\centering
\caption{The Dragon Disciple Breath Weapon}
\label{tab:DragonDiscipleBreathWeapon}
\begin{tabular}{ll}
\toprule
Dragon Variety$^1$&Breath Weapon\\
\midrule
Black		&Line of acid\\
Blue		&Line of lightning\\
Green		&Cone of corrosive gas (acid)\\
Red		&Cone of fire\\
White		&Cone of cold\\
Brass		&Line of fire\\
Bronze		&Line of lightning\\
Copper		&Line of acid\\
Gold		&Cone of fire\\
Silver		&Cone of cold\\
\bottomrule
\end{tabular}
\begin{enumerate}
 \item Other varieties of Dragon Disciple are possible, using other dragon varieties as ancestors.
\end{enumerate}
\end{tableonecolumn}

\paragraph{Form of the Dragon:}
At 7th level, the Dragon Disciple gains \nameref{Spell:FormDragon} as a bonus spell known.
If he does not have the ability to learn or cast 6th level spells when he gains this ability, he learns it immediately when he can do so.
If he already knows Form of the Dragon when reaching 7th level, he can select another spell in its stead he would have qualified for at the time he learned Form of the Dragon.
\paragraph{Wings (Ex):}
At 9th level, a Dragon Disciple grows a set of draconic wings. 
He may now fly at a speed equal to his normal land speed, with average maneuverability.
\paragraph{Dragon Apotheosis:}
At 10th level, a Dragon Disciple's transformation is complete. His type changes to dragon, and gains the augmented subtype (as normal).
This does not cause any loss of class features, despite the Dragon Disciple now no longer fulfilling the class preprequisites.
He gains +4 to Strength and +2 to Charisma.
His natural armor bonus increases to +4, and he acquires low-light vision, 60-foot darkvision, immunity to sleep and paralysis effects, and immunity to the energy type used by his breath weapon.