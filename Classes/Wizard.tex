\subsection[Wizard]{The Wizard}
\label{sec:Wizard}
\begin{quote}
\emph{Behold your fate, creatures of darkness! 
Your demise is at hand, for I wield arcane power beyond your feeble goblin reasoning! 
The forces of the very cosmos are mine to command, and yet still you cannot comprehend the dark dismal end in store for you and your wicked compatriots. 
Nay! Your little brains can only leave you gasping in horror as I bend reality to my very will. 
The magic I wield is capable of rending asunder the universe--nay, the whole of the multiverse, and in fact is wasted on such pitiful creatures as yourself. 
But I shall bring it to bear nonetheless, and you shall rue the day I chose to wreak such unimaginable havoc on your lives with the sheer volume of my arcane works. 
And lo, in days and years to come, when the children come to play in the smoking crater that once held your den of evil, 
they shall know nothing of your wicked ways but all shall feel the echoes of the power spent here today.}
- \href{http://www.giantitp.com/comics/oots0010.html}{Vaarsuvius}, elven Wizard
\end{quote}
A Wizard is a learned user of arcane magic.
\paragraph{Alignment:} Any
\paragraph{Hit Die:} d4
\paragraph{Class skills:}
The Wizard's class skills (and the key ability for each skill) are Concentration (Con), 
Knowledge (all skills, taken individually) (Int), Profession (Wis), and Spellcraft (Int). 
In addition, a Wizard gains access to additional class skills based on his specialization (see below):
\subparagraph{Abjurer:} Diplomacy (Cha), Heal (Wis), Speak Language (N/A), and Survival (Wis)
\subparagraph{Conjurer:} Appraise (int), Craft (Int), Disable Device (Int), and Forgery (Int)
\subparagraph{Diviner:} Decipher Script (Int), Gather Information (Cha), Listen (Wis), and Spot (Wis).
\subparagraph{Enchanter:} Bluff (Cha), Diplomacy (Cha), Gather Information (Cha) and Sense Motive (Wis).
\subparagraph{Evoker:} Autohypnosis (Wis), Disable Device (Int), Intimidate (Cha) and Tumble (Dex).
\subparagraph{Illusionist:} Disguise (Cha), Forgery (Int), Hide (Dex), and Move Silently (Dex)
\subparagraph{Necromancer:} Bluff (Cha), Disguise (Cha), Heal (Wis), and Sense Motive (Wis).
\subparagraph{Transmuter:} Balance (Dex), Climb (Str), Jump (Str) and Swim (Str).

\paragraph{Skill Points at 1st Level:} (2 + Int modifier) $\times$ 4.
\paragraph{Skill Points at each additional Level:} 2 + Int modifier.
\begin{table*}
\centering
\caption{The Wizard}
\label{tab:Wizard}
\makebox[\textwidth]{
\begin{tabular}{llccclccc}
\toprule
	&	&	&	&	&				&\multicolumn{3}{c}{Spellcasting}\\ \cmidrule(r){7-9}
Level	&BAB	&Fort 	&Ref 	&Will 	&Special			&SP/day	&Known&Max level\\
\midrule
1st	&+0	&+0	&+0	&+2	&Bonus feat, cantrips, 			&3	&2	&1st\\
	&	&	&	&	&specialization				&	&	&\\
2nd	&+1	&+0	&+0	&+3	&-					&6	&4	&1st\\
3rd	&+1	&+1	&+1	&+3	&-					&10	&6	&2nd\\
4th	&+2	&+1	&+1	&+4	&-					&16	&7	&2nd\\
5th	&+2	&+1	&+1	&+4	&Bonus feat				&24	&8	&3rd\\
6th	&+3	&+2	&+2	&+5	&-					&33	&10	&3rd\\
7th	&+3	&+2	&+2	&+5	&-					&43	&11	&4th\\
8th	&+4	&+2	&+2	&+6	&-					&55	&12	&4th\\
9th	&+4	&+3	&+3	&+6	&-					&69	&14	&5th\\
10th	&+5	&+3	&+3	&+7	&Bonus feat				&84	&15	&5th\\
11th	&+5	&+3	&+3	&+7	&-					&100	&16	&6th\\
12th	&+6/+1	&+4	&+4	&+8	&-					&118	&18	&6th\\
13th	&+6/+1	&+4	&+4	&+8	&-					&138	&19	&7th\\
14th	&+7/+2	&+4	&+4	&+9	&-					&159	&20	&7th\\
15th	&+7/+2	&+5	&+5	&+9	&Bonus feat				&181	&22	&8th\\
16th	&+8/+3	&+5	&+5	&+10	&-					&205	&23	&8th\\
17th	&+8/+3	&+5	&+5	&+10	&-					&231	&24	&9th\\
18th	&+9/+4	&+6	&+6	&+11	&-					&258	&26	&9th\\
19th	&+9/+4	&+6	&+6	&+11	&-					&286	&27	&9th\\
20th	&+10/+5	&+6	&+6	&+12	&Bonus feat				&316	&28	&9th\\
\bottomrule
\end{tabular}}
\end{table*}
\subsubsection{Class Features}
All the following are class features of the Wizard.

\paragraph{Weapon and Armor Proficiency:} Wizards are proficient with the club, dagger, heavy crossbow, light crossbow, quarterstaff, and shortspear. 
They are not proficient with any type of armor or shield. Armor does not, however, interfere with the casting of spells.

\paragraph{Spell Points/Day:} A Wizard's ability to cast spells is limited by the spell points he has available. 
His base daily allotment of spell points is given on \nameref{tab:Wizard} table. 
 In addition, he receives bonus spell points per day if he has a high Intelligence score.
%\footnote{\textbf{Variant: Sorcerer} 
% 
% Some users of magic are not the careful, studious folk that populate most mage guilds - they are simply born.
% These mages are called Sorcerers. Their magic comes intuitively, and thus they use their Charisma score rather than their Intelligence score
% to determine their spells' saving throw DCs, their bonus spell points, and the maximum spell level they have access to.
% 
% Since Sorcerers can devote their time to study things other than magic, they receive (4 + Int modifier) $\times$ 4 skill points at first level, 
% and 4 + Int modifier skill points at each level thereafter, as well as proficiency with all simple weapons.}.
His race may also provide bonus spell points per day, as may certain feats and items.

\paragraph{Specialization:} Every Wizard must decide at 1st level which school of magic he will specialize in. 
Choosing a specialization provides a Wizard with access to the class skills associated with that school (see above), as well as the spells restricted to that school. 
However, choosing a discipline also means that the Wizard cannot learn spells that are restricted to other schools. 
He can't even use such spells by employing magical items.

\paragraph{Spells Known:} A Wizard begins play knowing two Wizard spells of your choice. 
Each time he achieves a new level, he unlocks the knowledge of new spells.
Choose the spells known from the Sorcerer/Wizard spell list, and/or from the list of spells only available to Wizards of his specialization. He may not learn spells from the spell list of another specialization. (Exception: The feats Expanded Knowledge and Epic Expanded Knowledge do allow a Wizard to learn spells of other spell lists, including spell lists of other specialist wizards, and even the lists belonging to other classes.) 
A Wizard can cast any spell he knows that has a spell point cost equal to or lower than his caster level. A Wizard whose spells known have been progressed by a prestige class can continue to select spells of his specialization.
The number of times a Wizard can cast spells in a day is limited only by his daily spell points. 
A Wizard simply knows his spells; they are ingrained in his mind, though he must get a good night's sleep each day to regain all his spent spell points.
The Difficulty Class for saving throws against Wizard spells is 10 + one-half the number of spell points spent on the spell (round up) + the Wizard's Intelligence modifier. 

Spells learned via the Wizard class are arcane spells.
\paragraph{Maximum Spell Level Known:} A Wizard begins play with the ability to learn 1st-level spells. 
As he attains higher levels, a Wizard may gain the ability to master more complex spells, as shown on \nameref{tab:Wizard} table.
To learn or cast a spell, a Wizard must have an Intelligence score of at least 10 + the spell's level.
\paragraph{Bonus Feats:} A Wizard gains a bonus feat at 1st level, 5th level, 10th level, 15th level, and 20th level. 
This feat must be a magical feat, a metamagic feat, an item creation feat or the Familiar feat.
These bonus feats are in addition to the feats that a character of any class gains every three levels. 

\paragraph[Cantrips]{Cantrips (Su):}
\label{sec:Cantrips}
Cantrips are minor tricks that novice spellcasters use for practice.
Using a cantrip requires no expenditure of spell points.
You can perform one as a standard action whenever you are magically focused.
This does not expend your focus.

The cantrips you can perform are:
\begin{itemize}
 \item Lifting and moving up to 5 pounds of items from a distance.
This requires concentrating on the cantrip (a standard action).
You can manipulate the moved items as if you were using one hand for the task.
 \item Coloring, cleaning, or soiling items in a 1-foot cube.
 \item Chilling, warming, or flavoring 1 pound of nonliving material. 
 \item Lighting an unattended object within 30' on fire.
 \item Dimly illuminating a 5-foot radius around you, like a candle. 
The light emitted can be of any color, and usually appears as a small globe hovering near you. 
 \item Dealing 1d3 points of cold, electricity, fire or acid damage to a target within 30' by succeeding on a ranged touch attack against it.
 \item Creating small objects that look crude and artificial. 
The materials created by a cantrip are extremely fragile, and they cannot be used as tools or weapons. 
 \item Inscribing your personal rune or mark, which can consist of no more than six characters. 
The writing can be visible or invisible. 
You etch the rune upon any substance without harm to the material upon which it is placed. 
If an invisible mark is made, a \nameref{Spell:DetectMagic} spell causes it to glow and be visible.
\nameref{Spell:SeeInvisibility}, \nameref{Spell:TrueSeeing}, a gem of seeing, or a robe of eyes likewise allows the user to see an invisible mark of this kind. 
Unlike other cantrips, this one is permanent (but is removable by either a \nameref{Spell:DispelMagic} spell or a rigorous mundane cleaning).
 \item Magically deciphering magical inscriptions on objects - books, \nameref{Item:Scrolls}, weapons, and the like - that would otherwise be unintelligible.
 \item Increasing your reading speed to 250 pages per minute.
\end{itemize}

Cantrips lack the power to duplicate any other spell effects.
They cannot inflict status conditions, or affect the concentration of spellcasters.

Any actual change to an object (beyond just moving, cleaning, or soiling it, or creating a personal mark) persists only 1 hour.
\subsubsection{Magical Schools}
A school is one of eight groupings of spells, each defined by a common theme. 
The eight schools are Abjuration, Conjuration, Divination, Enchantment, Evocation, Illusion, Necromancy and Transmutation. 
The schools are described in detail in section \ref{sec:MagicalSchools}, and summarized below.

\paragraph{Abjuration:} 
Spells that protect, block, or banish. An abjuration specialist is called an abjurer.
\paragraph{Conjuration:}
Spells that bring creatures or materials to the caster. A conjuration specialist is called a conjurer.
\paragraph{Divination:}
Spells that reveal information. A divination specialist is called a diviner.
\paragraph{Enchantment:} 
Spells that imbue the recipient with some property or grant the caster power over another being. An enchantment specialist is called an enchanter.
\paragraph{Evocation:}
Spells that manipulate energy or create something from nothing. An evocation specialist is called an evoker.
\paragraph{Illusion:}
Spells that alter perception or create false images. An illusion specialist is called an illusionist.
\paragraph{Necromancy:} 
Spells that manipulate, create, or destroy life or life force. A necromancy specialist is called a necromancer.
\paragraph{Transmutation:}
Spells that transform the recipient physically or change its properties in a more subtle way. A transmutation specialist is called a transmuter.