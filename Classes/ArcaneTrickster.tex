\subsection{Arcane Trickster}
\begin{quote}
\emph{``I'm dishonest, and a dishonest man you can always trust to be dishonest. Honestly. It's the honest ones you want to watch out for.''}
- Ruby, gnome Arcane Trickster
\end{quote}
An Arcane Trickster is a spellcaster who has more than a healthy interest in mischief, subterfuge and larceny.
\paragraph{Hit Die:} d4
\paragraph{Requirements:}
To qualify to become an Arcane Trickster, a character must fulfill all the following criteria.
\subparagraph{Skills:} Bluff 8 ranks, Disable Device 8 ranks, Escape Artist 8 ranks, Knowledge (arcana) 4 ranks.
\subparagraph{Spells:} At least one arcane spell of 3rd level or higher.
\subparagraph{Special:} Sneak Attack +2d6 OR Sneak Attack +1d6 and \nameref{Feat:SilentSpell} and \nameref{Feat:StillSpell}.
\paragraph{Class Skills}
The Arcane Trickster's class skills (and the key ability for each skill) are Appraise (Int), Balance (Dex), Bluff (Cha), Climb (Str), Concentration (Con), Craft (Int), Decipher Script (Int), Diplomacy (Cha), Disable Device (Int), Disguise (Cha), Escape Artist (Dex), Gather Information (Cha), Hide (Dex), Jump (Str), Knowledge (all skills taken individually) (Int), Listen (Wis), Move Silently (Dex), Open Lock (Dex), Profession (Wis), Search (Int), Sense Motive (Wis), Sleight of Hand (Dex), Speak Language (None), Spellcraft (Int), Spot (Wis), Swim (Str), Tumble (Dex), and Use Rope (Dex).
\paragraph{Skill Points at each level:} 4 + Int modifier.

\begin{table*}
\centering
\caption{The Arcane Trickster}
\label{tab:ArcaneTrickster}
\makebox[\textwidth]{
\begin{tabular}{|l|l|c|c|c|l|l|}
\hline
\textbf{Level}&\textbf{BAB}&\textbf{Fort}&\textbf{Ref}&\textbf{Will}&\textbf{Special}&\textbf{Spellcasting}\\
\hline
1st	&+0	&+0	&+2	&+0	&Arcane Ability, Cantrips&+1 level of existing class\\
2nd	&+1	&+0	&+3	&+0	&Sneak Attack +1d6	&+1 level of existing class\\
3rd	&+2	&+1	&+3	&+1	&Arcane Ability		&+1 level of existing class\\
4th	&+3	&+1	&+4	&+1	&Sneak Attack +2d6	&+1 level of existing class\\
5th	&+3	&+1	&+4	&+1	&Arcane Ability		&+1 level of existing class\\
6th	&+4	&+2	&+5	&+2	&Sneak Attack +3d6	&+1 level of existing class\\
7th	&+5	&+2	&+5	&+2	&Arcane Ability		&+1 level of existing class\\
8th	&+6	&+2	&+6	&+2	&Sneak Attack +4d6	&+1 level of existing class\\
9th	&+6	&+3	&+6	&+3	&Arcane Ability		&+1 level of existing class\\
10th	&+7	&+3	&+7	&+3	&Sneak Attack +5d6	&+1 level of existing class\\
\hline
\end{tabular}
}
\end{table*}

\subsubsection{Class Features}
All the following are Class Features of the Arcane Trickster prestige class.

\paragraph{Weapon and Armor Proficiency:} Arcane Tricksters gain no proficiency with any weapon or armor.

\paragraph{Spellcasting:} When a new Arcane Trickster level is gained, the character gains spell points per day, an increase in caster level, spells known and maximum available spell level as if he had also gained a level in whatever arcane spellcasting class in which he could cast a 3rd-level arcane spell before he added the prestige class level.
He does not, however, gain any other benefit a character of that class would have gained. 
If a character had more than one arcane spellcasting class in which he could cast a 3rd-level arcane spell before he became a Arcane Trickster, he must decide to which class he adds each level of Arcane Trickster for the purpose of determining what spellcasting class gains the benefit of the spellcasting advancement.

\paragraph[Cantrips]{Cantrips (Su):} 
An Arcane Trickster can use \nameref{sec:Cantrips} as a \nameref{sec:Wizard} can.

\paragraph{Arcane Ability:}
Like Rogues, Arcane Tricksters learn a few tricks to set them apart from the rest.
At 1st level, and at every other level beyond that (3th, 5th, 7th, and 9th), the Arcane Trickster learns a new arcane ability. Unless otherwise noted, an arcane ability can only be selected once. Some arcane abilities have specific prerequisites.

\subparagraph{Component Switch (Su):} By expending his magical focus, an Arcane Trickster with this ability can replace a spell's verbal and somatic components with those of another spell of the same level he knows. This has no effect on the spell's results, but anyone using \nameref{sec:Spellcraft} to identify the spell by its components will be mislead.
This has no effect on anyone attempting to use \nameref{sec:Spellcraft} (or more obvious clues) to identify the spell by its effects.
This ability does not require specific activation, it is activated as part of casting the spell in question.

\subparagraph{Confounding Attack (Su):} By expending his magical focus, an Arcane Trickster with this ability can force an opponent he has successfully Sneak Attacked to succeed on a Will save (DC 10 + the Arcane Trickster's class level + the Arcane Trickster's Int modifier) or be \emph{confused} for five rounds.
An immunity or resistance to enchantment, mind-affecting, compulsion or confusion effects applies against this ability.
The Sneak Attack must deal damage for this ability to be usable. The decision of whether to use the ability is made after the Arcane Trickster knows whether the Sneak Attack was effective.
This ability does not require a specific activation action, its activation is a part of making the attack in question.
You must be a 7th-level Arcane Trickster to select this ability.

\subparagraph{Dramatic Disappearance (Su):} By expending his magical focus, an Arcane Trickster with this ability can create a diversion to hide (see the Bluff skill description) as a swift action rather than a standard action.

\subparagraph{Implausible Escape (Su):} By expending his magical focus, an Arcane Trickster with this ability can gain a +20 competence bonus on an Escape Artist check.
This ability does not require specific activation, it is activated as part of using the skill.

\subparagraph{Impromptu Sneak Attack (Su):} By expending his magical focus, an Arcane Trickster with this ability can declare one melee or ranged attack he makes to be a sneak attack (the target can be no more than 30 feet distant if the impromptu sneak attack is a ranged attack). The target of an impromptu sneak attack loses any Dexterity bonus to AC, but only against that attack. The power can be used against any target, but creatures that are not subject to critical hits take no extra damage (though they still lose any Dexterity bonus to AC against the attack). This ability does not require specific activation, it is activated as part of making the attack in question.
You must be a 3rd-level Arcane Trickster to select this ability.

\subparagraph{Ranged Legerdemain (Su):} While magically focused, an Arcane Trickster with this ability can perform the following skills at a range of 30 feet: Disable Device, Open Lock, or Sleight of Hand. An Arcane Trickster cannot take 10 on this check. Any object to be manipulated must weigh 5 pounds or less. He must have at least 1 rank in the skill being used.
This ability does not require specific activation, it is activated as part of using the skill in question.

\subparagraph{Surprise Spell (Su):} When an Arcane Trickster casts an illusion or enchantment spell that affects a foe who would be denied a Dexterity bonus to AC against the Arcane Trickster, the spell's save DC is increased by 2.
This ability is active continuously.
You must be a 5th-level Arcane Trickster to select this ability.

\paragraph{Sneak Attack:}
This is exactly like the Rogue ability of the same name. 
The extra damage dealt increases by +1d6 every other level (2nd, 4th, 6th, 8th, and 10th). 
If an Arcane Trickster gets a Sneak Attack bonus from another source the bonuses on damage stack.