\section{Feats}
\subsection{Magical feats}
Magical feats are available only to characters and creatures with the ability to cast spells. 
(In other words, they either have a spell point reserve or have spell-like abilities.)
Because magical feats are supernatural abilities - a departure from the general rule that feats do not grant supernatural abilities - 
they cannot be disrupted in combat (as spells can be) and generally do not provoke attacks of opportunity (except as noted in their descriptions). 
Supernatural abilities are not subject to spell resistance and cannot be dispelled; however, they do not function in areas where magic is suppressed, 
such as in an antimagic field. Leaving such an area immediately allows magical feats to be used.
Many magical feats can be used only when you are magically focused; others require you to expend your magical focus to gain their benefit. 
Expending your magical focus does not require an action; it is part of another action (such as using a feat). 
When you expend your magical focus, it applies only to the action for which you expended it.

\subsection{Metamagic Feats}
As a caster's knowledge of magic grows, he can learn to cast spells in ways slightly different from how the spells were originally designed or learned. 
Of course, casting a spell while using a metamagic feat is more expensive than casting the spell normally.
\begin{itemize}
\item Casting Time:
Spells cast using metamagic feats take the same time as casting the spells normally unless the feat description specifically says otherwise.
\item Casting Cost:
To use a metamagic feat, a caster must both expend his magical focus (see \nameref{sec:MagicFocus}) 
and pay an increased spell point cost as given in the feat description.
\item Limits on Use: 
As with all spells, you cannot spend more spell points on a spell than your caster level. 
Metamagic feats merely let you cast spells in different ways; they do not let you violate this rule. 
\item Effects of Metamagic Feats on a Spell: 
In all ways, a metamagic spell operates at its original spell level, even though it costs additional spell points. 
The modifications to a spell made by a metamagic feat have only their noted effect on the spell. 
A caster can't use a metamagic feat to alter a spell being cast from a matrix, wand, or other device.
\end{itemize}
Casting a spell modified by the Quicken Spell feat does not provoke attacks of opportunity.
Some metamagic feats apply only to certain spells, as described in each specific feat entry.

\emph{Magical Items and Metamagic Spells:} With the right item creation feat, you can store a metamagic spell in a matrix or wand. 
%Level limits for potions apply as if the spell point increase actually raised the level of the spell.
A character doesn't need the appropriate metamagic feat to activate an item in which a metamagic spell is stored, but does need the metamagic feat to create such an item.

Cost and other limits for matrices and wands apply as if the containing spell were of a level equal to the spell's original level plus one-half the spell point cost increase of the metamagic feat (if any) plus one. For example, a matrix of Extended \nameref{Spell:AlignedProtection} would cost as a spell of level $1 + 0/2 + 1 = 2$, and a matrix of Empowered \nameref{Spell:Fireball} would cost as a spell of level $2 + 2/2 + 1 = 4$.

You can not apply more than one metamagic feat to a spell within a matrix or wand.
\subsection{Companion Feats}
\label{sec:CompanionFeats}
The companions of spellcasters have focuses different from those of most adventurers, and a unique magical link to their master that grants them special powers.
Some of these are represented by companion feats. Only companions (familiars, spellstaffs, animal companions, and celestial mounts, but not their hosts) can take these feats.

When the description of a companion feat refers to the master's level, it refers to his number of levels in spellcasting classes.

\subsection{Leadership Feats}
\label{sec:LeadershipFeats}
A feat that grants the character a powerful, trusty companion is usually denoted as a Leadership feat.

A character may only have one Leadership feat at a time, even if he qualifies for multiple feats.\footnote{\textbf{Is Leadership a Leadership feat?} If the Leadership feat is allowed in your game, yes, it most likely should be.}